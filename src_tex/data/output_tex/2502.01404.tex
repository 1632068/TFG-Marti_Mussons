\documentclass[10pt]{amsart}
\usepackage{geometry}
\usepackage{mathtools}
\usepackage[utf8]{inputenc}
\usepackage[T1]{fontenc}
\usepackage{amsmath}
\usepackage{amsfonts}
\usepackage{amssymb}
\usepackage{mathtools}
\usepackage{extarrows}
\usepackage{amsthm}
\usepackage{mathrsfs}
\usepackage[mathcal]{eucal}
\usepackage{graphicx}
\usepackage{xcolor}
\usepackage[a-1b]{pdfx}
\usepackage[british]{babel}
\usepackage{tikz-cd}
\usepackage{csquotes}
\usepackage[style=alphabetic, backend=biber]{biblatex}
\usepackage[cmtip, all]{xy}
\usepackage{accents}

\usetikzlibrary{decorations.pathmorphing}
\addbibresource{bibliography.bib}
\theoremstyle{definition}
\newtheorem{defn}{Definition}[section]
\newtheorem{constr}[defn]{Construction}
\newtheorem{exmp}[defn]{Example}
\newtheorem{rmk}[defn]{Remark}
\newtheorem{univprop}[defn]{Universal property}
\newtheorem{exrc}[defn]{Exercise}
\newtheorem{defnprop}[defn]{Definition/Proposition}
\theoremstyle{plain}
\newtheorem{introthm}{Theorem}
\newtheorem{introprop}[introthm]{Proposition}
\newtheorem{thm}[defn]{Theorem}
\newtheorem{prop}[defn]{Proposition}
\newtheorem{corollary}[defn]{Corollary}
\newtheorem{lemma}[defn]{Lemma}
\numberwithin{equation}{section}
%%
\newcommand{\com}[1]{{\color{red}#1}}
\newcommand{\delete}[1]{{\color{green}#1}}
%%
\newcommand{\0}{\emptyset}
\newcommand{\sA}{{\mathcal A}}
\newcommand{\sB}{{\mathcal B}}
\newcommand{\sC}{{\mathcal C}}
\newcommand{\sD}{{\mathcal D}}
\newcommand{\sE}{{\mathcal E}}
\newcommand{\sF}{{\mathcal F}}
\newcommand{\sG}{{\mathcal G}}
\newcommand{\sH}{{\mathcal H}}
\newcommand{\sI}{{\mathcal I}}
\newcommand{\sJ}{{\mathcal J}}
\newcommand{\sK}{{\mathcal K}}
\newcommand{\sL}{{\mathcal L}}
\newcommand{\sM}{{\mathcal M}}
\newcommand{\sN}{{\mathcal N}}
\newcommand{\sO}{{\mathcal O}}
\newcommand{\sP}{{\mathcal P}}
\newcommand{\sQ}{{\mathcal Q}}
\newcommand{\sR}{{\mathcal R}}
\newcommand{\sS}{{\mathcal S}}
\newcommand{\sT}{{\mathcal T}}
\newcommand{\sU}{{\mathcal U}}
\newcommand{\sV}{{\mathcal V}}
\newcommand{\sW}{{\mathcal W}}
\newcommand{\sX}{{\mathcal X}}
\newcommand{\sY}{{\mathcal Y}}
\newcommand{\sZ}{{\mathcal Z}}
% Sonderbuchstaben mit Doppellinie
\newcommand{\A}{{\mathbb A}}
\newcommand{\B}{{\mathbb B}}
\newcommand{\C}{{\mathbb C}}
%\renewcommand{\D}{{\mathbb D}}
\newcommand{\E}{{\mathbb E}}
\newcommand{\F}{{\mathbb F}}
\newcommand{\G}{{\mathbb G}}
\renewcommand{\H}{{\mathbb H}}
%\renewcommand{\I}{{\mathbb I}}
\newcommand{\J}{{\mathbb J}}
\renewcommand{\L}{{\mathbb L}}
\newcommand{\M}{{\mathbb M}}
\newcommand{\N}{{\mathbb N}}
\renewcommand{\P}{{\mathbb P}}
\newcommand{\Q}{{\mathbb Q}}
\newcommand{\R}{{\mathbb R}}
\newcommand{\T}{{\mathbb T}}
\newcommand{\U}{{\mathbb U}}
\newcommand{\V}{{\mathbb V}}
\newcommand{\W}{{\mathbb W}}
\newcommand{\X}{{\mathbb X}}
\newcommand{\Y}{{\mathbb Y}}
\newcommand{\Z}{{\mathbb Z}}
\newcommand{\KQ}{{\operatorname{KQ}}}
\newcommand{\MGL}{{\operatorname{MGL}}}
\newcommand{\KGL}{{\operatorname{KGL}}}
\newcommand{\MSL}{{\operatorname{MSL}}}
\newcommand{\MSp}{{\operatorname{MSp}}}
\newcommand{\BGL}{{\operatorname{BGL}}}
\newcommand{\BSp}{{\operatorname{BSp}}}
\newcommand{\GL}{{\operatorname{GL}}}
\newcommand{\Sp}{{\operatorname{Sp}}}
\newcommand{\Gr}{{\operatorname{Gr}}}
\newcommand{\HGr}{{\operatorname{HGr}}}
\newcommand{\HP}{{\operatorname{HP}}}
\newcommand{\Spec}{{\operatorname{Spec}}}
\newcommand{\equals}{\mathop{=}} 
\newcommand{\GW}{{\operatorname{GW}}} 
\newcommand{\sGW}{{\mathcal{GW}}} 
\newcommand{\SH}{{\operatorname{SH}}} 
\newcommand{\DM}{{\operatorname{DM}}}
\newcommand{\Th}{{\operatorname{Th}}}
\newcommand{\Spc}{{\operatorname{Spc}}}
\newcommand{\Sm}{{\operatorname{Sm}}}
\newcommand{\CH}{{\operatorname{CH}}}
\newcommand{\Hom}{{\operatorname{Hom}}}
\newcommand{\Map}{{\operatorname{Map}}}
\renewcommand{\th}{{\operatorname{th}}} 
\renewcommand{\deg}{{\operatorname{deg}}}
\renewcommand{\dim}{{\operatorname{dim}}} 
\renewcommand{\top}{{\operatorname{top}}}
\newcommand{\Mod}{{\operatorname{Mod}}}
\newcommand{\rnk}{{\operatorname{rank}}}
\newcommand{\Aut}{{\operatorname{Aut}}}
\newcommand{\Sym}{{\operatorname{Sym}}}
\newcommand{\id}{{\operatorname{id}}}
\newcommand{\op}{{\operatorname{op}}}
\newcommand{\Zar}{{\operatorname{Zar}}}
\newcommand{\colim}{{\operatorname*{colim}}}
\newcommand{\Anan}{{\operatorname{Anan}}}
\newcommand{\Ext}{{\operatorname{Ext}}}
\newcommand{\cone}{{\operatorname{cone}}}
\newcommand{\CF}{{\operatorname{CF}}}
\newcommand{\chr}{{\operatorname{char}}}
\newcommand{\tors}{{\operatorname{tors}}}
\newcommand{\Laz}{{\mathbb{L} \text{az}}}
\newcommand{\Cor}{{\operatorname{Cor}}}
\newcommand{\Sch}{{\operatorname{Sch}}}
\newcommand{\Nat}{{\operatorname{Nat}}}
\newcommand{\Obj}{{\operatorname{Obj}}}
\newcommand{\Ker}{{\operatorname{Ker}}}
\newcommand{\Locfree}{{\operatorname{LocFree}}}
\newcommand{\SPP}{{\mathcal{S}\text{p}}}
\newcommand{\hs}{{\operatorname{hs}}}
\newcommand{\Fl}{{\operatorname{Fl}}}
\newcommand{\MU}{{\operatorname{MU}}}




\title{Generators of the Algebraic Symplectic Bordism Ring}
\date{\today}
\author{Pietro Gigli}
\address{Universit\"at Duisburg-Essen,
Fakult\"at Mathematik, Campus Essen, 45117 Essen, Germany}
\email{pietro.gigli@uni-due.de}

\begin{document}


\begin{abstract}
    Algebraic symplectic cobordism is the universal symplectically oriented cohomology theory for schemes, represented by the motivic commutative ring spectrum MSp constructed by Panin and Walter. The graded algebraic diagonal $\MSp^*$ of the coefficient ring of MSp is unknown. Through a symplectic version of the Pontryagin-Thom construction, one can associate any symplectic variety $X$ with a symplectic class $[X]_\MSp$ in $\MSp^{-\dim X}$. Still, the problem in using these classes to study the ring $\MSp^*$ is the paucity of non-trivial examples of symplectic varieties. We modify this construction to obtain elements in $\MSp^*$ from a large family of varieties that are not symplectic but carry a certain "symplectic twist". Then, using a strategy relying on the Adams spectral sequence for MSp, we find a criterion to select generators among these classes, after taking a completion along the motivic Hopf map $\eta$.
\end{abstract}

\thanks{The author was partially supported by the DFG through the grant  LE 2259/8-1 and by the ERC through the project QUADAG.  This paper is part of a project that has received funding from the European Research Council (ERC) under the European Union's Horizon 2020 research and innovation programme (grant agreement No. 832833).\\
\includegraphics[scale=0.08]{ERC_logo.pdf}}

\maketitle

\tableofcontents

\section{Introduction}

The content of this paper is the author's Ph.D thesis. The main goal of this work is to obtain a partial computation of the coefficient ring of algebraic symplectic cobordism, the universal symplectically oriented cohomology theory for schemes. A complete description of the coefficient ring is at present unknown.

\subsection*{Motivation}

We work over a perfect field $k$ of characteristic not 2. For us, a cohomology theory is more precisely a bigraded cohomology theory $\sE^{*,*}(-)$ for smooth $k$-schemes, and is represented by a commutative monoid object $\sE$ in the stable homotopy category $\SH(k)$ (see \S \ref{subsection:cohomologyTheories}). The category $\SH(k)$ \cite[Section 5]{voe:homotopy_theory} is the motivic analogue of the stable homotopy category of topological spectra $\SH$ (\cite{Boardman:Spectra}, \cite[Part III, Section 2]{Adams:homotopy}). If $\sE^{*,*}(-)$ is a cohomology theory, then for each smooth $k$-scheme $X$, the graded ring $\sE^{*,*}(X)$ is a $\sE^{*,*}(\Spec k)$-algebra. The graded ring $\sE^{*,*}:=\sE^{*,*}(\Spec k)$ is called the coefficient ring of $\sE$ (Definition \ref{defn:coefficientRing}).

The notion of a $\GL$-oriented cohomology theory, introduced in  \cite{Panin:pushforward}, is an algebraic version of the notion of a complex oriented cohomology theory, extensively studied in topology (\cite[Part II]{Adams:homotopy}, \cite[Chapter 1]{ravenel:cobordism}). In short, a cohomology theory $\sE^{*,*}(-)$ is $\GL$-oriented if it carries a system of Thom classes for vector bundles, that is, for each vector bundle $V$ over a smooth $k$-scheme $X$ we have an element $\th(V) \in \sE^{2\rnk V,\rnk V}(\Th(V))$ in the cohomology of the Thom space of $V$, in such a way that natural compatibilities are satisfied (see Definition \ref{defn:thomclasstheory}). Oriented cohomology theories are exactly the ones having a pushforward construction for proper maps (Construction \ref{constr:properpushforward}), as well as a Chern class theory and an associated formal group law (see \S \ref{section:Chern}). 

Voevodsky's algebraic cobordism spectrum $\MGL \in \SH(k)$ \cite[Section 6.3]{voe:homotopy_theory} is the most natural analogue of the complex cobordism spectrum $\MU \in \SH$ \cite[Chapter 1, Section 5]{C-F:Cobordism} in our setting (see also Levine-Morel's algebraic cobordism $\Omega^*(-)$ \cite{LevMor:Cob}). $\MGL$ has the structure of a commutative monoid \cite[Section 3]{Vezzosi:B.P.} and a natural $\GL$-orientation (see Construction \ref{constr:MGLthomclasses}). Just as $\MU$ represents the universal complex oriented cohomology theory (\cite[Lemma 4.1.13]{ravenel:cobordism}), similarly $\MGL$ is universal among $\GL$-oriented theories, that is, every $\GL$-oriented cohomology theory $\sE^{*,*}(-)$ comes with a map $\varphi_\sE: \MGL\to \sE$ of commutative monoids in $\SH(k)$ such that the $\sE$-Thom classes are induced by the $\MGL$-Thom classes by composition with $\varphi_\sE$ (\cite[Theorem 2.7]{Panin:algcob}). Hoyois \cite{Hoyois:AlgCob} and Spitzweck \cite{Spitzweck:AlgCob} showed that, after inverting $\chr k$ if $\chr k \neq 0$, the algebraic diagonal $\MGL^*:=\MGL^{2*,*}$ of the coefficient ring of $\MGL$ is isomorphic to the Lazard ring $\L$, the universal coefficient ring for formal group laws, which is known to be the polynomial ring in infinite variables $\Z[x_1,x_2,\ldots]$ \cite[Part II, Theorem 7.1]{Adams:homotopy}. Moreover, similar to the topological case, $\MGL^*$ has a nice geometric interpretation. In fact, through a Pontryagin-Thom construction, one can associate each smooth proper variety $X$ over $k$ with a class $[X]_\MGL \in \MGL^{-\dim X}$ (see \cite[Definition 3.1]{lev:ellcoh}), and $\MGL^*$ is generated by such classes (see \cite[Theorem 3.4 (4)]{lev:ellcoh}). In summary, we have the following.

\begin{introthm}
\label{introthm:MGL*}
    Let $p=\chr k$ if $\chr k>0$, and $p=1$ otherwise. Then we have an isomorphism of graded ring
    $$\MGL^*[1/p]\simeq \Z[1/p][x_1,x_2, \ldots]$$
    with $x_i$ in degree $-2i$, and $\MGL^*[1/p]$ is generated by classes $[X]_\MGL$,   $X$ a smooth and proper $k$-scheme.
\end{introthm}

Despite the appealing analogy between complex orientations and $\GL$-orientations, in $\A^1$-homotopy theory various "interesting" cohomology theories are not $\GL$-oriented, examples include Chow-Witt theory and Hermitian $K$-theory. This led to defining weaker notions of orientation (\cite[Section 3]{Ana:Slor},\cite[Section 2]{DegFas:Borel}), which consist of a theory of Thom classes for vector bundles with additional structures instead of all vector bundles. In particular, we focus on symplectic vector bundles (Definition \ref{defn:sympl.bundles}).

Panin and Walter \cite[Section 6]{Panwal-cobordism} constructed the motivic spectrum $\MSp$ of algebraic symplectic cobordism, a motivic analogue of quaternionic symplectic cobordism $\MSp^\top$ (\cite{Thomcompl},\cite[Chapter 1, Section 5]{C-F:Cobordism}), and proved that this is universal for having an orientation for symplectic vector bundles \cite[Theorem 4.5]{Panwal-cobordism}. Moreover, the above mentioned Pontryagin-Thom construction can be adapted to construct a class $[X]_\MSp \in \MSp^{-\dim X}$ for each smooth proper $k$-variety $X$ which is symplectic, namely, which has a symplectic structure on the tangent bundle. It is then natural to ask whether the coefficient ring of $\MSp$ has a description similar to that for $\MGL$ given by Theorem \ref{introthm:MGL*}. It turns out that for $\MSp$ the picture is much less clear, and in fact the structure of the coefficient ring of $\MSp$ is mostly unknown. There are mainly \underline{2 issues:}

1) It is known that $(\MSp^\top)^*$ is not polynomial (see e.g. \cite{Ray:SymplBordism}). Roughly, this arises from the fact that the sum of a non-trivial symplectic bundle with itself can be a trivial symplectic bundle, and this gives indecomposable torsion elements of order 2. For example, Ray \cite{Ray:Tors} gave explicit computations of indecomposable torsion elements $\phi_i \in (\MSp^\top)^{-8i+3}$ of order $2$ for all $i\ge 1$. Based on the situation in classical homotopy theory, we expect that $\MSp^*$ is also not polynomial and contains infinitely many torsion elements.

2) There are not many known non-trivial symplectic varieties in algebraic geometry. A few sporadic examples come from families of $K3$-symplectic surfaces and their Hilbert schemes of points (for these, see \cite{Beauville:K3}). Even quaternionic projective spaces, which have a symplectic structure on the tautological bundle, do not seem to have a symplectic structure on the tangent bundle. This makes it hard to compute $\MSp^*$ by purely algebraic methods. 

The topological spectrum $\MSp^\top$ has been computed in low dimension by using spectral methods based on the Adams and Adams-Novikov spectral sequences (see \cite{Ray:SymplBordism}, \cite{kochman:symplectic}, \cite{Anisimov:MSp}).

As counterpart to the first issue, Bachmann and Hopkins \cite{Bachmann:eta-periodic} studied the $\eta$-inverted symplectic cobordism spectrum $\MSp[\eta^{-1}]$, where $\eta:\Sigma^{1,1}1_k \to 1_k$ is the motivic Hopf Map induced by the map of schemes $\A^2\setminus \{0\}\to \P^1$ given by $(x,y)\mapsto[x:y]$, and showed that the coefficient ring of $\MSp[\eta^{-1}]$ is polynomial over the Witt ring of $k$ \cite[Theorem 8.7]{Bachmann:eta-periodic}. 

In this paper, inspired by the work of Levine, Yang and Zhao \cite{lev:ellcoh} for the special linear cobordism $\MSL$, we study the coefficient ring of $\MSp$ after taking $\eta$-completion (see Construction \ref{constr:eta-completion}). Our main goal is to show that the resulting ring is polynomial, to prove a criterion for an element of this ring to be a polynomial generator, and to construct explicitly a generating family. We follow the strategy adopted in \cite{lev:ellcoh}, which relies on the Adams spectral sequence.

 In order to produce classes in $\MSp^*$, we modify the symplectic Pontryagin-Thom construction by using an isomorphism of Thom spaces due to Ananyevskiy \cite[Lemma 4.1]{Ana:Slor}, in such a way that we obtain classes from particular varieties carrying a certain "symplectic twist", namely, varieties $Y$ for which the Thom space $\Th(T_Y)$ is isomorphic to the Thom space $\Th(T_Y')$ of a vector bundle $T_Y'$ admitting a symplectic structure.  

 Along the way, we explore some cellularity properties of $\MSp$, by studying both the geometry of quaternionic Grassmannians and the $H\Z$-module structure of $\MSp$.

\subsection*{Outline of the paper and main results}

\underline{Section 2} is a preliminary section describing the setting in which we will work and the language we will use, with no original contributions. We recall in particular the definition of the unstable motivic homotopy category $\sH_\bullet(S)$ and the stable motivic homotopy category $\SH(S)$, mainly from \cite{voe:homotopy_theory}. We recall bigraded cohomology theories for schemes and other related notions, such as commutative ring spectra and highly structured commutative ring spectra in $\SH(S)$, and we briefly discuss motivic cohomology, the cohomology theory represented by the motivic Eilengerg-MacLane spectrum $H\Z$. We also recall the six functors formalism on $\SH(-)$ developed by Ayoub \cite{ayoub:sixfunctors} and Cisinski-Déglise \cite{deglise:mixmot}.

In \underline{Section 3} we review the theory of $\GL$-oriented cohomology theory, and prove a few results, mostly technical, which will be needed for the continuation. In particular, by using a construction due to Ananyevskiy \cite[Lemma 4.1]{Ana:Slor}, we define a "twisted version" $\deg^\Anan$ of the degree map for motivic cohomology $\deg:H\Z^{2\dim X,\dim X}(X) \to \Z$ costructed by Voevodsky, and we compare it with the original degree. We use this machinery to compute compositions of motivic cohomology classes $c \in H\Z^{2 \dim X, \dim X}(\sE)$ with a certain kind of twisted classes $[X,v,\vartheta]_\sE \in \sE^{-2\dim X,-\dim X}(\Spec k)$, see Corollary \ref{cor:TwistClassComp}. We then give a picture of $\Sp$-oriented cohomology theories parallel to that of $\GL$-oriented cohomology theories, and we recall the algebraic symplectic cobordism spectrum $\MSp$, constructed by Panin and Walter \cite{Panwal-cobordism}, representing the universal $\Sp$-oriented cohomology theory.

In \underline{Section 4} we give some "cellularity results" for $\MSp$. In particular, following \cite{panwal:grass}, we study the geometry of quaternionic Grassmannians, and we prove that, if $v$ is in $K_0(\HGr(r,n))$ and $p:\HGr(r,n) \to \Spec k$ is the structure map, $p_\#\Sigma^v1_{\HP^n}$ is a cellular spectrum in the sense of Definition \ref{defn:motivicCellular}. We then obtain the following as a corollary.
\begin{introthm}
\label{introthm:MSpCellular}
    $\MSp$ is a cellular spectrum.
\end{introthm}
With similar techniques, we study the $H\Z$-module $\MSp \wedge H\Z$, getting the following description.
\begin{introprop}
\label{introprop:MSpMotive}
We have the expression
     \begin{equation*}
        \MSp \wedge H\Z \simeq \oplus_\alpha \Sigma^{4n_\alpha,2n_\alpha}H\Z,
    \end{equation*}
    for some non-negative integers $n_\alpha$, such that for each non-negative integers $n$, there are only finitely many indices $\alpha$ with $n_\alpha=n$.
\end{introprop}
See Proposition \ref{prop:MotiveOfMSp} for the precise statement we are interested in.

In \underline{Section 5}, we study the mod $\ell$ motivic Adams spectral sequence associated to $\MSp$, with $\ell$ an odd prime different from $\chr k$. In particular, we start by studying the $\Z/\ell$-algebra $\Ext_{A^{*,*}}(H^{*,*}(\MSp), H^{*,*})$, with $H^{*,*} :=(H\Z/\ell)^{*,*}$, and using a decomposition of $H^*(\MSp^\top)$ given by Novikov in \cite{Thomcompl}, we manage to work out the following presentation.
\begin{introprop}
\label{introprop:ExtMSp}
    $\Ext_{A^{*,*}}(H^{*,*}(\MSp), H^{*,*})$ is a trigraded polynomial algebra over $\Z/\ell$ generated by elements $z_{(2k)}$ in tridegree $(0,-4k,-2k)$, for $k\ge 1$ and $2k$ not of the form $\ell^i-1$, and elements $h_r'$ in tridegree $(1,1-2\ell^r,1-\ell^r)$, for $r\ge 0$.
\end{introprop}
See Proposition \ref{prop:pres.ExtAlgebra} for the complete picture. This gives an analogue for $\MSp$ to the results for $\MGL$ given in \cite[Section 5]{lev:ellcoh}. We also recall the completion of a motivic spectrum along the \emph{motivic Hopf map} $\eta: \Sigma^{1,1}1_k \to 1_k$ (Construction \ref{constr:eta-completion}) and the completion over $\ell$ (Definition \ref{defn:ellCompl}), and the combination of the two makes us define the $(\eta,\ell)$-completion $\MSp_{\eta,\ell}^\wedge$ of $\MSp$. We then use Theorem \ref{introthm:MSpCellular} and Proposition \ref{introprop:MSpMotive} to prove the following.
\begin{introprop}
\label{introprop:ConvergeceA.S.S.}
    The $E_2$-page of the Adams spectral sequence for $\MSp$ is of the form
    $$E_2^{s,t,u}= \Ext_{A^{*,*}}^{s,(t-s,u)}(H^{*,*}(\MSp),H^{*,*}).$$
    Moreover, the spectral sequence converges completely to $(\MSp_{\eta,\ell}^\wedge)^{2u,u}$ and degenerates at the second page.
\end{introprop}

In \underline{Section 6}, we study the structure of the coefficient ring $\MSp_\eta^*$. We define a symplectic Thom isomorphism and a symplectic version of the Pontryagin-Thom construction to obtain classes $[X]_\MSp \in \MSp^{-4d,-2d}$ for $X \in \Sm/k$ a proper variety of dimension $2d$ with a symplectic structure on the tangent bundle. Using Ananyevskiy's isomorphism, we modify this construction to obtain classes $[X]_\MSp$ from proper varieties $X \in \Sm/k$ carrying a stable symplectic twist, that is, an isomorphism $\Sigma^{-T_X}1_X \xrightarrow{\sim}\Sigma^{2r,r}\Sigma^{-T_X'}1_X$ for $(T'_X,\omega)$ some virtual symplectic vector bundle over $X$. We apply this approach to certain smooth codimension $2$ subvarieties $Y$ of projective varieties $X$ of the form $X=\prod_{i=1}^{2s}\P^{2n_i+1}$. We also discuss certain $H\Z$-characteristic numbers $s_{2d}$ (Definition \ref{defn:SegreNumber}), and by working out the numbers $s_{\dim Y}(Y)$ of the varieties $Y$ constructed before, we prove the following.
\begin{introthm}
\label{introthm:SegreNumbers}
    There exists a family $\{Y_{2d}\}_{d \ge 1}$ of proper irreducible varieties constructed as in the previous subsection satisfying
    $$\nu_{\ell}(s_{2d}(Y_{2d}))=
    \begin{cases}
        0, \; \; \; \text{if} \;\;  2d \neq \ell^i-1 \; \forall \; i \\
        1, \; \; \; \text{if} \; \; 2d=\ell^r -1, \; r \ge 1
    \end{cases}$$
    where $\nu_\ell$ is the $\ell$-adic valuation.
\end{introthm}
See Theorem \ref{thm:symplclasses2} for the precise statement. We then compare the $E_2$-page of the Adams spectral sequence for $\MSp$ given by Propositions \ref{introprop:ExtMSp}-\ref{introprop:ConvergeceA.S.S.} with that for $\MGL$ studied in \cite{lev:ellcoh}, and we show that $(\MSp^\wedge_{\eta,\ell})^*$ is a subalgebra of the $\ell$-adic completion $(\MGL^*)^\wedge_\ell$ of $\MGL^*$, with $(\MSp^\wedge_{\eta,\ell})^{2d} \simeq (\MGL^{2d})^\wedge_\ell$ and $(\MSp^\wedge_{\eta,\ell})^{2d+1}=0$. By a generating criterion for $(\MGL^*)^\wedge_\ell$ given in \cite{lev:ellcoh} in terms of Segre numbers, we deduce an analogous criterion for $(\MSp^\wedge_{\eta,\ell})^*$, and by using the machinery developed in Section 3 based on the twisted degree map, we see that the family $\{Y_{2d}\}_{d \ge 1}$ of Theorem \ref{introthm:SegreNumbers} fits this criterion. We thus get the following.
\begin{introthm}
\label{introthm:finalResult}
    A family of varieties $\{Y'_{2d}\}_{d \ge 1}$ which has an associated family of symplectic classes $[Y'_{2d}]_{\MSp}\in \MSp^{-2d}$ forms a family of polynomial generators of $(\MSp_{\eta,\ell}^\wedge)^*$ if and only if 
    $$\nu_{\ell}(s_{2d}(Y'_{2d}))=
    \begin{cases}
        0, \; \; \; \text{if} \;\;  2d \neq \ell^i-1 \; \forall \; i \\
        1, \; \; \; \text{if} \; \; 2d=\ell^r -1, \; r \ge 1.
    \end{cases}$$
    Moreover, Theorem \ref{introthm:SegreNumbers} gives such a family.
\end{introthm}
Finally, by working on the $\ell$-torsion of $(\MSp_{\eta,\ell}^\wedge)^*$ one prime $\ell$ at a time, we globalize Theorem \ref{introthm:finalResult} to the coefficient ring $(\MSp_\eta^\wedge)^*$. The complete statement is Theorem \ref{thm:FinalResult}.

\subsection*{Acknowledgements}

I wholeheartedly thank my Ph.D supervisor Marc Levine for his inextinguishable dedication, his infinite amount of patience in reexplaining concepts to me and answering my repetitive questions, and for the number of mathematical insights he offered me throughout all my Ph.D. His contribution to making this paper exist is invaluable. I would like to sincerely thank Alexey Ananyevskiy, who agreed to read my thesis and be my referee. I am incredibly grateful to the PhD students and postdocs of the ESAGA group in Essen, for creating an amazingly nice working environment, which is constantly welcoming and stimulating. Among them, I give special thanks to Chirantan for his continuous interest in this work and for the uncountably many stimulating conversations we had during the period while this paper was written. Lastly, I will be forever indebted to my family for their unbounded love and for always supporting my choices.



\section{Preliminaries}

\subsection{Motivic spaces and motivic spectra}

For this subsection, let us fix a Noetherian scheme $S$ of finite Krull dimension. We will briefly recall the construction of the unstable motivic homotopy category $\sH_{\bullet}(S)$ over $S$, originally constructed in \cite{morvoe:homotopytheory}.

Let $\Sch/S$ be the category of quasi-projective $S$-schemes, and $\Sm/S$ the full subcategory of $\Sch/S$ of smooth quasi-projective $S$-schemes.
\begin{defn}
    We denote by $\Spc(S)$ and $\Spc_{\bullet}(S)$ the categories of Nisnevich sheaves of simplicial sets and Nisnevich sheaves of pointed simplicial sets, respectively, on $\Sm/S$. Usually, objects in $\Spc(S)$ and $\Spc_{\bullet}(S)$ are simply called \emph{spaces} and \emph{pointed spaces} respectively.
\end{defn}

$\Sm/S$ can be embedded in $\Spc(S)$ by identifying a scheme $U \in \Sm/S$ with the Nisnevich sheaf associated to the constant representable presheaf $\text{Hom}_{\Sm/S}(-,U)$, and analogously, $U_+$ can be seen as a constant pointed sheaf.

For two pointed spaces $(X,x)$ and $(Y,y)$, the wedge product $(X,x) \vee (Y,y)$ is given by gluing the distinguished points, and one defines the \emph{smash product} $(X,x)\wedge (Y,y)\coloneqq X \times Y /(X \times y) \vee (x \times Y)$.

Let us denote by $\Delta_S^n$ the \emph{algebraic $n$-simplex} over $S$, that is, the closed subspace of $\A^{n+1}_S=S \times_{\Spec \Z}\Spec(\Z[t_0,\ldots,t_n])$ given by $\A_S^{n+1}/(\Sigma_{i=0}^nt_i-1) \simeq \A_S^n$. As explained in \cite[Section 3, p. 584]{voe:homotopy_theory}, there exists a realization functor $\mid - \mid_S:\Delta^{\text{op}}\text{Sets} \to \Spc(S)$ from the category of simplicial sets to $\Spc(S)$, characterized by the properties that it commutes with colimits and $\mid \Delta^n \mid_S=\Delta_S^n$. 

In \cite[Section 3, p. 585]{voe:homotopy_theory}, for $X \in \Spc(S)$, $U \in \Sm/S \subset \Spc(S)$, and $x \in \text{Hom}_{\Spc(S)}(U,X)$, Voevodsky defines the set $\pi^{\A^1}_{i,U}(X,x)$ as the analogue of the standard $i$-th homotopy group of a topological space. The category $\Spc(S)$ has a model structure described in \cite[Section 2]{morvoe:homotopytheory} or \cite[Section 3]{voe:homotopy_theory}, whose homotopy equivalences are as in the following:
\begin{defn}
    A morphism $f: X \to Y$ in $\Spc(S)$ is an \emph{$\A^1$-weak equivalence}, or just a \emph{weak equivalence}, if, for any $U \in \Sm/S$ and $x \in \Hom_{\Spc(S)}(U,X)$, the corresponding maps
    $$\pi^{\A^1}_{i,U}(X,x) \to \pi^{\A^1}_{i,U}(Y,f(x))$$
    are bijections for all $i \ge 0$.
\end{defn}

This definition of weak equivalences is \cite[Definition 3.4]{voe:homotopy_theory}, and is a bit different from the more technical one given in \cite[\S2 Definition 1.2 (1)]{morvoe:homotopytheory}, but the two definitions are equivalent by \cite[Theorem 3.6]{voe:homotopy_theory}. Let us recall the model structure.

\begin{thm}[\cite{voe:homotopy_theory}, Theorem 3.7]
\label{thm:model-spaces}
    The category $\Spc(S)$ has a proper closed model structure with weak equivalences as we have just defined, cofibrations being the monomorphisms, and fibrations being the maps satisfying the right lifting property with respect to the maps that are both cofibrations and weak equivalences.
\end{thm}

By replacing schemes and spaces by pointed schemes and pointed spaces, one gets the analogous definition of weak equivalences in $\Spc_{\bullet}(S)$, and the relative model structure. In particular, a morphism of pointed spaces is a weak equivalence if it is a weak equivalence of spaces, after forgetting the distinguished point.
\begin{defn}
    We denote by $\sH(S)$ and $\sH_{\bullet}(S)$ respectively, the localizations of the categories $\Spc(S)$ and $\Spc_{\bullet}(S)$ with respect to the relative classes of weak equivalences. The category $\sH_{\bullet}(S)$ is called the \emph{motivic unstable homotopy category over $S$}. Objects in $\sH_{\bullet}(S)$ are called \emph{motivic spaces over $S$}.  
\end{defn}

Again, every $U \in \Sm/S$ defines an object $U \in \sH(S)$ and $U_+ \in \sH_{\bullet}(S)$.

\begin{rmk}
\label{rmk:colims}
    The categories $\Spc(S)$ and $\Spc_{\bullet}(S)$ have all small colimits and limits. This just comes from general theory of sheaves on Grothendieck topologies. As an immediate consequence, the categories $\sH(S)$ and $\sH_{\bullet}(S)$ have all small colimits and limits.
\end{rmk}

\begin{rmk}
\label{rmk:unstablefunctors}
   Let $f:S\to S'$ be a morphism of schemes. Taking the inverse image and direct image of sheaves along $f$ defines two adjoint functors
   $$ \begin{tikzcd}
    f^*: \Spc(S') \arrow[r, shift left] & \Spc(S): f_* \arrow[l, shift left]
    \end{tikzcd}.$$
    Through localization, these functors induce analogue functors
    $$ \begin{tikzcd}
    f^*: \sH(S') \arrow[r, shift left] & \sH(S): f_* \arrow[l, shift left]
    \end{tikzcd}.$$
    On the other hand, if $f$ is smooth, $S$ is in $\Sm/S'$, then a space $X \in \Spc(S)$ gives a space $f_\# X \in \Spc(S')$ via $f$ that, if $X$ is representable, has the form $f_\#(X \xrightarrow{g} S) = (X \xrightarrow{fg} S')$. This gives the functor $f_\#:\Spc(S)\to \Spc(S')$, and one easily verifies that $f_\#$ is left adjoint to $f^*$. Again, $f_\#$ induces $f_\#:\sH(S) \to \sH(S')$ through localization. By replacing spaces and motivic spaces by pointed spaces and pointed motivic spaces, we obtain the analogue adjunctions $f^* \dashv f_*$ and $f_\# \dashv f^*$ for $\Spc_{\bullet}(-)$ and $\sH_{\bullet}(-)$.
\end{rmk}

The smash product $\wedge$ in $\Spc_{\bullet}(S)$ induces a smash product $\wedge$ in $\sH_{\bullet}(S)$. With this product, $\sH_{\bullet}(S)$ acquires a symmetric monoidal structure, with unit being the \emph{$0$-sphere} $S^0\coloneqq S_+$. This follows by the properties of weak equivalences (see \cite[Section 3, pp.585-586]{voe:homotopy_theory}).

\begin{defn}
    In $\sH_{\bullet}(S)$, we have the punctured line $\G_m \coloneqq (\A^1_S\setminus \{0\},1)$ and the \emph{simplicial circle} $S^1 \coloneqq \mid \Delta^1/\partial \Delta^1 \mid_S$, with distinguished point given by the collapsed boundary, and for $p \ge q \ge 0$, one defines the \emph{motivic spheres}
    $$S^{p,q} \coloneqq (S^1)^{\wedge (p-q)} \wedge \G_m^{\wedge q}.$$
    Also, for $p \ge q \ge 0$, we have the suspension endofunctor $\Sigma^{p,q}: \sH_{\bullet}(S) \to \sH_{\bullet}(S)$ that takes a motivic space $X$ to $\Sigma^{p,q}(X) \coloneqq S^{p,q} \wedge X$.
\end{defn}

There is a canonical isomorphism $(\P^1_S,\infty)\simeq S^{2,1}$ in $\sH_{\bullet}(S)$ given by the fact that both these spaces are homotopy pushouts of
$$
\begin{tikzcd}
    \G_m \arrow[r, hook] \arrow[d, hook] & \A^1 \\
    \A^1,
\end{tikzcd}
$$
in $\Spc_{\bullet}(S)$, with $\A^1$ pointed at $1$. We use the shorthand $\P^1 \coloneqq (\P^1_S,\infty)$.

\begin{defn}
    For $X \in \Spc_{\bullet}(S)$, and $p \ge q \ge 0$, one defines the \emph{unstable homotopy sheaf} $\pi_{p,q}^{\A^1}(X)$ as the Nisnevich sheafification of the presheaf $U \to \Hom_{\sH_{\bullet}(S)}(\Sigma^{p,q}U_+,X)$ on $\Sm/S$. One also defines $\pi_i^{\A^1}(X) \coloneqq \pi_{i,0}^{\A^1}(X)$.
\end{defn}

By \cite[Lemma 3.8]{voe:homotopy_theory}, one has $\pi_i^{\A^1}(X)(U) = \pi_{i,U}^{\A^1}(X,x)$, with $x$ the distinguished point of $X$.

\begin{defn}
    For $\pi: V\to S$ vector bundle, let $\P(V) \coloneqq \operatorname{Proj}(\Sym^*\sV)\xrightarrow{q}S$, with $\sV$ sheaf of sections of $V^\vee$, denote the associated projective space. The \emph{Thom space of $V$}, denoted by $\Th_S(V)$, or simply $\Th(V)$, is the quotient $\P(V \oplus \mathcal{O}_S)/\P(V)$ in $\sH_{\bullet}(S)$, with the obvious distinguished point, which exists because of Remark \ref{rmk:colims}. If $S \in \Sm/S'$ with structure map $p$, one usually refers to $p_\#\Th(V)$ still as the \emph{Thom space of $V$ in} $\sH_{\bullet}(S')$.
\end{defn}

\begin{rmk}
    Equivalently, one can define $\Th_S(V)$ as the quotient $V/V^0$, where $V^0 \coloneqq V - s_0(S)$, with $s_0:S \to V$ the zero section of the bundle, by considering $V$ and $V^0$ as spaces over $S$ through the structure morphism $\pi$. The quotient is naturally a pointed space.
\end{rmk}

\begin{rmk}
\label{rmk:Thomspaces}
    \begin{enumerate}
        \item From the properties of the smash product, it follows that, if $V_1,V_2 \to S$ are two vector bundles over $S$, one has 
\begin{equation}
\label{eq:smash-thom}
    \Th(V_1 \oplus V_2) \simeq \Th(V_1) \wedge \Th(V_2).
\end{equation}
\item Since $\P^1 \simeq \A^1/\G_m$, in $\sH_{\bullet}(S)$, we have in particular $\P^1 \simeq \Th(\mathcal{O}_S)$ in $\sH_{\bullet}(S)$. By the previous point, one also gets $\Th(\mathcal{O}_S^n)\simeq (\P^1)^{\wedge n} \simeq S^{2n,n}$. Technically, if $S \in \Sm/S'$ and we want to consider $\Th(\mathcal{O}_S^n)$ as a motivic space over $S'$, it corresponds to $S^{2n,n}\wedge S_+ \in \sH_{\bullet}(S')$.
\item Let $g:X \to Y$ be a map in $\Sm/S$, and $V\to X$, $W \to Y$ two vector bundles with a map $f:V \to W$ arising from a fiberwise injective map $f_X:V\to g^*W$. Then the Thom space construction gives naturally a map $\Th(g,f):\Th_X(V) \to \Th_Y(W)$ in $\sH_{\bullet}(S)$. If the isomorphism $g$ is obvious from the context, for instance if $g$ is the identity, we will just write $\Th(f)$.
    \end{enumerate}
\end{rmk}

We now need to discuss the stabilization of the unstable motivic homotopy category, still following \cite{voe:homotopy_theory}. The way of stabilizing the category $\sH_{\bullet}(S)$ is given by the theory of $T$-spectra.

\begin{defn}
    Let $T \in \Spc_{\bullet}(S)$ be a pointed space. A \emph{$T$-spectrum} $\E$ is a sequence of pointed spaces $(E_0,E_1, E_2, \ldots)$ with bonding maps $\epsilon_r:T \wedge E_r \to E_{r+1}$. A \emph{morphism of $T$-spectra} $f:\E \to \F$ is a collection of morphisms of pointed spaces $\{f_r:E_r \to F_r\}_{r \ge 0}$ such that all the obvious squares with the bonding maps commute. These data define the \emph{category of $T$-spectra} $\SPP_S(T)$. Also, If $\E$ is a $T$-spectrum, for every $X \in \Spc_{\bullet}(S)$ and any integer $n$, there is a functor $\E^n: \Spc_{\bullet}\to \text{Sets}$ given by
    $$\E^n(X)= \colim_{i \ge \text{max}{0,-n}}\Map_{\sH_{\bullet}(S)}(T^{\wedge i} \wedge X, T^{\wedge i +n}\wedge E_i),$$
    where $\Map_{\mathcal{C}}(-,-)$ is the notation for the mapping set, and the maps in the direct system are given by the bonding maps of $\E$. A morphism of $T$-spectra is called a \emph{stable weak equivalence} if the induced natural transformations of functors $\E^n(-) \to \F^n(-)$ are natural isomorphisms for all $n$.
\end{defn}

The category of $T$-spectra has all small limits and colimits, which are defined termwise thanks to Remark \ref{rmk:colims}. In particular, the coproducts $\oplus_\alpha \E_\alpha$ of a family of $T$-spectra $\E_\alpha$ is the $T$-spectrum $(\vee_\alpha E_{\alpha,0}, \vee_\alpha E_{\alpha, 1}, \ldots)$, with bonding maps $\epsilon_r\coloneqq \delta_r \circ (\vee_\alpha \epsilon_{\alpha,r})$, where $\delta_r$ is the canonical isomorphism $T \wedge (\vee_\alpha E_{\alpha,i}) \to \vee_\alpha (T \wedge E_{\alpha,i})$.

There is an obvious model structure on the category of $T$-spectra where weak equivalences are termwise weak equivalences of pointed spaces. It is non trivial to show that one can take a Bousfield localization of this model structure with respect to the class of stable weak equivalences (see \cite[Appendix A]{Jardine:SymSpectra} or \cite[Section 3]{Hovey:Spectra}). 

We now consider $T=\P^1$.

\begin{defn}[\cite{voe:homotopy_theory}, Definition 5.7; \cite{Jardine:SymSpectra}, Theorem 2.9; \cite{Hovey:Spectra}, Definition 3.3]
    The \emph{Motivic stable homotopy category over $S$}, denoted by $\SH(S)$, is the left Bousfield localization of the category of $\P^1$-spectra over $S$ with respect to the class of stable weak equivalences. Objects in $\SH(S)$ are called \emph{motivic $\P^1$-spectra over $S$}.
\end{defn}

Clearly, the category $\SH(S)$ has again all small limits and colimits.

\begin{defn}
    The \emph{infinite $\P^1$-suspension functor} is the functor $\Sigma_{\P^1}^\infty:\sH_{\bullet}(S)\to \SH(S)$ that takes a motivic space $X$ to  
    $$\Sigma_{\P^1}^\infty(X) \coloneqq (X, \P^1 \wedge X, (\P^1)^{\wedge 2} \wedge X, \ldots, (\P^1)^{\wedge r}\wedge X, \ldots),$$
    with bonding maps $\epsilon_r$ being the identities maps on $(\P^1)^{\wedge r+1}\wedge X$. $\Sigma_{\P^1}^\infty(X)$ is called the \emph{infinite $\P^1$-suspension spectrum associated to $X$}.
\end{defn}

In this way, any motivic space $X$ gives a motivic $\P^1$-spectrum. Let us note that the functor $\Sigma_{\P^1}^\infty$ takes weak equivalences to stable weak equivalences.

When working in $\SH(S)$, we will usually refer to stable weak equivalences simply as weak equivalences, and we will say that two motivic spectra are weakly equivalent if there is a weak equivalence between them.

It is useful to remember the following:
\begin{lemma}[\cite{voe:homotopy_theory}, Theorem 5.2]
\label{lemma:stablehom}
    For $X \in \sH_{\bullet}(S)$ and $\E \in \SH(S)$, one has
    $$\Map_{\SH(S)}(\Sigma^{\infty}_{\P^1}X, \E) = \colim_i \Map_{\sH_{\bullet}(S)}((\P^1)^{\wedge i}\wedge X, E_n),$$
    where the maps in the direct system are given by the bonding maps of $\E$.
\end{lemma}
\begin{defn}
    In $\SH(S)$, we have the \emph{sphere spectrum} $\mathbb{S}_S \coloneqq \Sigma_{\P^1}^\infty S_+$.
\end{defn}

The category $\SH(S)$ has a rich structure resumed in the following two theorems.

\begin{thm} [\cite{voe:homotopy_theory}, Proposition 5.4]
    $\SH(S)$ is an additive category, and has a triangulated structure defined as follows. The shift functor takes a spectrum $\E=(E_0,E_1,\ldots)$, with bonding maps $\epsilon_r$, to the spectrum $\E[1]=(E_0 \wedge S^1, E_1 \wedge S^1, \ldots)$, with bonding maps $\epsilon_r \wedge \id_{S^1}$. The distinguished triangles are of the form
    $$\E \xrightarrow{f} \F \to \cone (f) \to \E[1],$$
    where $f$ is a morphism of $\P^1$-spectra, and $\cone (f)$ is the colimit of the diagram
    $$
    \begin{tikzcd}
        \E \arrow[rr, "1_{\Delta_S^1} \wedge \id_\E"] \arrow[d, swap, "f"] & & \E \wedge \Delta_S^1 \\
        \F & &
    \end{tikzcd}
    $$
    of $\P^1$-spectra. Here $\E \wedge \Delta_S^1$ is the spectrum $(E_0 \wedge \Delta_S^1, E_1 \wedge \Delta_S^1, \ldots)$ with bonding maps $\epsilon_i \wedge \id_{\Delta_S^1}$. 
\end{thm}

\begin{thm}[\cite{voe:homotopy_theory}, Theorem 5.6]
\label{thm:SHsymmonoidal}
    $\SH(S)$ has a symmetric monoidal structure, with a monoidal product $\wedge$, called again \emph{smash product}, and monoidal unity $1_S \coloneqq \mathbb{S}_S$ the sphere spectrum, such that the following properties hold:
    \begin{enumerate}
        \item For $\E \in \SH(S)$ and $X \in \Spc_{\bullet}(S)$, the motivic spectrum $\E \wedge \Sigma_{\P^1}^\infty X$ is canonically isomorphic to the spectrum $(E_0 \wedge X, E_1 \wedge X, E_2 \wedge X, \ldots)$ with bonding maps $\epsilon_i \wedge \id_X$.
        \item For $\E \in \SH(S)$ and $\{\F_\alpha\}_\alpha$ a collection of motivic spectra in $\SH(S)$, we have a canonical isomorphism
        $$(\oplus_\alpha \F_\alpha) \wedge \E \xrightarrow{\sim} \oplus_\alpha(\F_\alpha \wedge \E).$$
    \end{enumerate}
\end{thm}

Intuitively, the smash product $\wedge$ on $\SH(S)$ is induced by taking termwise the smash product of spaces, even if this is clearly not a good definition in the category of $\P^1$-spectra. It is extremely non trivial to prove that localizing at weak equivalences kills all the ambiguities. One way to avoid this issue is using the theory of symmetric spectra instead of $\P^1$-spectra. For these objects, the smash product induces a symmetric monoidal model structure, and the associated stable homotopy category is equivalent to $\SH(S)$ as a triangulated category. This approach is developed in \cite{Jardine:SymSpectra} (we give some details in \S \ref{subsection:HighlyStructuredRings}).

On the other hand, the coproduct $\oplus$ in $\SH(S)$ is induced by the wedge product $\vee$ of pointed spaces. Let us note that, if $X \in \Sm/S$ is a disjoint union $X_1 \sqcup X_2$ of two schemes, one has $\Sigma^\infty_{\P^1}X_+ = \Sigma^\infty_{\P^1}(X_{1,+}\vee X_{2,+}) \simeq \Sigma_{\P^1}^\infty X_{1,+}\oplus \Sigma_{\P^1}^\infty X_{2,+}$.

We have the $\P^1$-suspension endofunctor $\Sigma_{\P^1}\coloneqq \P^1 \wedge (-):\SH(S) \to \SH(S)$. Basically by construction of $\SH(S)$, the functor $\Sigma_{\P^1}$ is invertible. Its inverse is given by $\Sigma_{\P^1}^{-1}\coloneqq (\P^1)^{-1}\wedge (-)$, with $(\P^1)^{-1}$ being the $\P^1$-spectrum $(S_+,S_+,\P^1,(\P^1)^{\wedge 2}, \ldots)$, with the obvious bonding maps. Indeed, by Property (1) of Theorem \ref{thm:SHsymmonoidal}, one has 
$$(\P^1)^{-1} \wedge \P^1 \simeq (\P^1,\P^1,(\P^1)^2, (\P^1)^3, \ldots),$$ and this is in turn weakly equivalent to $(S_+,\P^1,(\P^1)^2,(\P^1)^3, \ldots)=1_S$.

In the same way, one defines the suspension endofunctors $\Sigma_{S^1}$ and $\Sigma_{\G_m}$. In particular, $\Sigma_{S^1}$ gives the shift functor $(-)[1]$. Since $\P^1 \simeq S^1 \wedge \G_m$, one lets $\Sigma_{S^1}^{-1} \coloneqq \Sigma_{\G_m} \circ \Sigma_{\P^1}^{-1}$, and $\Sigma_{\G_m}^{-1} \coloneqq \Sigma_{S^1} \circ \Sigma_{\P^1}^{-1}$, and it is easy to see that these functors are inverse to $\Sigma_{S^1}$ and $\Sigma_{\G_m}$ respectively. This allows us to define $\Sigma_{\P^1}^n$, $\Sigma_{S^1}^n$, $\Sigma_{\G_m}^n$ for any $n \in \Z$. By adopting a notation analogous to the unstable motivic spheres, one writes $\Sigma^{p,q}\coloneqq \Sigma_{S^1}^{p-q}\circ \Sigma_{\G_m}^q$, and we have $\Sigma_{\P^1} \simeq \Sigma^{2,1}$.

Let us also note that the symmetric monoidal category $\SH(S)$ has internal-Hom objects. This ultimately comes from the fact that the category of motivic spaces, with product $\wedge$, is closed monoidal by \cite[Theorem 3.7]{voe:homotopy_theory}.

\subsubsection{Cohomology theories}
\label{subsection:cohomologyTheories}

Every motivic spectrum $\E \in \SH(S)$ defines bigraded cohomology functors $\E^{p,q}(-): \Spc(S)^{\text{op}} \to \text{Ab}$ and homology functors $\E_{p,q}(-): \Spc(S) \to \text{Ab}$ for any integer $p,q$, in the category $\text{Ab}$ of abelian groups, through:
\begin{align*}
    \E^{p,q}(X) \coloneqq & \Hom_{\SH(S)}(\Sigma_{\P^1}^\infty X_+, \Sigma^{p,q}\E),\\
    \E_{p,q}(X) \coloneqq & \Hom_{\SH(S)}(1_S, \Sigma^{-p,-q}\E \wedge \Sigma_{\P^1}^\infty X_+).
\end{align*}
For $f:X \to Y$ in $\Spc(S)$, composing with $\Sigma_{\P^1}^\infty f$ defines the ordinary cohomology pullback $\E^{p,q}(f)=f^*$ and the ordinary homology pushforward $\E_{p,q}(f)=f_*$.

Let us note that $\E^{p,q}(S)=\E_{-p,-q}(S)$.

We let $\E^n(X) \coloneqq \E^{2n,n}(X)$ and $\E_n(X) \coloneqq \E_{2n,n}(X)$. Also, we use the notations $\E^{*,*}(X)\coloneqq \oplus_{p,q \in \Z}\E^{p,q}(X)$, $\E_{*,*}(X)\coloneqq \oplus_{p,q \in \Z}\E_{p,q}(X)$, $\E^*(X)\coloneqq \oplus_{n \in \Z}\E^n(X)$, $\E_*(X)\coloneqq \oplus_{n \in \Z}\E_n(X)$. Occasionally, we use the shorthand $\E^{*,*}$ for $\E^{*,*}(S)$, and similarly $\E_{*,*}$, $\E^*$, $\E_*$. 

\begin{defn}
    A commutative monoid object $\sE$ in the symmetric monoidal category $\SH(S)$ is called a \emph{motivic commutative ring spectrum}.
\end{defn}
In particular, a motivic commutative ring spectrum $\sE$ is equipped with a multiplication map $\mu_{\sE}:\sE \wedge \sE \to \sE$ and a unit map $\epsilon_\sE:1_S \to \sE$.

For $\alpha \in \sE^{p,q}(X)$ and $\beta \in \sE^{p',q'}(X)$, with $\sE$ a motivic commutative ring spectrum, we have the cup product $\alpha \cup \beta \in \sE^{p+p',q+q'}(X)$ defined by
\begin{multline*}
    \alpha \cup \beta:\Sigma_{\P^1}^\infty X_+ \xrightarrow{\Sigma_{\P^1}^\infty \delta_X} \Sigma_{\P^1}^\infty X_+ \wedge \Sigma_{\P^1}^\infty X_+ \xrightarrow{\alpha \wedge \beta} \Sigma^{p,q}\sE \wedge \Sigma^{p',q'}\sE = \Sigma^{p+p',q+q'}\sE \wedge \sE \\
    \xrightarrow{\Sigma^{p+p',q+q'}\mu_\sE} \Sigma^{p+p',q+q'} \sE,
\end{multline*}
that makes $\sE^{*,*}(X)$ a graded ring.

Also, for $\alpha \in \sE^{p,q}(X)$ and $\gamma \in \sE_{p_1,q_1}(X)$, we have a cap product $\alpha \cap \gamma \in \sE_{p_1-p,q_1-q}(X)$ defined by
\begin{multline*}
\alpha \cap \gamma: 1_S \xrightarrow{\gamma} \Sigma^{-p_1,-q_1}\sE \wedge \Sigma_{\P^1}^\infty X_+ 
\xrightarrow{\id\wedge\Sigma_{\P^1}^\infty\delta_X}
\Sigma^{-p_1,-q_1}\sE \wedge \Sigma_{\P^1}^\infty X_+ \wedge\Sigma_{\P^1}^\infty X_+\\ 
\xrightarrow{\id\wedge \alpha\wedge\id}
\Sigma^{p-p_1,q-q_1}\sE \wedge \sE \wedge\Sigma_{\P^1}^\infty X_+ 
\xrightarrow{\mu_\sE\wedge\id}\Sigma^{p-p_1,q-q_1}\sE \wedge\Sigma_{\P^1}^\infty X_+ .
\end{multline*}

For $\E \in \SH(S)$, we have a canonical pairing 
\begin{equation}\label{eqn:Pairing}
\langle -,-\rangle:\sE^{p,q}(\E)\times \sE_{p_1, q_1}(\E)\to \sE_{p_1-p, q_1-q}(S)
\end{equation}
defined by sending $\alpha:\E \to \Sigma^{p,q}\sE$ and $\gamma:1_S\to \Sigma^{-p_1, -q_1}\sE\wedge \E$ to the composition
\[
\langle \alpha,\gamma\rangle: 1_S\xrightarrow{\gamma}\Sigma^{-p_1, -q_1}\sE\wedge \E
\xrightarrow{\id_\sE\wedge\alpha}\Sigma^{p-p_1,q-q_1}\sE\wedge\sE
\xrightarrow{\mu_\sE} \Sigma^{p-p_1,q-q_1}\sE.
\]

If $\gamma$ is of the form $\epsilon_\sE\wedge \gamma_0$ for some map $\gamma_0:1_S\to \Sigma^{-p_1, -q_1}\E$, where $\epsilon_\sE:1_S\to \sE$ is the unit map, then $\langle\alpha,\gamma\rangle$ is just the composition $(\Sigma^{-p_1,-q_1} \alpha)\circ \gamma_0$. 

\begin{rmk}\label{rmk:HurewiczMap} If $\sE\in \SH(S)$ is a commutative ring spectrum, then for any $\E \in \SH(S)$ we have the {\em motivic $\sE$-Hurewicz map}
\[
h_\sE:\E_{a,b}(S)\to \sE_{a,b}(\E)
\]
defined by composition with the unit map $\epsilon_\sE:1_S\to \sE$,
\[
h_\sE(f:1_S\to \Sigma^{-a,-b}\sF):=(\epsilon_\sE\wedge\id)\circ f:1_k\to \Sigma^{-a,-b}\sE\wedge \E.
\]
For $\alpha\in \sE^{p,q}(\E)$, $\gamma_0\in \E_{p_1,q_1}(S)$, we then have
\[
\langle \alpha,h_\sE(\gamma_0)\rangle =\alpha\circ \gamma_0\in \sE_{p_1-p,q_1-q}(S).
\]
\end{rmk}

\begin{defn}
\label{defn:coefficientRing}
    For $\sE \in \SH(S)$ motivic commutative ring spectrum, the graded ring $\sE^{*,*}(S)=\sE_{*,*}(S)$ is called the \emph{coefficient ring of $\E$}.
\end{defn}

Note that, if $X \in \Spc(S)$ has structure map $p$, the cohomology pullback $p^*$ makes $\sE^{*,*}(X)$ a bigraded module over $\sE^{*,*}(S)$. The $\sE$-cohomology $\sE^{*,*}(-)$ can then be seen as a functor from $\Spc(S)^\op$ to bigraded $\sE^{*,*}(S)$-algebras.

One of the highlights of stable motivic homotopy theory, is that every "sufficiently good" cohomology theory, such as Weil cohomology theories, is represented by a motivic commutative ring spectrum in $\SH(S)$. For this matter, one can see for instance \cite{CisDeg:Weil}.

\begin{exmp}[Algebraic $K$-theory]
\label{exmp:KGL}
For a scheme $X$ we have the Thomason-Trobaugh connective $K$-theory spectrum $K(X)$, with $n$-th homotopy group the algebraic $K$-theory of $X$, $K_n(X)$, $n\ge0$, \cite[Definition 3.1]{Thomason:K-theory}, that is, the Waldhausen $K$-theory of the complicial category of perfect complexes on $X$. If $X$ is Noetherian and admits an ample family of line bundles, then by \cite[Exercise 5.7]{Thomason:K-theory}, $K_n(X)$ agrees with the Quillen $K$-theory of the exact category of  locally free coherent sheaves. By \cite[3.1.1]{Thomason:K-theory}, $K_0(X)$ agrees with the Grothendieck group of the triangulated category of perfect complexes on $X$. 

Let $S$ be a regular Noetherian scheme of finite Krull dimension. Then Morel and Voevodsky \cite[\S 4 Theorem 3.13]{morvoe:homotopytheory} have shown that the functor $K_*$ from $\Sm/S$ to $\N$-graded abelian groups is represented in $\sH_\bullet(S)$ via a natural isomorphism (for $n\ge m\ge0$)
 \begin{equation}
    \label{eq:repr.K-theory}
        \Map_{\sH_\bullet(S)}(\Sigma^{n-m,m}X_+, \coprod_{i \in \Z}\BGL_+)\simeq K_{n-m}(X),
    \end{equation}
where $\BGL \in \sH_{\bullet}(S)$ is a colimit of infinite Grassmannians (see \cite[Section 6.2]{voe:homotopy_theory} or \cite[\S4 Section 4.3]{morvoe:homotopytheory} or our \S \ref{section:MGL}). 
 
 For $S$ as above a Noetherian scheme of finite Krull dimension, Voevodsky \cite[Section 6.2]{voe:homotopy_theory} has constructed 
a $\P^1$-spectrum $\KGL \in \SH(S)$, building on \eqref{eq:repr.K-theory} (Voevodsky denoted 
 this spectrum by $\mathbf{BGL}$, but it is now more often denoted $\KGL$). For $S$ regular,  $\KGL$ represents $K_*(-)$ on $\Sm/S$ by natural isomorphisms
 \begin{equation}\label{eqn:KThIso}
 \KGL^{a,b}(X)\simeq K_{2b-a}(X),
 \end{equation}
 where as above  $K_n(X)=0$ for $n<0$. Panin-Pimenev-R\"ondigs \cite{PPR} have constructed a multiplicative structure for $\KGL$, refining $\KGL$ to a commutative monoid in symmetric $\P^1$-spectra, making $\KGL\in \SH(S)$ a highly structured motivic commutative ring spectrum. Moreover, by \cite[Corollary 2.2.4]{PPR}, the resulting  multiplication in $\KGL^{*,*}(X)$ is compatible via \eqref{eqn:KThIso} with the product structure on the Waldhausen-Thomason-Trobaugh $K$-theory $K_*(X)$. Thus, $\KGL$ represents $K_*$ as a cohomology theory on $\Sm/S$.
 
By \cite[Theorem 6.8]{voe:homotopy_theory}, algebraic $K$-theory satisfies Bott periodicity $\KGL \simeq \Sigma^{2,1}\KGL$, representing $\KGL$ as a $(2,1)$-periodic cohomology theory, and the corresponding isomorphism  $ \KGL^{a,b}(-)\simeq  \KGL^{a+2,b+1}(-)$ goes over to the identity on $K_*(-)$ via \eqref{eqn:KThIso}.
\end{exmp}

\subsubsection{Highly structured motivic ring spectra}
\label{subsection:HighlyStructuredRings}

In the next sections we will need to consider modules over a motivic ring spectrum $\sE$ in a homotopical sense, that is, as objects in the homotopy category of $\sE$-modules. The problem with our definition of $\SH(S)$ is that the category $\SPP_S(T)$ of $T$-spectra is not a symmetric monoidal model category. However, there exist various models for $\SH(S)$ that are "highly structured", in the sense that they come from some symmetric monoidal model category $\SPP^{\text{hs}}_S$ with monoidal product $\wedge$, such that the homotopy category $\text{Ho}(\SPP_S^{\text{hs}})$ is equivalent to $\SH(S)$ as a symmetric monoidal category. One such model is given by the already mentioned symmetric spectra, which we now recall for the sake of exposition.

\begin{defn}[\cite{Jardine:SymSpectra}, \cite{Hovey:SymmSpectra}]
     Let $T \in \Spc_{\bullet}(S)$ be a pointed space. A \emph{symmetric $T$-spectrum} $\E$ is a sequence of pointed spaces $(E_0,E_1, E_2, \ldots)$, where each $E_r$ is equipped with an action $\Sigma_r \times E_r \to E_r$ of the symmetric group $\Sigma_r$, with bonding maps $\epsilon_r:T \wedge E_r \to E_{r+1}$, such that the induced maps $T^{\wedge m}\wedge E_r \to E_{r+m}$ are $(\Sigma_m \times \Sigma_r)$-equivariant for all $r,m \ge 0$. A \emph{morphism of symmetric $T$-spectra} $f:\E \to \F$ is a collection $\{f_r\}_{r \ge 0}$, with $f_r:E_r \to F_r$ $(\Sigma_r)$-equivariant map of pointed spaces, such that all the obvious squares with the bonding maps commute. This defines the \emph{category of symmetric $T$-spectra} $\SPP^\Sigma_S(T)$. We call a commutative monoid object in $\SPP^\Sigma_S(T)$ a \emph{commutative $T$-monoid}.
\end{defn}

As shown in \cite{Jardine:SymSpectra} and \cite{Hovey:SymmSpectra}, the category $\SPP^\Sigma_S(T)$ is endowed with a closed model structure analogous to the one in $\SPP_S(T)$, the smash product $\wedge$ of pointed spaces induces a smash product $\wedge$ on $\SPP^\Sigma_S(\P^1)$ that makes it into a symmetric monoidal category, and the homotopy category $\text{Ho}(\SPP^\Sigma_S(\P^1))$ of symmetric $\P^1$-spectra is equivalent to $\SH(S)$ as a symmetric monoidal category (see \cite[\S 4.4]{Jardine:SymSpectra} or \cite[\S 4]{Hovey:SymmSpectra}).

Moreover, one can prove that $\text{Ho}(\SPP^\Sigma_S(\P^1))$ is equivalent, as symmetric monoidal category, to the homotopy category $\text{Ho}(\SPP^\Sigma_S((\P^1)^{\wedge 2}))$ of symmetric $(\P^1)^{\wedge 2}$-spectra (see for instance \cite[Theorem 3.2]{Panwal-cobordism}).

\begin{defn}
\label{defn:highlyStructuredRings}
    A motivic commutative ring spectrum $\sE \in \SH(S)$ is said to be a \emph{highly structured motivic commmutative ring spectrum} if it admits a model $\sE^{\hs}$ as a commutative monoid object in a certain symmetric monoidal model category $\SPP^{\hs}_S$ such that $\text{Ho}(\SPP^{\hs}_S) \simeq \SH(S)$ as symmetric monoidal categories. For $\sE$ a highly structured motivic commutative ring spectrum with model $\sE^{\hs}\in \SPP^\hs_S$, the homotopy category $\text{Ho}(\Mod_{\sE^{\hs}})$ of $\sE^{\hs}$-module objects in $\SPP^{\hs}_S$ gives the \emph{category of $\sE$-modules} $\Mod_\sE$. Moreover, the usual free-forget adjunction on $\sE^{\hs}$-modules in $\SPP^\hs_S$ induces the adjunction
\[
\text{Free}_\sE:\SH(S)\xymatrix{\ar@<3pt>[r]&\ar@<3pt>[l]}\Mod_\sE:\text{Forget},
\]
where, for a motivic spectrum $\E \in \SH(S)$, $\text{Free}_\sE(E)$ is the $\sE$-module $\sE \wedge \E$ with module map $\mu_\sE \wedge \id_\E:\sE \wedge \sE \wedge \E \to \sE \wedge \E$.
\end{defn}

As shown above, $\SPP_S^\Sigma(\P^1)$ and $\SPP^\Sigma_S((\P^1)^{\wedge 2})$ offer models for $\SPP^\hs_S$, hence, commutative monoids in  $\SPP_S^\Sigma(\P^1)$ and $\SPP^\Sigma_S((\P^1)^{\wedge 2})$ define highly structured motivic commutative ring spectra in $\SH(S)$. 
\begin{exmp}
\label{exmp:CommMonoids}
    \begin{enumerate}
        \item The motivic Eilenberg-MacLane spectrum $H\Z$ representing motivic cohomology is a commutative $\P^1$-monoid (see \S \ref{subsection:HZ}).
        \item The algebraic cobordism spectrum $\MGL$ is a commutative $\P^1$-monoid (see \S \ref{section:MGL}).
        \item The symplectic cobordism spectrum $\MSp$ is a commutative $(\P^1)^{\wedge 2}$-monoid (see \S \ref{subsection:MSp}).
    \end{enumerate}
\end{exmp}

Other models for $\SPP^\hs_S$ include orthogonal spectra (see \cite{OrthogonalSpectra}) and $\mathbb{S}_S$-modules (see \cite{EKMM:Modules} and \cite{Mandell:S-mod}). However, every time we will consider a highly structured motivic commutative ring spectrum $\sE$, we will implicitly fix a specific model, and the category $\Mod_\sE$ will be the category arising from that model. In particular, we can consider $\Mod_{H\Z}$ as the category arising from the model of Example \ref{exmp:CommMonoids} (1).   

Before moving on, we point out that there exists in the literature a different setting where the construction of the stable motivic homotopy category $\SH(S)$ takes place.

\begin{rmk}
    For any quasi-compact quasi-separated scheme $S$, there exists a model for the motivic stable homotopy category $\SH(S)$ as an $(\infty,1)$-category, defined in \cite[Appendix C]{Hoy:Grothendieck} and revisited in Khan's thesis \cite{khan:thesis}. As detailed in Hoyois' paper, for $S$ Noetherian of finite dimension, the homotopy category of this $(\infty,1)$-category is equivalent, as a triangulated category, to the category $\SH(S)$ constructed by Voevodsky in \cite{voe:homotopy_theory}. In this setting, the notion of highly structured motivic commutative ring spectra corresponds to the notion of $E_\infty$-ring spectra, explained in \cite{May:Einfty}.
\end{rmk}

\subsubsection{Motivic cohomology}
\label{subsection:HZ}

Motivic cohomology is a cohomology theory for schemes that provides, in many ways, a motivic analogue of singular cohomology in the classical setting. Its definition has been refined multiple times over the years. Historically, the first attempt in this direction is Bloch's definition of higher Chow groups for smooth schemes (see \cite{Bloch:cycles}), later extended by Levine in \cite{Lev:cycles}. Let us recall the definition from \cite{Bloch:cycles}. Let $X$ be a smooth scheme over a field $k$. Let $z^i(X)$ be the group of $i$-th algebraic cycles, namely the free abelian group generated by the irreducible closed subschemes of $X$. The algebraic simplices $\Delta_k^n$ over $k$, with the obvious coface and codegeneracy maps, give the cosimplicial $k$-scheme $\Delta_k^{\bullet}$. One takes the simplicial abelian group $z^p(X,\bullet)$ to be the subgroup of $z^p(X \times \Delta_k^{\bullet})$ generated, in simplicial degree $q$, by the cycles that intersect properly with $X \times F$, for all faces $F$ of $\Delta_k^q$. Finally, Bloch defines the \emph{higher Chow groups of $X$} as the homotopy groups $\text{CH}^p(X,q) \coloneqq \pi_q(z^p(X,\bullet))$. It turns out that the groups $\text{CH}^p(X,0)$ recover the ordinary Chow groups $\text{CH}^p(X)$ defined by Fulton in \cite{Fulton}.

Voevodksy defined motivic cohomology in the setting of stable homotopy theory, that is, as the cohomology associated to a certain motivic spectrum $H\Z \in \SH(S)$, called the \emph{motivic Eilenberg-MacLane spectrum}. $H\Z$ can be written as a $\P^1$-spectrum
$$H\Z \coloneqq (\Z, K(\Z(1),2), K(\Z(2),4), \ldots),$$
where the spaces $K(\Z(n),2n)$, called \emph{motivic Eilenberg-MacLane spaces}, are the motivic analogue of the topological Eilenberg-MacLane spaces. Over a field of characteristic zero, the construction is described in \cite[Section 6.1]{voe:homotopy_theory}. The definitions has been extended over more general bases in many ways (see works of R\"{o}ndigs-{\O}stvaer \cite{rond:modules}, Hoyois-Kelly-{\O}stvaer \cite{Hoy:Steenrod}, Spitzweck \cite{Spitzweck:HZ}, Cisinki-Déglise \cite{deglise:mixmot}, Hoyois \cite{Hoyois:localization}). We will not recall these constructions here. 

The motivic cohomology represented by $H\Z$ is related with Bloch's higher Chow groups through the following result.

\begin{thm}[\cite{MazWei:lectures}, Theorem 19.1]
\label{thm:chow-motiviccohomology}
    For all $X \in \Sm/k$, with $k$ a perfect field, and $p,q \ge 0$, we have 
    $$H\Z^{p,q}(X) \simeq \CH^q(X, 2q-p).$$
\end{thm}

Since $\text{CH}^p(X,0)=\text{CH}^p(X)$, we also have the following.

\begin{corollary}
\label{cor:HZ,CH}
    In the same setting as Theorem \ref{thm:chow-motiviccohomology}, we have
    $$H\Z^{2p,p}(X) \simeq \CH^p(X).$$
\end{corollary}

We recall the following well known result.

\begin{thm}
    The motivic Eilenberg-MacLane spectrum $H\Z$ is a highly structured motivic commutative ring spectrum.
\end{thm}

For a proof, one can see \cite{DunRon:Functors}. In fact, in the paper, they use a different model for $H\Z$, as a motivic functor factorizing through the category of motivic spaces with transfer, and in \cite[Example 3.4]{DunRon:Functors} they show that this is a commutative monoid in the category of motivic functors. But this motivic functor is represented by Voevodsky's motivic Eilenberg-MacLane spectrum by \cite[Lemma 4.6]{DunRon:Functors}. 

This allows one to define the category $\Mod_{H\Z}$ of $H\Z$-modules. For $S=\Spec k$, $k$ a field, we let $\DM(k) \coloneqq \Mod_{H\Z}$.

\begin{rmk}
     Usually, $\DM(k)$ denotes Voevodsky's triangulated category of motives constructed in \cite{voev:MotivicHomology}. However, for $k$ a field of characteristic zero, the main result of \cite{rond:modules} shows that this is equivalent to $\Mod_{H\Z}$, and for $\chr(k)=p >0$, the equivalence holds after inverting $p$ by \cite[Theorem 5.8]{Hoy:Steenrod}. So for our purposes in this work there should be no confusion.
\end{rmk}

As in Definition \ref{defn:highlyStructuredRings}, there is a $\text{Free}_{H\Z}$-Forgetful adjunction 
$$\begin{tikzcd}
    M:\SH(\Spec k) \arrow[r, shift left] & \DM(k): \text{EM}, \arrow[l, shift left]
\end{tikzcd}
$$
where $\text{EM}$, called the \emph{motivic Eilenberg-MacLane functor}, forgets the $H\Z$-module structure, and $M$ is the functor $\E \mapsto M(\E)\coloneqq \E \wedge H\Z$. The $H\Z$-module $M(\E)$ is sometimes called the \emph{motive of $\E$}.

We recall a partial computation of motivic cohomology:

\begin{thm}[\cite{MazWei:lectures}, Theorem 19.1, Theorem 19.3, Corollary 4.2]
\label{thm:MazWeiHZ}
    For $X \in \Sm/k$, with $k$ a perfect field, $H\Z^{p,q}(X)=0$ if either $q<0$, $p > 2q$ or $p>q +\dim(X)$. Moreover, $H\Z^{0,0}(\Spec k)=\Z$, $H\Z^{p,0}(\Spec k)=0$ for $p \neq 0$, and $H\Z^{p,1}(\Spec k)=0$ for $p \neq 1$.
\end{thm}

 \subsection{The six functors}

Let us fix a base scheme $B$ Noetherian of finite Krull dimension, and recall that $\Sch/B$ denotes the category of quasi-projective $B$-schemes.

As the scheme $S$ varies in $\Sch/B$, the categories $\SH(S)$ acquire the formalism of Grothendieck's six operations, as discussed in \cite{ayoub:sixfunctors}, and revisited in \cite{deglise:mixmot}. In this subsection we recall the main features of the six-functors formalism.

\begin{defnprop}
\label{prop:fourfunctors}
    For every morphism $f:S \to T$ in $\Sch/B$, we have two adjoint functors
    $$
    \begin{tikzcd}
    f^*: \SH(T) \arrow[r, shift left] & \SH(S): f_*, \arrow[l, shift left]
    \end{tikzcd}
    $$
    where $f^*$ is symmetric monoidal and compatible with the unstable functor $f^*$ through infinite $\P^1$-suspension functors. $f^*$ and $f_*$ are called, respectively, \emph{inverse image along $f$} and \emph{direct image along $f$}. Furthermore, we have another pair of adjoint functors
    $$\begin{tikzcd}
    f_!: \SH(S) \arrow[r, shift left] & \SH(T): f^!, \arrow[l, shift left]
    \end{tikzcd}
    $$
    called, respectively, \emph{exceptional direct image along $f$} and \emph{exceptional inverse image along $f$}.
\end{defnprop}

If one lets $S$ vary in a more general category than $\Sch/B$, the functors $f_!,f^!$ are only defined for $f$ separated of finite presentation. In our setting, these assumptions automatically hold for all morphisms.

Sending $S \in \Sch/B$ to $\SH(S)$, and a map of schemes $f$ to the functor $f^*$, defines a pseudofunctor
$$\SH(-): \Sch/B^{\text{op}} \to \mathbf{Tr}^\otimes$$
from $\Sch/B$ to the category of symmetric monoidal triangulated categories. 

The construction of the functors $f^*,f_,f^!,f_!$ is discussed in \cite[Chapter 1]{ayoub:sixfunctors}, as are the properties listed in the following proposition.

\begin{prop}
\label{prop:sixfunctors}
    The four functors for $\SH(-)$ defined in Proposition \ref{prop:fourfunctors} satisfy the following properties:
    \begin{enumerate}
        \item For every $f$, there is a natural transformation $f_! \to f_*$ that is invertible if $f$ is proper.
        \item If $f$ is an open immersion, there is a natural isomorphism $f^* \xrightarrow{\sim} f^!$.
        \item If $f$ is smooth, $f^*$ has a left adjoint, denoted by $f_{\#}$.
        \item (Projection formula) The exceptional direct image satisfies projection formula against the inverse image. Namely, for $f:S \to T$, $\E \in \SH(S)$, $\F \in \SH(T)$, there is a canonical isomorphism
        $$\F \wedge f_! \E \xrightarrow{\sim} f_!(f^* \F \wedge \E)$$
        in $\SH(T)$.
        \item (Base change) For a cartesian square
        $$\begin{tikzcd}
         S' \arrow[d, "g'"] \arrow[r, "f'"] & T' \arrow[d, "g"] \\
         S \arrow[r, "f"] & T, \\
        \end{tikzcd}$$
        there are canonical natural isomorphisms
        $$g^* f_! \xrightarrow{\sim} f'_!g'^* \; \; \; \text{and} \; \; \; g'_*f'^! \xrightarrow{\sim}f^!g_*.$$
    \end{enumerate}
\end{prop}

In addition to this, the $\A^1$-invariance property of $\SH(S)$ can be stated in terms of the four functors as follows.

\begin{prop}[$\A^1$-invariance]
\label{prop:hom-invariance}
    If $\pi:E \to X$ is a vector bundle, $\pi^*$ is fully faithful. In particular, the unit map $\id_{\SH(X)} \to \pi_*\pi^*$ and the counit map $\pi_\# \pi^* \to \id_{\SH(X)}$ are invertible.
\end{prop}

\begin{defn}
    The family of functors $(f^*,f_*,f_!,f^!)$, together with the monoidal product bifunctor $\wedge$ and the internal-Hom bifunctor $\underline{\Hom}$, are called the \emph{six funtors} for $\SH(-)$.
\end{defn}

\begin{rmk}
    Let $p:X \to S$ a smooth map, and $\E \in \SH(S)$. By the description of the unstable $p_\#$ of Remark \ref{rmk:unstablefunctors}, we have, in $\Spc_\bullet(S)$, $p_\#((\id_X)_+)=(X\xrightarrow{p}S)_+=X_+$. Since the stable $f^*$ is compatible with the unstable $f^*$ through $\Sigma^\infty_{\P^1}$, the same is true for their left adjoints $p_\#$. Thus, in $\SH(S)$, $p_\# 1_X = p_\# (\Sigma^\infty_{\P^1}(\id_X)_+) \simeq \Sigma_{\P^1}^\infty X_+$. 
\end{rmk}

We also mention a well known reformulation of Morel-Voevodsky localization theorem in the language of six functors:

\begin{prop}
\label{prop:localization}
    Let $i:Z \to S$ be a closed immersion, with complementary open immersion $j:U \to S$. Then we have the distinguished triangles
    $$i_!i^! \xrightarrow{\eta_{(i_!,i^!)}} \id_{\SH(S)} \xrightarrow{\epsilon_{(j^*,j_*)}} j_*j^* \to i_!i^![1] \; \; \; \text{and} \; \; \; j_!j^! \xrightarrow{\eta_{(j_!,j^!)}} \id_{\SH(S)} \xrightarrow{\epsilon_{(i^*,i_*)}} i_*i^* \to j_!j^![1]$$
    of endofunctors of $\SH(S)$, called the localization sequences, which give distinguished triangles once applied to any spectrum $\E \in \SH(S)$. The notations $\eta_{(F,G)}$ and $\epsilon_{(F,G)}$ stand for the counit and the unit of the adjunction $F \dashv G$.
\end{prop}

For a proof see for instance \cite[Corollary 3.2.3]{ayoub:sixfunctors}.

\subsubsection{Thom equivalences}
\label{subsection:thomequivalences}

Let $\pi:V \to S$ be a vector bundle, with zero section $s_0:X \hookrightarrow V$. Then the adjunction
$$
\begin{tikzcd}
    \Sigma^V \coloneqq \pi_\# s_{0!}: \SH(S) \arrow[r, shift left] & \SH(S): s_0^! \pi^* =: \Sigma^{-V} \arrow[l, shift left]
\end{tikzcd}
$$
is a self-equivalence of $\SH(S)$.
\begin{defn}
    The endofuntors $\Sigma^V$ and $\Sigma^{-V}$ are called the \emph{$V$-suspension} and the \emph{$V$-desuspension} respectively. Often, one calls them generically the \emph{Thom equivalences}.
\end{defn}

All the results about Thom equivalences that we are going to state here are discussed in \cite[Section 1.5]{ayoub:sixfunctors} and recalled in \cite[Section 2.1]{deglise-jin-khan}.

Thom equivalences are compatible with the monoidal structure, in the sense that, for $V\to S$ vector bundle, and $\E,\F \in \SH(S)$, one has $\Sigma^V\E \wedge \F \simeq \Sigma^V(\E \wedge \F)$, and analogously for $\Sigma^{-V}$. From this, Thom equivalences can be read as
$$\Sigma^V(-)=\Sigma^V1_S \wedge (-) \; \; \; \text{and} \; \; \; \Sigma^{-V}(-)=\Sigma^{-V}1_S \wedge (-).$$
Thom equivalences are also compatible with the four functors $(f^*,f_*,f_!,f^!)$, in the sense that one has canonical isomorphisms
\begin{equation}
\label{eq:thomcompatibilities}
    f^*\Sigma^V \simeq \Sigma^{f^*V}f^*, \; \; \Sigma^Vf_* \simeq f_*\Sigma^{f^*V}, \; \; \Sigma^Vf_! \simeq f_!\Sigma^{f^*V}, \; \; f^!\Sigma^V \simeq \Sigma^{f^*V}f^!.
\end{equation}

Let $K(S)$ be the Thomason-Trobaugh $K$-theory space of perfect complexes on $S$ (\cite[Definition 3.1]{Thomason:K-theory}), and let $\sK(S)$ denote the fundamental groupoid of $K(S)$. Let $\sV$ be a locally free sheaf of finite rank over $S$, and let us consider the associated vector bundle $V =\V(\sV) \coloneqq \Spec_{\mathcal{O}_X}\Sym^*(\sV)$. Again, we have endofunctors of $\Sigma^\sV \coloneqq \Sigma^V$ and $\Sigma^{-\sV}\coloneqq \Sigma^{-V}$ of $\SH(S)$. Clearly, one also recovers $\Sigma^V$ as $\Sigma^\sV$ with $\sV$ being the sheaf of sections of $V^\vee$.

The assignment $\sV \to \Sigma^\sV$ naturally extends to a functor of groupoids
\begin{equation}
    \label{eq:SigmaFunctorOfGroupoids}
    \Sigma^{(-)}: \sK(S) \to \Aut(\SH(S)),
\end{equation}
where $\Aut(\SH(S))$ is the groupoid of autoequivalences of $\SH(S)$ with their natural isomorphisms as morphisms.

\begin{rmk}
    Let us note that a distinguished triangle $\sV' \to \sV \to \sV''\to \sV'[1]$ of perfect complexes on $S$ induces canonically an identification $\sV \simeq \sV' + \sV''$ in $\sK(S)$, inducing the isomorphism:
$$\Sigma^\sV \simeq \Sigma^{\sV' +\sV''} \simeq \Sigma^{\sV'}\circ \Sigma^{\sV''}.$$
In particular, this gives the isomorphism
\begin{equation}
\label{eq:smash-stablethom}
    \Sigma^\sV 1_S \simeq \Sigma^{\sV'}1_S \wedge \Sigma^{\sV''}1_S.
\end{equation}
The canonical distinguished triangles $\sV\to \sV\oplus\sV'\to \sV'\to \sV[1]$ and $\sV'\to \sV\oplus\sV'\to \sV\to \sV'[1]$ give canonical isomorphisms $\Sigma^\sV\circ\Sigma^{\sV'}\simeq \Sigma^{\sV\oplus \sV'}\simeq \Sigma^{\sV'}\circ \Sigma^\sV$. Similarly, the distinguished triangle $\sV\to 0\to \sV[1]\to \sV[1]$ gives the canonical isomorphism $\Sigma^{\sV}\circ \Sigma^{\sV[1]}\simeq\Sigma^0=\id$, so, for $\sV$ a locally free sheaf, we have $\Sigma^{-\sV}=(\Sigma^{\sV})^{-1}=\Sigma^{\sV[1]}$. We can thus extend the notation $\Sigma^\sV$ to virtual perfect complexes by setting $\Sigma^{\sV-\sV'}:=\Sigma^{\sV}\circ\Sigma^{\sV'[1]}$.
\end{rmk}

The Morel-Voevodsky relative purity theorem \cite[\S 3 Theorem 2.23]{morvoe:homotopytheory} has a reformulation in the stable setting in terms of six functors:

\begin{thm}[\cite{deglise-jin-khan}, item 2.1.8]
\label{thm:MorVoePurity}
    Let $i:Z \hookrightarrow X$ be a closed immersion of smooth schemes over $S$ locally of finite type, with structure maps $p_Z$, $p_X$. Then there are natural isomorphisms
    $$p_{X\#}i_* \simeq p_{Z\#}\Sigma^{N_i} \; \; \; \text{and} \; \; \; \Sigma^{-N_i}p_Z^* \simeq i^!p_X^*,$$
    where $N_i$ is the normal bundle of $i$.
\end{thm}

On the other hand, Ayoub's purity theorem \cite[Section 1.6]{ayoub:sixfunctors} implies the following:

\begin{thm}[\cite{deglise-jin-khan}, item 2.1.7]
\label{thm:ayoubpurity}
    If $f$ is a smooth morphism locally of finite type, there is a canonical isomorphism
    \begin{equation}
    \label{eq:AyoubPurity}
        \Sigma^{T_f}f^* \xrightarrow{\sim} f^!,
    \end{equation}
    where $T_f$ is the relative tangent bundle of $f$.
\end{thm}

\begin{rmk}
\label{rmk:smoothsharp}
If $f$ is a smooth morphism, taking the left adjoints of both sides of the natural isomorphism \eqref{eq:AyoubPurity} gives the natural isomorphism
$$f_\#\Sigma^{-T_f}\simeq f_!.$$
\end{rmk}

We immediately note the following.

\begin{rmk}
\label{rmk:smoothsharp-properties}
    Since in general $\Sigma^V(\F \wedge \E)=\Sigma^V\E \wedge \F$, Proposition \ref{prop:sixfunctors}(4) and Remark \ref{rmk:smoothsharp} imply that, for $f$ smooth, $f_\#$ satisfies a projection formula against $f^*$. Also, $f_{\#}$ verifies the same compatibility as $f_!$ in \eqref{eq:thomcompatibilities} with Thom equivalences, since Thom equivalences commute with each other.
\end{rmk}

\begin{rmk}
\label{rmk:etalemaps}
    If $f$ is étale, $T_f$ is the zero bundle $S \to S$, so $\Sigma^{T_f}=\id_{\SH(S)}$. Thus, Theorem \ref{thm:ayoubpurity} gives an isomorphism $f^* \xrightarrow{\sim} f^!$. If $f$ is an open immersion, this is the isomorphism of Proposition \ref{prop:sixfunctors} (2) . Moreover, in this case we also have $f_\# \simeq f_!$ by Remark \ref{rmk:smoothsharp}.
\end{rmk}

For $\pi:V \to S$ a vector bundle, let $j:V^0 \to V$ be the open complement of the zero section $s_0:S \to V$. By Proposition \ref{prop:localization}, we have a distinguished triangle 
\begin{equation}
\label{eq:thom-loc}
    j_!j^!1_V \to 1_V \to s_{0*}s_0^*1_V \to j_!j!1_V[1].
\end{equation}
Let $\pi^0:V^0 \to S$ be the restriction of $\pi$. By applying $\pi_\#$ to \eqref{eq:thom-loc}, and using $j_! \simeq j_\#$ from Remark \ref{rmk:etalemaps} and $j^! \simeq j^*$ from Proposition \ref{prop:sixfunctors}(2), we obtain the distinguished triangle
\begin{equation}
\label{eq:thom-loc2}
    \pi^0_\# 1_{V^0} \to \pi_\# 1_V \to \pi_\# s_{0*}1_S \simeq \Sigma^V 1_S \to \pi^0_\# 1_{V^0}[1],
\end{equation}
where the isomorphism follows because $s_{0*}\simeq s_{0!}$ by Proposition \ref{prop:sixfunctors}(1). The distinguished triangle \eqref{eq:thom-loc2} can be rewritten as
\begin{equation}
\label{eq:thom-loc3}
    \Sigma_{\P^1}^\infty V_+^0 \to \Sigma_{\P^1}^\infty V_+ \to \Sigma^V1_S \to \Sigma_{\P^1}^\infty V_+^0[1],
\end{equation}
from which we see that $\Sigma^V1_S \simeq \Sigma^\infty _{\P^1}\Th(V)$. For this reason, the motivic spectrum $\Sigma^V 1_S$ is also called the \emph{stable Thom space of $V$}.

In view of this discussion on stable Thom spaces, we can say that the relation \eqref{eq:smash-stablethom} is a generalization of \eqref{eq:smash-thom} to non-split sequences.

\begin{rmk}
By recalling Remark \ref{rmk:Thomspaces} (2), we have $\Sigma^{\mathcal{O}_S} \simeq \P^1 \wedge (-) \simeq \Sigma^{2,1}$ as endofunctors of $\SH(S)$. By taking wedge powers of $\P^1$, one also gets $\Sigma^{\mathcal{O}_S^r}\simeq \Sigma^{2r,r}$.
\end{rmk}

\section{$\GL$-oriented and $\Sp$-oriented theories}

\subsection{$\GL$-oriented cohomology theories}

From now on, $k$ will be a perfect field of characteristic different from $2$. We use the shorthand $\SH(k) \coloneqq \SH(\Spec k)$, and $\sH_\bullet (k) \coloneqq \sH_\bullet(\Spec k)$.

While working over a fixed base field $k$, if $X$ is in $\Sm/k$ with structure map $p$, and $V \to X$ is a vector bundle, it is common to denote the space $p_\# \Th_X(V) \in \sH_\bullet(k)$ simply by $\Th(V)$, if there is no ambiguity, and call it the \emph{Thom space of $V$ over $k$}. So will we. Let us note that $\Sigma_{\P^1}^\infty p_\# \Th_X(V) \simeq p_\# \Sigma_{\P^1}^\infty \Th_X(V) \simeq p_\#\Sigma^V 1_X$ in $\SH(k)$.

For $\E,\F \in \SH(S)$, we will use the shorthand $[\E,\F]_{\SH(S)} \coloneqq \Hom_{\SH(S)}(\E,\F)$, and similarly for $\Mod_\sE$, with $\sE$ highly structured motivic ring spectrum in $\SH(S)$.

\begin{defn}
\label{defn:thomclasstheory}
A \emph{$\GL$-orientation}, or simply an \emph{orientation}, for a commutative ring spectrum $\sE\in \SH(k)$ consists of the assignment of Thom classes $\th^\sE(V)\in\sE^{2r,r}(\Th(V))=[p_\#\Sigma^V1_X,\Sigma^{2r,r}\sE]_{\SH(k)}$ for each vector bundle $V\to X$, $X\in \Sm/k$, with $r=\rnk(V)$, satisfying the axioms of \cite[Definition 3.3]{Ana:Slor}: 
\begin{equation}\label{enum:ThomClassAxioms}
\end{equation}
\begin{enumerate} 
\item Normalization: For $V=\V(\sO_X^r)$ over some smooth scheme $p:X \to \Spec k$, $\th^\sE(V)\in \sE^{2r, r}(\Th(V))$ is the image of the unit $u_X\in \sE^{0,0}(X)$ under the suspension isomorphism $\sE^{0,0}(X) = \sE^{0,0}(p_\#1_X)  \xrightarrow{\sim} \sE^{2r, r}(p_\#\Sigma^{2r,r}1_X) = \sE^{2r, r}(\Th(V))$. 
\item Naturality: Given vector bundles $V\to X$, $W\to Y$, $X,Y\in \Sm/k$, a morphism $f:Y\to X$ and an isomorphism $\alpha:W\xrightarrow{\sim} f^*V$ of vector bundles on $Y$, let $\Th(\alpha,f): \Th(W)\to \Th(V)$ be the obvious induced morphism of motivic spaces. Then
$$\Th(\alpha,f)^*(\th^\sE(V))=\th^\sE(W).$$ 
\item Multiplicativity: Given an exact sequence of vector bundles over $X\in \Sm/k$
\[
0\to V'\to V\to V''\to0
\]
of ranks $r', r,r''$, respectively, one has the corresponding cup product map $\cup:\sE^{2r', r'}(\Th(V'))\otimes \sE^{2r'', r''}(\Th(V''))\to \sE^{2r, r}(\Th(V))$ thanks to the relation \eqref{eq:smash-stablethom}. Then
\[
\th^\sE(V)=\th^\sE(V')\cup\th^\sE(V'').
\]
\end{enumerate}
\end{defn}

In \cite[Definition 3.3]{Ana:Slor} there are four axioms, but axioms $(1)$ and $(2)$ are covered by our axiom $(2)$.

Let $V\to X$ be a rank $r$ vector bundle on $X\in \Sm/k$, with structure morphism $p:X\to \Spec k$, and let $\sV$ be the sheaf of sections of $V^\vee$. Since $\Sigma_{\P^1}^\infty \Th(V) \simeq p_\#(\Sigma^\sV1_X)$, as seen in \S \ref{subsection:thomequivalences}, we may use the adjunction $p_\#\dashv p^*$ to view $\th^\sE(V)$ as an element $\th_\sE(\sV)\in \Map_{\SH(X)}(\Sigma^{\sV-\sO_X^r}1_X, p^*\sE)$.

\begin{lemma}
\label{lemma:ThomClassNaturalTransformation}
    The assignment $\Locfree(X) \ni \sV\mapsto \th_\sE(\sV) \in \Map_{\SH(X)}(\Sigma^{\sV-\sO_X^r}1_X, p^*\sE),$ with $r=\rnk \sV$, extends to a natural transformation of functors of groupoids from $\sK(X)$ to $\SH(X)_\simeq$
$$
[\th_\sE(-):(\Sigma^{(-)-\sO_X^{\rnk(-)}}1_X)\to c_{p^*\sE}]:\sK(X)\to \SH(X)_\simeq,
$$
where $\SH(X)_\simeq$ is the underlying groupoid of $\SH(X)$, and $c_{p^*\sE}$ is the constant functor with value $p^*\sE$.
\end{lemma}

\begin{proof}
    The functor $\Sigma^{(-)}:\sK(X)\to \Aut(\SH(X))$ of \eqref{eq:SigmaFunctorOfGroupoids} gives rise to a functor of groupoids 
    \begin{equation}
    \label{eq:FunctorOfGroupoids}
        \Sigma^{(-)-\sO_X^{\rnk(-)}}1_X:\sK(X)\to \SH(X)_\simeq.
    \end{equation}
Composing \eqref{eq:FunctorOfGroupoids} with the canonical functor of groupoids $i:\Locfree(X) \to \sK(X)$, we get a functor of groupoids $\Locfree(X) \to \SH(X)_\simeq$. Let $\alpha:\sW\to \sV$ be an isomorphism in  $\Locfree(X)$, giving the induced isomorphism $\th(\alpha):\Sigma^{\sW-\sO_X^{\rnk(\sW)}}1_X\to \Sigma^{\sV-\sO_X^{\rnk(\sV)}}1_X$. By \eqref{enum:ThomClassAxioms}(2), we see that $\th_\sE(\sV)\circ \th(\alpha)= \th_\sE(\sW)$, and we can then see the assignment $\Locfree(X) \ni \sV \mapsto \th_\sE(\sV)$ as a natural transformation of functors of groupoids 
\begin{equation}
    \label{eq:NaturalTransformation}
    [\th_\sE(-):(\Sigma^{(-)-\sO_X^{\rnk(-)}}1_X)\circ i\to c_{p^*\sE}]: \Locfree(X)\to \SH(X)_\simeq,
\end{equation}
from the functor $(\Sigma^{(-)-\sO_X^{\rnk(-)}}1_X)\circ i$ to the constant functor $c_{p^*\sE}$ with value $p^*\sE$.

If $q:\tilde{X}\to X$ is a morphism in $\Sm/k$, the two functors commute with $q^*$. Thus, taking $q$ as a Jouanolou cover of $X$, we can assume that $X$ is affine. 

If $\sV \in \Locfree(X)$, with $X$ affine, we have a canonical isomorphism $$\sV\oplus \sO_X^r-\sO_X^{\rnk(\sV\oplus \sO_X^r)} \xrightarrow{\sim} \sV-\sO_X^{\rnk(\sV)}$$
in $\sK(X)$, which gives the corresponding isomorphism in $\SH(X)$ after applying $\Sigma^{(-)}1_X$. We then have a diagram
   
$$
\begin{tikzcd}
        \Sigma^{\sV\oplus \sO_X^r-\sO_X^{\rnk(\sV\oplus \sO_X^r)}}1_X \arrow[rr, "\sim"] \arrow[dr, swap, "\th_\sE(\sV\oplus \sO_X^r)"] & & \Sigma^{\sV-\sO_X^{\rnk(\sV)}}1_X \arrow[dl, "\th_\sE(\sV)"] \\
        & p^*\sE, &
\end{tikzcd}
$$
It follows by \eqref{enum:ThomClassAxioms}(1,3) that this diagram commutes. Thus, we can extend $\th_\sE(-)$ to differences $\sV-\mathcal{O}_X^r$ by letting $\th_\sE(\sV-\mathcal{O}_X^r)\coloneqq \th_\sE(\sV)$.

We recall from \cite{Gra:Quillen} that the groupoid $\sK(X)$ is obtained from the groupoid $\Locfree(X)$ by inverting the autoequivalence $\oplus \mathcal{O}_X$. Therefore, we can extend the natural transformation \eqref{eq:NaturalTransformation} to a natural transformation
$$[\th_\sE(-):(\Sigma^{(-)-\sO_X^{\rnk(-)}}1_X)\to c_{p^*\sE}]:\sK(X)\to \SH(X)_\simeq.$$
\end{proof}

The natural transformation $\th_\sE(-)$ is multiplicative in the following sense: given $v,v'\in \sK(X)$, we have a canonical isomorphism
\[
\Sigma^{v+v'-\sO_X^{\rnk(v+v')}}1_X\simeq \Sigma^{v-\sO_X^{\text{rank}(v)}}1_X\wedge_X\Sigma^{v'-\sO_X^{\rnk(v')}}1_X.
\]

\begin{lemma}\label{lem:ThomMult}
For $v,v'\in \sK(X)$, $\th_\sE(v+v')$ is the composition
\begin{multline}\label{multline:ThomMult}
\Sigma^{v+v'-\sO_X^{\rnk(v+v')}}1_X\simeq \Sigma^{v-\sO_X^{\rnk(v)}}1_X\wedge_X\Sigma^{v'-\sO_X^{\rnk(v')}}
1_X\\\xrightarrow{\th_\sE(v)\wedge\th_\sE(v')}p^*\sE\wedge_Xp^*\sE\xrightarrow{\mu_{p^*\sE}} p^*\sE\notag
\end{multline}
where $\mu_{p^*\sE}$ is the multiplication on $p^*\sE$. 

Consequently, letting $r:=\rnk(v)$ and $r'=\rnk(v')$, we have the following commutative diagram
\[
\xymatrix{
\Sigma^{v+v'-\sO_X^{r+r'}}1_X\ar[ddrrr]^{\th_\sE(v+v')}\ar[d]_-\wr\ar[r]^-\sim&\Sigma^{v-\sO_X^r}(\Sigma^{v'-\sO_X^{r'}}1_X)\ar[rr]^-{\Sigma^{v-\sO_X^r}(\th_\sE(v'))}&&\Sigma^{v-\sO_X^r}p_X^*\sE\ar[d]^-{\id\wedge\th_\sE(v)}\\
\Sigma^{v'-\sO_X^{r'}}(\Sigma^{v-\sO_X^r}1_X)
\ar[d]_-{\Sigma^{v'-\sO_X^{r'}}(\th_\sE(v))}&&&p_X^*\sE\wedge_Xp^*\sE\ar[d]^-{\mu_{p^*\sE}}\\
\Sigma^{v'-\sO_X^{r'}}p^*\sE\ar[r]^-{\th_\sE(v')\wedge\id}&p_X^*\sE\wedge_Xp^*\sE\ar[rr]^-{\mu_{p^*\sE}}&&p^*\sE.
}
\]
\end{lemma}
The proof is a direct consequence of the multiplicativity of Thom classes \eqref{enum:ThomClassAxioms}(3).

\begin{defn} For $v\in \sK(X)$, $X\in \Sm/k$, let $\th_{\sE}^f(v):\Sigma^{v-\sO_X^{\rnk(v)}}p^*\sE\to p^*\sE$ be the composition  
\[
\Sigma^{v-\sO_X^{\rnk(v)}}p^*\sE=\Sigma^{v-\sO_X^{\rnk(v)}}1_X\wedge_Xp^*\sE\xrightarrow{\th_\sE(v)\wedge\id_{p^*\sE}}p^*\sE\wedge_X p^*\sE\xrightarrow{\mu_{p^*\sE}} p^*\sE.
\]
Also, for $\sE$ a highly structured motivic commutative ring spectrum, let $\th^f_{\sE\Mod}(v):\Sigma^{v-\sO_X^{\rnk(v)}}p^*\sE\to p^*\sE$ denote the map in $\Mod_{p^*\sE}$ arising from the morphism $\th_\sE(v)$ in $\SH(X)$ by the Free-Forgetful adjunction between $\SH(X)$ and $\Mod_{p^*\sE}$.
\end{defn}

\begin{rmk}
\label{rmk:th(-)&th_E(-)}
Applying the forgetful functor to $\th^f_{\sE\Mod}(v)$ gives back the map $\th_{\sE}^f(v)$. For arbitrary $\sE$, we recover $\th_\sE(v)$ from $\th_{\sE}^f(v)$ by composing with the unit $\epsilon_{p^*\sE}\in \sE^{0,0}(X)=[1_X,p^*\sE]_{\SH(X)}$:
\[
\th_\sE(v)=\th_{\sE}^f(v)\circ \Sigma^{v-\mathcal{O}_X^{\rnk(v)}}\epsilon_{p^*\sE}
\]
\end{rmk}

\begin{lemma}\label{lem:ThomIso} For $X\in \Sm/k$, $v\in \sK(X)$, $\sE\in \SH(k)$ oriented, the map $\th_{\sE}^f(v):\Sigma^{v-\sO_X^{rnk(v)}}p^*\sE\to p^*\sE$ in $\SH(X)$ is an isomorphism in $\SH(k)$, and for $\sE$ highly structured, $\th^f_{\sE\Mod}(v)$ is an isomorphism in $\Mod_{p^*\sE}$.
\end{lemma}

\begin{proof} Since the forgetful functor $\Mod_{p^*\sE}\to\SH(X)$ reflects isomorphisms, it suffices to see that $\th_{\sE}^f(v)$ is an isomorphism in $\SH(X)$. It follows from the normalization axiom (2) that for $v=[\sO_X^r]$, $\th_\sE(v):1_X\to p^*\sE$ is the unit $u_{\sE,X}\in \sE^{0,0}(X)$, which easily implies the lemma for $v=[\sO_X^r]$. In general, we take a Zariski open cover $\sU=\{U_i\}_i$ for $X$ for which the restriction of $v$ to each $U_i$ is $r_i[\sO_{U_i}]$ for some integer $r_i$. The result then follows from the gluing axiom for the sheaf $\SH(-)$ on $X_\Zar$.
\end{proof}

Analogously to $\th_\sE(-)$, $\th_{\sE}^f(-)$ can be viewed as a natural isomorphism of functors of groupoids
$$[\th_{\sE}^f(-):(\Sigma^{(-)-\sO_X^{\rnk(-)}}p_X^*\sE)\to c_{p_X^*\sE}]:\sK(X)\to \SH(X)_\simeq.$$

\begin{prop}
\label{prop:K_0ExtensionTh(-)}
Let $\sE\in \SH(k)$ be oriented and take $X\in \Sch/B$. \\[5pt]
1. The functor of groupoids
\[
\Sigma^{(-)}p_X^*\sE:\sK(X)\to \SH(X)_\simeq,\ v\mapsto \Sigma^vp_X^*\sE,
\]
factors, up to natural isomorphism, through the quotient map $\sK(X)\to \pi_0\sK(X)=K_0(X)$, to define a map
\[
\Sigma^{(-)}p_X^*\sE:K_0(X)\to \Obj(\SH(X)),\ v\mapsto \Sigma^vp_X^*\sE.
\]
2. The natural isomorphism $[\th_{\sE}^f(-):(\Sigma^{(-)-\sO_X^{\rnk(-)}}p_X^*\sE)\to c_{p_X^*\sE}]:\sK(X)\to \SH(X)_\simeq$ descends to a natural isomorphism
\[
[\th_{\sE}^f(-):(\Sigma^{(-)-\sO_X^{\rnk(-)}}p_X^*\sE\to c_{p_X^*\sE}]:K_0(X)\to \SH(X)_\simeq. \]
\end{prop}

\begin{proof} (2) follows directly from (1).

For (1), let $\phi:v\to v'$ be an isomorphism in $\sK(X)$,  let $r=\rnk(v)=\rnk(v')$ and let $\Sigma^\phi(p_X^*\sE):\Sigma^{v}p_X^*\sE\simeq \Sigma^{v'}p_X^*\sE$ be the isomorphism induced from $\Sigma^\phi:\Sigma^v \simeq \Sigma^{v'}$. This gives us the commutative diagram of isomorphisms in $\SH(X)$
\[
\xymatrixcolsep{40pt}
\xymatrix{
\Sigma^{v-\sO_X^r}p_X^*\sE\ar[d]_{\Sigma^{-\sO_X^r}\Sigma^\phi(p_X^*\sE)}\ar[r]^-{\th_{\sE}^f(v)}&p_X^*\sE\\
\Sigma^{v'-\sO_X^r}p_X^*\sE\ar[ru]_{\th_{\sE}^f(v')}
}
\]
In other words $\Sigma^{\phi}(p_X^*\sE)=\Sigma^{\sO_X^r}(\th_{\sE}^f(v)^{-1}\circ \th_{\sE}^f(v))$. This shows that $\Sigma^{\phi}(p_X^*\sE)$ is independent of the choice of $\phi:v\to v'$, proving (1). 
\end{proof}

For $V_1, V_2$ vector bundles on $X\in \Sm/k$, we have the Thom class $\th_{\sE}^f(V_1-V_2)$ defined as $\th_{\sE}^f([\sV_1]-[\sV_2])$, where $\sV_i$ is the sheaf of sections of $V_i^\vee$.
\begin{lemma}\label{lem:ThomClassVirtualBundles} Let $V_1\to X$, $V_2\to X$ vector bundles on $X\in \Sm/k$ of rank $r_1, r_2$, and let $\sV_1, \sV_2$ be the respective locally free sheaves of sections of $V_1^\vee, V_2^\vee$. Then $\th_{\sE}^f(V_1-V_2)\in [\Sigma^{[\sV_1]-[\sV_2]-(r_1-r_2)[\sO_X]}p^*\sE, p^*\sE]_{\SH(X)}$ is the composition
\begin{multline*}
    \Sigma^{[\sV_1]-[\sV_2]-(r_1-r_2)[\sO_X]}p^*\sE\xrightarrow{\Sigma^{r_2[\sO_X]-[\sV_2]}\th_{\sE}^f(V_1)}\Sigma^{-[\sV_2]+r_2[\sO_X]}p^*\sE \\ \xrightarrow{(\Sigma^{-[\sV_2]+r_2[\sO_X]}\th_{\sE}^f(V_2))^{-1}}p^*\sE
\end{multline*}
In particular, for $V\to X$ a vector bundle of rank $r$, we have
$$\th_{\sE}^f(-V)=(\Sigma^{-[\sV]+r[\sO_X]}\th_{\sE}^f(V))^{-1}.$$
\end{lemma} 

The proof is a direct computation.

An important feature characterizing oriented cohomology theories is being endowed with the data of pushforward maps for maps that are smooth and proper. The construction is related with the one of proper dual maps.

\begin{defn}
\label{defn:DualMap}
    Let $f:X \to Y$ a proper map in $\Sch/k$, and $p_X, p_Y$ the structure maps of the schemes. By Proposition \ref{prop:sixfunctors} (1), we have $f_*\simeq f_!$. We than have the natural transformation
  $$p_{Y!}\xrightarrow{p_{Y!}\epsilon_{f^*,f_*}}p_{Y!}f_*f* \simeq p_{Y!}f_!f^*=p_{X!}f^*.$$
    By applying this to $1_Y$, we obtain the map
    $$f^\vee: p_{Y!}1_Y \to p_{X!}1_X,$$
    in $\SH(k)$, that we call the \emph{dual map of $f$}.
\end{defn}

\begin{rmk}
    The term \emph{dual map} comes from the fact that, in the symmetric monoidal category $\SH(k)$, the object $p_{X\#}1_X \simeq \Sigma^\infty_{\P^1}X_+$, with $p_X:X \to \Spec k$ smooth and proper, is strongly dualizable, with dual $p_{X!}1_X$, and the map $f^\vee$ corresponds to the dual map of $\Sigma^\infty_{\P^1}f:p_{X\#}1_X \to p_{Y\#}1_Y$. This follows from stable ambidexterity \cite[Theorem 6.9]{Hoyois:6functors} and Atiyah duality \cite[Corollary 6.13]{Hoyois:6functors}. In Hoyois' paper \cite{Hoyois:6functors}, the statements are proved in the equivariant stable homotopy category, but the result clearly specializes to the classical category $\SH(k)$.    
\end{rmk}

\begin{constr}[Proper pushforward]
\label{constr:properpushforward}
    Let $\sE \in \SH(k)$ oriented. Let $X,Y$ smooth schemes of dimension $d_X,d_Y$ over $k$ respectively, and structure maps $p_X,p_Y$, and let $f:X \to Y$ be a proper map in $\Sm/k$ of relative dimension $d$.
    
    The dual map $f^\vee:p_{Y!}1_Y \to p_{X!}1_X$ of Definition \ref{defn:DualMap} gives rise to a pullback
    $$(f^\vee)^*:\sE^{a,b}(p_{X!}1_X) \to \sE^{a,b}(p_{Y!}1_Y)$$
    for all $a,b \in \Z$. Since $p_X$ is smooth, by Remark \ref{rmk:smoothsharp} we have $p_{X!}\simeq p_{X\#}\Sigma^{-T_{p_X}}=p_{X\#}\Sigma^{-T_X}$. Then:
    \begin{multline*}
        \sE^{a,b}(p_{X!}1_X) = [p_{X!}1_X,\Sigma^{a,b}\sE]_{\SH(k)} \simeq [p_{X\#}\Sigma^{-T_X}1_X, \Sigma^{a,b}\sE]_{\SH(k)} \simeq [1_X,\Sigma^{T_X}p_X^*\Sigma^{a,b}\sE]_{\SH(X)} \\ \simeq [1_X,\Sigma^{a,b}\Sigma^{T_X}p_X^*\sE]_{\SH(X)} \xrightarrow[\sim]{\circ \Sigma^{a,b}\th_{\sE}^f(T_X)}[1_X,\Sigma^{a,b}\Sigma^{2d_X,d_X}p_X^*\sE]_{\SH(X)} \\ \simeq [1_X,p_X^*\Sigma^{a+2d_X,b+d_X}\sE]_{\SH(X)} \simeq [p_{X\#}1_X,\Sigma^{a+2d_X,b+d_X}\sE]_{\SH(k)}=\sE^{a+2d_X,b+d_X}(X).
    \end{multline*}
    In the same way, we have $\sE^{a,b}(p_{Y!}1_Y) \simeq \sE^{a+2d_Y,b+d_Y}(Y).$ Thus, by reindexing $a \mapsto a-2d_X$ and $b \mapsto b- d_X$, we have that $(f^\vee)^*$ gives a map
    $$f_*:\sE^{a,b}(X) \to \sE^{a-2d,b-d}(Y),$$
    for all $a,b \in \Z$, where $d \coloneqq d_X-d_Y$. The map $f_*$ just defined is called the \emph{proper pushforward for $\sE$ along $f$}.
\end{constr}

Pullback and proper pushforward for an oriented cohomology 
satisfy the following compatibilities.

\begin{prop}\label{prop:PushPull} Let $\sE\in \SH(k)$ be an oriented commutative ring spectrum.\\[5pt]
1.(Push-pull formula) Let
\[
\xymatrix{
Y'\ar[r]^{g'}\ar[d]^{f'}&Y\ar[d]^f\\
X'\ar[r]^g&X
}
\]
be a transverse cartesian square, with $X,X', Y,Y'$ in $\Sm/k$ and with $f$ proper. Then $f'$ is also proper and $g^*f_*=f'_*g^{\prime *}$.\\[2pt]
2. (Projection formula) Let $f:Y\to X$ be a proper map in $\Sm/k$. Then for $x\in \sE^{*,*}(X)$, $y\in \sE^{*',*'}(Y)$, we have $f_*(f^*(x)\cdot y)=x\cdot f_*(y)$.
\end{prop}

The push-pull formula is a consequence of the well known analogue push-pull formula for coherent sheaves discussed in \cite[Section 9]{EGA1}. The projection formula is a direct computation.
 
\begin{defn}\label{defn:stwist} For $Y\in \Sm/k$ of dimension $d$ over $k$, a {\em stable  twist of $-T_Y$} is a pair  $(v, \vartheta)$, where $v\in K_0(Y)$ is a virtual vector bundle of virtual rank $-d$, and $\vartheta$ is an isomorphism $\Sigma^{-T_Y}1_Y\xrightarrow{\sim} \Sigma^v1_Y$ in $\SH(Y)$.  
\end{defn}

\begin{defn}\label{defn:twistclass} Let $\sE\in \SH(k)$ be an oriented commutative ring spectrum. For $p_Y:Y\to \Spec k\in \Sm/k$ proper of dimension $d$ over $k$, let $(v, \vartheta)$ be a stable twist of $-T_Y$. Let $\th_\sE(v)':\Sigma^{2d,d}p_{Y\#}\Sigma^{v}1_Y\to
\sE$ be the map corresponding to $\th_\sE(v):\Sigma^{2d,d}\Sigma^v1_Y\to p_Y^*\sE$ by adjunction. We define the class $[Y, v,\vartheta]_\sE\in \sE^{-2d,-d}(\Spec k)$ as the composition
\[
1_k\xrightarrow{p_Y^\vee}p_{Y\#}\Sigma^{-T_Y}1_Y\xrightarrow{p_{Y\#}(\vartheta)}
p_{Y\#}\Sigma^{v}1_Y\xrightarrow{\Sigma^{-2d,-d}\th_\sE(v)'}\Sigma^{-2d,-d}\sE,
\]
where $p_Y^\vee$ is the dual map of $p_Y$ of Definition \ref{defn:DualMap}, composed with the purity isomorphism $p_{Y!}1_Y\simeq p_{Y\#}\Sigma^{-T_Y}1_Y$ (Remark \ref{rmk:smoothsharp}). We use the shorthand $[Y]_\sE :=[Y,-T_Y, \id]_\sE$.
\end{defn}

\subsection{Chern classes}
\label{section:Chern}

Analogously to a theory of Thom classes as in Definition \ref{defn:thomclasstheory}, one can characterize oriented cohomology theories by using a theory of Chern classes, as in the following definition.

\begin{defn}
\label{def:ChernClassTheory}
    A theory of Chern classes for a commutative ring spectrum $\sE \in \SH(k)$ consists in the datum of cohomology classes $c_i(V) \in \sE^{2i,i}(X)$, $i=0,1,\ldots$, for each vector bundle $V \to X, X \in \Sm/k$, $c_i(V)$ being called the \emph{$i$-th Chern class of $V$}, satisfying the axioms of \cite[Definition 3.26]{Pan:oriented}:

\begin{equation}\label{enum:ChernClassAxioms}
\end{equation}
\begin{enumerate} 
\item Bounds: $c_0(V)=1$ and $c_i(V)=0$ for all $i > \rnk(V)$.
\item Whitney sum formula: Let us define the \emph{total Chern class} of $V$ as $c(V) \coloneqq \sum_{i=0}^{\rnk(V)}c_i(V) \in \sE^{2*,*}(X)$. Then for every short exact sequence
$$0 \to V' \to V \to V'' \to 0$$
of vector bundles over $X$, we have
$$c(V)=c(V')c(V'').$$
\item Naturality: Given vector bundles $V\to X$, $W\to Y$, $X,Y\in \Sm/k$, a morphism $f:Y\to X$ and an isomorphism $W\xrightarrow{\sim} f^*V$ of vector bundles on $Y$, then we have
$$c(W)=f^*c(V).$$ 
\end{enumerate}
\end{defn}

In \cite[Theorem 3.27]{Pan:oriented}, Panin shows that the Chern classes $c_i(V)$ for $i>0$ are  nilpotent elements in the graded ring $\oplus_n\sE^{2n,n}(X)$. For $\sV$ the sheaf of sections of a vector bundle $V$, we define $c_i(\sV)$ to be $c_i(V)$.

A classical complete treatment of the theory of Chern classes is given in \cite{Gro:Chernclasses}.

Panin has shown (see\cite[Lemma 3.33, Theorems 3.5,  3.27,  3.36]{Pan:oriented}) that, for a commutative ring spectrum $\sE$, giving a theory of Chern classes is equivalent to giving a theory of Thom classes. Under this correspondence, given a vector bundle $V\to X$, the top Chern class $c_{\rnk(V)}(V)\in\sE^{2{\rnk(V)}, {\rnk(V)}}(X)$ is the pullback $z_0^*\th^\sE(V)$ of $\th^\sE(V)$ along the zero section $z_0:X_+ \to \Th(V)$ in $\sH_\bullet(k)$. This yields in particular an expression for the unique non-trivial Chern class $c_1(L)\in \sE^{2, 1}(X)$ for a line bundle $L\to X$.

 One can also describe $c_{\rnk(V)}(V)$ using the proper pushforward along the zero-section $s_0:X\to V$.
\begin{lemma}\label{lem:FirstChernClassFacts} Let $V\to X$ be a rank $r$ vector bundle on $X\in \Sm/k$, and let $\sE\in \SH(k)$ be an oriented ring spectrum. \\[5pt]
1. Let $s_0:X\to V$ be the zero section and let $s:X\to V$ be an arbitrary section. Then  
\begin{equation}\label{eqn:ChernEuler}
c_r(V)=s^*(s_{0*}(1^\sE_X)),
\end{equation}
where $1^\sE_X\in \sE^{0,0}(X)$ is the unit.\\[2pt]
2. In the same setting as in (1), let us suppose that the cartesian square
\[
\xymatrix{
Y:=s_0^*V\ar[r]^i\ar[d]^i&X\ar[d]^s\\
X\ar[r]^{s_0}&V
}
\]
is transverse, that is, $i:Y\to X$ is a codimension $r$ closed subscheme of $X$, smooth over $k$. Then 
\begin{equation}\label{eqn:ChernEuler2}
c_r(V)=i_*(1^\sE_Y)), 
\end{equation}
where $1^\sE_Y\in \sE^{0,0}(Y)$ is the unit.
\end{lemma}

\begin{proof} For (1), we first note that $s^*:\sE^{*,*}(V)\to \sE^{*,*}(X)$ is an isomorphism, and is equal to the isomorphism $s_0^*:\sE^{*,*}(V)\to \sE^{*,*}(X)$. In fact, $s_0$ and $s$ are both inverse of the structure map $V \to X$ in $\sH(k)$, which is a weak equivalence by homotopy invariance. Thus, they induce the same isomorphism in $\sE$ cohomology by pullback. Now, by construction of the proper pushforward for $s_0$, we have that $s_{0*}(1^\sE_X)\in \sE^{2r,r}(V)$ is the image of $\th^\sE(V)\in \sE^{2r,r}(\Th(V))$ under the map 
$\pi^*:\sE^{2r,r}(\Th(V))\to \sE^{2r,r}(V)$ induced by the quotient map $\pi:V_+\to \Th(V)$ in $\sH_\bullet(k)$. Finally, by considering the commutative diagram
\[
\xymatrix{
\sE^{*,*}(\Th(V))\ar[r]^{\pi^*}\ar[dr]_{z_0^*}&\sE^{*,*}(V)\ar[d]^{s_0^*}\\
&\sE^{*,*}(X),
}
\]
we get

$$c_r(V) =z_0^*(\th(V))
=s_0^*(\pi^*\th(V))
=s_0^*(s_{0*}(1^\sE_X))
=s^*(s_{0*}(1^\sE_X)).$$
This proves (1).

We prove (2). Using (1) and applying Proposition \ref{prop:PushPull}(2) to the transverse cartesian diagram
\[
\xymatrix{
Y\ar[r]^i\ar[d]^i&X\ar[d]^s\\
X\ar[r]^{s_0}&V,
}
\]
gives
$$c_r(V)=s^*(s_{0*}(1^\sE_X))=i_*(i^*(1^\sE_X))=i_*(1^\sE_Y).$$
\end{proof}

An important property of oriented cohomology theories consists in satisfying a \emph{Projective bundle formula}. In order to state the projective bundle formula for a vector bundle $V\to X$, let $\sV$ be the sheaf of sections of $V$, and let $\P(V)$ denote the projective vector bundle $\operatorname{Proj}(\Sym^*\sV)\xrightarrow{q}X$, with tautological quotient invertible sheaf $q^*\sV\twoheadrightarrow\sO_V(1)$.

\begin{thm}[Projective Bundle Formula]\label{thm:PBF} Let $\sE\in \SH(k)$ be an oriented commutative ring spectrum. Let $X$ be in $\Sm/k$, and let $V\to X$ be a rank $n+1$ vector bundle on $X$. Let $\xi:=c_1(\sO_V(1))\in \sE^{2,1}(X)$. Then $\sE^{*,*}(\P(V))$ is a free $\sE^{*,*}(X)$-module, via the pullback $q^*$, with basis $1, \xi,\ldots, \xi^n$.
\end{thm}
This is \cite[Theorem 3.9]{Pan:oriented}.

The lower Chern classes $c_i(V)\in \sE^{2i,i}(X)$, $0<i<{\rnk(V)}$, for a vector bundle $V\to X$ are determined using Grothendieck's formula
\begin{equation}\label{eqn:GrothFormula}
\xi^{\rnk(V)}+\sum_{i=1}^{\rnk(V)}(-1)^ic_i(V)\cdot \xi^{\rnk(V)-i}=0;\quad \xi:=c_1(\sO_V(1))\in \sE^{2,1}(\P(V)).
\end{equation}

\begin{rmk}\label{rem:PBFAlgebra} Combining the projective bundle formula (Theorem \ref{thm:PBF}) with the Grothendieck formula \eqref{eqn:GrothFormula} gives a description of $\sE^{*,*}(\P(V))$ as a $\sE^{*,*}(X)$-algebra, as follows. Let $\sE^{*,*}(X)[x]$ be the polynomial algebra over $\sE^{*,*}(X)$ with generator $x$ in bidegree $(2,1)$, let $q:\P(V)\to X$ be the structure morphism and let $r=\rnk(V)$. Then sending $\alpha\cdot x^i$ to  $q^*\alpha\cdot \xi^i$, with $\xi:=c_1(\sO_V(1))\in \sE^{2,1}(\P(V))$, gives an isomorphism of bigraded rings
\[
\psi_V:\sE^{*,*}(X)[x]/(x^r+\sum_{i=1}^r(-1)^ic_i(V)\cdot x^{r-i})\xrightarrow{\sim} \sE^{*,*}(\P(V)).
\]
Indeed, the Grothendieck formula shows that sending $\alpha\cdot x^i$ to $q^*\alpha\cdot \xi^i$ descends to a well-defined bigraded ring homomorphism $\psi_V$, and, since $\sE^{*,*}(X)[x]/(x^{r}+\sum_{i=1}^{r}(-1)^ic_i(V)\cdot x^{r-i})$ is a free $\sE^{*,*}(X)$-module with basis $1, x,\ldots, x^{r-1}$, the  projective bundle formula implies that $\psi_V$ is an isomorphism.
\end{rmk}

From now on, for $V=O_X^{n+1}$, considered as rank $n+1$ vector bundle over $X\in \Sm/k$, we write $\P^n_X$ for $\P(V)$ and $\sO_{\P^n_X}(1)$ for $\sO_{O_X^{n+1}}(1)$.

In order to complete the picture about Chern classes, we recall that to any Chern class theory on a commutative ring spectrum $\sE$, one can associate a formal group law $F \in \sE^{2*,*}(\Spec k)[[x,y]]$ such that, for every pair of line bundles $L_1,L_2$ over $X \in \Sm/k$, one has $c_1(L_1 \otimes L_2)=F(c_1(L_1),c_1(L_2))$. Let us recall the definition of a formal group law.

\begin{defn}
    A \emph{formal group law} over a ring $R$ is an element $F=F(x,y)$ in the ring $R[[x,y]]$ of formal power series on two variables over $R$, satisfying:
    \begin{enumerate}
        \item Commutativity: $F(x,y)=F(y,x)$.
        \item Identity element: $F(x,0)=x$ and $F(0,y)=y$.
        \item Associativity: $F(F(x,y),z)=F(x,F(y,z))$.
    \end{enumerate}
\end{defn}

Following \cite[Section 3.9]{Pan:oriented}, using the projective bundle formula gives an isomorphism of $\Z$-graded rings $\sE^{2*,*}(\P^\infty)\cong \sE^{2*,*}(k)[[x]]$, with $x$ mapping to $c_1(\sO(1))$ and similarly $\sE^{2*,*}(\P^\infty\times\P^\infty)\cong \sE^{2*,*}(k)[[x,y]]$, with $x$ mapping to $p_1^*(c_1(\sO(1)))$, and $y$ mapping to $p_2^*(c_1(\sO(1)))$. This then gives the uniquely defined $F(x, y)\in  \sE^{2*,*}(k)[[x, y]]$, with 
\[
F(p_1^*(c_1(\sO(1))), p_2^*(c_1(\sO(1))))=c_1(p_1^*(c_1(\sO(1)))\otimes p_2^*(c_1(\sO(1)))).
\]

Next, Panin showed (see \cite[Proposition 3.37]{Pan:oriented}) that, given two line bundles $L_1, L_2$ on $X\in \Sm/k$, one has
\[
c_1(L_1\otimes L_2)=F(c_1(L_1), c_1(L_2)),
\]
where the right-hand side makes sense because $c_1(L_1), c_1(L_2)$ are nilpotent by \cite[Lemma 3.29]{Pan:oriented}. The proof follows by replacing $X$ with a Jouanolou cover $p:\tilde{X}\to X$, that is, an affine space bundle over $X$ such that $\tilde{X}$ is affine, inducing an isomorphism $p^*:\sE^{*,*}(X)\to \sE^{*,*}(\tilde{X})$ by the homotopy invariance of $\sE$. Then, since $\tilde{X}$ is affine, we may assume from the beginning that $L_1, L_2$ are globally generated, giving morphisms $f_i:X\to \P^\infty$, $i=1, 2$, with the invertible sheaf $f_i^*(\sO(1))$ being isomorphic to the invertible sheaf of sections of $L_i$. Using the relation defining $F$ and the naturality of $c_1$ concludes the proof.

Finally, by \cite[Proposition 3.38]{Pan:oriented}, $F(x, y)\in  \sE^{2*,*}(k)[[x, y]]$ is a rank one formal group law over $\sE^{2*,*}(k)$.

\begin{defn}
    The formal group law $F$ obtained via the oriented commutative ring spectrum $\sE$ as in the previous paragraph, is called the \emph{formal group law associated to the oriented spectrum $\sE$}. 
\end{defn}

\begin{exmp}
\label{exmp:orientedHZ}
    Motivic cohomology $H\Z$ has a canonical orientation with additive formal group law $F(x,y)=x + y$. See for example \cite[Section 11.3]{deglise:mixmot}.  
\end{exmp}

In this text we will always consider $H\Z$ as an oriented commutative ring spectrum with the canonical orientation of Example \ref{exmp:orientedHZ}.

\subsection{Cohomology classes valued in motivic cohomology}
\label{sec:motiviccohomology}

\begin{defn}
\label{defn:HZclasses}
For $\sE\in \SH(k)$ an oriented ring spectrum, $c\in H\Z^{a,b}(\sE)=[\sE, \Sigma^{a,b}H\Z]_{\SH(k)}$, and $v\in K_0(X)$ for $X\in \Sm/k$, we have $c^\sE(v)\in H\Z^{a,b}(X)$ defined as follows. 

Let $p:X \to \Spec k$ be the structure map. Applying $p^*:\SH(k)\to \SH(X)$ to the class $c$ gives the map $p^*(c):p^*\sE\to \Sigma^{a,b}p^*H\Z$. 
The Thom classes for $\sE$ give us the map $\th_\sE(v):\Sigma^{v-\rnk(v)[\sO_X]}1_X\to p^*\sE$.  
We have the map $\tilde{c}^\sE(v)'$ defined as the composition
\[
\Sigma^{v-\rnk(v)[\sO_X]}1_X\xrightarrow{\th_\sE(v)} p^*\sE\xrightarrow{p^*(c)}\Sigma^{a,b}p^*H\Z
\]
in $\SH(X)$, and by the adjunction $p_\# \dashv p^*$, the map
\[
\tilde{c}^\sE(v):p_\#\Sigma^{v-\rnk(v)[\sO_X]}1_X\to p_{\#}p^*\Sigma^{a,b}H\Z \to \Sigma^{a,b}H\Z.
\]
In other words, we have constructed a class
\[
[\tilde{c}^\sE(v)]\in H\Z^{a,b}(p_\#\Sigma^{v-\rnk(v)[\sO_X]}1_X).
\]
 Now, using the fact that $H\Z$ is also oriented, we have the isomorphism
\[
\Sigma^{a,b}\th_{H\Z}^f(-v):\Sigma^{\rnk(v)[\sO_X]-v}p^*\Sigma^{a,b}H\Z\xrightarrow{\sim} \Sigma^{a,b}p^*H\Z,
\]
inducing the isomorphism $H\Z^{a,b}(p_\#\Sigma^{v-\rnk(v)[\sO_X]}1_X)\simeq H\Z^{a,b}(X)$, so
$[\tilde{c}^\sE(v)]$ gives us the class
\[
c^\sE(v)\in H\Z^{a,b}(X).
\]
\end{defn}

The construction \ref{constr:properpushforward} of proper pushforward yields the following degree map for proper varieties.

\begin{defn}[Degree map]\label{defn:Degree}  Let $p_X:X\to \Spec k$ be the structure map of $X\in \Sm/k$, with $X$ proper of dimension $d_X$ over $k$. Define the  {\em degree map}
\[
\deg_k:H\Z^{2d_X,d_X}(X)\to H\Z^{0,0}(\Spec k)=\Z,
\]
by $\deg_k:=p_{X*}$.  
\end{defn}

In particular, for $X\in \Sm/k$, proper over $k$ of dimension $d_X$, for $\sE\in \SH(k)$,   $c\in H\Z^{2d_X, d_X}(\sE)$ and  $v\in K_0(X)$ for $X\in \Sm/k$, we may apply $\deg_k$ to  $c^\sE(v)\in H\Z^{2d_X,d_X}(X)$
to yield an integer $\deg_k(c^\sE(v))\in \Z$.

\subsubsection{A twisted degree map}
\label{subsec:TwistedDegree}

For any $E$ vector bundle on $X \in \Sm/S$ and any $L$ line bundle over $X$, Ananyevskiy in \cite[Lemma 4.1]{Ana:Slor} gives an isomorphism of unstable Thom spaces
$$\rho: \Th(E\oplus L)\xrightarrow{\sim} \Th(E \oplus L^{\vee})$$
in $\sH_{\bullet}(S)$. Through $\Sigma^{\infty}_{\P^1}$, the isomorphism passes to the stable setting. Then, for line bundles $L_1, \ldots, L_n$ over $X$, we denote by $\Anan_{L_1, \ldots, L_n}$ the isomorphism
\begin{equation}
    \label{eqn:AnanIso}
    \Anan_{L_1,\ldots, L_n}: \Sigma^{\oplus_{i=1}^n L_i}1_X \xrightarrow{\sim} \Sigma^{\oplus_{i=1}^n L_i^\vee}1_X
\end{equation}
in $\SH(X)$ obtained through Ananyevskiy's construction, with $S=X$. In the same way, for all $e \in K_0(X)$ and line bundles $L_i$ as before, we obtain an isomorphism
\begin{multline}
\label{eq:Anaconstruction}
    \Anan_{e;-L_1, \ldots, -L_n} \coloneqq \Sigma^{e-\oplus_{i=1}^nL_i^\vee-
    \oplus_{i=1}^nL_i} \Anan_{L_1^\vee,\ldots, L_n^\vee}: \Sigma^{e-\sum_{i=1}^n[L_i]}1_X\xrightarrow{\sim} \\ \Sigma^{e-\sum_{i=1}^n[L_i^\vee]}1_X.
\end{multline}

\begin{lemma}\label{lem:Anan} Let us suppose $X \in \Sm/k$ irreducible, and use the shorthand $\Anan \coloneqq \Anan_{e;-L_1,\ldots,-L_n}$ for the isomorphism \eqref{eq:Anaconstruction}. Let also $d\coloneqq \rnk(e)-n$. For $p:X\to \Spec k$ in $\Sm/k$, the diagram in $\Mod_{p^*H\Z}$
$$
\begin{tikzcd}
\Sigma^{e- \sum_{i=1}^n[L_i] - r[\mathcal{O}_X]}1_X\wedge p^*H\Z \arrow[rr, "\Sigma^{-2d,-d} \Anan\wedge\id_{p^*H\Z}", "\sim"'] \arrow[dr, swap, "\th_{H\Z\Mod}^f (e- \sum_{i=1}^nL_i)", "\sim"'] && \Sigma^{e-\sum_{i=1}^n[L_i^\vee]-r[\mathcal{O}_X]}1_X\wedge p^*H\Z \arrow[dl, "(-1)^n\th^f_{H\Z\Mod} (e-\sum_{i=1}^nL_i^\vee)", "\sim"'] \\
& p^*H\Z &
\end{tikzcd}
$$
commutes.
\end{lemma}

\begin{proof}
    By using a Jouanolou cover of $X$, we may assume that $X$ is affine. In that case, there exists a vector bundle $E$ over $X$ such that $-e=[E]-s[\sO_X]$, and let $r \coloneqq s-d$. In this way, we are reduced to prove the commutativity of the diagram
    \begin{equation}
    \label{eq:anatriangle}
         \begin{tikzcd}
        \Sigma^{2r,r}\Sigma^{-[E +\oplus_i^nL_i]}1_X \wedge p^*H\Z \arrow[rrrr, "\Sigma^{2r,r}\Anan_{E;L_1,\ldots,L_n} \wedge \id_{p^*H\Z}", "\sim"'] \arrow[drr, swap, "\th^f_{H\Z \Mod} (-E-\oplus_i^nL_i)", "\sim"'] &&&&  \Sigma^{2r,r}\Sigma^{-[E +\oplus_i^nL_i^{\vee}]}1_X \wedge p^*H\Z \arrow[dll, "(-1)^n\th^f_{H\Z \Mod} (-E-\oplus_i^nL_i^\vee)", "\sim"'] \\
        &&p^*H\Z&&
    \end{tikzcd}
    \end{equation}
    in $\Mod_{p^*H\Z}$. Let $\rho:\Th(E +\oplus_{i=1}^nL_i)\xrightarrow{\sim}\Th(E + \oplus_{i=1}^nL_i^\vee)$ the isomorphism in $\sH_{\bullet}(k)$ defined by Ananyevskiy.
    
    The commutativity of \eqref{eq:anatriangle} is equivalent to
    $$(\Sigma^{2r,r}\Anan_{E;L_1,\ldots,L_n} \wedge \id_{p^*H\Z})^*\th_{H\Z}^f (E\oplus_i^nL_i^\vee) =(-1)^n \th_{H\Z}^f (E+\oplus_i^nL_i),$$
    and this is in turn equivalent to
    $$\rho^*\th^{H\Z} (E+\oplus_i^nL_i^\vee) = (-1)^n \th^{H\Z} (E+\oplus_i^nL_i).$$
    Lemma \ref{lemma:anatwists} below shows in detail the computation for $n=1$. The result immediately follows by induction on $n$.
\end{proof}

\begin{lemma}
\label{lemma:anatwists}
    Let $X \in \Sm/k$ irreducible. Let $E \to X$ a vector bundle of finite rank over $X$ and $L \to X$ a line bundle over $X$, with dual bundle $L^{\vee}\to X$. Let $\rho: \Th(E\oplus L) \xrightarrow{\sim} \Th(E \oplus L^\vee)$ the isomorphism in $\SH(k)$ defined as in \cite[Lemma 4.1]{Ana:Slor}. Then $$\rho^*\th^{H\mathbb{Z}}(E + L^{\vee})=-\th^{H\mathbb{Z}}(E+L).$$
\end{lemma}

\begin{proof}
    Since we are working in the stable setting, we simplify the notation by writing $\rho$ for $\Sigma^{\infty}_{\P^1}\rho:p_\#\Sigma^{E +L}1_X \xrightarrow{\sim} p_\#\Sigma^{E+L^\vee}1_X$, with $p\coloneqq p_X:X \to \Spec k$ the structure map. Let us briefly review the construction of $\rho$ from \cite{Ana:Slor}.

    Let $\pi_L:L\to X$ and $\pi_{L^\vee}:L^\vee \to X$ the structure morphisms of the two vector bundles. Let us denote by $L^0$ and $(L^\vee)^0$ the complements of the zero sections, and by $\pi_{L^0}$, $\pi_{(L^\vee)^0}$ the restrictions of the relative morphisms. Let us note that $\A^1$-invariance (Proposition \ref{prop:hom-invariance}) implies $\pi_{L\#}1_L \simeq 1_X \simeq \pi_{L^\vee \#}1_{L^\vee}$. The localization sequence associated to the inclusion $L^0 \hookrightarrow L$ as in \eqref{eq:thom-loc2} gives a distinguished triangle 
    \begin{equation}
    \label{eq:locseq1}
        \pi_{L^0\#}1_{L^0} \to 1_X \to \Sigma^L1_X \to \pi_{L^0\#}1_{L^0}[1], 
    \end{equation}
    and analogously, we have the distinguished triangle
    \begin{equation}
    \label{eq:locseq2}
        \pi_{(L^\vee)^0\#}1_{(L^\vee)^0} \to 1_X \to \Sigma^{L^\vee}1_X \to \pi_{(L^\vee)^0\#}1_{(L^\vee)^0}[1].
    \end{equation}
    There is a unique isomorphism of $X$-schemes $f:L^0\xrightarrow{\sim}(L^{\vee})^0$ given by taking, fiberwise, the vector $v$ to the functional $f_v$ such that $f_v(v)=1$. By taking $\Sigma_{\P^1}^\infty f$ over $X$, this gives an isomorphism between the right and terms of \eqref{eq:locseq1} and \eqref{eq:locseq2}, that, together with $\id_X$, induces the isomorphism $\Anan_L:\Sigma^L1_X \xrightarrow{\sim} \Sigma^{L^\vee}1_X$ on the cofibers. Applying $p_\# \circ \Sigma^E$ to $\Anan_L$ gives the isomorphism $\rho: p_\#\Sigma^{E+L}1_X \xrightarrow{\sim} p_\#\Sigma^{E +L^\vee}1_X$.

    From this construction, we see that $\rho=\id_{p_\# \Sigma^E1_X}\wedge \rho_L$, with $\rho_L \coloneqq p_\# \Anan_L$, and since Thom classes are additive, we have $\rho^*\th^{H\mathbb{Z}}(E + L^{\vee})=\th^{H\Z}(E) \cup \rho_L^*(L)$. This reduces to the case $E=0$ and $\rho=\rho_L$.

    Next, we reduce to the case $X=\Spec k(X)$. The Thom isomorphism in $H\mathbb{Z}$-cohomology makes $H\mathbb{Z}^{**}(p_{\#}\Sigma^L1_X)$ and $H\mathbb{Z}^{**}(p_{\#}\Sigma^{L^{\vee}}1_X)$ two free $H\mathbb{Z}^{**}(X)$-modules of rank $1$, generated respectively by $\th^{H\mathbb{Z}}(L)$ and $\th^{H\mathbb{Z}}(L^{\vee})$, both in bidegree $(2,1)$. $\rho_L^*$, as an isomorphism of $H\mathbb{Z}^{**}(X)$-modules, sends generators to generators, then $\rho_L^*\th^{H\mathbb{Z}}(L^{\vee})=\alpha \cdot \th^{H\mathbb{Z}}(L)$, with $\alpha \in H\mathbb{Z}^{0,0}(X)=\mathbb{Z}.$ Since $X$ is irreducible, there is a unique generic point $i:\Spec k(X) \to X$. Let then
    $$\rho_{i^*L}: p_{k(X)\#}\Sigma^{i^*L}1_{k(X)} \xrightarrow{\sim} p_{k(X)\#}\Sigma^{i^*L^{\vee}}1_{k(X)}$$
    be the isomorphism obtained in the same way as $\rho$, with $p_{k(X)}\coloneqq p \circ i =\Spec(k \hookrightarrow k(X))$. We have the commutative diagram
\[
 \xymatrix{
 H\Z^{0,0}(X)\ar@{=}[d]\ar[r]^{i^*}_\sim&H\Z^{0,0}(k(X))\ar@{=}[d]\\
  [1_X, p^*H\Z]_{\SH(X)}\ar[r]^{i^*}\ar[d]_\wr^{\cup \; \th^{H\mathbb{Z}}(L)}&[1_{k(X)}, p_{k(X)}^*H\Z]_{\SH(k(X))}\ar[d]_\wr^{\cup \; \th^{p_{k(X)}^*H\mathbb{Z}}(i^*L)}\\
 [\Sigma^L1_X, \Sigma^{2,1}p^*H\Z]_{\SH(X)}\ar@{=}[d]\ar[r]^{i^*}&[\Sigma^{i^*L}1_{k(X)}, \Sigma^{2,1}p_{k(X)}^*H\Z]_{\SH(k(X))}\ar@{=}[d]\\
 H\Z^{2,1}(\Th(L))\ar[r]^{i^*}&H\Z^{2,1}(\Th(i^*L)) 
 }
 \]
 and a similar diagram for $L^\vee$. Let $\alpha_{k(X)}\in \Z=H\Z^{0,0}(k(X))$ be the element such that $\rho_{i^*L}(\th^{p_{k(X)}^*H\mathbb{Z}}(i^*L^{\vee}))=\alpha_{k(X)} \cdot \th^{p_{k(X)}^*H\mathbb{Z}}(i^*L)$. The maps $\rho_L$ and $\rho_{i^*L}$ define a map of the above diagram to the corresponding one for $L^\vee$, so we have $i^*\alpha=\alpha_{k(X)}$. But since $i^*:\Z=H\Z^{0,0}(X)\to H\Z^{0,0}(k(X))=\Z$ is the identity map on $\Z$, we have $\alpha_{k(X)}=\alpha$. This reduces to the case $X =\Spec k(X)$. 
 
 Changing notation, we may simply assume that $X=\Spec k$. $L$ is then a one dimensional $k$-vector space with zero section being the origin. The total space of $L$ is (non canonically) isomorphic to $\mathbb{A}^1$. Let us choose any isomorphism $L\xrightarrow{\sim} \mathbb{A}^1$. This also induces an isomorphism $L^{\vee}\xrightarrow{\sim}\mathbb{A}^1$ through $f\mapsto f(1)$. Through these isomorphisms, we identify $L^0$ and $(L^{\vee})^0$ with $\mathbb{G}_m$. The isomorphism of schemes $L^0 \xrightarrow{\sim}(L^{\vee})^0$ given by $v \mapsto f_v$, induces the automorphism $t \mapsto t^{-1}$ of $\mathbb{G}_m$, because $1=f_t(t)=t\cdot f_t(1)$. The isomorphisms $L\xrightarrow{\sim}\mathbb{A}^1\xleftarrow{\sim}L^{\vee}$ induce
    $$\Sigma^L1_X \xrightarrow{\sim}S^1\wedge \mathbb{G}_m \xleftarrow{\sim} \Sigma^{L^{\vee}}1_X,$$
    and trough these, the isomorphism $\rho_L$ can be read as the automorphism $\id_{S^1}\wedge(t\mapsto t^{-1})$ of $S^1\wedge\G_m$, where $t$ is the canonical coordinate on $\G_m$. We have
    $$H\Z^{2,1}(\P^1) \simeq H\mathbb{Z}^{2,1}(S^1\wedge\G_m)\simeq  H\mathbb{Z}^{1,1} (\G_m) \xleftarrow{\sim} \mathbb{Z},$$
    with the last isomorphism sending $n\in\Z$ to $t^n\in H\mathbb{Z}^{1,1}(\G_m))$. Thus, $\rho_L$ induces the isomorphism $n\mapsto -n$ on $\Z\cong H\mathbb{Z}^{2,1}(\P^1)$.

    $\th^{H\mathbb{Z}(L)}$ and $\th^{H\mathbb{Z}}(L^{\vee})$ belong to $H\mathbb{Z}^{2,1}(\mathbb{P}^1)\simeq \mathbb{Z}$, and they are generators. Since $L\simeq L^\vee\simeq \A^1_F$ as line bundles over $\Spec F$, we have $\th^{H\mathbb{Z}}(L)=\th^{H\mathbb{Z}}(L^{\vee})$, and since $\rho_L^*(n)=-n$, we conclude $\rho_L^*(\th^{H\mathbb{Z}}(L^{\vee}))= -\th^{H\mathbb{Z}}(L)$.
\end{proof}

Let us now consider $X \in \Sm/k$ and $e=\sum_{i=1}^n[L_i]-[T_X]+d_X[\sO_X] \in K_0(X)$, with $d_X \coloneqq \dim_k(X)$. Let us suppose $v \in K_0(X)$ being an element satisfying the identity
\begin{equation}
\label{eq:comparisontangentbundle}
    v=e - \sum_{i=1}^n[L_i^\vee] - d_X[\sO_X].
\end{equation}
In particular, $\rnk(v)=-\rnk(T_X)=-d_X$. In this case, the construction \eqref{eq:Anaconstruction} gives the isomorphism
\begin{equation}
\label{eq:anatwist}
    \Anan_{-L_1, \ldots, -L_n}(v): \Sigma^{-T_X+\rnk(T_X)[\sO_X]}1_X \xrightarrow{\sim} \Sigma^{v-\rnk(v)[\sO_X]}1_X,
\end{equation}
where $\Anan_{-L_1, \ldots, -L_n}(v):=\Anan_{e;-L_1, \ldots, -L_n}$.

\begin{constr}
\label{constr:AnaDegree}
With $p:X\to \Spec k$ smooth and proper of dimension $d_X$, and $v \in K_0(X)$ as in \eqref{eq:comparisontangentbundle}, we have the \emph{twisted degree}
\[
\deg^\Anan_k:[1_X, \Sigma^{2d_X, d_X}p^*H\Z]_{\SH(X)}\to 
\Z
\]
defined as the composition \eqref{eq:anandeg} below, where we write $\Anan$ for the isomorphism $\Anan_{-L_1, \ldots, -L_n}(v)$ in \eqref{eq:anatwist}:
\begin{align*}
[1_X, \Sigma^{2d_X, d_X}p^*H\Z]_{\SH(X)}&\xrightarrow{\sim}
[p^*H\Z, \Sigma^{2d_X, d_X}p^*H\Z]_{\Mod_{p^*H\Z}}\\
&\xrightarrow[\sim]{\th_{H\Z\Mod}^f(v)^*}[\Sigma^{v-\rnk(v)[\sO_X]}1_X\wedge p^*H\Z, \Sigma^{2d_X, d_X}p^*H\Z]_{\Mod_{p^*H\Z}}\\
&\xrightarrow[\sim]{(\Anan \wedge\id)^*}[\Sigma^{-T_X+\rnk(T_X)[\sO_X]}1_X\wedge  p^*H\Z, \Sigma^{2d_X, d_X}p^*H\Z]_{\Mod_{p^*H\Z}}\\
&=[\Sigma^{-T_X}1_X\wedge  p^*H\Z, p^*H\Z]_{\Mod_{p^*H\Z}}\\
&=[\Sigma^{-T_X}1_X, p^*H\Z]_{\SH(X)}\\
&=[p_\#(\Sigma^{-T_X}1_X), H\Z]_{\SH(k)}\\
&\rightarrow{(p^\vee)^*}[1_k,H\Z]_{\SH(k)}\\
&=H\Z^{0,0}(\Spec k)=\Z.
\end{align*}
\begin{equation}
    \label{eq:anandeg}
\end{equation}
\end{constr}

\begin{prop}
\label{prop:comparisondegrees}
Suppose we have $X\in \Sm/k$ irreducible, $v\in K_0(X)$ and line bundles $L_1,\ldots, L_r$ such that Ananyevskiy's isomorphism $\Anan_{L_1,\ldots, L_r}$ is defined. Suppose that $X$ is smooth and proper of dimension $d_X$ over $k$. Then, for $\alpha\in H\Z^{2d_X, d_X}(X)$ we have
\[
\deg_k(\alpha)=(-1)^r\deg^\Anan_k(\alpha).
\]
\end{prop}

\begin{proof} By rewriting the construction of the proper pushforward map (Construction~\ref{constr:properpushforward}), and recalling the definition of the degree map (Definition~\ref{defn:Degree}), we have that the degree map $\deg_k:H\Z^{2d_X, d_X}(X)\to \Z$ is given by the composition
\begin{align*}
H\Z^{2d_X, d_X}(X)&=[1_X, \Sigma^{2d_X, d_X}p^*H\Z]_{\SH(X)}\\
&\xrightarrow{\sim}
[p^*H\Z, \Sigma^{2d_X, d_X}p^*H\Z]_{\Mod_{p^*H\Z}}\\
&\xrightarrow{\th_{H\Z\Mod}^f(-T_X)^*}[\Sigma^{-T_X+d_X[\sO_X]}1_X\wedge p^*H\Z, \Sigma^{2d_X, d_X}p^*H\Z]_{\Mod_{p^*H\Z}}\\
&=[\Sigma^{-T_X}1_X\wedge  p^*H\Z, p^*H\Z]_{\Mod_{p^*H\Z}}\\
&=[\Sigma^{-T_X}1_X, p^*H\Z]_{\SH(X)}\\
&=[p_\#(\Sigma^{-T_X}1_X), H\Z]_{\SH(k)}\\
&\xrightarrow{(p^\vee)^*}[1_k,H\Z]_{\SH(k)}=H\Z^{0,0}(\Spec k)=\Z.
\end{align*}
By comparing this composition with \eqref{eq:anandeg} and applying Lemma~\ref{lem:Anan}, we complete the proof.
\end{proof}

In particular, we have the following result.

\begin{corollary}
\label{cor:TwistClassComp}
Let $Y\in \Sm/k$ be proper of dimension $d$ over $k$ and irreducible, and let $(v,\vartheta)$ be a stable twist of $-T_Y$. Let us suppose that, for a certain $e \in K_0(Y)$ and $L_1, \ldots, L_n$ line bundles on $Y$, we have $v=e-\sum_{i=1}^s[L_i^\vee]$, $-[T_Y]=e-\sum_{i=1}^n[L_i]$ in $K_0(Y)$, and $\vartheta$ is the isomorphism
\[
\Sigma^{-2d,-d}\Anan_{e, -L_1,\ldots, -L_n}(v):\Sigma^{-T_Y}1_Y\xrightarrow{\sim} \Sigma^v1_Y.
\]
Let us now consider $\sE \in \SH(k)$ an oriented commutative ring spectrum, and $c$ a class in $H\Z^{2d,d}(\sE)$, so that we have $c^\sE(v)\in H\Z^{2d,d}(Y)$ and $[Y,v,\vartheta]_\sE\in \sE_{2d,d}(\Spec k)$.
Then 
\[
\deg_k c^\sE(v)=(-1)^n c\circ [Y,v,\vartheta]_\sE\in H\Z^{0,0}(\Spec k)=\Z.
\]
\end{corollary}

\begin{proof}
    Let $p:=p_Y:Y \to \Spec k$ be the structure map. We write $\Anan$ for $\Anan_{e, -L_1,\ldots, -L_n}(v)$.
    
    By looking at the definition of $[Y,v,\vartheta]_\sE$ (Definition \ref{defn:twistclass}), and recalling that $\vartheta$ is the isomorphism $\Sigma^{-2d,-d}\Anan$, we see that $c\circ [Y,v,\vartheta]$ is the composition
    $$1_k\xrightarrow{p^\vee}p_{\#}\Sigma^{-T_Y}1_Y\xrightarrow[\sim]{p_\#\Sigma^{-2d,-d}\Anan} p_{\#}\Sigma^{v}1_Y\xrightarrow{\Sigma^{-2d,-d}\th_\sE(v)'} \Sigma^{-2d,-d}\sE\xrightarrow{\Sigma^{-2d-d}c}H\Z,$$
where $\th_\sE(v)'$ is the map corresponding to $\th_\sE(v)$ by adjunction $p_\# \dashv p^*$.

On the other hand, by looking at the definition of $c^\sE(v)$ (Definition \ref{defn:HZclasses}), we note that the composition $$p_{\#}\Sigma^{v}1_Y\xrightarrow{\Sigma^{-2d,-d}\th_\sE(v)'} \Sigma^{-2d,-d}\sE\xrightarrow{\Sigma^{-2d-d}c}H\Z$$
is the class $\Sigma^{-2d,-d}\tilde{c}^\sE(v)$, where $\tilde{c}^\sE(v)$ is the class that induces $c^\sE(v)$ through the isomorphism $H\Z^{2d,2}(p_\#\Sigma^{v+d[\sO_Y]}1_Y)\simeq H\Z^{2d,d}(Y)$. Then
$$c\circ [Y,v,\vartheta]=(p^\vee)^*(\Sigma^{-2d,-d}(p_\#\Anan)^*(\Sigma^{-2d,-d}\tilde{c}^\sE(v))).$$

We also see from Definition \ref{defn:HZclasses} that the isomorphism $H\Z^{2d,2}(p_\#\Sigma^{v+d[\sO_Y]}1_Y)\simeq H\Z^{2d,d}(Y)$ giving $c^\sE(v)$ is induced by the isomorphism $\th_{H\Z}^f(v)$, or equivalently, by $\th_{H\Z}^f(v)^*$.

Lastly, the isomorphism $p_\#\Anan$ in $\SH(k)$ corresponds to the isomorphism 
$$\Anan \wedge \id_{p^*H\Z}:\Sigma^{-T_Y+d[\sO_Y]}1_Y \wedge p_*H\Z \xrightarrow{\sim} \Sigma^{v +d[\sO_Y]}1_Y \wedge p^*H\Z$$
in $\Mod_{p^*H\Z}$ through the free-forgetful adjunction. 

Thus, by looking at the composition \eqref{eq:anandeg} defining the twisted degree $\deg_k^\Anan$, we see that $c\circ [Y,v,\vartheta]$ is the class $\deg_k^\Anan(c^\sE(v))$ in $H\Z^{0,0}(\Spec k)$. Applying Proposition \ref{prop:comparisondegrees} concludes the proof. 
\end{proof}

\subsection{Algebraic Cobordism}
\label{section:MGL}

The universal oriented cohomology theory is Voevodsky's algebraic cobordism $\MGL\in \SH(k)$. Here, we briefly recall a definition of $\MGL$ and a sketch of the construction of the Thom classes.

\begin{constr}
    Let $\Gr(r,n)$ denote the classical Grassmannian variety of $r$-planes in $\A_k^n$, and let $E_{r,n} \to \Gr(r,n)$ be the tautological rank $r$ vector subbundle of the trivial bundle $\mathcal{O}^n_{\Gr(r,n)}$. We have canonical inclusions $\Gr(r,n) \to \Gr(r, n+1)$, and we consider the space $\BGL_r:=\colim_n\Gr(r, n)$ in $\sH(k)$. Now, let $E_r \to \BGL_r$ be the vector bundle defined by $E_r \coloneqq \colim_nE_{r,n}$, and
    $$\MGL_r \coloneqq \Th(E_r) \in \sH_{\bullet}(k).$$
    Let $i_r:\BGL_r\to \BGL_{r+1}$ be the canonical inclusion. Then we have a canonical isomorphism $i_r^*E_{r+1}\simeq E_r\oplus \mathcal{O}_{\BGL_r}$ which induces an isomorphism of Thom spaces $\Th(i_r^*E_{r+1})\simeq \Sigma_{\P^1}\Th(E_r)$. The map $\tilde{i}_r:i_r^*E_{r+1}\to E_{r+1}$ over $i_r$ gives the composition
\[
\epsilon_r : \Sigma_{\P^1}\Th(E_r)\cong \Th(i_r^*E_{r+1}) \xrightarrow{\Th(\tilde{i}_r)} \Th(E_{r+1}).
\]
The \emph{algebraic cobordism spectrum} is then the $\P^1$-spectrum
$$\MGL \coloneqq (\MGL_0, \MGL_1, \ldots, \MGL_r, \ldots),$$
with bonding maps $\epsilon_r:\Sigma_{\P^1}\MGL_r\to\MGL_{r+1}$.
\end{constr}

\begin{rmk}
    It is shown in \cite[Section 2.1]{Panin:algcob} that $\MGL$ has the structure of a highly structured motivic commutative ring spectrum.
\end{rmk}

$\MGL$ has a theory of Thom classes constructed as follows.

\begin{constr}
\label{constr:MGLthomclasses}
The identification $\Th(E_r) = \MGL_r$ gives the map $\P^1 \wedge \Th(E_r) \simeq \P^1 \wedge \MGL_r \xrightarrow{\epsilon_r} \MGL_{r+1}$, and inductively, for all $n \ge r$, a map
\begin{multline*}
    (\Sigma_{\P^1}^\infty \Th(E_r))_n= (\P^1)^{\wedge n} \wedge \MGL_r \to (\P^1)^{\wedge (n-1)}\wedge \MGL_{r+1} \to \ldots \\ \to (\P^1)^{\wedge r} \wedge \MGL_n= (\Sigma^{2r,r}\MGL)_n.
\end{multline*}
These maps naturally define a map in $\SH(k)$
\begin{equation}
\label{eq:universalmapsMGL}
    \alpha_r: \Sigma_{\P^1}^\infty\Th(E_r)\to \Sigma^{2r,r}\MGL.
\end{equation}

Also, a rank $r$ vector bundle $V\to X$ on  some affine $X\in \Sm/k$ gives a classifying map $\eta_V:X\to \BGL_r$ that presents $V$ as the pullback $\eta_V^* E_r$. Indeed, one can always pick $n$ generating sections of the dual bundle $V^\vee$ and obtain in this way the epimorphism $\mathcal{O}_X^n \twoheadrightarrow V^\vee$ of vector bundles over $X$. Thus, dualizing gives $V$ as a subbundle $V \hookrightarrow \mathcal{O}_X^n$, which induces, by classifying the fibers of $X$, a map $X \to \Gr(r,n)$, and by composing with the natural map $\Gr(r,n) \to \BGL_r$, we get the map $\eta_V:X \to \BGL_r$. Although this construction depends on the choice of generators for $V^\vee$, two different choices yield $\A^1$-homotopic maps.

We naturally get a pullback diagram
\[
\xymatrix{
V\ar[r]^{\tilde{\eta}_V}\ar[d]&E_r\ar[d]\\
X\ar[r]^{\eta_V}&\BGL_r
}
\]
classifying $V$. This in turn gives the map of Thom spaces $\Th(\tilde{\eta}_V):\Th(V)\to \Th(E_r)$, and the Thom class $\th^\MGL(V) \in \MGL^{2r,r}(\Th(V))$ defined by
\begin{equation}
    \label{eq:MGLThomclasses}
    \th^\MGL(V) \coloneqq \alpha_r \circ \Sigma_{\P^1}^\infty \Th(\tilde{\eta}_V):\Sigma^\infty_{\P^1}\Th(V)\to  \Sigma^{2r,r}\MGL.
\end{equation}

For $V\to X$ vector bundle on a general $X\in \Sm/k$, one still gets maps $\eta_V$ and $\tilde{\eta}_V$ by using a Jouanolou cover and the construction in the affine case, so the class $\th^{\MGL}(V)$ is always defined.

All the axioms for a theory of Thom classes are straightforward to check.
\end{constr}

\begin{rmk}
\label{rmk:MGLcolim}
    We can note that the maps $\Sigma^{-2r,-r}\alpha_r$, with $\alpha_r$ as in \eqref{eq:universalmapsMGL}, exhibit $\MGL$ as the colimit
    $$\colim(\Sigma_{\P^1}^\infty \MGL_0 \to \Sigma^{-2,-1}\Sigma_{\P^1}^\infty \MGL_1 \to \ldots \to \Sigma^{-2r,-r}\Sigma_{\P^1}^\infty \MGL_r \to \ldots) \in \SH(k),$$
    where the maps $\Sigma^{-2r,-r}\Sigma_{\P^1}^\infty \MGL_r \to \Sigma^{-2r-2,-r-1}\Sigma_{\P^1}^\infty \MGL_{r+1}$ are given by $\Sigma^{-2r-2,-r-1}\Sigma^\infty_{\P^1}\epsilon_r$.
\end{rmk}

Considering the tautological bundle $E_r \to \BGL_r$, its Thom class $\th^\MGL(E_r)$ is a map $\Sigma_{\P^1}^\infty \MGL_r \to \Sigma^{2r,r}\MGL$ that, by construction, coincides with the map $\alpha_r$ in \eqref{eq:universalmapsMGL}. This Thom class is usually called \emph{$r$-th universal Thom class}. Algebraic cobordism $\MGL$, with its universal Thom classes, plays a central role among oriented cohomology theories, because of the following theorem.

\begin{thm}[Universality Theorem]
    For any oriented commutative ring spectrum $\sE$, there exists a morphism of commutative ring spectra in $\SH(k)$, $\varphi:\sE \to \MGL$, such that, for any vector bundle $V\to X$ of rank $r$ over $X \in \Sm/k$, the Thom class $\th^\sE(V):p_{X\#}\Sigma^V1_X \to \Sigma^{2r,r}\sE$ factors through $\varphi$, i.e. 
    \begin{equation}
    \label{eq:defthomclass}
        \th^\sE(V)= \Sigma^{2r,r}\varphi \circ \th^\MGL(V),
    \end{equation}
    and the assignment
    \begin{multline*}
        \{\text{morphisms of commutative ring spectra} \;  \MGL \to \sE \; \text{in} \; \; \SH(k)\} \\ \to \{\text{orientations on the ring spectrum} \; \sE\}
    \end{multline*}
    that takes a map $\varphi: \MGL \to \sE$ to the orientation of $\sE$ defined by \eqref{eq:defthomclass}, gives a bijection between morphisms $\MGL \to \sE$ and $GL$-orientations on the commutative ring spectrum $\sE$. 
\end{thm}

The universality Theorem is \cite[Theorem 2.7]{Panin:algcob}.

We mentioned in the previous subsection that one can associate any oriented commutative ring spectrum $\sE \in \SH(k)$ with a formal group law on the ring $\sE^{2*,*}(\Spec k)$. Let us recall the construction of the universal formal group law. 
\begin{constr}
\label{constr:lazardring}
    Let $\Laz \coloneqq \Z[a_{ij}]/R$ be the ring of integer polynomials on variables $\{a_{ij}\}_{i,j \in \N}$ subject to the following set $R$ of relations:
    \begin{enumerate}
        \item $a_{ij}=a_{ji}$,
        \item $a_{10}=1=a_{01}$,
        \item $a_{i0}=0=a_{0i} \; \forall \; i \neq 1$,
        \item the relations imposed by $F(F(x,y),z)=F(x,F(y,z))$ with 
        $$F(x,y)=\Sigma_{i,j \in \N}a_{ij}x^iy^j.$$
    \end{enumerate}
We define $F^{\text{univ}}$ as the image of the formal group law $F\in \Z[a_{ij}][[x,y]]$ of relation $4$, under the surjection $\Z[a_{ij}][[x,y]]\to \Laz[[x,y]]$. In this way, we see that $F^{\text{univ}}\in \Laz[[x,y]]$ is the universal commutative, rank one formal group law, which means, for every commutative ring $R$ with a one dimensional commutative formal group law $F \in R[[x,y]]$, there exists a unique ring homomorphism $\phi_F:\Laz \to R$ such that $F=\phi_F(F^{\text{univ}})$. 
\end{constr}

The ring $\Laz$ of Construction \ref{constr:lazardring} is called the \emph{Lazard ring}. It has a natural grading defined by $\deg(a_{i,j})=1-i-j$, which makes it a graded ring. Quillen proved that the Lazard ring is isomorphic, as a graded ring, to the polynomial ring $\Z[x_1, x_2, \ldots]$ in variables $x_i$, with $x_i$ in degree $-i$. A proof is spelled out, for example, in \cite[Part II, Theorem 7.1]{Adams:homotopy} or \cite[Theorem 4.4.9]{koch:bordism}

The universal orientation of $\MGL$ has an associated formal group law in $\MGL^{2*,*}(\Spec k)$, thus, there is a graded ring homomorphism $\phi_{\MGL}:\Laz \to \MGL^{2*,*}(\Spec k)$ that takes $F^{\text{univ}}$ to the formal group law of $\MGL$.

\begin{thm}
\label{thm:comparisonLazard}
    Let $p= \chr k$ if $\chr k>0$, and $p=1$ otherwise. After inverting $p$, $\phi_{\MGL}:\Laz \to \MGL^{2*,}(\Spec k)$ is an isomorphism of graded rings. Moreover, $\MGL^{2n+m,n}(\Spec k)[1/p]=0$ for all $m>0$.
\end{thm}
The formal group law of $\MGL$ is then the universal formal group law. This result is due to Spitzweck (\cite{Spitzweck:AlgCob}) and Hoyois (\cite{Hoyois:AlgCob}). A detailed proof of this theorem can be found in \cite[Theorem 2.1]{lev:ellcoh}.

Through a well-established algebraic version of the Pontryagin-Thom construction (see \cite[Section 3]{lev:ellcoh}), we can associate to any smooth proper variety over $k$ a class in the algebraic cobordism ring $\MGL^{2*,*}(\Spec k)$. We already have all the ingredients for the definition.  

\begin{defn}
    \label{defn:MGLclasses}
If $X$ is a smooth proper variety $X \in \Sm/k$ of dimension $d$, we call the \emph{$\MGL$-class of $X$} the class $[X]_\MGL=[X,-T_X,\id]_\MGL \in \MGL_{2d,d}(\Spec k)=\MGL^{-2d,-d}(\Spec k)$ constructed as in Definition \ref{defn:twistclass}. Explicitly, if $p:X \to \Spec k$ is the structure map, $[X]_\MGL$ is given by the composition
\begin{multline*}
    1_k \xrightarrow{p^\vee} p_\# \Sigma^{-T_X}1_X \xrightarrow{p_\#\Sigma^{-T_X}\epsilon_{p^*\MGL}} p_\#\Sigma^{-T_X}p^*\MGL \xrightarrow{p_\#\Sigma^{-2d,-d}\th_{\MGL}^f(-T_X)} \\ p_\#\Sigma^{-2d,-d} p^*\MGL \simeq p_\#p^*\Sigma^{-2d,-d}\MGL \xrightarrow{\eta_{p_\#,p^*}(\Sigma^{-2d,-d}\MGL)}\Sigma^{-2d,-d}\MGL.
 \end{multline*}
\end{defn}

\begin{rmk}
    The class $[X]_\MGL$ of Definition \ref{defn:MGLclasses} can be seen as the class $p_*\th^{\MGL}(-T_X)$, where $p_*:\MGL^{-2d,-d}(p_{X\#}\Sigma^{-T_X}1_X) \simeq \MGL^{0,0}(X) \to \MGL^{-2d,-d}(\Spec k)$ is the proper pushforward along $p$, and $\th^{\MGL}(-T_X)$ is the $\MGL$ Thom class for the virtual vector bundle $-T_X$. Even if the assignment $\sV \mapsto \th^{\sE}(\sV)$ does not descend to a natural transformation $\th^\sE(-)$ of functors from $K_0(X)$ to $\SH(X)_\simeq$, it gives a map of spectra for each object in $\sK(X)$, and Proposition \ref{prop:K_0ExtensionTh(-)} says that automorphisms of $\sV$ in $\sK(X)$ act trivially on $\Sigma^{\sV-\mathcal{O}_X^{\rnk(\sV)}}p^*\sE$, which is all we need.
\end{rmk}

Let us recall that the algebraic cobordism ring is generated by $\MGL$-classes of $k$-schemes, in the following sense.
\begin{prop}[\cite{lev:ellcoh}, Theorem 3.4 (4)]
\label{prop:MGL-classes}
    After inverting  $\chr k$ if $\chr k>0$, $\MGL_{2d,d}(\Spec k)$ is generated by the classes $[X]_\MGL$, for $X$ smooth projective $k$-scheme with $\dim_k(X)=d$.
\end{prop}

Now, let us write $\BGL \coloneqq \colim_r \Sigma_{\P^1}^{\infty}\BGL_{r \; +}$, and let us briefly recall a computation for the motivic cohomology of $\BGL$ and $\MGL$.

\begin{prop}
\label{prop:MotCohOfMGL}
    We have
    $H\Z^{*,*}(\MGL)\simeq H\Z^{*,*}(\BGL)\simeq H\Z^{*,*}(k)[c_1,c_2, \ldots],$
    with $c_r$ in bidegree $(2r,r)$.
\end{prop}

\begin{proof}
As is well known, see for example \cite[Theorem 2.2]{Panin:algcob}, one has
$$H\Z^{*,*}(\BGL_r)=H\Z^{*,*}(k)[c_1,\ldots, c_r],$$
with $c_r$ in bidegree $(2r,r)$ being the usual $r$-th Chern class $c_r(E_m)$ for $m\ge r$, defined by the Grothendieck formula.

For $i_r:\BGL_r\to\BGL_{r+1}$ and for all $a,b$, the map
$$i_r^*:H\Z^{a,b}(\BGL_{r+1})=H\Z^{a,b}(k)[c_1, \ldots, c_{r+1}] \to H\Z^{a,b}(k)[c_1, \ldots, c_r]=H\Z^{a,b}(\BGL_r)$$
is surjective, because $i_r^*(c_i)=c_i$ for $i\le r$ and $i_r^*(c_{r+1})=0$, by the functoriality of Chern classes. Thus, the inverse system $\{H\Z^{a,b}(\BGL_r), i_r^*\}_r$ satisfies the Mittag-Leffler condition, and its $\lim^1$-term vanishes. We get the identity: 
\begin{equation}
\label{eq:cohomologyBGL}
    H\Z^{a,b}(\BGL):=[\colim_r\Sigma^\infty_{\P^1}\BGL_{r,+}, \Sigma^{a,b}H\Z]_{\SH(k)}\simeq
\lim_rH\Z^{a,b}(\BGL_r).
\end{equation}
Thus, because $H\Z^{a,b}(k)=0$ for $b<0$, it follows that
$$H\Z^{*,*}(\BGL)=H\Z^{*,*}(k)[c_1,\ldots, c_r, \ldots].$$

Since $\MGL_r=\Th(E_r)$, the fact that $H\Z$ is oriented gives us the Thom isomorphism
\begin{equation}
\label{eq:thomisoBGL}
    H\Z^{a+2r, b+r}(\MGL_r)\simeq H\Z^{a,b}(\BGL_r).
\end{equation}
Through this, we have a similar description of $H\Z^{a,b}(\MGL)$, namely,
\begin{align*}
H\Z^{a,b}(\MGL)&:=[\MGL, \Sigma^{a,b}H\Z]_{\SH(k)}\\
&=[\colim_r\Sigma^{-2r, -r}\MGL_r, \Sigma^{a,b}H\Z]_{\SH(k)}\\
&\simeq \lim_r H\Z^{2r+a, r+b}(\MGL_r)\\
&\overset{\eqref{eq:thomisoBGL}}{\simeq}\lim_r H\Z^{a,b}(\BGL_r)\\
&\overset{\eqref{eq:cohomologyBGL}}{\simeq} H\Z^{a,b}(\BGL).
\end{align*}
\end{proof}

In particular, via the description of $H\Z^{a,b}(\MGL)$ given above, we see that each element $c \in H\Z^{2d,d}(\MGL)$ corresponds to a unique weighted degree $d$ homogeneous polynomial $P_c \in \Z[x_1, \ldots, x_2]$, with $x_i$ of degree $i$, such that $c=P_c(c_1,\ldots, c_d)$.

\subsection{Sp-oriented cohomology theories}

The axioms defining a $\GL$-orientation as in Definition \ref{defn:thomclasstheory} can be used to define weaker kinds of orientation, namely a theory of Thom classes for vector bundles carrying additional structures (see \cite[Section 3]{Ana:Slor} or \cite[Section 2]{DegFas:Borel}). We are interested in defining a theory of Thom classes for symplectic vector bundles.
\begin{defn}
\label{defn:sympl.bundles}
    A rank $2r$ \emph{symplectic vector bundle} over $X \in \Sm/k$ is a pair $(V,\omega)$, where $V \to X$ is a rank $2r$ vector bundle, and $\omega: V \otimes V \to \mathcal{O}_X$ is a symplectic form on it, namely a non-degenerate antisymmetric bilinear form.
\end{defn}

The trivial $2r$ symplectic vector bundle $\mathcal{O}_X^{2r}$ can always be equipped with the standard rank $2r$ symplectic form $\phi_{2r}$, represented by the $2r \times 2r$ matrix
$$\phi_{2r} =
\begin{pmatrix}
     & & &  &  & 1\\
     & &  &  &\ldots & \\
     & & & 1& &\\
     & & -1 &  & &  \\
     & \ldots &  & & & \\
     -1 & & & & &
\end{pmatrix}.$$
Sometimes, in the literature, one refers to the standard $2r$ symplectic form as the symplectic form $\psi_{2r}$ represented by the $2r \times 2r$ matrix
$$\psi_{2r} =
\begin{pmatrix}
     \psi_2& & &\\
     & \psi_2 &  & \\
     & & \ldots &\\
     & & & \psi_2
\end{pmatrix}, \; \; \; \text{with} \; \; \; \psi_2=\phi_2 =
\begin{pmatrix}
    0 & 1 \\
    -1 & 0
\end{pmatrix}.$$

\begin{rmk}
    For $V\to X$ any rank $r$ vector bundle, the rank $2r$ vector bundle $V \oplus V^\vee \to X$ can always be equipped with the rank $2r$ standard symplectic form $\phi_{2r}$. Concretely, this is the map
    $$\phi_{2r}((v,v^*),(w,w^*)) \coloneqq <w^*,v> -<v^*,w>,$$
    where $<-,->:V^\vee \times V \to \mathcal{O}_X$ is the canonical pairing.
\end{rmk}

\begin{defn}
    An $\Sp$-orientation for a commutative ring spectrum $\sE \in \SH(k)$ consists in a theory of Thom classes $\th_{\Sp}^\sE(V,\omega) \in \sE^{4r,2r}(\Th(V))$ for each rank $2r$ symplectc vector bundle $(V,\omega)\to X$, $X \in \Sm/k$, satisfying the axioms of Definition \ref{defn:thomclasstheory}, after replacing the trivial vector bundle $\mathcal{O}^r_X$ with the trivial symplectic bundle $(\mathcal{O}_X^{2r},\phi_{2r})$, morphisms and exact sequences of vector bundles with morphisms and exact sequences of symplectic vector bundles.
\end{defn}

It is immediate to note that any $\GL$-orientation induces an $\Sp$-orientation, thus, any $\GL$-oriented commutative ring spectrum $\sE$ with a theory of Thom classes $\th^\sE(V)$ is also $\Sp$-oriented with an induced theory of Thom classes $\th_\Sp^\sE(V)$ for symplectic vector bundles.

We now want to get a theory of symplectic Thom isomorphisms as in the case of $\GL$-oriented theories. First, we need to define suitable sources of the suspension maps and Thom isomorphisms, giving symplectic analogs of the fundamental groupoid $\sK(X)$ and the Grothendieck group $K_0(X)$.

\begin{defn} Let us fix a scheme $X\in \Sm/k$.\\[5pt]
1. Let $\sS{p}(X)$ denote the groupoid whose objects are the symplectic vector bundles $(V,\omega)$ on $X$, and whose morphisms are isomorphisms of vector bundles compatible with the given symplectic forms.\\[2pt]
2. Let  $\sS{p}(X)_{st}$ be the category where objects are pairs of symplectic vector bundles on $X$ $((V,\omega), (V', \omega'))$, and where a morphism $((V_1,\omega_1), (V_1', \omega_1'))\to
((V_2,\omega_2), (V_2', \omega_2'))$ is a triple $((V,\omega), f,f')$ with $(V,\omega)$ a
symplectic vector bundle on $X$, and
$$f:(V_1\oplus V,\omega_1\oplus \omega)\xrightarrow{\sim} (V_2,\omega_2), \; \; f':(V'_1\oplus V,\omega'_1\oplus \omega)\xrightarrow{\sim} (V'_2,\omega'_2)$$ are isomorphisms of symplectic vector bundles. If we have two morphisms
$$((V_1,\omega_1),(V_1',\omega_1'))\xrightarrow{((V_f,\omega_f),f,f')} ((V_2,\omega_2),(V_2',\omega_2')) \xrightarrow{((V_g,\omega_g),g,g')} ((V_3,\omega_3),(V_3',\omega_3')),$$
their composition is the triple $((V_f \oplus V_g,\omega_f \oplus \omega_g),g \circ (f\oplus \id_{V_g}), g' \circ (f' \oplus \id_{V_g}))$. \\ [2pt]
3. Let $\sK^\Sp(X)$ be the localization of $\sS{p}(X)_{st}$ with respect to the morphisms 
$$((V'',\omega''), \id_{V\oplus V''},\id_{V'\oplus V''}):((V,\omega), (V', \omega'))\to ((V\oplus V'',\omega\oplus \omega''), (V'\oplus V'', \omega'\oplus\omega'')).$$
4. We have the commutative monoid $(\sS{p}(X)/\text{iso},\oplus)$ of isomorphism classes in $\sS{p}(X)$, with operation induced by the direct sum of symplectic  bundles. Let $K_0^\Sp(X)$ denote its group completion.
\end{defn}

\begin{rmk} 1. $\sK^\Sp(X)$ is a symmetric monoidal groupoid: the inverse of $((V,\omega), f,f'):((V_1,\omega_1), (V_1', \omega_1'))\to
((V_2,\omega_2), (V_2', \omega_2'))$ is $((V,\omega), \id_{V_1 \oplus V},\id_{V_1'\oplus V})^{-1}\circ (0, f^{-1}, f^{\prime-1})$, and the symmetric monoidal product is induced by direct sum. \\[2pt]
2. Sending $((V_1,\omega_1), (V_1', \omega_1'))$ to the formal difference of classes $[(V_1,\omega_1)]- [(V_1', \omega_1')]\in K_0^\Sp(X)$ descends to an isomorphism of abelian groups $ \sK^\Sp(X)/\text{iso}\cong  K_0^\Sp(X)$. \\[2pt]
3. Sending $((V_1,\omega_1), (V_1', \omega_1'))$ to  the complex $V_1\oplus V_1'[1]\in \sK(X)$ extends to a functor of symmetric monoidal groupoids $\sK^\Sp(X)\to \sK(X)$, which induces the corresponding map of abelian groups $K_0^\Sp(X)\to K_0(X)$ by passing to isomorphism classes. In particular, sending $((V_1,\omega_1), (V_1', \omega_1'))$ to $\Sigma^{V_1-V_1'}1_X\in \SH(X)$ extends to a functor of groupoids
\[
\Sigma^{(-)}:\sK^\Sp(X)\to \SH(X)_{\text{iso}}.
\]
\end{rmk}

Let $\sE\in \SH(k)$ be an $\Sp$-oriented ring spectrum. Given a rank $2r$ symplectic bundle, $(V,\omega)$ on $X\in \Sm/k$,  the Thom class $\th^\sE_\Sp(V,\omega)$ induces a morphism
\[
\th^{\Sp}_\sE(V,\omega):\Sigma^{V-\sO_X^{2r}}1_X\to p_X^*\sE
\]
in $\SH(X)$, where $p_X:X\to \Spec k$ is the structure map, which in turn induces the morphism
\[
\th^{\Sp,f}_{\sE}(V,\omega):\Sigma^{V-\sO_X^{2r}}p_X^*\sE\to p_X^*\sE
\]
in $\SH(X)$. If $\sE$ is highly structured, $\th^{\Sp}_\sE(V,\omega)$ also induces the map 
\[
\th^{\Sp, f}_{\sE\Mod}(V,\omega):\Sigma^{V-\sO_X^{2r}}p_X^*\sE\to p_X^*\sE 
\]
in $\Mod_{p^*\sE}$, by the free-forget adjunction.

\begin{lemma}
\label{lemma:beforethom}
    The map $\th^{\Sp,f}_{\sE}(V,\omega)$ is an isomorphism in $\SH(X)$, and if $\sE$ is highly structured, the map $\th^{\Sp, f}_{\sE\Mod}(V,\omega)$ is an isomorphism in $\Mod_{p^*\sE}$.
\end{lemma}

\begin{proof}
    The statement for $\th^{\Sp, f}_{\sE\Mod}(V,\omega)$ follows immediately from that for $\th^{\Sp,f}_{\sE}(V,\omega)$, by applying the forgetful functor. We prove the statement for $\th^{\Sp,f}_{\sE}(V,\omega)$.

    Let us suppose that $(V,\omega)$ is a trivial symplectic vector bundle, namely, a trivial rank $2r$ vector bundle $V$, equipped with the standard symplectic form $\omega = \phi_{2r}$. In this case $\th^{\Sp,f}_{\sE}(V,\omega)$ is an isomorphism. Indeed, we have $\Sigma^V=\Sigma^{4r,2r}$, and the map $\th_{\sE}^\Sp(V,\omega): \Sigma^V1_X \to \Sigma^{4r,2r}p_X^*\sE$ in $\SH(X)$ is obtained by applying $\Sigma^{4r,2r}$ to the unit map $\epsilon_{p_X^*\sE}$ of $p_X^*\sE$ in $\SH(X)$. In particular, the map 
    $$\th^{\Sp,f}_{\sE}(V,\omega)=\Sigma^{4r,2r}\mu_{p_X^*\sE} \circ \Sigma^{4r,2r}(\epsilon_{p_X^*\sE} \wedge 1) = \Sigma^{4r,2r}\id_{p_X^*\sE}$$
    is an isomorphism.

We now claim that every symplectic vector bundle is Zariski-locally isomorphic to the trivial symplectic vector bundle. We can prove this by applying a skew-symmetric version of the Gram-Schmidt process to find (not uniquely) a Zariski-local trivialization of the bundle, as follows.

Let $(V,\omega)$ be a symplectic vector bundle over $X$ of rank $2r$, and let us consider $x \in X$ a point. Let $v_1$ be a local section such that $v_1(x)\neq 0$. $v_1$ will be a nowhere zero section on a Zariski open neighbourhood $U_1$ of $x$. Since $\omega$ is non-degenerate, there exists a section $v_{2r}$ such that $\omega(v_1,v_{2r})(x)\neq 0$. The closed subset of $U_1$ where $\omega(v_1,v_{2r})=0$ is a Zariski closed subset. Then, after removing this closed subset from $U_1$, we can suppose $\omega(v_1,v_{2r}) \neq 0$ on $U_1$. Moreover, by rescaling $v_{2r}$ by the function $1/\omega(v_1,v_{2r})$, we can suppose $\omega(v_1,v_{2r})=1$ on $U_1$. Let us consider the restriction of $V$ to $U_1$, and let $V_1 \coloneqq \langle v_1, v_{2r} \rangle$ be the rank $2$ subbundle of $V\mid _{U_1}$ spanned by the two sections $v_1,v_{2r}$. The restriction of $\omega$ to the subbundle $V_1$ is then represented by the standard $2 \times 2$ symplectic matrix $\phi_2$ with respect to the two generating sections $v_1,v_{2r}$. So, $(V_1,\omega\mid_{V_1})$ is a rank $2$ symplectic subbundle of $V$ on $U_1$. Let us consider its perpendicular symplectic bundle
$$V_1 ^{\perp} \coloneqq \{v' \in V \mid_{U_1} \; : \; \omega(v',v)=0 \; \; \forall \; v \in V_1\}.$$
This gives us the orthogonal decomposition of the restriction of $(V,\omega)$ to $U_1$ as $(V_1,\omega_1)\perp(V_1^\perp,\omega_1^\perp)$, where $\omega_1$ is the restriction of $\omega$ to $V_1$ and $\omega_1^\perp$ is defined similarly. By proceeding inductively on the rank $2r$ of $V$, we find a neighbourhood $U_2 \subset U_1$ of $x$ over which $V_1^\perp$ admits a trivialization as a symplectic vector bundle. As a result, we have found a Zariski open neighbourhood $U_x$ of $x$ and a decomposition 
$$V \mid_U=V_1 \oplus \ldots \oplus V_r = \langle v_1, \ldots, v_{2r} \rangle$$
such that $\omega$ is represented by the standard $2r \times 2r$ symplectic matrix $\phi_{2r}$ with respect to the generating sections $v_1, \ldots v_{2r}$. This proves that $(V,\omega)$ is Zariski-locally isomorphic to the trivial symplectic bundle, whence the claim.

Thus, there will exist a Zariski open cover $\{U_{\alpha}\}_{\alpha}$ of $X$ such that, for all $j_{\alpha}:U_{\alpha}\to X$, the restriction of $(V,\omega)$ on $U_{\alpha}$ is a trivial symplectic vector bundle. If we write $(V_{\alpha},\omega_{\alpha}) \coloneqq j_{\alpha}^*(V,\alpha)$, we have that $\th_{\sE}^{\Sp,f}(V_\alpha,\omega_\alpha)$ is an isomorphism in $\SH(U_{\alpha})$ for all $\alpha$. Clearly, the restriction of $(V,\omega)$ to the intersection of finitely many such opens is trivial as well.

We will conclude by using a Mayer-Vietoris argument to show that $\th^{\Sp,f}_{\sE}(V,\omega)$ is an isomorphism as follows. To simplify the notation, we write $\Psi$ for $\th^{\Sp,f}_{\sE}(V,\omega)$. 

Since $X$ is quasi-compact, there are  $\alpha_1,\ldots, \alpha_n$ such that $X=\cup_{i=1}^nU_{\alpha_i}$, and thus this gives a finite open cover of $X$ trivializing the symplectic bundle $(V,\omega)$. 

Let $X_i=\cup_{j=1}^iU_{\alpha_i}$ with open immersion $j_i:X_i\to X$. We show by induction on $i$ that $j_i^*\Psi:j_i^*\Sigma^Vp_X^*\sE\to j_i^*\Sigma^{4r,2r}p_X^*\sE$ is an isomorphism, the case $i=1$ following by our construction of $U_{\alpha_1}$.

Let us take $i>1$ and write $X_i=X_{i-1}\cup U_{\alpha_i}$.  Let $j:X_{i-1}\cap U_{\alpha_i}\hookrightarrow X_i$, $j_1:X_{i-1}\hookrightarrow X_i$ and $j_2:U_{\alpha_i}\hookrightarrow X_i$ be the evident open immersions. Lemma \ref{lemma:Mayer-Vietoris} below shows that there exists a Mayer-Vietoris  distinguished triangle 
\begin{equation}
\label{eq:Mayer-Vietoris}
    j_\#j^* \to j_{1\#}j_1^* \oplus j_{2\#}j_2^* \to \id_{\SH(X_i)}\to  j_\#j^*[1]
\end{equation}
of endofunctors of $\SH(X_i)$, inducing the corresponding distinguished triangle of endofunctors of $\Mod_{p_{X_i}^*\sE}$. We apply this to the map of spectra $j_i^*\Psi$, giving the commutative diagram
\[
\xymatrixcolsep{60pt}
\xymatrix{
j_\#j^*j_i^*\Sigma^Vp_X^*\sE\ar[d]\ar[r]^-{j_\#j^*j_i^*\Psi}&j_\#j^*j_i^*\Sigma^{(4r,2r)}p_X^*\sE\ar[d]\\
\hbox{$\begin{matrix}j_{1\#}j_{i-1}^*\Psi\\\oplus\\ j_{2\#}j_2^*j_i^*\Psi\end{matrix}$}\ar[d]\ar[r]^-{\tiny\begin{matrix}j_{1\#} j_{i-1}^*\Psi\\
\oplus\\
 j_{2\#}j_2^*j_i^*\Psi\end{matrix}}&\hbox{$\begin{matrix}j_{1\#}j_{i-1}^*\Sigma^{(4r,2r)}p_X^*\sE\\\oplus \\j_{2\#}j_2^*j_i^*\Sigma^{(4r,2r)}p_X^*\sE\end{matrix}$}\ar[d]\\
j_i^*\Sigma^Vp_X^*\sE\ar[d]\ar[r]^-{j_i^*\Psi}&j_i^*\Sigma^{(4r,2r)}p_X^*\sE\ar[d]\\
j_\#j^*j_i^*\Sigma^Vp_X^*\sE[1]\ar[r]^-{j_\#j^*j_i^*\Psi[1]}&j_\#j^*j_i^*\Sigma^{(4r,2r)}p_X^*\sE[1], 
}
\]
with columns distinguished triangles in $\Mod_{p_{X_i}^*\sE}$.

Noting that the trivialization of restriction of $(V,\omega)$ to $U_{\alpha_i}$ induces a trivialization of the restriction to $X_{i-1}\cap U_{\alpha_i}$ and using the induction hypothesis, we see that the horizontal maps in the first, second, and fourth rows are isomorphisms, hence
$j_i^*\Psi$ is an isomorphism and the induction goes through. 
\end{proof}

\begin{lemma}
\label{lemma:Mayer-Vietoris}
    Let $X=U_1 \cup U_2$ be the union of two Zariski open subsets $U_1, U_2$, and let $j_1:U_1 \to X$, $j_2: U_2 \to X$ and $j:U_1 \cap U_2 \to X$ the respective open inclusions. Then we have a Mayer-Vietoris distinguished triangle
    $$j_\#j^* \to j_{1\#}j_1^* \oplus j_{2\#}j_2^* \to \id_{\SH(X)} \to j_\#j^* \to j_{1\#}j_1^*[1]$$
    of endofunctors of $\SH(X)$.
\end{lemma}

\begin{proof}
    Let us consider the diagram
$$
\begin{tikzcd}
    U_1 \cap U_2 \arrow[r, "t_2"] \arrow[d, swap, "t_1"] \arrow[dr, "j"] & U_2\arrow[d, "j_2"] \\
    U_1 \arrow[r, "j_1"] & U_1 \cup U_2 =X.
\end{tikzcd}
$$
Let $i_1: X-U_1 \to X$ be the closed complement of $j_1$, and similarly, let $i_2$ be the closed complement of $t_2$. From Proposition \ref{prop:localization}, the localization sequence for the pair $(j_1,i_1)$ is the distinguished triangle 
\begin{equation}
\label{eq:MV1}
    j_{1!}j_1^*\to \id_{\SH(X)} \to i_{1*}i_1^* \to j_{1!}j_1^*[1]
\end{equation}
of endofunctors of $\SH(X)$. Analogously, the localization sequence for the pair $(t_2,i_2)$ is the distinguished triangle
$$t_{2!}t_2^* \to \id_{\SH(X)} \to i_{2*}i_2^* \to t_{2!}t_2^*[1].$$
By applying $j_{2!}(-)j_2^*$ to the last triangle, we get the distinguished triangle
$$j_{2!}t_{2!}t_2^*j_2^* \to j_{2!}j_2^* \to j_{2!}i_{2*}i_2^*j_2^* \to j_{2!}t_{2!}t_2^*j_2^*[1],$$
that can be rewritten as 
\begin{equation}
\label{eq:MV2}
     j_!j^* \to j_{2!}j_2^* \to j_{2!}i_{2*}i_2^*j_2^* \to j_!j^*[1].
\end{equation}
Moreover, we have a canonical distinguished triangle
\begin{equation}
\label{eq:MV3}
    j_{1!}j_1^* \to j_{1!}j_1^* \oplus j_{2!}j_2^* \to j_{2!}j_2^* \to j_{1!}j_1^*[1]. 
\end{equation}
Since $i_1 \simeq j_2 \circ i_2$ on $U_2-(U_1\cap U_2)$, we have that $j_{2!}i_{2*}i_2^*j_2^* \simeq i_{1*}i_1^*$, that is, the right-hand terms of \eqref{eq:MV1} and \eqref{eq:MV2} are isomorphic. Thus, we have the conditions for applying the octahedral axiom to the triangles \eqref{eq:MV1}, \eqref{eq:MV2} and \eqref{eq:MV3}, and we get the distinguished triangle
$$j_!j^* \to j_{1!}j_1^* \oplus j_{2!}j_2^* \to \id_{\SH(X)} \to j_!j^* \to j_{1!}j_1^*[1].$$
Finally, for any $f$ open immersion, we have $f_!\simeq f_\#$ by Remark \ref{rmk:etalemaps}. This concludes the proof.
\end{proof}

Let $(V,\omega)$ be a symplectic vector bundle of rank $2r$. We have proved that $\th^{\Sp,f}_{\sE}(V,\omega)$ is an isomorphism. We may then define 
\[
\th^{\Sp,f}_{\sE}(-(V,\omega)):\Sigma^{\sO_X^{2r}-V}p_X^*\sE\to p_X^*\sE
\]
by
\[
\th^{\Sp,f}_{\sE}(-(V,\omega)):=\Sigma^{\sO_X^{2r}-V}(\th^{\Sp}_{\sE\Mod}(V,\omega)^{-1}),
\]
and define
\[
\th^{\Sp}_{\sE}(-(V,\omega)):\Sigma^{\sO_X^{2r}-V}1_X\to p_X^*\sE
\]
by composing $\th^{\Sp,f}_{\sE}(-(V,\omega))$ with the map $\Sigma^{\sO_X^{2r}-V}1_X\to
\Sigma^{\sO_X^{2r}-V}p_X^*\sE$ induced by the unit $\epsilon_\sE$ of $\sE$. Finally, if $(V',\omega')$ is another symplectic vector bundle of rank $2r'$, we define
\[
\th^{\Sp}_{\sE}((V,\omega)-(V',\omega')):\Sigma^{V-V'- (\sO_X^{2r}-\sO_X^{2r'})}1_X\to p_X^*\sE
\]
as the composition
\begin{multline*}
\Sigma^{V-V'- (\sO_X^{2r}-\sO_X^{2r'})}1_X\simeq
\Sigma^{V- \sO_X^{2r}}(\Sigma^{-V'+ \sO_X^{2r'}}1_X)\\\xrightarrow{\Sigma^{V- \sO_X^{2r}}(\th^{\Sp}_{\sE}(-(V',\omega'))}
\Sigma^{V- \sO_X^{2r}}p_X^*\sE\xrightarrow{\th^{\Sp,f}_{\sE}(V,\omega)}
p_X^*\sE.
\end{multline*}
Again, we also have the $p_X^*\sE$-extension
\[
\th^{\Sp,f}_{\sE}((V,\omega)-(V',\omega')):\Sigma^{V-V'- (\sO_X^{2r}-\sO_X^{2r'})}p^*_X\sE\to p_X^*\sE;
\]
given by 
$\th^{\Sp,f}_{\sE}(V,\omega)\circ \Sigma^{V-V'-\sO_X^{2r}-\sO_X^{2r'}}(\th^{\Sp,f}_{\sE}(V',\omega')^{-1})$, and similarly, if $\sE$ is highly structured, the map $\th^{\Sp, f}_{\sE\Mod}((V,\omega)-(V',\omega'))$ of $p_X^*\sE$-modules.

The multiplicativity of the symplectic Thom isomorphisms, as stated in the next lemma, follows as for the $\GL$-oriented case from the multiplicativity of the symplectic Thom classes, following the symplectic version of  \eqref{enum:ThomClassAxioms}(3).

\begin{lemma}\label{lem:SympThomMult} Let $\sE\in \SH(k)$ be symplectically oriented, and take $X\in \Sm/k$. 
For $v,v'\in \sK^\Sp(X)$, $\th_\sE^\Sp(v+v')$ is the composition
\begin{multline}\label{multline:SympThomMult}
\Sigma^{v+v'-\sO_X^{\rnk(v+v')}}1_X\simeq \Sigma^{v-\sO_X^{\rnk(v)}}1_X\wedge_X\Sigma^{v'-\sO_X^{\rnk(v')}}
1_X\\\xrightarrow{\th_\sE^\Sp(v)\wedge\th^\Sp_\sE(v')}p^*\sE\wedge_Xp^*\sE\xrightarrow{\mu_{p^*\sE}} p^*\sE\notag
\end{multline}
where $\mu_{p^*\sE}$ is the multiplication on $p^*\sE$. 

In consequence, the following diagram commutes (with $2r=\rnk(v)$, $2r'=\rnk(v'))$:
\[
\xymatrix{
\Sigma^{v+v'-\sO_X^{2r+2r'}}1_X\ar[ddrrr]^{\th^\Sp_\sE(v+v')}\ar[d]_-\wr\ar[r]^-\sim&\Sigma^{v-\sO_X^{2r}}(\Sigma^{v'-\sO_X^{2r'}}1_X)\ar[rr]^-{\Sigma^{v-\sO_X^{2r}}(\th^\Sp_\sE(v'))}&&\Sigma^{v-\sO_X^{2r}}p_X^*\sE\ar[d]^-{\id\wedge\th^\Sp_\sE(v)}\\
\Sigma^{v'-\sO_X^{2r'}}(\Sigma^{v-\sO_X^{2r}}1_X)
\ar[d]_-{\Sigma^{v'-\sO_X^{2r'}}(\th^\Sp_\sE(v))}&&&p_X^*\sE\wedge_Xp^*\sE\ar[d]^-{\mu_{p^*\sE}}\\
\Sigma^{v'-\sO_X^{2r'}}p^*\sE\ar[r]^-{\th^\Sp_\sE(v')\wedge\id}&p_X^*\sE\wedge_Xp^*\sE\ar[rr]^-{\mu_{p^*\sE}}&&p^*\sE
}
\]
\end{lemma}

\begin{prop}\label{prop:K_0SpExtensionTh(-)} Let $\sE\in\SH(k)$ be a symplectically oriented ring spectrum and let $p_X:X\to \Spec k$ be in $\Sm/k$.\\[5pt]
1. Let $c_{p_X^*\sE}:\sK^\Sp(X)\to \SH(X)_{\id}$ be the constant functor with value $ p_X^*\sE$, with $\SH(X)_{\id}$ considered as a category with only identity morphisms. Then sending $(V,\omega)-(V',\omega')$ to $\th^{\Sp}_{\sE}((V,\omega)-(V',\omega'))$ extends to a natural transformation of functors
\[
(\th^{\Sp}_{\sE}(-):\Sigma^{(-)-\sO_X^{\rnk(-)}}1_X\to c_{p_X^*\sE}):\sK^\Sp(X)\to \SH(X). 
\]
2. The functor of groupoids $\Sigma^{(-)-\sO_X^{\rnk(-)}}p^*_X\sE:\sK^\Sp(X)\to (\SH(X))_\simeq$ descends (up to equivalence) to a functor $\Sigma^{(-)-\sO_X^{\rnk(-)}}p^*_X\sE:K_0^\Sp(X)\to \SH(X)$, where we consider 
$K_0^\Sp(X)$ as a category with only identity morphisms.\\[2pt]
3. For each $(v,\omega)\in K^\Sp_0(X)$, let $\th^{\Sp,f}_{\sE}(v, \omega):
\Sigma^{v-\sO_X^{\rnk(v)}}p^*_X\sE\to p_X^*\sE$ be the $\sE$-linear extension of  
$\th^{\Sp,f}_{\sE}(v, \omega)$. Then the maps $\th^{\Sp,f}_{\sE}(v, \omega)$ define a natural isomorphism of functors of groupoids
\[
\th^{\Sp,f}_{\sE}(-):\Sigma^{(-)-\sO_X^{\rnk(-)}}p^*_X\sE\to c_{p_X^*\sE}:K_0^\Sp(X)\to (\SH(X))_\simeq. 
\]
\end{prop}

The proof is essentially the same as that of Proposition~\ref{prop:K_0ExtensionTh(-)} for the $\GL$-oriented case, where one uses
Lemma~\ref{lem:SympThomMult} to show that the maps $\th^{\Sp}_{\sE}(-)$  are natural in $\sK^\Sp(X)$.

\begin{proof}
 We start by proving the following extra statement: \\[5pt]
(3${}'$) The maps $\th^{\Sp,f}_{\sE}(v, \omega)$ define a natural isomorphism of functors
\[
\th^{\Sp,f}_{\sE}(-):\Sigma^{(-)-\sO_X^{\rnk(-)}}p^*_X\sE\to c_{p_X^*\sE}:\sK^\Sp(X)\to (\SH(X))_\simeq. 
\]
We have already shown that the individual morphisms $\th^{\Sp,f}_{\sE}(v,\omega)$ are isomorphisms in $\SH(X)$, so we only need to show the naturality with respect to the morphisms in $\sK^\Sp(X)$. 

We first prove the naturality with respect to morphisms in $\sS{p}(X)^2=\sS{p}(X) \times \sS{p}(X)$. Let $f:(V_1,\omega_1)\to (V_2,\omega_2)$, and $f':(V'_1,\omega_1')\to (V'_2,\omega_2')$ be isomorphisms of symplectic vector bundles (of respective ranks $2r,2r'$) on $X$. We need to show that the diagram
\[
\xymatrixcolsep{100pt}
\xymatrix{
\Sigma^{V_1-\sO_X^{2r}-V_1'+\sO_X^{2r'}}p_X^*\sE\ar[d]^-\sim\ar[r]^{\Sigma^{f-f'}}&
\Sigma^{V_2-\sO_X^{2r}-V_2'+\sO_X^{2r'}}p_X^*\sE\ar[d]^-\sim\\
\Sigma^{V_1-\sO_X^{2r}}(\Sigma^{-V_1'+\sO_X^{2r'}}p_X^*\sE)\ar[d]^{\Sigma^{V_1-V_1'-\sO_X^{2r}-\sO_X^{2r'}}(\th^{\Sp,f}_{\sE}(V_1',\omega')^{-1})}\ar[r]^{\Sigma^f(\Sigma^{-f'})}&
\Sigma^{V_2-\sO_X^{2r}}(\Sigma^{-V_2'+\sO_X^{2r'}}p_X^*\sE)\ar[d]^{\Sigma^{V_2-V_2'-\sO_X^{2r}-\sO_X^{2r'}}(\th^{\Sp,f}_{\sE}(V_2',\omega')^{-1})}\\
\Sigma^{V_1-\sO_X^{2r}}p_X^*\sE\ar[d]^{\th^{\Sp,f}_{\sE}(V_1,\omega)}\ar[r]^{\Sigma^f}&
\Sigma^{V_2-\sO_X^{2r}}p_X^*\sE\ar[d]^{\th^{\Sp,f}_{\sE}(V_2,\omega)}\\
\sE\ar@{=}[r]&\sE
}
\]
commutes. The commutativity of the top square is clear. The commutativity of the bottom one follows from the naturality of the isomorphisms $\th^{\Sp,f}_{\sE}((V,\omega_V))$ with respect to isomorphisms of symplectic bundles $(V,\omega_V)\simeq (W,\omega_W)$.

Next, we check the naturality with respect to morphisms in $\sS{p}(X)_{st}$. Since we have already checked  the naturality with respect to morphisms in $\sS{p}(X)^2$, we only need to check  naturality with respect to morphisms of the form
$$((V'',\omega''),\id_{V\oplus V''},\id_{V'\oplus V''}):((V,\omega), (V',\omega'))\to((V\oplus V'',\omega\oplus \omega''), (V'\oplus V'',\omega'\oplus \omega'')).$$
This is equivalent to the commutativity of the diagram
\[
\xymatrix{
\Sigma^{(V+V'')-(V'+V'')-\sO_X^{2r}+\sO_X^{2r'}}p_X^*\sE\ar[d]^\wr\ar[r]^-{can}_-\sim&\Sigma^{V-V'-\sO_X^{2r}+\sO_X^{2r'}}p_X^*\sE\ar[d]^\wr\\
p_X^*\sE\ar@{=}[r]&p_X^*\sE,
}
\]
where the vertical maps are the respective Thom isomorphisms. This is the identity
\begin{multline}\label{mult:Identity}
\th^{\Sp,f}_{\sE}(V,\omega)\circ \Sigma^{V-V'-\sO_X^{2r}-\sO_X^{2r'}}(\th^{\Sp,f}_{\sE}(V',\omega')^{-1})\circ {can}\\=
\th^{\Sp,f}_{\sE}(V\oplus V'',\omega\oplus \omega'')\circ \Sigma^{V-V'-\sO_X^{2r}-\sO_X^{2r'}}(\th^{\Sp,f}_{\sE}(V'\oplus V'',\omega'\oplus \omega'')^{-1}).
\end{multline}
To check this identity, we note that Lemma~\ref{lem:SympThomMult} gives us the commutativity of the diagrams
\[
\xymatrix{
 \Sigma^{(V\oplus V'')-(V'\oplus V'')-\sO_X^{2(r+r'')}+\sO_X^{2(r'+r'')}}p_X^*\sE\ar[rrr]^-{can}_-\sim
\ar[d]^-{can}_-\wr&&& \Sigma^{-V'+\sO_X^{2r'}}\Sigma^{V-\sO_X^{2r}}p_X^*\sE
\ar[dd]_-{\Sigma^{-V'+\sO_X^{2r'}}\th^{\Sp,f}_{\sE}(V,\omega)} \\
 \Sigma^{-V'+\sO_X^{2r'}}\Sigma^{-V''+\sO_X^{2r''}}\Sigma^{V\oplus V''-\sO_X^{2(r+r'')}}p_X^*\sE
 \ar[d]^-{\Sigma^{-V'+\sO_X^{2r'}}\Sigma^{-V''+\sO_X^{2r''}}\th^{\Sp,f}_{\sE}(V\oplus V'',\omega\oplus\omega'')}\\
\Sigma^{-V'+\sO_X^{2r'}}\Sigma^{-V''+\sO_X^{2r''}}p_X^*\sE&&& \Sigma^{-V'+\sO_X^{2r'}}p_X^*\sE\ar[lll]^-{\Sigma^{-V'+\sO_X^{2r'}}\Sigma^{-V''+\sO_X^{2r''}}\th^{\Sp,f}_{\sE}(V'',\omega'')}
}
\]
and
\[
\xymatrixcolsep{30pt}
\xymatrix{
\Sigma^{-V'+\sO_X^{2r'}}\Sigma^{-V''+\sO_X^{2r''}}p_X^*\sE\ar[d]^-{can}_-\wr&&&& \Sigma^{-V'+\sO_X^{2r'}}p_X^*\sE\ar[llll]_-{\Sigma^{-V'+\sO_X^{2r'}}\Sigma^{-V''+\sO_X^{2r''}}\th^{\Sp,f}_{\sE}(V'',\omega'')}\\
\Sigma^{-(V'\oplus V'')+\sO_X^{2r'+2r''}}p_X^*\sE&&&&
\ar[llll]^-{\Sigma^{-(V'\oplus V'')+\sO_X^{2r'+2r''}}\th^{\Sp,f}_{\sE}(V'\oplus V'',\omega'\oplus \omega'')}p_X^*\sE,\ar[u]^-{\Sigma^{-V'+\sO_X^{2r'}}\th^{\Sp,f}_{\sE}(V',\omega')}
}
\]
which together yield the identity \eqref{mult:Identity}.

Since $\sK^\Sp(X)$ is formed from $\sS{p}(X)_{st}$ by localization, the naturality on 
$\sK^\Sp(X)$ follows from the naturality on $\sS{p}(X)_{st}$. We have then proved the extra statement ($3'$).

Having proved (3${}'$), (1) follows by composing $\th^{\Sp,f}_{\sE}(-)$ with the unit map 
\[
\Sigma^{(-)-\sO_X^{\rnk(-)}}\epsilon_{p_X^*\sE}:\Sigma^{(-)-\sO_X^{\rnk(-)}}1_X\to \Sigma^{(-)-\sO_X^{\rnk(-)}}p^*_X\sE. 
\]

For (2), let us suppose we are given two isomorphisms of symplectic vector bundles $f_1, f_2:(V,\omega)\to (V', \omega')$ of rank $2r$. It follows from (3${}'$) that
$$\Sigma^{f_i}:\Sigma^{V-\sO_X^{2r}}p_X^*\sE\to \Sigma^{V'-\sO_X^{2r}}p_X^*\sE,$$
for $i=1,2$, are both equal to $(\th^{\Sp,f}_{\sE}(V',\omega'))^{-1}\circ \th^{\Sp,f}_{\sE}(V,\omega)$. This shows that the functor  $\Sigma^{(-)-\sO_X^{\rnk(-)}}p^*_X\sE$ descends to $K_0^\Sp(X)$. 

Finally, (3) follows directly from (3${}'$) and (2).
 \end{proof}

\begin{rmk} For $\sE$ highly structured, we may refine Proposition~\ref{prop:K_0SpExtensionTh(-)} by replacing $\th^{\Sp,f}_{\sE}$ with $\th^{\Sp, f}_{\sE\Mod}$ and the target category $\SH(X)$ with $\Mod_\sE$. All the rest in the proof remains the same.
\end{rmk}
 
\subsubsection{Borel Classes}

It is worth at this point to define also a symplectic analogue of Chern classes, namely Borel classes, following \cite{panwal:grass}. We start from the first Borel class of a rank 2 symplectic bundle.

\begin{defn}
\label{defn:FirstBorelClass}
    Let $\sE \in \SH(k)$ be an $\Sp$-oriented commutative ring spectrum. Let $(V,\omega)$ be a rank $2$ symplectic vector bundle over $X \in \Sm/k$, and let $z_0: X_+ \to \Th(V)$ be the zero section of the Thom space in $\sH_\bullet(k)$. The \emph{first Borel class} of $(V,\omega)$ is the $\sE$-cohomology class $b_1(V,\omega) \coloneqq -z_0^* \th_\Sp^\sE(V,\omega) \in \sE^{4,2}(X)$.
\end{defn}

To any symplectic vector bundle $(V,\omega) \to X$, one has the associated quaternionic projective bundle $\HP(V,\omega) \to X$ defined by $\HP(V,\omega):=\Gr(2,V)/\Gr\Sp(2,V,\omega)$, where $\Gr(2,V)$ is the Grassmannian of rank $2$ vector subbundles of the vector bundle $V$, and $\Gr\Sp(2,V,\omega)$ is the closed subscheme of $\Gr(2,V)$ parametrizing rank $2$ vector subbundles of $V$ over which the restriction of $\omega$ is degenerate. $\HP(V,\omega)$ is the symplectic analogue of the projective bundle $\P(V)$. The analogue of the projective bundle formula is the following important theorem.

\begin{thm}[Quaternionic projective bundle formula]
\label{thm:QuatProjBundle}
    Let $\sE \in \SH(k)$ be an $\Sp$-oriented commutative ring spectrum. Let $(V,\omega) \to X$ be a rank $2r$ symplectic vector bundle over $X \in \Sm/k$, and $\HP(V,\omega) \xrightarrow{q}X$ the associated quaternionic projective bundle. Let $(\mathcal{U}, \omega \mid_{\mathcal{U}})\to \HP(V,\omega)$ be the tautological rank $2$ symplectic vector bundle over $\HP(V,\omega)$, and $\xi \coloneqq b_1(\mathcal{U}, \omega \mid_{\mathcal{U}}) \in \sE^{4,2}(\HP(V,\omega))$ its first Borel class. Then $\sE^{*,*}(\HP(V,\omega))$ is a free $\sE^{*,*}(X)$-module via the pullback $q^*$, with basis $1, \xi, \ldots, \xi^r$. Also, for $1\le i \le r$, there exist unique elements $b_i(V,\omega) \in \sE^{4i,2i}(X)$ 
    such that
    $$\xi^r + \sum_{i=1}^r(-1)^ib_i(V,\omega) \cup \xi^{r-i}=0,$$
    and if $(V,\omega)$ is the trivial, then $b_i(V,\omega)=0$ for all $1\le i \le r$. 
\end{thm}

This is \cite[Theorem 8.2]{panwal:grass}. In particular, if $r=1$, one deduces from the definition of the first Borel class and the discussion on symplectic Thom isomorphisms that the element $b_1(V,\omega)$ given by Theorem \ref{thm:QuatProjBundle} coincides with the first Borel class of $(V,\omega)$ of Definition \ref{defn:FirstBorelClass}. This leads to the following definition.
\begin{defn}
    For an $\Sp$-oriented cohomology theory $\sE$ or a vector bundle $(V,\omega)$ over $X \in \Sm/k$, we call the elements $b_i(V,\omega)$ given by Theorem \ref{thm:QuatProjBundle} the \emph{Borel classes} of $(V,\omega)$.
\end{defn}
Alternatively, one can also define the set of Borel classes $b_i(V,\omega)$ axiomatically, as we have done for the Chern classes $c_i(V)$ in Definition \ref{def:ChernClassTheory}. The set of axioms for the Borel classes is in \cite[Definition 14.1]{panwal:grass}, and follows the same lines as for Chern classes. This definition then does not need the quaternionic projective bundle formula. The two approaches are equivalent by \cite[Theorem 14.4]{panwal:grass}.

\begin{rmk}
    The sign in the definition of the first Borel class is purely conventional, and it is chosen in order to satisfy the traditional relation $b_i(V)=(-1)^ic_{2i}$ for cohomology theories that are both $\Sp$-oriented and $\GL$-oriented. In fact, for an oriented cohomology theory $\sE$, we had seen in our discussion on Chern classes, following \cite{Pan:oriented}, that for a vector bundle $V\to X$ we have $c_{\rnk(V)}(V)=z_0^*\th^\sE(V)$, with $z_0:X \to \Th(V)$ the zero section. Thus, considering the induced $\Sp$-orientation on $\sE$, for a rank $2$ symplectic vector bundle $(V,\omega)$, we have that $b_1(V,\omega)=-c_2(V)$, and more in general, one can see that the classes $(-1)^ic_{2i}$ for $i=1, \ldots, r$ satisfy the axioms for Borel classes in \cite[Definition 14.1]{panwal:grass}.
\end{rmk}

 \subsubsection{Algebraic Symplectic Cobordism}
 \label{subsection:MSp}

A symplectic analogue of the algebraic cobordism spectrum $\MGL$ is the spectrum $\MSp$ constructed in \cite[Section 6]{Panwal-cobordism}, which we now recall.

\begin{constr}
\label{constr:MSp}
    If we equip $\A_k^{2n}$ with the standard symplectic form $\phi_{2n}$, we can define the \emph{quaternionic Grassmannian} $\HGr(r,n)$ as the open subscheme $j:\HGr(r,n) \xhookrightarrow{j} \Gr(2r,2n)$ parametrizing the $2r$-dimensional subspaces of $\A_k^{2n}$ on which the restriction of $\phi_{2n}$ is non-degenerate. We denote by $E_{2r,2n}^\Sp$ the restriction $j^*E_{2r,2n}$ to $\HGr(r,n)$ of the tautological bundle over $\Gr(2r,2n)$. In particular, $E_{2r,2n}^\Sp$ is a subbundle of the trivial bundle $\mathcal{O}_{\HGr(r,n)}^{2n}$. $\phi_{2n}$ defines a symplectic form on $\mathcal{O}_{\HGr(r,n)}^{2n}$, and, by construction, its restriction to $E_{2r,2n}^\Sp$ is non-degenerate, hence, it is still a symplectic form. The symplectic bundle $(E_{2r,2n}^\Sp,\phi_{2n} \mid_{E_{2r,2n}^\Sp})$ is called the tautological rank $2r$ symplectic subbundle. We will occasionally omit the symplectic form in the notation, and write $E_{2r,2n}^\Sp$ for both the tautological symplectic subbundle and the underlying vector bundle, depending on the context.

    We now focus on the schemes $\HGr(r,rN)$, noting that
    $$\HGr(r,rN)=\HGr(r,(\A^{2r}_k,\phi_{2r})^{\oplus N}).$$
    There are canonical inclusions $\HGr(r,rN) \hookrightarrow \HGr(r,r(N +1))$, and we can then define 
    $$\BSp_{2r} \coloneqq \colim_N \HGr(r,rN) \in \sH(k).$$
    We can also consider the vector bundle $E_{2r}^\Sp \to \BSp_{2r}$ defined by $E_{2r}^\Sp \coloneqq \colim_N E_{2r,2rN}^\Sp$, and
    $$\MSp_{2r} \coloneqq \Th(E_{2r}^\Sp) \in \sH_{\bullet}(k).$$
    The canonical inclusion $i_r:\BSp_{2r} \to \BSp_{2r+2}$ gives a canonical isomorphism $i^*E_{2r+2}^\Sp \simeq \mathcal{O}_{\BSp_{2r}} \oplus E_{2r}^\Sp \oplus \mathcal{O}_{\BSp_{2r}}$, which induces an isomorphism of Thom spaces $\Th(i_r^*E_{2r+2}^\Sp)\simeq \Sigma^{4,2}\Th(E_{2r}^\Sp)$. The map $\tilde{i}_r:i_r^*E_{2r+2}^\Sp \to E_{2r+2}^\Sp$ over $i_r$ gives the composition 
    $$\epsilon_r:\Sigma^{4,2}\Th(E_{2r}^\Sp) \simeq \Th(i^*_r E_{2r+2}^\Sp) \xrightarrow{\Th(\tilde{i}_r)}\Th(E_{2r+2}^\Sp),$$
    and
    $$\Sigma^{-4r-4,-2r-2}\Sigma_{\P^1}^\infty \epsilon_r:\Sigma^{-4r,-2r}\MSp_{2r} \to \Sigma^{-4r-4,-2r-2}\MSp_{2r+2}.$$
    Finally, by analogy with Remark \ref{rmk:MGLcolim}, one defines
    $$\MSp \coloneqq \colim_r \Sigma^{-4r,-2r}\MSp_{2r} \in \SH(k).$$
\end{constr}

\begin{rmk}
    Even if $\MSp$ is an object of $\SH(k)$, the construction that we showed, as a spectrum
    $$\MSp=(\MSp_0, \MSp_2, \ldots, \MSp_{2r}, \ldots)$$
    with bonding maps $\epsilon_r:\Sigma^{4,2}\MSp_{2r} \to \MSp_{2r+2}$, fits more naturally into the category of $(\P^1)^{\wedge 2}$-spectra rather than $\P^1$-spectra. However, this is not very relevant, since as already mentioned in \S \ref{subsection:HighlyStructuredRings}, the homotopy category of $(\P^1)^{\wedge 2}$-spectra is equivalent to the homotopy category $\SH(k)$ of $\P^1$-spectra, as symmetric monoidal categories, by \cite[Theorem 3.2]{Panwal-cobordism}.
\end{rmk}

The $\MSp$ cohomology theory is called \emph{symplectic cobordism}, and the respective homology theory \emph{symplectic bordism}.  

As highlighted in \cite[Section 6]{Panwal-cobordism}, $\MSp$ lifts to a commutative monoid in the category of symmetric $\P^2$-spectra. Thus, $\MSp$ is a highly structured commutative ring spectrum in $\SH(k)$ as well.

Analogously to Construction \ref{constr:MGLthomclasses}, the identification $\Th(E_{2r}^\Sp)=\MSp_{2r}$ induces a map $\alpha_r^\Sp: \Sigma_{\P^1}^\infty \MSp_{2r}\to \Sigma^{4r,2r}\MSp$, which can be equivalently defined as the $\Sigma^{4r,2r}$-suspension of the canonical map $\Sigma^{-4r,-2r}\MSp_{2r} \to \MSp$ in the definition of $\MSp$ as a colimit. This map gives the \emph{symplectic $r$-th universal Thom class}. For any rank $2r$ symplectic vector bundle $(V,\omega)\to X$ over an affine scheme $X \in \Sm/k$, we have the classifying map $\eta^\Sp_{(V,\omega)}:X \to \BSp_{2r}$ and the induced map $\tilde{\eta}_{V,\omega}:(V,\omega)\to E_{2r}^\Sp$ as in Construction \ref{constr:MGLthomclasses}, replacing $\mathcal{O}_X^n$ with $(\mathcal{O}_X^{2n},\phi_{2n})$ and $\Gr(r,n)$ with $\HGr(r,n)$. By the usual Jouanolou trick, one extends the construction to general $X \in \Sm/k$. This then defines the symplectic Thom class for $\MSp$ as:
\begin{equation}
    \label{eq:MSpThomclass}
    \th_\Sp^\MSp(V,\omega) \coloneqq \alpha_r^\Sp \circ \Sigma_{\P^1}^\infty \Th(\tilde{\eta}_{V,\omega}): \Sigma_{\P^1}^\infty \Th(V) \to \Sigma^{4r,2r}\MSp.
\end{equation}

This theory of Thom classes for symplectic bundles defines the universal $\Sp$-orientation, because of the following theorem.

\begin{thm}[\cite{PanWal:MSpKtheory}, Theorem 4.5]
\label{thm:universalitySp}
    For any $\Sp$-oriented commutative ring spectrum $\sE$, there exists a morphism of commutative ring spectra in $\SH(k)$, $\varphi:\sE \to \MSp$, such that, for any rank $2r$ symplectic vector bundle $(V,\omega)\to X$ over $X \in \Sm/k$, the Thom class $\th_\Sp^\sE(V):p_{X\#}\Sigma^V1_X \to \Sigma^{4r,2r}\sE$ factors through $\varphi$,
    and in general, the assignment
    $$\varphi \mapsto \th_\Sp^\sE(V,\omega)\coloneqq \Sigma^{4r,2r}\varphi \circ \th_\Sp^\MSp(V,\omega)$$
    gives a bijection of sets
    \begin{multline*}
        \{\text{morphisms of commutative ring spectra} \;  \MSp \to \sE \; \text{in} \; \; \SH(k)\} \\ \xrightarrow{\sim} \{\text{$\Sp$-orientations on the ring spectrum} \; \sE\}.
    \end{multline*}
\end{thm}

\begin{rmk}
\label{rmk:MSp-MGLThomClasses}
    The inclusions $\HGr(r,n) \hookrightarrow \Gr(2r,2n)$ induce maps $\BSp_{2r} \to \BGL_{2r}$, and, since $E_{2r}^\Sp$ is the pullback of $E_{2r}$, we also get maps of Thom spaces $\MGL_{2r} \to \MSp_{2r}$ in $\sH_{\bullet}(k)$. These maps in turn define a natural map $\Phi:\MSp \to \MGL$ in $\SH(k)$. By Theorem \ref{thm:universalitySp}, this induces an $\Sp$-orientation on $\MGL$ with symplectic Thom classes $\th_\Sp^\MGL(V,\omega)=\Sigma^{4r,2r} \Phi \circ \th_\Sp^\MSp(V,\omega)$, with $2r= \rnk(V)$. By construction of the Thom classes $\th^\MGL(V)$ and $\th_\Sp^\MSp(V,\omega)$ in \eqref{eq:MGLThomclasses} and \eqref{eq:MSpThomclass} respectively, we see that
    $$\th^\MGL(V) = \th_\Sp^\MGL(V,\omega): p_{X \#}\Sigma^V 1_X \to \Sigma^{4r,2r}\MGL$$
    as maps in $\SH(k)$. In other words, the $\Sp$-orientation on $\MGL$ given by $\Phi$ is the $\Sp$-orientation induced by the natural $\GL$-orientation of $\MGL$.
\end{rmk}

For $\sE$ an $\Sp$-oriented commutative ring spectrum in $\SH(k)$, Panin and Walter computed the $\sE^{*,*}$-cohomology of $\BSp_{2r}$ and $\MSp_{2r}$ (see \cite[Section 9]{Panwal-cobordism}) as follows.

\begin{prop} \label{prop:CohBSpMSp} Let $E^\Sp_{2r}\to \BSp_{2r}$ be the tautological rank $2r$ symplectic bundle as in Construction \ref{constr:MSp}, with Thom space $\MSp_{2r}$, and let  $\sE$ be a $\Sp$-oriented commutative ring spectrum in $\SH(k)$.\\[5pt]
1. $\sE^{*,*}(\BSp_{2r})$ is the ring of homogeneous elements in the power series ring $\sE^{*,*}(\Spec k)[[b_1,\ldots, b_r]]_h$, where $b_i:=b_i(E^\Sp_{2r})\in \sE^{4i,2i}(\BSp_{2r})$ is the $i$-th Borel class of $E^\Sp_{2r}$.\\[2pt]

2. Let $z_{2r}:\BSp_{2r}\to \MSp_{2r}$ be the map induced by the zero-section of $E^\Sp_{2r}$. Then $z_{2r}^*:\sE^{*,*}(\MSp_{2r})\to \sE^{*,*}(\BSp_{2r})$ is injective, with image the ideal generated by $b_r$.
\end{prop}

The first statement in Proposition \ref{prop:CohBSpMSp} is \cite[Theorem 9.1]{Panwal-cobordism}, and the second is \cite[Theorem 9.3]{Panwal-cobordism}. The classes $b_i$ of the first statement are sometimes called the \emph{universal Borel classes}.

From this, similarly to how we have shown in Proposition \ref{prop:MotCohOfMGL} for the motivic cohomology of $\MGL$, they obtain a computation of the $\sE^{*,*}$-cohomology of $\MSp$, which we now recall. 

\begin{thm}[\cite{Panwal-cobordism}, Theorem 13.1]
\label{thm:cohomologyofmsp}
    Let $\sE \in \SH(k)$ any motivic commutative ring spectrum with an $\Sp$-orientation. Then we have an isomorphism of bigraded rings:
    $$\sE^{*,*}(\MSp)\simeq \sE^{*,*}(\Spec k)[[b_1,b_2,\ldots]]_h$$
    with the variables $b_i$ in bidegree $(4i,2i)$.
\end{thm}

\begin{rmk}
\label{rmk:homogeneous}
If there is a $k_0$ such that $\sE^{a,b}(\Spec k)=0$ for $a<-4k_0$, $b<-2k_0$, then $\sE^{*,*}(\Spec k)[[b_1,\ldots, b_r]]_h$ is just the polynomial ring $\sE^{*,*}(\Spec k)[b_1,\ldots, b_r]$. For instance, this is the case for $\sE=H\Z$ or $\sE=H\Z/\ell$.
\end{rmk}

This concludes our discussion on symplectic cobordism.

We do not have as nice description of the coefficient ring of $\MSp$ as we had for $\MGL$ through Theorem \ref{thm:comparisonLazard}. The main goal of this paper is a partial description of the graded ring $\MSp^{4*,2*}$. 

\section{Cellularity of $\MSp$}

\subsection{Cellular Spectra}
\begin{defn}
\label{defn:motivicCellular}
    Let $E \in \SH(k)$. We denote by $\langle E \rangle$ the full localizing subcategory of $\SH(k)$ generated by the suspensions $\Sigma^{a,b}E$ for $a,b \in \Z$, or in other words, the smallest full subcategory containing $\mathbb{S}_k$ and closed with respect to coproducts $\oplus$, the operation $E \wedge (-)$, taking homotopy colimits, and suspensions $\Sigma^{a,b}$ for $a,b \in \Z$. We will say that a motivic spectrum $\E \in \SH(k)$ is \emph{$E$-cellular} if it belongs to the subcategory $\langle E \rangle$. As a special case, we will simply say that $\E$ is \emph{cellular} if it is $\mathbb{S}_k$-cellular.
\end{defn}

\begin{prop}
\label{prop:s.e.s.-cellularity}
    Let 
    $$\E' \to \E \to \E''$$
    be a cofiber sequence in $\SH(k)$ such that two of $\E'$, $\E$ and $\E''$ are $E$-cellular. Then so is the third.
\end{prop}

\begin{proof}
    First, let us note that $\E''$ is the homotopy cofiber of $\E' \to \E$. Thus, if $\E'$ and $\E$ are cellular, so is $\E''$ since $\langle E \rangle$ is closed with respect to homotopy colimits, and in particular homotopy cofibers.

    For the other two cases, we can simply observe that $\E'$ is the homotopy cofiber of $\Sigma^{-1,0}\E \to \Sigma^{-1,0}\E''$, and $\E$ is the homotopy cofiber of $\Sigma^{-1,0}\E'' \to \E'$, and the $\Sigma^{-1,0}$-suspension of a cellular spectrum is cellular by definition. In both cases, the result follows because $\langle E \rangle$ is closed by homotopy cofibers.
\end{proof}

It is worth recalling the classical notion of cellularity for schemes.

\begin{defn}
\label{defn:CellularSchemes}
    A smooth scheme $X \in \Sm/k$ is \emph{cellular} if it admits a a filtration by closed subsets 
$$X =F^0X \supseteq F^1X \supseteq \ldots \supseteq F^dX \supseteq F^{d+1}X = \0,$$
with $d=\dim_kX$, such that, for all $i=0,\ldots, d$, $F^iX \setminus F^{i+1}X$ is either $\emptyset$ or a disjoint union of a finite number of copies of $\A^{d-i}$.
\end{defn}

The motivic notion of cellularity from Definition \ref{defn:motivicCellular} covers the classical notion of cellularity for schemes, in the following sense.

\begin{lemma}
\label{lemma:cellularSchemes}
    If $X \in \Sm/k$ is a cellular scheme, $\Sigma_{\P^1}^\infty X_+ \in \SH(k)$ is $\mathbb{S}_k$-cellular.
\end{lemma}

\begin{proof} Let $i_0(X)$ be the maximum index $i$ such that $F^iX$ is not empty; we proceed by induction on $i_0(X)$. We may also assume that $X$ is irreducible and is not the empty scheme. Let $d \coloneqq \dim_kX\ge0$.

If $i_0(X)=0$, then $X=F^0X=F^0X \setminus F^1X$ is a disjoint union of a finite number $n$ of copies of $\A^d_k$. Thus,
$$\Sigma_{\P^1}^\infty X_+ = \Sigma_{\P^1}^\infty (\A^d _{k+})^{\vee n} \simeq\Sigma_{\P^1}^\infty (\Spec k _+)^{\vee n} \simeq \oplus_{i=1}^n \mathbb{S}_k$$
is $\mathbb{S}_k$-cellular.

Suppose now that $i_0:=i_0(X)>0$. Then $F^{i_0}X=F^{i_0}X\setminus F^{i_0+1}X$ is the disjoint union of a finite number $n$ of copies of $\A^{d-i_0}_k$, and as above, we find that $\Sigma_{\P^1}^\infty F^{i_0}X_+\simeq \oplus_{i=1}^n \mathbb{S}_k$ is $\mathbb{S}_k$-cellular. Let $j:U\to X$ be the open subscheme $X\setminus F^{i_0}X$, with closed complement $i:Z:=F^{i_0}X\to X$. Let $F^iU:= F^iX\setminus Z$. This defines a filtration
$$U=F^0U\supset F^1U\supset\ldots\supset F^{d+1}U=\0$$
which exhibits $U$ as a cellular scheme in $\Sm/k$, with $i_0(U)<i_0(X)$. By induction, $\Sigma_{\P^1}^\infty U_+ \in \SH(k)$ is $\mathbb{S}_k$-cellular.

On the other hand, by Proposition \ref{prop:localization}, we have a distinguished triangle
    $$j_!j^!1_X \to 1_X \to i_*i^* 1_X \to j_!j^!1_X[1]$$
    in $\SH(X)$. Let $p:X \to \Spec k$ be the structure map. By applying $p_\#$ to the distinguished triangle above, we get the distinguished triangle
    \begin{equation}
    \label{eq:cellularschemes}
        p_\# j_! j^! 1_X \to \Sigma_{\P^1}^\infty X_+ \to p_\#i_*i^*1_X \to  p_\# j_! j^! 1_X[1],
    \end{equation}
      Since a distinguished triangle is in particular a cofiber sequence, by Proposition \ref{prop:s.e.s.-cellularity}, it is enough to prove that the left hand term and the right hand term of \eqref{eq:cellularschemes} are $\mathbb{S}_k$-cellular.

    Let us look at the left hand term $p_\# j_! j^! 1_X$. Since $j$ is an open immersion, we have $j^! \simeq j^*$ and $j_! \simeq j_\#$ by Remark \ref{rmk:etalemaps}. Then $p_\# j_! j^! 1_X \simeq p_{U\#}1_U \simeq \Sigma_{\P^1}^\infty U_+$, where $p_U:U \to \Spec k$ is the structure map, and we have already shown that $\Sigma^\infty_{\P^1}U_+$ is $\mathbb{S}_k$-cellular.

    Let us now look at the right-hand term $p_\#i_*i^*1_X \simeq p_\#i_*1_Z$. We note that $Z$ is a disjoint union of affine spaces $\A^{d-i_0}_k$, is in $\Sm/k$. By Theorem \ref{thm:MorVoePurity}, $p_\#i_*1_Z \simeq p_{Z\#}\Sigma^{N_i}1_Z$, where $p_Z:Z \to \Spec k$ is the structure map, and $N_i$ is the normal bundle of $i$. Now, by $\A^1$-homotopy invariance of $K_0$ \cite[Corollary to Theorem 8, p. 122]{Quillen:K}, we have $K_0(\A_k^n)\simeq K_0(\Spec k)$ for all $n\ge 0$, and $K_0(\Spec k)$ is isomorphic to $\Z$ through the rank homomorphism. Then on each component $\A_k^{d-i_0}$ of $Z$, we have $[N_i\mid_{\A_k^{d-i_0}}]=i_0[\sO_{\A_k^{d-i_0}}]$ in $K_0(\A_k^{d-i_0})\simeq \Z$. Thus, $[N_i]=i_0[\sO_Z]$ in $K_0(Z)$. Thus, since the isomorphism class of 
    $p_{Z\#}\Sigma^{N_i}1_Z$ in $\SH(k)$ only depends on the $K_0$-class of $N_i$, we get
    $$p_{Z\#}\Sigma^{N_i}1_Z \simeq p_{Z\#}\Sigma^{2i_0, i_0}1_Z \simeq \Sigma^{2i_0, i_0}\Sigma^\infty_{\P^1}Z_+\simeq  \oplus_{i=1}^n\Sigma^{2i_0,i_0} \mathbb{S}_k.$$
    Therefore, $p_{Z\#}\Sigma^{N_i}1_Z$ is $\mathbb{S}_k$-cellular. 
\end{proof}

We conclude our discussion of cellular spectra by giving a proof of the following theorem, which is \cite[Theorem 6.4]{dugisa:cellstructures}.

\begin{thm}
\label{thm:mglcellular}
    The algebraic cobordism spectrum $\MGL \in \SH(k)$ is cellular.
\end{thm}
We need a preliminary lemma.

\begin{lemma}\label{lemma:cellularSchemes2} Let $X\in \Sm/k$ be a cellular scheme, let $V\xrightarrow{\pi} X$ be a rank $r$ vector bundle over $X$, and let $V^0\hookrightarrow V$ be the open complement of the zero-section, namely $V^0:=V\setminus\{s_0(X)\}$. Then $\Sigma^\infty_{\P^1}V_+$ and $\Sigma^\infty_{\P^1}V^0_+$ are $\mathbb{S}_k$-cellular.
\end{lemma}

\begin{proof} For $\Sigma^\infty_{\P^1}V_+$, we have $\Sigma^\infty_{\P^1}V_+\simeq \Sigma^\infty_{\P^1}X_+$ by homotopy invariance, and $\Sigma^\infty_{\P^1}X_+$ is $\mathbb{S}_k$-cellular by Lemma~\ref{lemma:cellularSchemes}.

For $\Sigma^\infty_{\P^1}V^0_+$, we retain the notation of the proof of Lemma \ref{lemma:cellularSchemes} and use a similar induction on $i_0(X)$.

If $i_0(X)=0$, $X=F^0X=F^0X \setminus F^1X$ is a disjoint union of a finite number $n$ of copies of $\A^d_k$, with $d=\dim_kX$. By reasoning as in the last part of the proof of Lemma \ref{lemma:cellularSchemes}, with $V$ instead of $N_i$, we find that $[V]=r[\sO_{\A_k^d}]$ in $K_0(X)$, and 
$$\Sigma_{\P^1}^\infty\Th(V)\simeq p_{X\#}\Sigma^V1_X \simeq \oplus_{i=1}^n\Sigma^{2r,r}\mathbb{S}_k$$
is $\mathbb{S}_k$-cellular. Also, $\Sigma_{\P^1}^\infty V_+^0$ fits in the distinguished triangle
$$\Sigma_{\P^1}^\infty V_+^0 \to \Sigma_{\P^1}^\infty V_+ \to \Sigma^\infty_{\P^1}\Th(V) \to \Sigma_{\P^1}^\infty V_+^0[1]$$
in $\SH(k)$ (see triangle \eqref{eq:thom-loc3}). Thus, $\Sigma_{\P^1}^\infty V_+^0$ is $\mathbb{S}_k$-cellular by Proposition \eqref{prop:s.e.s.-cellularity}.

Let now $i_0:=i_0(X)>0$, and let us assume that the statement is true for any $X'$ with $i_0(X')<i_0$. Following the proof of Lemma~\ref{lemma:cellularSchemes}, we have the closed immersion $i:Z:=F^{i_0(X)}\hookrightarrow X$ with open complement $j:U:=X\setminus Z\hookrightarrow X$, with $i_0(U)<i_0(X)$. $V^0$ maps to $X$ through $\pi^0\coloneqq \pi\mid_{V^0}$. This gives us the open subscheme $\tilde{j}:j^*V^0\to V^0$ of $V^0$ with closed complement $\tilde{i}:i^*V^0\to V^0$. We have the cartesian square
$$
\begin{tikzcd}
    i^*V \arrow[r, hook, "\tilde{i}"] \arrow[d, swap, "\pi^0_Z"] & V^0 \arrow[d, "\pi^0"] \\
    Z \arrow[r, hook, "i"] & X
\end{tikzcd}
$$
As in the proof of Lemma~\ref{lemma:cellularSchemes}, the $K_0$-class of the normal bundle $N_i$ of $i$ in $K_0(Z)$ is the $K_0$-class of the trivial rank $i_0$ vector bundle on $Z$, which implies that the $K_0$-class of the normal bundle $N_{\tilde{i}}$ of $\tilde{i}$ in $K_0(i^*V^0)$ is the $K_0$-class of the trivial rank $i_0$ vector bundle on $i^*V^0$, since $N_{\tilde{i}}$ is the pullback $(\pi^0_Z)^*N_i$. This gives us the distinguished triangle
\[
\Sigma^\infty_{\P^1}(j^*V^0)_+\to \Sigma^\infty_{\P^1}V^0_+\to \Sigma^{2i_0, i_0}\Sigma^\infty_{\P^1}(i^*V^0)_+\to  \Sigma^\infty_{\P^1}(j^*V^0)_+[1]
\]
 in $\SH(k)$ by Proposition \ref{prop:localization}. Because of Proposition \ref{prop:s.e.s.-cellularity}, it is enough to prove that $\Sigma^\infty_{\P^1}(j^*V^0)_+$ and $\Sigma^{2i_0, i_0}\Sigma^\infty_{\P^1}(i^*V^0)_+$ are $\mathbb{S}_k$-cellular. 

$\Sigma^\infty_{\P^1}(j^*V^0)_+$ is $\mathbb{S}_k$-cellular by the induction hypothesis.

Using once again the $\A^1$-homotopy invariance of $K_0$ and the fact that $Z$ is a disjoint union of a finite number $m$ of copies of $\A^{d-i_0}$, we find that the $K_0$-class of $i^*V$ in $K_0(Z)$ is the $K_0$-class of the trivial rank $r$ vector bundle on $Z$. Thus, $\Sigma^\infty_{\P^1}(i^*V)_+\cong \Sigma^\infty_{\P^1}Z_+$ and $\Sigma^{i^*V}1_Z\cong \Sigma^{2r, r}1_Z$, so $\Th(i^*V)\cong \Sigma^{2r,r}\Sigma^\infty_{\P^1}Z_+$. Using again the distinguished triangle \eqref{eq:thom-loc3} we see that $\Sigma^{2i_0, i_0}\Sigma^\infty_{\P^1}(i^*V^0)_+$ is  $\mathbb{S}_k$-cellular. This concludes the proof.
\end{proof}

We can now prove the cellularity of $\MGL$.

\begin{proof}[Proof of Theorem \ref{thm:mglcellular}]
    By Remark \ref{rmk:MGLcolim}, $\MGL$ is isomorphic to the homotopy colimit of the system
    $$\Sigma^\infty_{\P^1}\MGL_0 \to \Sigma^{-2,-1}\Sigma^\infty_{\P^1}\MGL_1 \to \Sigma^{-4,-2}\Sigma^\infty_{\P^1}\MGL_2 \to \ldots \; .$$ 
    Thus, we need to prove that $\Sigma_{\P^1}^\infty \MGL_r$ is cellular for all $r$. $\MGL_r=\Th(E_r)$ is in turn the colimit $\colim_n\Th(E_{r,n})$, with $E_{r,n}$ the tautological rank $r$ vector bundle over $\Gr(r,n)$. Thus, we have $\Sigma_{\P^1}^\infty \MGL_r \simeq \colim_n \Sigma_{\P^1}^\infty \Th(E_{r,n})$, and such colimit, as a filtered colimit, is a homotopy colimit. We then need to prove that $\Sigma_{\P^1}^\infty \Th(E_{r,n})$ is cellular. 
    
    $\Th(E_{r,n})$ fits in the cofiber sequence
    $$\Sigma_{\P^1}^\infty E_{r,n \; +}^0 \to \Sigma_{\P^1}^\infty E_{r,n \; +} \to \Sigma_{\P^1}^\infty \Th(E_{r,n}),$$
    where $E_{r,n}^0$ is the open complement of the zero section. Thus, by Proposition \ref{prop:s.e.s.-cellularity}, it is enough to prove that the spectra $\Sigma_{\P^1}^\infty E_{r,n \; +}^0$ and $\Sigma_{\P^1}^\infty E_{r,n \; +}$ are cellular. 

    The classical Schubert cell decomposition for Grassmannians through Schubert varieties gives the Grassmannian $\Gr(r,n)$ the structure of a cellular scheme. The decomposition is described in \cite[Chapter 1, Section 5]{Griffith:AlgGeom} over $k=\C$, but the description works over any field $k$, and it is straightforward to see that it satisfies the condition of Definition \ref{defn:CellularSchemes}. Therefore, $\Sigma_{\P^1}^\infty E_{r,n \; +}^0$ and $\Sigma_{\P^1}^\infty E_{r,n \; +}$ are cellular spectra by Lemma \ref{lemma:cellularSchemes2}.
\end{proof}

The proof of Theorem \ref{thm:mglcellular} does not apply directly to the case of the symplectic cobordism spectrum, due to the different geometry of quaternionic Grassmannians. The cellular structure on $\MGL_r$ is induced by the Schubert filtration on the ordinary Grassmannians $\Gr(r,n)$, but the analogous filtration on quaternionic Grassmannians does not work as well. In the following subsections we study the geometry of quaternionic Grassmannians. We start by studying the special case of quaternionic projective spaces.

\subsection{Quaternionic projective spaces}

In Construction \ref{constr:MSp}, we defined the quaternionic Grassmannians $\HGr(r,n)$. Now, as in \cite{panwal:grass}, we define the \emph{quaternionic projective space} $\HP^n$ as the quaternionic Grassmannian $\HP^n \coloneqq \HGr(1,n+1)$. We start our discussion on cellularity of quaternionic Grassmannians by the special case of quaternionic projective spaces.

Let $V$ be a vector space of dimension $2n+2$ over $k$, and $\phi$ the standard symplectic form
$$\phi = \phi_{2n+2} =
\begin{pmatrix}
     & & &  &  & 1\\
     & &  &  &\ldots & \\
     & & & 1& &\\
     & & -1 &  & &  \\
     & \ldots &  & & & \\
     -1 & & & & &
\end{pmatrix}$$
over $V$ with respect to a fixed basis $\langle x_1, \ldots x_{n+1}, y_{n+1}, \ldots y_1 \rangle$. Over $\Spec k$, $\HP^n$ can be seen as the classifying space of the $2$-dimensional vector subspaces of $V$ that are symplectic with respect to $\phi$. 

If we let $E_i$ denote the subspace of $V$ generated by $x_1, \ldots, x_i$, and $E_i^{\perp}$ its perpendicular subspace with respect to $\phi$, we can define closed subschemes $\overline{X}_{2i}\coloneqq \Gr(2,E_i^{\perp}) \cap \HP^n$. Then $\HP^n$ has a filtration by closed subschemes
$$ \emptyset = \overline{X}_{2n+2} \subset \overline{X}_{2n} \subset \ldots \overline{X}_0=\HP^n.$$

For $i=0,\ldots, n$, let $X_{2i}$ be the open subset $\overline{X}_{2i}\setminus \overline{X}_{2i +2}$ of $\overline{X}_{2i}$. We have $\overline{X}_{2n}=X_{2n} \simeq \mathbb{A}^{2n}$ (see \cite[Corollary 3.3]{panwal:grass}). We want to show, following the ideas of \cite{panwal:grass}, that these open strata are all $\mathbb{A}^1$-contractible.

\begin{prop}
\label{prop:contractibleopencellofHPn}
    For $i=0, \ldots, n$, the locally closed stratum $X_{2i}$ in $\HP^n$ is smooth of dimension $4n-2i$ over $k$, and is $\mathbb{A}^1$-contractible as an objects in $\sH(k)$.  
\end{prop}

\begin{proof}
    Let us fix $n$ and $i$. Since $X_{2n}=\mathbb{A}^{2n}$, we can suppose $i<n$. By construction, $X_{2i}$ is the subspace of $\HP^n$ classifying the $2$-dimensional subspaces $\Pi$ of $V=\langle x_1, \ldots x_{n+1},y_{n+1}, \ldots y_1 \rangle$ symplectic with respect to $\phi$ that are also subspaces of $\langle x_1, \ldots x_{n+1},y_{n+1}, \ldots y_{i+1} \rangle$ but are not subspaces of $\langle x_1, \ldots x_{n+1}, y_{n+1}, \ldots, y_{i+2}\rangle$. Therefore, a point in $X_{2i}$ is defined (not uniquely) by two linearly independent vectors $e=(e_1, \ldots, e_{2n+2-i})$ and $f=(f_1, \ldots, f_{2n+2-i})$ in $V$ such that $e_{2n+2-i}$ and $f_{2n+2-i}$ are not both zero and $\phi(e,f)\neq 0$.

    Consider now the restriction of the rank $2$ universal bundle $\pi_{\mathcal{U}}: \mathcal{U} \to X_{2i}$, with $\mathcal{U}\coloneqq E_{2,2n}^\Sp \mid_{X_{2i}}$. A point in the total space of $\mathcal{U}$ over $X_{2i}$ is then a pair $u=(\Pi,v)$, with $\Pi \in X_{2i}$ identifying a certain two-dimensional vector space, and $v=(v_1,\ldots, v_{2n+2-i})$ a vector of $\Pi$. Let 
\begin{align*}
    \lambda : \mathcal{U} & \to \mathcal{O}_S \\
    u & \mapsto v_{2n+2-i}
\end{align*}
be the map taking the coefficient of $y_{i+1}$ of the vector $v$. Let $Y \coloneqq \lambda^{-1}(1)$, and consider the induced map $\pi :=\pi_Y: Y \to X_{2i}$. We want to use this map to define a homotopy equivalence between $X_{2i}$ and a contractible space.

First, we need to show that $\pi: Y \to X_{2i}$ is an affine space bundle, with fibre $\A^1$.

We note that the map $\lambda$ is surjective. In fact, for any $0\neq t \in \A^1$, we can take $e \in V$ as the vector with $e_{2n+2-i}=T$ and $e_j=0$ for $j \neq 2n+2-i$, and $f \in V$ as any vector with $f_{i+1} \neq 0$ and $f_j=0$ for all $j \neq i+1$. In this way, we have that $\lambda(\langle e,f \rangle,e)=t$, and $\lambda (\langle e,f \rangle,f)=0$. Thus, $\lambda$ is surjective. Therefore, $\Ker(\lambda)$ is a vector subbundle of $\mathcal{U}$ of rank 1, and thus $\lambda^{-1}(1)=Y \to X_{2i}$ is a torsor for the line bundle $\Ker(\lambda)$. In particular, $\pi:Y \to X_{2i}$ is a Zariski-locally trivial affine space bundle with fibre $\A^1$.

We now want to find an isomorphism between $Y$ and an $\A^1$-contractible scheme.

Let us define $Y' \coloneqq \{(e,f) \in \mathbb{A}^{2n+2-i} \times \mathbb{A}^{2n+2-i} \mid \lambda(e)=0, \; \lambda(f)=1, \; \phi(e,f)=1\}$, where $\lambda$ is the projection on the $(2n+2-i)$-th component, and let us consider the map $\beta: Y' \to Y$ given by $\beta((e,f)):=(\langle e,f \rangle,f).$ We now show that $\beta$ is an isomorphism.


For any $\Pi \in X_{2i}$ and $f$ a vector of $\Pi$ with $f_{2n+2-i}=1$, there is a unique vector $e(\Pi,f)$ in $\Pi$ such that $(e(\Pi,f))_{2n+2-i}=0$ and $\phi(e(\Pi,f),f)=1$. Indeed, $e(\Pi, f)$ can be obtained from any generator $e$ of the one-dimensional vector subspace $\Pi \cap \{y_{i+1}=0\}$ of $\Pi$, noting that $\phi(e,f) \neq 0$ because $\langle e,f \rangle =\Pi$, and then replacing $e$ by $(1/\phi(e,f))\cdot e$. Arguing as for $Y$, we see that sending $(e,f)\in Y'$ to $\langle e,f \rangle\in X_{2i}$ also makes $Y'\to X_{2i}$ a Zariski-locally trivial affine space bundle with fibre $\A^1$. We can show that the map $\beta^{-1}:Y \to Y'$ defined by $\beta^{-1}(\Pi, f):=(e(\Pi,f),f)$ is the inverse of $\beta$. We have
$$\beta(\beta^{-1}(\Pi,f))=\beta(e(\Pi,f),f)=(\langle e(\Pi,f),f \rangle, f),$$
but $\langle e(\Pi,f),f \rangle=\Pi$ because $e(\Pi,f)$ and $f$ belong to $\Pi$ and are linearly independent. On the other hand
$$\beta^{-1}(\beta(e,f))=\beta^{-1}(\langle e,f \rangle, f)=(e(\langle e,f \rangle,f),f),$$
but $e(\langle e,f \rangle,f)=e$ because $e$ is the only vector of $\langle e,f \rangle$ such that $e_{2n+2-1}=0$ and $\phi(e,f)=1$. Since $\beta$ and $\beta^{-1}$ are affine linear maps on the fibers of $Y$ and $Y'$ over $X_{2i}$, we conclude that $\beta$ is an isomorphism.

Finally, we observe that $Y'$ is isomorphic to $\A^{4n+1-2i}$ through 
$$\varphi:(e,f) \mapsto (e_1,\ldots, e_i, e_{i+1}, \ldots, e_{2n+1-i}, f_1, \ldots , f_{2n+1-i}).$$
In fact, once fixed these $4n+1-2i$ components for the vectors $(e,f)$, the remaining $3$ components are determined by definition of $Y'$. In particular, $e_{2n+2-1}=0$, $f_{2n+2-i}=1$, and $e_i$ is determined by the condition $\phi(e,f)=1$.

We have then proved that there exists a map $\A^{4n+1-2i} \to X_{2i}$ which is a Zariski-locally trivial $\A^1$-bundle. Thus, $X_{2i}$ is smooth of dimension $4n-2i$ over $k$, and is $\A^1$-contractible in $\sH(k)$.
\end{proof}

We can now prove the cellularity result for $\HP^n$.

\begin{prop}
\label{prop:vectorbundlesonHPn}
Let $p:\HP^n\to \Spec k$ be the structure map. Then for $v\in K_0(\HP^n)$, $p_{\#}\Sigma^v1_{\HP^n}$ belongs to $\langle \mathbb{S}_k \rangle$.
\end{prop}

\begin{proof}
We can prove the statement for all the closed strata $\overline{X}_{2n-2i}$ by induction on $i$. Since $\HP^n=\overline{X}_0$, the case $i=n$ will give us the result.

For $i=0$, we have $X \coloneqq \overline{X}_{2n}\simeq \mathbb{A}^{2n}$. Let $p :=p_X$. By the already mentioned $\A^1$-invariance of $K_0$ \cite[Corollary to Theorem 8]{Quillen:K}, the map $\rnk:K_0(\A^{2n})\to \Z$ is a ring isomorphism. Thus, for $v \in K_0(\A^{2n})$ of virtual rank $r$, we have $p_\#\Sigma^v1_X \simeq \Sigma^{2r,r}p_\#1_X \simeq \Sigma^{2r,r}\Sigma_{\P^1}^\infty \A^{2n}_+$, and since $\A^{2n}$ is a $\A^1$-contractible, this belongs to $\langle \mathbb{S}_k \rangle$.

For $i>0$, we have $X \coloneqq \overline{X}_{2n-2i}=X_{2n-2i}\sqcup X_{2n-2i+2} \sqcup \ldots \sqcup X_{2n}$. In particular, $X$ is the union of the closed stratum $Z \coloneqq \overline{X}_{2n-2i+2}$ and the open cell $U \coloneqq X_{2n-2i}$, which is $\A^1$-contractible by Proposition \ref{prop:contractibleopencellofHPn}.

Let $i:Z \hookrightarrow X$ and $j:U \hookrightarrow X$ the two immersions, with $p_Z:Z \to \Spec k$ and $p_U:U \to \Spec k$ the structure maps. Let us take again $v \in K_0(X)$ of virtual rank $r$. By Proposition \ref{prop:localization}, we have the distinguished triangle
$$p_{\#}j_!j^!\Sigma^v1_X \to p_{\#}\Sigma^v1_X \to p_{\#}i_*i^*\Sigma^v1_X \to p_{\#}j_!j^!\Sigma^v1_X[1]$$
in $\SH(k)$. By proposition \ref{prop:s.e.s.-cellularity}, in order to get $p_{\#}\Sigma^v1_X \in \langle \mathbb{S}_k \rangle$ it is sufficient to show that the left-hand term and the right-hand term of the above sequence belong to $\langle \mathbb{S}_k \rangle$.

 Let us see the left-hand term $p_{\#}j_!j^!\Sigma^v 1_X$. Since $j$ is an open immersion, we have $j_!=j_{\#}$ and $j^! \simeq j^*$ by Remark \ref{rmk:smoothsharp}, and since $U$ is smooth over the base, we have
 $p_{\#}j_!j^!\Sigma^v1_X\simeq p_{U\#}j^*\Sigma^v1_X.$ Now, since $U$ is contractible and algebraic $K$-theory is representable in $\SH(k)$ (see Example \ref{exmp:KGL}), we have again that the map
 $\rnk:K_0(U) \to \Z$ is an isomorphism of abelian groups, and $j^*\Sigma^v1_X=\Sigma^{j^*v}1_U \simeq \Sigma^{2r,r}1_U$. Thus,
 $$p_{U\#}j^*\Sigma^v1_X \simeq p_{U\#}\Sigma^{2r,r}1_U \simeq \Sigma^{2r,r}\Sigma^\infty_{\P^1}\Spec k_+$$
 belongs to $\langle \mathbb{S}_k \rangle$.

 For the right-hand term, we recall that $\overline{X}_{2n-2i+2}=\Gr(2, E_{n-i+1}^\perp)\cap \HP^n$, and as $\HP^n$ is an open subscheme of $\Gr(2,2n+2)$, $Z=\overline{X}_{2n-2i+2}$ is an open subscheme of the smooth $k$-scheme $\Gr(2, E_{n-i+1}^\perp)$, and hence is smooth. Thus, by Theorem \ref{thm:MorVoePurity} we have:
 $$p_{\#}i_*i^*\Sigma^v1_X \simeq p_{Z\#}\Sigma^{N_i}i^*\Sigma^v1_X \simeq p_{Z\#}\Sigma^{N_i+i^*v}1_Z.$$
We can now use our induction hypothesis and we get that $p_{Z\#}\Sigma^{N_i + i^*v}1_Z$ belongs to $\langle \mathbb{S}_k \rangle$.

 Therefore, we conclude that $p_{\#}\Sigma^v1_{\HP^n}$ belongs to $\langle \mathbb{S}_k \rangle$.
\end{proof}

\subsection{Quaternionic Grassmannians}

We now want to generalize the previous picture to quaternionic Grassmannians. We remark that a similar procedure was used in \cite[Proposition 3.1]{Rond:Cellularity}.

 Let $V$ be a $(2n)$-dimensional vector space over $k$, and $\phi$ the symplectic form
  $$\phi = \phi_{2n}=
\begin{pmatrix}
     & & &  &  & 1\\
     & &  &  &\ldots & \\
     & & & 1& &\\
     & & -1 &  & &  \\
     & \ldots &  & & & \\
     -1 & & & & &
\end{pmatrix}$$
with respect to a fixed basis $\langle x_1, \ldots, x_n,y_n, \ldots, y_1 \rangle$.
Over $\Spec k$, $\HGr(r,n)$ can be seen as the classifying space of the $(2r)$-dimensional symplectic vector subspaces of the symplectic vector space $(V,\phi)$. 

We still denote by $E_i$ the $i$-dimensional subspace $\langle x_1, \ldots , x_i \rangle$ of $V$. Now, following \cite[Theorem 4.1]{panwal:grass}, we define $N^+\coloneqq \HGr(r,n)\cap \Gr(2r,E_1^{\perp})$. $N^+$ is a closed subscheme of of $\HGr(r,n)$, and it is a rank $2r$ subbundle of the normal bundle $N_i$ of the obvious inclusion $i:\HGr(r,n-1) \to \HGr(r,n)$. More specifically, \cite[Theorem 4.1 (b)]{panwal:grass} gives $N_i=N^+\oplus N^-$, where $N^-=\HGr(r,n)\cap \Gr(2r,y_1^{\perp})$. Moreover, by \cite[Theorem 4.1 (c)]{panwal:grass}, both $N^+$ and $N^-$ are isomorphic to the tautological rank $2r$ symplectic bundle over $\HGr(r,n-1)$. In particular
$$\dim(N^+)=\dim(\HGr(r,n-1))+2r=4nr-2r-4r^2.$$

\subsubsection{The geometry of the open stratum}
\label{sub:opencells}

In this subsection, we generalize Proposition \ref{prop:contractibleopencellofHPn}.

Let us denote by $Y$ the open complement of the closed subscheme $N^+$ in $\HGr(r,n)$. $Y$ can then be seen as the classifying space of $2r$-dimensional subspaces of $V$, symplectic with respect to the form $\phi$, that are not subspaces of $\langle x_1, \ldots, x_n,y_n, \ldots y_2 \rangle$.  

Let us consider the restriction of the tautological rank $2r$ bundle $\mathcal{U} \to Y$, with $\mathcal{U}\coloneqq E^\Sp_{2r,2n} \mid_Y$. Note that a point in $\mathcal{U}$ is now a pair $(y,v)$ where $y$ is a point of $Y$ and $v$ an element of the vector space $[y]$. Unlike before, two linearly independent vectors in $[y]$ do not specify $[y]$. Let $\lambda:\mathcal{U} \to \mathcal{O}_S$ be the map that takes $(y,(v_1,\ldots, v_{2n})\in \mathcal{U}$ to $v_{2n}$, and let $Y_1 \coloneqq \lambda^{-1}(1)$. Let $g_1:Y_1 \to Y$ be the induced map.

Let now
$$Y_2 \coloneqq \{(y,e,f) \mid y \in Y, \;e \in [y],\; f \in [y], \; e_{2n}=0, \; f_{2n}=1, \; \phi(e,f)=1 \},$$
and let $g_2:Y_2 \to Y_1$ be the map defined by $(y,e,f) \mapsto (y,f)$.

\begin{lemma}
    The maps $g_1$ and $g_2$ just defined are Zariski-locally trivial affine space bundles.
\end{lemma}

\begin{proof}
Since $Y_1=\lambda^{-1}(1)$, the map $g_1$ is the restriction of the projection $\mathcal{U}\to Y$ to a Zariski closed subscheme of $\mathcal{U}. $ $Ker\lambda$ is a subbundle of rank $2r-1$ of $\mathcal{U}$, and fiberwise, it acts freely on $\mathcal{U}$ by translations. Since this action preserves the image of $\lambda$, it induces an action on $Y_1$. Since the action is fiberwise it preserves $Y$, and $Y$ is obtained as the quotient of $Y_1$ by this action. By construction, $\lambda$ is surjective. For example, for $0 \neq t \in \A^1$, on can take $[y]:= \langle x_1, \ldots, x_r,y_r,\ldots, t \cdot y_1 \rangle$, and we have $\lambda(y, t \cdot y_1)=t$, and $\lambda(y,x_1)=0$. Thus, $Y_1 \to Y$ is a torsor for the vector subbundle $\Ker\lambda$, and since $\Ker\lambda \to Y$ is a rank $(2r-1)$ vector bundle, $g_1$  is a Zariski-locally trivial affine space bundle with fibre $\mathbb{A}^{2r-1}$.

In order to obtain the statement for $g_2$, let us consider the rank $2r-1$ vector bundle $g_1^*\Ker\lambda$ over $Y_1$. A point in this bundle will then be identified by a point $(y,f)$ in $Y_1$ and a point $e$ in its fiber, or in other words, will be a triple $(y,e,f)$ with $y$ in $Y$, and $e,f$ two vectors in $[y]$, with $\lambda(f)=1$ and $\lambda(e)=0$. We then get a surjective map of vector bundles 
\begin{align*}
    \lambda':g_1^*\Ker\lambda & \to \mathcal{O}_S\\
    (y,e,f) &\mapsto \psi(e,f),
\end{align*}
and $Y_2=\lambda'^{-1}(1)$. $Y_2$ is then a Zariski closed subscheme of the vector bundle $g_1^{-1}Ker\lambda$ over $Y_1$, and $Ker\lambda'$ is a subbundle of rank $2r-2$ that acts fiberwise on $g_1^{-1}Ker\lambda$ by translations. The induced map $Y_2 \to Y_1$ is the map $(y,e,f) \mapsto (y,f)$, that is, the map $g_2$. By applying the previous argument to this situation, we obtain that $g_2$ is a Zariski locally trivial affine space bundle with fiber $\mathbb{A}^{2r-2}$. 
\end{proof}

\begin{lemma}
\label{lemma:smallerGrassmannian}
    Let $\Tilde{Y}$ be the space
    $$\Tilde{Y} \coloneqq\{(e,f)\in V \times V \mid e_{2n}=0, \; f_{2n}=1, \; \phi (e,f)=1\}.$$
    There is an isomorphism of $k$-schemes $Y_2 \simeq \Tilde{Y}\times \HGr(r-1,n-1)$.
\end{lemma}

\begin{proof}
    Firstly, let us note that the vector space $V$ can be written as $\langle x_1 \rangle \oplus V' \oplus \langle y_1 \rangle$, with $V' = \langle x_2, \ldots, x_n, y_n, \ldots, y_2 \rangle$. $V'$ equipped with the symplectic structure induced by $\phi$ is a symplectic vector subspace of $V$, and we can identify $\HGr(r-1,n-1)$ with $\HGr(r-1,V')$. A point of $\Tilde{Y}\times \HGr(r-1,n-1)$ is then a triple $(e,f,x)$ with $e,f$ vectors of $V$ satisfying the three conditions requested for $\Tilde{Y}$, and $x$ classifying a symplectic subspace $[x]$ of $V'$ of rank $2r-2$. One can associate to the pair of vectors $(e,f)$ two invertible matrices
$$\rho_e=
\begin{pmatrix}
    1 & 0 & 0 & & \ldots  &  & & & 0 \\
    e_2 & 1 & 0 & & & & & & 0 \\
    e_3 & 0 & 1 & 0 & & 0 & & & \ldots \\
    \ldots  & & & 1 & & & & & \ldots \\
    \ldots & & & & 1 & & & & \ldots \\
    \ldots & & 0 & & & 1 & & & \ldots \\
    e_{2n-2} & & & & & & 1 & & 0 \\
    e_{2n-1} & & & & & \ldots & 0 & 1 & 0 \\
    0 & -e_{2n-1} & -e_{2n-2} & \ldots & -e_{n+1} & e_{n} & \ldots & e_{2} & 1
\end{pmatrix},$$

$$\rho_f=
\begin{pmatrix}
    1 & -f_{2n-1} & -f_{2n-2} & \ldots & -f_{n+1} & f_n & \ldots & f_2 & f_1 \\
    0 & 1 & 0 & & & & & & f_2 \\
    0 & 0 & 1 & 0 & & & 0 & & f_3 \\
    \ldots  & & & 1 & & & & & \\
    \ldots & & & & 1 & & & & \ldots \\
    \ldots & & 0 & & & 1 & & & \ldots \\
    \ldots & & & & & & 1 & & \\
    0 & & & & & & 0 & 1 & f_{2n-1} \\
    0 & & & \ldots & & & & 0 & 1
\end{pmatrix}.
$$
Let now $\rho_{e,f}\coloneqq \rho_f \cdot \rho_e$. We get $\rho_{e,f}(x_1)=\rho_f(1, e_2, \ldots, e_{2n-1},0)=e$, and $\rho_{e,f}(y_1)=\rho_f(y_1)=f$. It is immediate to see that $\rho_{e,f}$ preserves the form $\phi$: 
$$\rho_{e,f}^T \cdot \phi \cdot \rho_{e,f}=\rho_e^T\cdot \rho_f^T \cdot \phi \cdot \rho_f \cdot \rho_e = \rho_e^T \cdot \phi \cdot \rho_e = \phi.$$
Thus, for each symplectic subspace $W$ of $V$, $\rho_{e,f}$ maps $W^{\perp}$ isometrically to $(\rho_{e,f}(W))^{\perp}$. In particular, $\rho_{e,f}(V')=\rho_{e,f}(\langle x_1,y_1\rangle^{\perp}) \simeq \langle e,f \rangle^{\perp}$. We can then define a map:
\begin{align*}
    \gamma:\Tilde{Y}\times \HGr(r-1,n-1) & \to Y_2 \\
    ((e,f),x) & \mapsto (\rho_{e,f}([x])\oplus \langle e \rangle \oplus \langle f \rangle, e, f),
\end{align*}
and note that this is an isomorphism with inverse
\begin{align*}
    \gamma^{-1}: \; \;  Y_2 & \to \Tilde{Y}\times \HGr(r-1,n-1)\\
    (y,e,f) & \mapsto ((e,f), \rho_{e,f}^{-1}([y])\cap V').
\end{align*}
\end{proof}

We get a map $q:Y_2 \to \HGr(r-1,n-1)$ by identifying $Y_2$ with $\Tilde{Y}\times \HGr(r-1,n-1)$ thanks to Lemma \ref{lemma:smallerGrassmannian}, and applying the canonical projection. Moreover, as in the end of the proof of Proposition \ref{prop:contractibleopencellofHPn}, there is an isomorphism $\varphi:\Tilde{Y} \xrightarrow{\sim}\A^{4n-3}$ with inverse
\begin{align*}
    \varphi^{-1}:\mathbb{A}^{4n-3} & \to \Tilde{Y}\\
    (x_1,\ldots,x_{4n-3}) &\mapsto (e_1,x_1,\ldots,x_{2n-2},0,x_{2n-1}, x_{2n}, \ldots, x_{4n-3},1),
\end{align*}
where $e_1=1-\phi_{2n-2}((x_1,\ldots,x_{2n-2}),(x_{2n},\ldots,x_{4n-3}))$. The map $q$ is then a trivial $\mathbb{A}^{4n-3}$-bundle, and as such, it is an $\mathbb{A}^1$-equivalence.

We summarize the results of this subsection in the following proposition.
\begin{prop} \label{prop:summary}
Let $Y=\HGr(r,n)\setminus N^+$. Then  we have a diagram
\[
\xymatrix{
 &&Y_2 \ar[dl]_{g_2} \ar[dr]^q& \\
    &Y_1 \ar[dl]_{g_1}&&\HGr(r-1,n-1)\\
    Y &&&
 }
 \]
where the maps $g_1,g_2$ and $q$ are Zariski locally trivial affine space bundles, hence $\mathbb{A}^1$-weak equivalences. Moreover, $q$ is a trivial affine space bundle with fiber $\mathbb{A}^{4n-3}$.
\end{prop}  

\subsubsection{Vector bundles over quaternionic Grassmannians}

We are ready to prove the main result of this section:

\begin{prop}
\label{prop:bundlesonHGr}
    For $v\in K_0(\HGr(r,n))$, $p_{\HGr(r,n)\#}\Sigma^v1_{\HGr(r,n)}\in \SH(k)$ is cellular.
\end{prop}

\begin{proof}
 We can prove this statement for all $n \ge r \ge 0$ by induction on $r$.

For $r=0$ the statement is trivial, and for $r=1$, $\HGr(r,n)=\HP^{n-1}$, so the statement follows from Proposition \ref{prop:vectorbundlesonHPn}. We can then assume $r>1$ and proceed by induction on $n \ge r$.

For $n=r$ and $n=r+1$, $\HGr(r,n)$ is isomorphic to $\Spec k$ and $\HP^r$ respectively. So we can assume $n>r+1$.

Let us denote by $Z$ and $Y$ respectively the closed subscheme $N^+$ of $\HGr(r,n)$ and its open complement. Since $N^+\to \HGr(r,n-1)$ is a rank $2r$ vector bundle, $Z$ is smooth over $k$ of codimension $2r$ in $\HGr(r,n)$.

From the diagram
\begin{equation}
\label{diag:loc.sequence.HGr}
    \begin{tikzcd}
    Z \arrow[r, hookrightarrow, "i"] \arrow[dr, "p_Z", swap] & \HGr(r,n) \arrow[d, "p"] & Y \arrow[l, hook', "j", swap] \arrow[dl, "p_Y"] \\
    & \Spec k &
\end{tikzcd}
\end{equation}
we get the localization distinguished triangle
$$j_!j^! \to \id_{\SH(\HGr(r,n))} \to i_*i^* \to j_!j^![1]$$
of endofunctors of $\SH(\HGr(r,n))$, by Proposition \ref{prop:localization}. By composing this sequence with $p_{\#}$ and applying it to $\Sigma^v1_{\HGr(r,n)}$, we get the cofiber sequence in $\SH(k)$
\begin{equation}
    \label{eq:s.e.s.ofThomSpaces}
    p_{\#}j_!j^!\Sigma^v1_{\HGr(r,n)} \to p_{\#}\Sigma^v1_{\HGr(r,n)} \to p_{\#}i_*i^*\Sigma^v1_{\HGr(r,n)}.
\end{equation}
 
By Proposition \ref{prop:s.e.s.-cellularity}, we just need to show that the left-hand term and the right-hand term of \eqref{eq:s.e.s.ofThomSpaces} belong to $\langle \mathbb{S}_k \rangle$.

Let us see the left-hand term first. Because $j$ is an open immersion, we have $j_!\simeq j_\#$ and $j^!\simeq j^*$, then $$p_{\#}j_!j^!\Sigma^v1_{\HGr(r,n)}\simeq p_{Y\#}j^*\Sigma^v1_{\HGr(r,n)} \simeq p_{Y\#}\Sigma^{j^*v}1_Y.$$

By Proposition~\ref{prop:summary}, we have the diagram
$$\begin{tikzcd}
    &&Y_2 \arrow[dl, swap, "g_2"] \arrow[dr, "q"]& \\
    &Y_1 \arrow[dl, swap, "g_1"]&&\HGr(r-1,n-1)\\
    Y &&&
    \end{tikzcd}$$
where the maps $g_1,g_2$ and $q$ are $\mathbb{A}^1$-weak equivalences, in fact, affine space bundles. Moreover, $q$ is a trivial $\mathbb{A}^{4n-3}$-bundle, hence, it has a zero section $s_0:\HGr(r-1,n-1) \to Y_2$, which is an inverse of $q$ in $\sH(k)$. The composition $h \coloneqq g_1 \circ g_2 \circ s_0: \HGr(r-1,n-1) \to Y$ is thus an $\mathbb{A}^1$-weak equivalence.

Now, $g_1,g_2$ and $q$ are affine space bundles, and $Y_1$, $Y_2$ and 
$\HGr(r-1,n-1)$ are smooth, hence, by $\A^1$-homotopy invariance of $K$-theory on regular schemes \cite[\S7 paragraph 1, and Proposition 4.1]{Quillen:K}, the maps $q^*:K_0(\HGr(r-1,n-1))\to K_0(Y_2)$ and $(g_1g_2)^*:K_0(Y)\to K_0(Y_2)$ are isomorphisms. Thus, there is an element $v'\in K_0(\HGr(r-1,n-1))$ such that $q^*v'=(g_1g_2)^*j^*v$ in $K_0(Y_2)$, and, since $g_1,g_2$ and $q$ are $\A^1$ weak equivalences, we have
$$
p_{Y\#}\Sigma^{j^*v}1_Y \simeq p_{(\HGr(r-1,n-1)\#}\Sigma^{v'}1_{\HGr(r-1,n-1)}
$$
in $\SH(k)$. 
By the induction hypothesis on $r$, we conclude that $p_{Y\#} \Sigma^{j^*v}1_Y$ belongs to $\langle \mathbb{S}_k \rangle$.

Let us now prove the claim for the right-hand term of \eqref{eq:s.e.s.ofThomSpaces}. Let us recall that $Z$ is isomorphic to the tautological rank $2r$ symplectic bundle over $\HGr(r,n-1)$, and as such, it is a smooth scheme with projection $q:Z\to \HGr(r,n-1)$ realizing $Z$ as a vector bundle over $\HGr(r,n-1)$. For simplicity, let us call $X \coloneqq \HGr(r,n)$. By Theorem \ref{thm:MorVoePurity}, we have
$$p_{\#}i_*i^*\Sigma^v1_X \simeq p_{Z\#}\Sigma^{N_i}i^*\Sigma^v1_X \simeq p_{Z\#}\Sigma^{N_i+i^*v}1_Z.$$

Since the morphism $q:Z\to \HGr(r,n-1)$ realizes $Z$ as a vector bundle over $\HGr(r,n-1)$, there is an element $v''\in K_0(\HGr(r,n-1))$ such that $q^*v''=i^*v+N_i$ in $K_0(Z)$. Since $q$ is an $\A^1$-weak equivalence,
$$p_{Z\#}\Sigma^{N_i + i^*v}1_Z\simeq p_{\HGr(r,n-1)\#}\Sigma^{v''}1_{\HGr(r,n-1)}.$$
By our induction hypothesis, we have that $p_{Z\#}\Sigma^{N_i + i^*v}1_Z$ belongs to $\langle \mathbb{S}_k \rangle$.

Therefore, we conclude that $p_{\#}\Sigma^V1_{\HGr(r,n)}$ belongs to $\langle \mathbb{S}_k \rangle $.
\end{proof}

\begin{corollary}
\label{cor:cellularityofMSp}
    The symplectic cobordism spectrum $\MSp$ is cellular.
\end{corollary}

\begin{proof}
    By definition, $\MSp$ is isomorphic to the filtered colimit of the system
    $$\Sigma^\infty_{\P^1}\MSp_0 \to \Sigma^{-4,-2}\Sigma^\infty_{\P^1}\MSp_2 \to \Sigma^{-8,-4} \Sigma_{\P^1}^\infty \MSp_4 \to \ldots,$$
    which is in particular a homotopy colimit. Thus, it is enough to show that $\Sigma^\infty_{\P^1}\MSp_{2r}$ is cellular for all $r$. $\MSp_{2r}$ is in turn the homotopy colimit $\colim_n\Th(E_{r,n}^{\Sp})$, where $E_{r,n}^{\Sp}$ is the tautological rank $2r$ symplectic bundle over the quaternionic Grassmannian $\HGr(r,n)$. It is then enough to prove that $\Sigma^\infty_{\P^1}\Th(E_{r,n}^\Sp)$ is cellular for all $r,n$. Since $\Sigma^\infty_{\P^1}\Th(E_{r,n}^\Sp)=p_{\HGr(r,n)\#}\Sigma^{E_{r,n}^\Sp}1_{\HGr(r,n)}$, this follows from Proposition \ref{prop:bundlesonHGr}.
\end{proof}

\subsection{The motive of $\MSp$}

By adapting the procedure adopted in the previous subsection, one can express the cellularity of $\MSp$ in terms of its motive. We recall that the motive of a motivic spectrum $\E \in \SH(k)$ is the $H\Z$-module $M(\E)=\E \wedge H\Z\in \Mod_{H\Z}=\DM(k)$, and we want to compute $M(\MSp)$. In order to do that, let us introduce some notation. For a free graded $\Z$-module $\Z \cdot x$, with $x$ in degree $n$, we write $\Z \cdot x \otimes \sE$ for the motivic spectrum $\Sigma^{-2n,-n}\sE \in \SH(k)$. This naturally extends to define $M_* \otimes \sE$ for $M_*=\oplus_{n \in \Z}M_n$ a direct sum of free graded $\Z$-modules.

\begin{prop}
\label{prop:MotiveOfMSp}
    The motive of $\MSp$ in $\DM(k)$ has the form
    $$\MSp \wedge H\Z \simeq \Z[b'_1,b_2',\ldots] \otimes H\Z,$$
    with $b_i'$ a polynomial generator of degree $-2i$.
\end{prop}

\begin{proof}
    First, by adapting the proof of Proposition \ref{prop:bundlesonHGr}, we prove that for all $n \ge r \ge 0$, the $H\Z$-module $\Sigma^\infty_{\P^1} \HGr(r,n)_+ \wedge H\Z$ is of the form $\oplus_s \Sigma^{4m_s,2m_s}H\Z$, for a finite number of non-negative integers $m_s$. 

    For $r=0$ or $r=n$, we have $\Sigma^\infty_{\P^1} \HGr(r,n)_+\simeq \Sigma^\infty_{\P^1}\Spec k_+=\mathbb{S}_k$, and the statement trivially holds. We can then take $n>r>0$ and suppose the claim true for all quaternionic Grassmannians $\HGr(r',n')$ with $r'\le r$ and $n'<n$, or $r'<r$ and $n'\le n$. 

    As in the diagram \eqref{diag:loc.sequence.HGr}, we consider $i:Z \hookrightarrow \HGr(r,n)$ the inclusion of the closed subscheme $Z \coloneqq N^+$, and $j:Y \hookrightarrow \HGr(r,n)$ its open complement, and we get the localization distinguished triangle
    \begin{equation*}
        j_!j^! \to \id_{\SH(\HGr(r,n))} \to i_*i^* \to j_!j^![1].
    \end{equation*}

    Let $X\coloneqq \HGr(r,n)$ and let $p:X \to \Spec k$ be the structure map. By applying the sequence above to $1_X$, and applying $p_\#(-)\wedge H\Z$ to the resulting distinguished triangle in $\SH(X)$, we get the distinguished triangle
    \begin{equation}
        \label{eq:loc.seq.HGr}
        p_\#j_!j^!1_X \wedge H\Z \to \Sigma^\infty_{\P^1}X_+ \wedge H\Z \to p_\# i_*i^*1_X \wedge H\Z \xrightarrow{\alpha} j_!j^!1_X[1] \wedge H\Z
    \end{equation}
    in $\DM(k)$. By the same arguments as in the proof of Proposition \ref{prop:bundlesonHGr}, we have $p_\#j_!j^!1_X \simeq p_{Y\#}1_Y$ and $p_\#i_*i^*1_X \simeq p_{Z\#}\Sigma^{N_i}1_Z$, with $N_i$ the normal bundle of $i$.
    
    For the remainder of the proof, except where indicated to the contrary, we will work in $\DM(k)$.

  Let us look at the right-hand term $p_{Z\#}\Sigma^{N_i}1_Z \wedge H\Z$ of \eqref{eq:loc.seq.HGr}. Since $p_{Z\#}$ satisfies projection formula against $p_Z^*$ by Remark \ref{rmk:smoothsharp-properties}, we have that $$p_{Z\#}\Sigma^{N_i}1_Z \wedge H\Z \simeq p_{Z\#}(\Sigma^{N_i}1_Z \wedge p_Z^*H\Z) \simeq p_{Z\#}\Sigma^{N_i}p_Z^*H\Z,$$
    and through the isomorphism $\th^f_{H\Z \Mod}(N_i)$, we have $p_{Z\#}\Sigma^{N_i}p_Z^*H\Z \xrightarrow{\sim} \Sigma^{4c,2c}p_{Z\#}p_Z^*H\Z$, with $2c$ the codimension of $Z$ in $X$. By the projection formula again, we have $p_{Z\#}p_Z^*H\Z \simeq p_{Z\#}1_Z \wedge H\Z \simeq \Sigma_{\P^1}^\infty Z_+\wedge H\Z$, but $Z$ is a vector bundle over $\HGr(r,n-1)$, so $\Sigma_{\P^1}^\infty Z_+ \simeq \Sigma_{\P^1}^\infty \HGr(r,n-1)_+$. We can then apply the induction hypothesis to get
    $$p_{Z\#}\Sigma^{N_i}1_Z \wedge H\Z\simeq \Sigma^{4c,2c}\oplus_i \Sigma^{4m_i^Z,2m_i^Z}H\Z$$
    for a finite number of non-negative integers $m_i^Z$.

    Let us look at the left-hand term $p_{Y\#}1_Y \wedge H\Z$ of \eqref{eq:loc.seq.HGr}. As seen in the proof of Proposition \ref{prop:bundlesonHGr}, there is an $\A^1$-weak equivalence $h:\HGr(r-1,n-1) \to Y$, then $p_{Y\#}1_Y \wedge H\Z \simeq \Sigma^\infty_{\P^1}Y_+\wedge H\Z \simeq \Sigma_{\P^1}^\infty \HGr(r-1,n-1)_+ \wedge H\Z$. We can then use again our induction hypothesis to get
    $$p_{Y\#}1_Y \wedge H\Z\simeq \oplus_{i'} \Sigma^{4m_{i'}^Y,2m_{i'}^Y}H\Z$$
    for a finite number of non-negative integers $m_{i'}^Y$.

    Now, the boundary map $\alpha$ in \eqref{eq:loc.seq.HGr} is an element in
    \begin{multline*}
         [p_\# i_*i^*1_X \wedge H\Z,j_!j^!1_X[1] \wedge H\Z]_{\DM(k)}\simeq \\
         [\Sigma^{4c,2c}\oplus_i \Sigma^{4m_i^Z,2m_i^Z}H\Z,\Sigma^{1,0}\oplus_{i'} \Sigma^{4m_{i'}^Y,2m_{i'}^Y}H\Z]_{\DM(k)} \simeq \\
         \oplus_{i,i'}[\Sigma^{4(c+m_i^Z),2(c+m_i^Z)}H\Z, \Sigma^{4m_{i'}^Y+1,2m_{i'}^Y}H\Z]_{\DM(k)} \simeq \\
         \oplus_{i,i'}[H\Z, \Sigma^{4(m_{i'}^Y-m_i^Z-c)+1,2(m_{i'}^Y-m_i^Z-c)}H\Z]_{\DM(k)}.
    \end{multline*}
    By the free-forgetful adjunction, this corresponds to an element in
    $$\oplus_{i,i'}[1_k, \Sigma^{4(m_{i'}^Y-m_i^Z-c)+1,2(m_{i'}^Y-m_i^Z-c)}H\Z]_{\SH(k)} = \oplus_{i,i'}H\Z^{4(m_{i'}^Y-m_i^Z-c)+1,2(m_{i'}^Y-m_i^Z-c)}(\Spec k),$$
    but each abelian group $H\Z^{4(m_{i'}^Y-m_i^Z-c)+1,2(m_{i'}^Y-m_i^Z-c)}(\Spec k)$ is trivial by Theorem \ref{thm:MazWeiHZ}. We conclude that $\alpha=0$, and the sequence \eqref{eq:loc.seq.HGr} splits. Hence 
    $$\Sigma^\infty_{\P^1}\HGr(r,n)_+ \wedge H\Z \simeq \oplus_s\Sigma^{4m_s,2m_s}H\Z$$
    for a finite number of non-negative integers $m_s$, proving the claim.

    Now, let us recall from Construction \ref{constr:MSp} that there are canonical inclusions $\HGr(r,rN) \hookrightarrow \HGr(r,r(N+1))$, and $\BSp_{2r}=\colim_N \HGr(r,rN)$. Since the $\text{free}_{H\Z}$ functor $(-)\wedge H\Z$ is a left adjoint, it commutes with small colimits, hence 
    \begin{equation}
    \label{eq:MotiveOfBsp}
        \Sigma_{\P^1}^\infty \BSp_{2r \, +}\wedge H\Z \simeq \colim_N(\Sigma_{\P^1}^\infty \HGr(r,rN)_+\wedge H\Z).
    \end{equation}

    By the inductive proof of the identity $\Sigma^\infty_{\P^1} \HGr(r,n)_+ \wedge H\Z \simeq \oplus_s \Sigma^{4m_s,2m_s}H\Z$, we can see that, once we fix $r$, for each non-negative integer $m$ the number of summands in the right-hand side for which $m_s=m$ is eventually constant for $n>>0$. Indeed, we can assume that this is the case for $\Sigma^\infty_{\P^1} \HGr(r-1,n)_+ \wedge H\Z$ by induction. Noting that $Z$ has codimension $c=2r$, independent of $n$, the above argument shows that the pairs $(4m,2m)$ that occur for $\HGr(r,n+1)$ are the pairs that occur for $\HGr(r-1,n)$ together with pairs of the form $(4(r+m'), 2(r+m'))$ where $(4m',2m')$ is a pair occurring for $\HGr(r,n)$. This proves the claim on the coefficients $m_s$.
    
    We can then rewrite \eqref{eq:MotiveOfBsp} as
    \begin{equation}
    \label{eq:MotiveBsp2}
        \Sigma_{\P^1}^\infty \BSp_{2r \, +}\wedge H\Z \simeq \oplus_s \Sigma^{4m_s,2m_s}H\Z,
    \end{equation}
    where now the sum will be infinite, but for all $m$ there is only a finite number of summands for which $m_s=m$.

    As before, by using projection formula of $p_{\BSp_{2r}\#}$ against $p_{\BSp_{2r}}^*$ and the isomorphism $\th^f_{H\Z \Mod}(E_{2r}^\Sp):\Sigma^{E_{2r}^\Sp}p_{\BSp_{2r}}^*H\Z \xrightarrow{\sim}\Sigma^{4r,2r}p_{\BSp_{2r}}^*H\Z$, we get
    $$p_{\BSp_{2r}\#}\Sigma^{E_{2r}^\Sp}1_{\BSp_{2r}}\wedge H\Z \simeq \Sigma^{4r,2r}p_{\BSp_{2r}\#}1_{\BSp_{2r}}\wedge H\Z.$$
    Thus, since $\Sigma_{\P^1}^\infty \MSp_{2r}\simeq p_{\BSp_{2r}\#}\Sigma^{E_{2r}^\Sp}1_{\BSp_{2r}}$, we can write
    $$\Sigma_{\P^1}^\infty \MSp_{2r} \wedge H\Z \simeq \Sigma^{4r,2r}\oplus_s \Sigma^{4m_s,2m_s}H\Z,$$
    with $m_s$ as in \eqref{eq:MotiveBsp2}. Since $\MSp =\colim_r(\Sigma^{-4r,-2r}\Sigma^\infty_{\P^1}\MSp_{2r})$, we can use again that $(-)\wedge H\Z$ commutes with small colimits and that, for each integer $m$, the number of summands in the right hand side of \eqref{eq:MotiveBsp2} such that $m_s=m$ becomes constant for $r>>0$, to get
    \begin{equation}\label{eqn:MSpDecom}
  \MSp \wedge H\Z \simeq \oplus_\alpha \Sigma^{4n_\alpha,2n_\alpha}H\Z,
  \end{equation}
    where, for all integers n, there are only finitely many indices $\alpha$ with $n_\alpha=n$, and each $n_\alpha$ is non-negative.

    We have then the following chain of isomorphisms:
  \begin{align*}H\Z^{*,*}(\MSp)&= [\MSp, \Sigma^{*,*}H\Z]_{\SH(k)}\\
        &\simeq [\MSp \wedge H\Z, \Sigma^{*,*}H\Z]_{\DM(k)}\\
        &\overset{\eqref{eqn:MSpDecom}}{\simeq} [\oplus_\alpha \Sigma^{4n_\alpha,2n_\alpha}H\Z,\Sigma^{*,*}H\Z]_{\DM(k)} \\
       & \simeq \prod_\alpha[H\Z,\Sigma^{*-4n_\alpha,*-2n_\alpha}H\Z]_{\DM(k)}\\
       & \simeq \prod_\alpha[1_k,\Sigma^{*-4n_\alpha,*-2n_\alpha}H\Z]_{\SH(k)}\\
       &=\oplus_\alpha H\Z^{*,*}(\Spec k)\cdot b_\alpha,
   \end{align*}
with $b_\alpha$ in bidegree $(4n_\alpha,2n_\alpha)$. But from Theorem \ref{thm:cohomologyofmsp}, with Remark \ref{rmk:homogeneous}, we know that 
    $$H\Z^{*,*}(\MSp) \simeq H\Z^{*,*}(\Spec k)[b_1,b_2,\ldots]$$
    with $b_i$ in bidegree $(4i,2i)$, in other words, $H\Z^{*,*}(\MSp)$ is a free bigraded $H\Z^{*,*}$-module with basis the monomials in $b_1, b_2,\ldots$. Thus, the bidegrees $(4n_\alpha,2n_\alpha)$ that occur in our decomposition \eqref{eqn:MSpDecom} are exactly the bidegrees of the monomials in $b_1, b_2,\ldots$.
    
    From this we see that 
    $$\MSp \wedge H\Z \simeq \Z[b_1',b_2',\ldots] \otimes H\Z,$$
with $b_i'$ a polynomial generator of degree $-2i$.
\end{proof}

\begin{corollary}
\label{cor:product_map}
    For any integer $m \ge 1$, the canonical map
    $$H\Z^{*,*}(\MSp)^{\otimes_{H\Z^{*,*}(\Spec k)}m} \to H\Z^{*,*}(\MSp^{\wedge m})$$
    is an isomorphism.
\end{corollary}

\begin{proof}
    In the proof of Proposition \ref{prop:MotiveOfMSp} we have seen that we can write the motive of $\MSp$ as
    \begin{equation}
    \label{eq:MotiveMSp}
        \MSp \wedge H\Z \simeq \oplus_\alpha \Sigma^{4n_\alpha,2n_\alpha}H\Z,
    \end{equation}
    for some non-negative integer $n_\alpha$, and for each non-negative integers $n$, there are only finitely many indices $\alpha$ such that $n_\alpha=n$.

    By induction on $m$, we will also have 
    \begin{equation}
    \label{eq:MotiveMSp^n}
       (\MSp^{\wedge m}) \wedge H\Z \simeq (\MSp \wedge H\Z)^{\wedge_{H\Z} m} \simeq \oplus_\beta \Sigma^{4m_\beta,2m_\beta}H\Z, 
    \end{equation}
    with analogous conditions on the non-negative integers $m_\beta$. Thus, computing the motivic cohomology of $\MSp^{\wedge m}$ we have
    \begin{multline}
    \label{eq:chain1}
        H\Z^{p,q}(\MSp^{\wedge m})=[\MSp^{\wedge m},\Sigma^{p,q}H\Z]_{\SH(K)} \simeq [\MSp^{\wedge m} \wedge H\Z, \Sigma^{p,q}H\Z]_{\DM(k)} \\
        \simeq [\oplus_\beta \Sigma^{4m_\beta,2m_\beta}H\Z,\Sigma^{p,q}H\Z]_{\DM(k)}\simeq [1_k,\Sigma^{p-4m_\beta,q-2m_\beta}H\Z]_{\SH(k)} \\ \simeq \prod_\beta H\Z^{p-4m_\beta,q-2m_\beta}(\Spec k).
    \end{multline}
    Since $H\Z^{a,b}(\Spec k)$ vanishes for $b<0$ by Theorem \ref{thm:MazWeiHZ}, we also have 
    \begin{equation}
    \label{eq:chain2}
        \prod_\beta H\Z^{p-4m_\beta,q-2m_\beta}(\Spec k) \simeq \oplus_\beta H\Z^{p-4m_\beta,q-2m_\beta}(\Spec k).
    \end{equation}

    The expressions \eqref{eq:MotiveMSp} and \eqref{eq:MotiveMSp^n} give the isomorphism of $H\Z$-modules
    $$(\oplus_\alpha \Sigma^{4n_\alpha,2n_\alpha}H\Z)^{\wedge_{H\Z} m} \simeq \oplus_\beta \Sigma^{4m_\beta,2m_\beta} H\Z.$$
    With the product in motivic cohomology, this induces the isomorphism
    $$(\oplus_\alpha H\Z^{*-4n_\alpha,*-2n_\alpha}(\Spec k))^{\otimes_{H\Z^{*,*}(\Spec k)} m} \xrightarrow{\sim} \oplus_\beta H\Z^{*-4m_\beta,*-2m_\beta}(\Spec k),$$
    which can be rewritten, through \eqref{eq:chain2} and \eqref{eq:chain1}, as
    $$H\Z^{*,*}(\MSp)^{\otimes_{H\Z^{*,*}(\Spec k)} m} \xrightarrow{\sim} H\Z^{*,*}(\MSp^{\wedge m}).$$
\end{proof}

\begin{rmk}
\label{rmk:HZ/ell}
    It is immediate to observe that Theorem \ref{thm:MazWeiHZ} remains true if we replace $H\Z$ with $H\Z/\ell$ for any prime $\ell$. Thus, the proofs of Proposition \ref{prop:MotiveOfMSp} and Corollary \ref{cor:product_map} also apply after replacing $H\Z$ with $H\Z/\ell.$
\end{rmk}

\section{The Adams spectral sequence for $\MSp$}
\label{chapter:A.S.S}

\subsection{The motivic Steenrod Algebra}

For this section, and for the rest of this work, $\ell$ will denote an odd prime different from $\text{char} k$.

Let us briefly review the construction of the motivic mod-$\ell$ Steenrod algebra.

Let us recall that a bistable cohomology operation of bidegree $(i,j)$ on a motivic spectrum $\E \in \SH(k)$ is a family $\{\phi_{p,q}\}_{p,q \in \Z}$ of natural transformations of functors in $\SH(k)$
$$\phi_{p,q}:\E^{p,q}(-) \to \E^{p+i,q+j}(-)$$
that commute with arbitrary suspensions. The set of all bistable cohomology operations on $\E$ has a natural ring structure induced by composition, which make is a bigraded $\Z/\ell$-algebra. The bistable operations on $\sE$ form a 
$\E^{*,*}(\Spec k)$-bimodule, where the left (resp. right) module structure is by post-composition (resp. pre-composition) with the left multiplication $\E^{*,*}(\Spec k)\times \E^{*,*}(-)\to
\E^{*,*}(-)$ (resp. the right multiplication $\E^{*,*}(-)\times \E^{*,*}(\Spec k)\times \to
\E^{*,*}(-)$).

We recall that mod $\ell$ motivic cohomology is represented in $\SH(k)$ by the motivic spectrum $H\Z/\ell$.

By \cite[Theorem 1.1]{Hoy:Steenrod}, the algebra of bistable cohomology operations on $H\Z/\ell$ is naturally isomorphic to the $\Z/\ell$-algebra of shifted endomorphisms of $H\Z/\ell$, $\text{End}_{\SH(k)}^{*,*}(H\Z/\ell):=\Hom_{\SH(k)}(H\Z/\ell,\Sigma^{*,*}H\Z/\ell)$, where $\Phi:H\Z/\ell\to \Sigma^{i,j}H\Z/\ell$ defines the cohomology operation $\phi:=\{\phi_{p,q}:=\Sigma^{p,q}\Phi_*\}_{p,q\in\Z}$.

The short exact sequence
$$0 \to \Z/\ell \xrightarrow{\cdot \ell} \Z /\ell^2 \to \Z/\ell \to 0$$
induces the distinguished triangle
\begin{equation}
    \label{eq:Bockstein}
    H\Z/\ell \to H\Z/\ell^2\to  H\Z/\ell \xrightarrow{\beta_{0,0}} \Sigma^{1,0}H\Z/\ell
\end{equation}
in $\SH(k)$. The resulting map $\beta_{0,0}:H\Z/\ell\to \Sigma^{1,0}H\Z/\ell$ in $\SH(k)$ is called the \emph{Bockstein homomorphism}. This defines a bistable cohomology operation $\beta=\{\beta_{p,q}:=\Sigma^{p,q}\beta_{0,0}\}_{p,q}$ of bidegree $(1,0)$ on $H\Z/\ell$, called the \emph{Bockstein operator}, which satisfies the same properties as the Bockstein operator in ordinary cohomology:
\begin{enumerate}
    \item $\beta \circ \beta = 0,$
    \item $\beta (xy)=\beta(x)\cup y + (-1)^p x \cup \beta(y)$, for $x \in (H\Z/\ell)^{p,*}(-)$ 
\end{enumerate}
(see \cite[Section 8]{Voev-power}).

In \cite[Section 9]{Voev-power}, for $i \ge 0$, Voevodsky defines reduced power operations $P^i$, with $P^0=1$, which we can consider as bistable cohomology operations of bidegree $(2i(\ell-1),i(\ell-1))$ on $H\Z/\ell$. 

\begin{defn}
    The \emph{motivic mod-$\ell$ Steenrod algebra} $A^{*,*}=A^{*,*}(k,\Z/\ell)$ is the subalgebra of the algebra of bistable cohomology operations on the mod $\ell$ motivic cohomology $H\Z/\ell$ over $k$ generated by the Bockstein operator $\beta$, the reduced power operations $\{P^i\}_{i \ge0}$ and operations $x \mapsto a \cdot x$ for $a \in (H\Z/\ell)^{*,*}(\Spec k)$.
\end{defn}

As a subalgebra of $\text{End}_{\SH(k)}^{*,*}(H\Z/\ell)$, $A^{*,*}$ acts on $(H\Z/\ell)^{*,*}(\E)$ for each $\E \in \SH(k)$. In fact, $A^{*,*}$ is isomorphic to the entire algebra $\text{End}_{\SH(k)}^{*,*}(H\Z/\ell)$ of bistable cohomology operations on $H\Z/\ell$ (see \cite[Therem 1.1 (1)]{Hoy:Steenrod}). For $k$ of characteristic $0$, this was proved by Voevodksy in \cite[Theorem 3.49]{Voe:EM}.

From now on, for simplicity, we shall use the notation $H^{*,*}(-) \coloneqq (H\Z/\ell)^{*,*}(-)$, and $H^{*,*} \coloneqq H^{*,*}(\Spec k)$.

The bigraded algebra $A^{*,*}$ is a left $H^{*,*}$-module via $(a\cdot\phi)(x):=a\cdot(\phi(x))$ for $a\in H^{*,*}$, $\phi\in A^{*,*}$ and $x\in H\Z^{*,*}(\E)$. There is also a right $H^{*,*}$-module structure given by $(\phi\cdot a)(x):=\phi(a\cdot x)$. The left and right actions do not agree in general, but they do agree on $\Z/\ell=H^{0,0}(k)$, making $A^{*,*}$ a bigraded $\Z/\ell$-algebra. Unless stated to the contrary, we will always consider $A^{*,*}$ with its left $H^{*,*}$-module structure.

Also, $A^{*,*}$ is equipped with a map $\Delta: A^{*,*} \to A^{*,*} \otimes_{H^{*,*}} A^{*,*}$ defined by
$\Delta (\gamma)= \sum_i \alpha_i \otimes \beta_i$, where the bistable operations $\alpha_i$, $\beta_i$ are determined by the relation
$$\gamma(x \cup y)= \sum_i(-1)^{ab_i} \alpha_i(x) \cup \beta_i(y),$$
for $x\in H^{a,a'}(\sE)$ and $\beta_i$ of bidegree $(b_i,b'_i)$. Since the left and right $H^{**}$-module actions on $A^{**}$ do not agree, $A^{*,*} \otimes_{H^{*,*}} A^{*,*}$ does not have a natural ring structure. Voevodsky then defines an $H^{*,*}$-submodule $(A^{*,*} \otimes_{H^{*,*}} A^{*,*})_r$ for which the formula $(x\otimes y)\cdot (x'\otimes y')=xx'\otimes yy'$ does give a ring structure, and which contains the image of $\Delta$. This gives a co-associative map 
\begin{equation}\label{eqn:ACoproduct}
\Psi:A^{*,*}\to (A^{*,*} \otimes_{H^{*,*}} A^{*,*})_r, 
\end{equation}
which Voevodsky calls, by an abuse of notation, a coproduct. All the details for this are in \cite[Lemmas 11.6-11.9]{Voev-power}.


In \cite[Section 13]{Voev-power}, Voevodsky defines the \emph{Milnor basis} for $A^{*,*}$, given by cohomology operations $\rho(E,R)$, for $E=(\epsilon_0, \epsilon_1, \epsilon_2, \ldots)$ a sequence of zeros and ones that are almost all zeros, and $R=(r_1, r_2, r_3, \ldots)$ a sequence of non-negative integers that are almost all zero. Let us write $Q(E) \coloneqq \rho(E, \underaccent{\bar}{0})$ and $\mathcal{P}^R \coloneqq \rho(\underaccent{\bar}{0},R)$. For the sequence $e_i=(0,0, \ldots, 0,1,0, \ldots)$ having $1$ only in the $i$-th place, he defines $Q_i\coloneqq Q(e_i)$. In this basis, we have $P^i = \mathcal{P}^{i\cdot e_1}$ and $Q_0 = \beta$.

The motivic Steenrod algebra $A^{*,*}$ that we have just described is the algebraic analogue of the classical Steenrod algebra $A^{\text{top}} \coloneqq A^{\top}_{\ell}$ described in \cite{mil:Steenrod}. Similarly to the algebraic situation, $A^{\text{top}}$ acts on the singular cohomology $H^*(-,\Z/\ell)$, it has a Bockstein operator $\beta_{\top}$, discussed in \cite[Chapter IV]{steenrod:cohomology}, and power operations $P^i_{\top}$. Also, $A^{\top}$ is a $\Z/\ell$ algebra with generators $\rho^{\top}(E,R)$, with $E$ and $R$ as before. We let $Q^{\top}(E) \coloneqq \rho^{\top}(E, \underaccent{\bar}{0})$, $\mathcal{P}_{\top}^R \coloneqq \rho^{\top}(\underaccent{\bar}{0},R)$ and define the special elements $Q_i^{\top} \coloneqq Q^{\top}(e_r)$. We have $P^i_{\top}=\mathcal{P}^{i \cdot e_1}_{\top}$, $P_{\top}^0=1$, and $Q_0^{\top}=\beta_{\top}$.

Let us now recall a few properties of the power operations from \cite{Voev-power}.

\begin{lemma}
\label{lemma:someproperties}
\begin{enumerate}
    \item[(1)] $P^0= \id$, and, for any $i<0$, $P^i=0$.
    \item[(2)] For $x \in H^{2n,n}(\Spec k)$, $P^n(x)=x^\ell$.
    \item[(3)] For $x \in H^{p,q}(\Spec k)$, $n > p-q$ and $n\ge q$, we have $P^n(x)=0$.
    \item[(4)] $Q_i$ has bidegree $(2 \ell^i -1, \ell^i-1)$, and for each $k \ge 1$, $P^{k \cdot e_i}$ has bidegree $(2k(\ell^i-1),k(\ell^i-1))$.
\end{enumerate}
\end{lemma}

\begin{proof}
    $(1)$ is \cite[Theorems 9.4-9.5]{Voev-power}. $(2)$ is \cite[Lemma 9.8]{Voev-power}. $(3)$ is \cite[9.9]{Voev-power}. The formulas for the bidegrees of $Q_i$ and $P^{k \cdot e_i}$ come directly from their definition in \cite[Section 13]{Voev-power}.
\end{proof}

Finally, let us define $B \subset A^{*,*}$ as the $\Z/\ell$-subalgebra generated by $\{Q_i\}_{i\ge 0}$, and $M_B \coloneqq A^{*,*}/A^{*,*}(Q_0,Q_1, \ldots)$ as the quotient of $A^{*,*}$ by its left ideal generated by the special elements $Q_i$. Analogously, we have $B^{\top} \subset A^{\top}$ and $M_B^{\top} \coloneqq A^{\top}/A^{\top}(Q_0^{\top}Q_1^{\top}, \ldots)$. 

\begin{rmk}\label{rmk:MBDescent} Taken $\E \in \SH(k)$  and  $u\in H\Z^{2n,n}(\E)$, the left $A^{*,*}$-module structure on $H\Z^{*,*}(\E)$ gives the bigraded map  $r_u:A^{*, *}\cdot u\to H\Z^{*,*}(\E)$ of left $A^{*,*}$-modules, defined by $r_u(\alpha\cdot u):= \alpha(u)$. If $H\Z^{2*+1,*}(\E)=0$, then $r_u$ descends to a map of $A^{*,*}$-modules $r_u:M_B\cdot u\to H\Z^{*,*}(\E)$. Indeed, since $Q_i$ has bidegree $(2\ell^i+1, \ell^i)$, $Q_i(u)$ lands in $H\Z^{2(\ell^i+n)+1, \ell^i+n}(\E)=0$ for each $i$. An analogous statement in the topological setting holds for $M_B^{\top}$.
\end{rmk}

\subsection{The $\text{Ext}_{A^{**}}$-algebra of $H^{*,*}(\MSp)$}

Let us remind the reader that in this section we use the notation $H^{*,*}(-) \coloneqq (H\Z/\ell)^{*,*}(-)$ and $H^{*,*} \coloneqq H^{*,*}(\Spec k)$.

For two bigraded $A^{*,*}$-modules $N^{**}, M^{**}$, we have the trigraded abelian group  $\Ext_{A^{**}}(N^{**}, M^{**}):=\Ext_{A^{**}}^{*,(*,*)}(N^{**}, M^{**})$ defined as follows. Let $D_{\text{bgr}}(A^{**})$ denote the derived category of bigraded $A^{**}$-modules. We write $$\Ext_{A^{**}}^{a,(b,c)}(N^{**}, M^{**}):=\Hom_{D_{\text{bgr}}(A^{**})}^{b,c}(N^{**}, M^{**}[a]),$$
where $\Hom_{D_{\text{bgr}}(A^{**})}^{b,c}(-,-)$ is the $(b,c)$ bigraded component of $\Hom_{D_{\text{bgr}}(A^{**})}(-,-)$. This gives a natural tri-grading on the sum $\Ext_{A^{**}}^{*,(*,*)}(N^{**}, M^{**}) = \oplus_{a,b,c \in \Z}\Ext_{A^{**}}^{a,(b,c)}(N^{**}, M^{**})$. We will omit the tri-grading from the notation when it is not needed.

Now, let us consider two motivic spectra $\E,\E' \in \SH(k)$, and let us observe that two maps $H^{*,*}(\E) \to H^{*,*}[a]$ and $H^{*,*}(\E') \to H^{*,*}[b]$ in $D_{\text{bgr}}(A^{*,*})$ give a map 
$$H^{*,*}(\E) \otimes_{H^{*,*}}H^{*,*}(\E')\to H^{*,*} \otimes_{H^{*,*}}H^{*,*}[a][b]=H^{*,*}[a+b]$$
in the derived category of $(A^{*,*}\otimes_{H^{*,}}A^{*,*})_r$-modules. Thus, through the map $\Psi$ introduced in \eqref{eqn:ACoproduct}, we can see this map in the derived category $D_{\text{bgr}}(A^{*,*})$. In particular, if one of $H^{*,*}(\E)$, $H^{*,*}(\E')$ is flat over $H^{*,*}$, this defines a product
\begin{multline}
\label{eq:tensorproduct}
    \Ext_{A^{*,*}}(H^{*,*}(\E),H^{*,*})\otimes_{H^{*,*}}\Ext_{A^{*,*}}(H^{*,*}(\E'),H^{*,*}) \to \\ \Ext_{A^{*,*}}(H^{*,*}(\E)\otimes_{H^{*,*}}H^{*,*}(\E'),H^{*,*}).
\end{multline}
Now, if the canonical map
\begin{equation}
\label{eq:tensor_to_wedge}
    H^{*,*}(\E) \otimes_{H^{*,*}}H^{*,*}(\E') \to H^{*,*}(\E \wedge \E')
\end{equation}
is an isomorphism, we can turn \eqref{eq:tensorproduct} into the product
\begin{equation}
    \label{eq:tensor_product_ext}
    \Ext_{A^{*,*}}(H^{*,*}(\E),H^{*,*})\otimes_{H^{*,*}}\Ext_{A^{*,*}}(H^{*,*}(\E'),H^{*,*}) \to \Ext_{A^{*,*}}(H^{*,*}(\E \wedge \E'), H^{*,*})
\end{equation}
by composition with the inverse of \eqref{eq:tensor_to_wedge}. The co-associativity of the coproduct $\Psi$ makes both the products \eqref{eq:tensorproduct} and \eqref{eq:tensor_product_ext} associative.

We are interested in studying the structure of $\Ext_{A^{*,*}}(H^{*,*}(\MSp),H^{*,*})$.

Let us then consider $\E=\E'=\MSp$. The description of the cohomology of $\MSp$ given in Theorem \ref{thm:cohomologyofmsp} shows that $H^{*,*}(\MSp)$ is a free $H^{*,*}$-module, so, in particular, it is flat. Also, by Corollary \ref{cor:product_map}, we have that the map
$$H^{*,*}(\MSp) \otimes_{H^{*,*}}H^{*,*}(\MSp) \to H^{*,*}(\MSp \wedge \MSp)$$
is an isomorphism, so we have the product \eqref{eq:tensor_product_ext} for $\E=\MSp$. Now, the multiplication map $\mu_{\MSp}:\MSp \wedge \MSp \to \MSp$ gives the pullback $\mu_{\MSp}^*:H^{*,*}(\MSp) \to H^{*,*}(\MSp \wedge \MSp)$ in cohomology, and composition with $\mu_{\MSp}^*$ gives the map
$$(\mu_{\MSp}^*)^*:\Ext_{A^{*,*}}(H^{*,*}(\MSp \wedge \MSp),H^{*,*}) \to \Ext_{A^{*,*}}(H^{*,*}(\MSp),H^{*,*}).$$
By composing the product \eqref{eq:tensor_product_ext} for $\MSp$ with $(\mu_{\MSp}^*)^*$, we finally get the product
$$\Ext_{A^{*,*}}(H^{*,*}(\MSp),H^{*,*})\otimes_{H^{*,*}}\Ext_{A^{*,*}}(H^{*,*}(\MSp),H^{*,*}) \to \Ext_{A^{*,*}}(H^{*,*}(\MSp), H^{*,*}).$$
Since the product \eqref{eq:tensor_product_ext} was associative, this product is still associative, and thus induces a natural structure of associative $\Z/\ell$-algebra on the trigraded abelian group $\Ext_{A^{*,*}}(H^{*,*}(\MSp),H^{*,*})$, which becomes a trigraded $\Z/\ell$-algebra.

In order to study the convergence of the Adams spectral sequence associated to $\MSp$, we need to find a presentation of the trigraded $\Z/\ell$-algebra
$$\Ext_{A^{*,*}}(H^{*,*}(\MSp),H^{*,*}) \coloneqq \bigoplus_{s,t,u}\Ext^{s,(t-s,u)}_{A^{*,*}}(H^{*,*}(\MSp),H^{*,*}).$$
We will follow the strategy used in \cite{lev:ellcoh} for the case of MGL.

\begin{defn}
    Let us recall that a \emph{partition} $\omega$ of an integer $n\ge0$ is a sequence of integers $(\omega_1,\ldots, \omega_r)$ with $\omega_1\ge \omega_2\ge\ldots\ge \omega_r > 0$ such that $|\omega|:=\sum_i\omega_i=n$. We say that a partition $\omega$ is \textit{even} if all the terms $\omega_i$ are even.
\end{defn}

For a positive integer $n$, let us extend the natural bigrading on $H^{**}$ to a bigrading on the polynomial algebra $H^{*,*}[t_1,\ldots t_k]$ by giving $t_i$ bidegree $(2,1)$. As before, we have the polynomial ring $H^{*,*}[b_1, \ldots, b_k]$ with $b_i$ of bidegre $(4i,2i)$, and we get a map from $H^{*,*}[b_1, \ldots, b_k]$ to the subalgebra of symmetric polynomials in $t_1^2,\ldots, t_k^2$ by taking $b_s$ to the elementary symmetric polynomial
$$f_s \coloneqq \sum_{i_1,\ldots, i_s}t_{i_1}^2 \cdot \cdot \cdot t_{i_s}^2$$ 
in $t_1^2,\ldots, t_k^2$ for all $s \leq k$. Moreover, to each even partition $\omega=(2q_1,\ldots, 2q_s)$ of an even integer $2q$, we can associate the symmetric polynomial
$$f_{\omega} \coloneqq \sum_{i_1,\ldots, i_s}t_{i_1}^{2q_1} \cdot \cdot \cdot t_{i_s}^{2q_s}.$$
Since the $H^{*,*}$-algebra of symmetric polynomials in $t_1^2,\ldots, t_k^2$ is the polynomial algebra over $H^{*,*}$ in the  elementary symmetric polynomials in the same variables, we have a uniquely defined element $P_\omega\in H^{*,*}[b_1, \ldots, b_k]$ corresponding to $f_\omega$ under the association $b_s \mapsto f_s$.

For each even partition $\omega$, let $u_{\omega}$ denote the element in $H^{*,*}(\MSp)$ corresponding to $P_{\omega}$, using Theorem~\ref{thm:cohomologyofmsp} to identify $H^{*,*}(\MSp)$ with $H^{*,*}[b_1, b_2,\ldots]$. In $H^{*,*}(\MSp)$, $u_{\omega}$ has bidegree $(2 |\omega|, |\omega|)$.

Since $H^{a,b}=0$ if $a>b$ or if $b<0$ by Theorem \ref{thm:MazWeiHZ}, we have $H^{2*+1, *}=0$. As $b_i$ has bidegree $(4i, 2i)$, it follows that $H^{2*+1,*}(\MSp)=(H^{*,*}[b_1, b_2,\ldots])^{2*+1,*}=0$. Thus, using Remark~\ref{rmk:MBDescent}, for each even partition $\omega$ we can define a map of left $A^{**}$-modules
\begin{align*}
    \Phi_{\omega}:M_Bu_{\omega} & \to H^{*,*}(\MSp)\\
    \alpha \cdot u_{\omega} & \mapsto \alpha(u_{\omega}).
\end{align*}

\begin{defn}
\label{defn:l-adicpartition}
    We say that a partition $\omega=(\omega_1, \ldots, \omega_s)$ is \textit{$\ell$-adic} if at least one of its terms $\omega_r$ is of the form $\omega_r=\ell^m-1$, for some $m\ge1$. 
\end{defn}

\begin{lemma}
\label{lemma:decompmsp}
    Let $P$ be the set of all even partitions $\omega$ that are not $\ell$-adic. Then the map
    $$\Phi \coloneqq \bigoplus_{\omega \in P} \Phi_{\omega}:\bigoplus_{\omega \in P}M_Bu_{\omega}\to H^{*,*}(\MSp)$$
    is an isomorphism of left $A^{**}$-modules.
\end{lemma}

\begin{proof}
    We start by briefly reviewing the topological analogue for $\MSp^{\top}$, studied in \cite[Chapter 1]{Thomcompl}.

    Let $H^*(-)$ denote the mod $\ell$ singular cohomology $H^*(-,\mathbb{Z}/\ell)$. We take $\Sp_{2r}^\top:=\Sp_{2r}(\C)$, with the classical topology. This gives the classifying space $\BSp_{2r}^\top$, as well as the tautological rank $2r$ symplectic bundle $E^{\Sp,\top}_{2r}\to \BSp_{2r}^\top$ and the Thom space $\MSp_{2r}^\top \coloneqq \Th(E^{\Sp,\top}_{2r})$. The $S^4$-spectrum $\MSp^{\top}$ is the sequence 
    $$\MSp^\top=(\MSp_0^\top, \MSp_2^\top, ,\ldots,\MSp_{2r}^\top,\ldots).$$ Equivalently, $\MSp^{\top}$ is the colimit of the shifted suspension spectra $\Sigma^{-4r}_{S^1}\Sigma^\infty_{S_1}\MSp_{2r}^\top$. Thus, the cohomology $H^*(\MSp^\top)$ is the limit of the cohomologies $H^{*+4r}(\MSp_{2r}^\top)$, for each fixed degree $*$. This limit is eventually constant. 
 
 The cohomology algebra $H^{*}(\BSp_{2r}^\top)$ is described for instance in \cite{BorelSerre:Steenrod}: it is the polynomial algebra over $\Z/\ell$ with generators the mod $\ell$ quaternionic Borel classes $$k_{4j}:=b_j^\top(E^{\Sp,\top}_{2r})\in H^{4j}(\BSp_{2r}^\top), \; j=1,\ldots, r.$$
 
 Novikov \cite[Lemma 3]{Thomcompl} showed that the pullback by the zero section $z_{2r}^{\top*}:H^{*+4r}(\MSp_{2r}^\top)\to
 H^{*+4r}(\BSp_{2r}^\top)$ is injective, with image the ideal generated by the top Borel class $k_{4r}$. 
 
Writing $\BSp^\top$ as $\colim_r\BSp_{2r}^\top$ gives the isomorphism $H^*(\BSp^\top)\simeq \lim_rH^*(\BSp_{2r}^\top)$, and similarly, we have the isomorphism 
 $H^*(\MSp^\top)\simeq \lim_rH^{*+4r}(\MSp_{2r}^\top)$. Together with the Thom isomorphisms $H^*(\BSp_{2r}^\top)\simeq H^{*+4r}(\MSp_{2r}^\top)$, this  gives the commutative diagram of isomorphisms
 $$
 \begin{tikzcd}
     H^*(\BSp^\top) \arrow[d, "\wr"] \arrow[r, dashed, "\sim"] & H^*(\MSp^\top) \arrow[d, "\wr"] \\
     \lim_rH^*(\BSp_{2r}^\top) \arrow[r, "\sim"] & \lim_r H^{*+4r}(\MSp_{2r}^\top)
 \end{tikzcd}
 $$
 Under the isomorphism $H^*(\BSp^\top)\xrightarrow{\sim}H^*(\MSp^\top)$, we can write
 $$H^*(\MSp^\top)\simeq H^*(\BSp^\top)= \Z/\ell[k_4, k_8,\ldots].$$
 
 Now, since the operations in the topological mod $\ell$ Steenrod algebra $A^\top$ are stable with respect to $S^1$-suspension, the action on $H^*(\MSp^\top)$ is induced by the action on $H^{*+4r}(\MSp_{2r}^\top)$ for $r\gg0$, and thus by the action on $H^{*+4r}(\BSp_{2r}^\top)$ via $z_{2r}^{\top*}$. Let us note that this does not identify with the action of $A^\top$ on $H^*(\BSp^\top)$ via the above isomorphism $H^*(\MSp^\top)\simeq H^*(\BSp^\top)$.
 
 Let $\HGr(r,n)^\top \coloneqq \HGr(r,n)(\C)$, again with the classical topology. We can approximate $H^{*+4r}(\BSp_{2r}^\top)$ by $H^{*+4r}(\HGr(r,n)^\top)$, and similarly, we can approximate $H^*(\MSp^\top)$ as $A^\top$-module by the $A^\top$-module $H^{*+4r}(\HGr(r,n)^\top)$.
 
Explicitly, $k_{4j}\in H^{4j}(\BSp^\top_{2r})$ corresponds to the elementary symmetric polynomial $\sum_{i_1,\ldots, i_j}t_{i_i}^2 \cdot \ldots \cdot t_{i_j}^2$ in $t_1^2,\ldots, t_r^2$, and to each even partition $\omega=(2q_1,\ldots,2q_s)$, we can assign the element
    $$v^{(r)}_{\omega}\coloneqq \sum_{i_1,\ldots, i_s}t_{i_1}^{2q_1}\cdot \ldots \cdot t_{i_s}^{2q_s}\in
 H^{2|\omega|}(\BSp_{2r}^\top).$$
The $v^{(r)}_{\omega}$ are compatible in $r$, giving the elements $v_\omega\in H^{2|\omega|}(\BSp^\top)$, and we let $u_\omega\in H^{2|\omega|}(\MSp^\top)$ be the element corresponding to $v_\omega$ via the isomorphism $H^*(\MSp^\top)\simeq H^*(\BSp^\top)$. Via the Thom isomorphism, $v^{(r)}_\omega$ maps to an element $v_\omega^{(r)'}\in H^{*+4r}(\MSp_{2r}^\top)$, and then $z_{2r}^*v_\omega^{(r)'}=v^{(r)}_{\omega}\cdot k_{4r}\in
 H^{2|\omega|+4r}(\BSp_{2r}^\top)$. We can drop the ${}^{(r)}$ from the notation, and we see that the action of the Steenrod algebra on  $u_\omega\in H^{2|\omega|}(\MSp^\top)$ is given by the action on $v_{\omega}\cdot k_{4r}\in
 H^{2|\omega|+4r}(\BSp_{2r}^\top)$ for $r\gg0$.
    
 The natural product $\mu_{\MSp^\top}: \MSp^\top\wedge \MSp^\top\to \MSp^\top$ on the spectrum $\MSp^\top$ is induced by the embeddings $\Sp(m)\times \Sp(n) \subset \Sp(m+n)$. Together with the K\"unneth formula, $\mu_{\MSp^\top}$ gives rise to the ``diagonal map''
 \[
 \delta^\top:=\mu_{\MSp^\top}^*:H^*(\MSp^\top)\to H^*(\MSp^\top)\otimes_{\Z/\ell}H^*(\MSp^\top).
 \] 
 
   By \cite[Lemma 7]{Thomcompl} the diagonal morphism on the generators $u_{\omega}^{\top}$ has the following form:
    \begin{equation}
    \label{eq:comultiplication}
        \delta^\top(u_{\omega}^{\top})= \sum_{{(\omega_1,\omega_2)=\omega} \atop{\omega_1 \neq \omega_2}}(u_{\omega_1}^{\top} \otimes u_{\omega_2}^{\top}+u_{\omega_2}^{\top}\otimes u_{\omega_1}^{\top}) + \sum_{(\omega_1,\omega_1)=\omega}(u_{\omega_1}^{\top}\otimes u_{\omega_1}^{\top}).
    \end{equation}
    
    The fact that $H^*(\MSp^{\top})$, as a module over $A^{\top}$, is the direct sum of modules $M_B^{\top}u_{\omega}^{\top}$ over even non-$\ell$-adic partitions $\omega$ follows by \cite[Lemma 4]{Thomcompl}. In Novikov's paper, the $Q_i^{\top}$ are the Adams elements $e_r'$ of Cartan type $1$ defined in \cite[Section 2.2]{Thomcompl}, and the $P_{\top}^i$ are a subfamily of the Adams elements $e_{r,k}$ of Cartan type $0$ defined in the same section. In other words, the map
    $$\Phi^{\top} \coloneqq \bigoplus_{\omega \in P} \Phi_{\omega}^{\top}:\bigoplus_{\omega \in P}M_B^{\top}u_{\omega}^{\top}\to H^*(\MSp^{\top})$$
    where $\Phi_{\omega}^{\top}$ is defined in the same way as we defined $\Phi_{\omega}$, is an isomorphism of graded left  $A^{\top}$-modules.

    Now let us recall that, by \cite[Chapter 6, Section 1]{steenrod:cohomology}, the $P_{\top}^i$ and $\beta_{\top}$ satisfy the Adem relations and $\beta_{\top}^2=0$, and these relations determine $A^{\top}$ as $\mathbb{Z}/\ell$-algebra. By \cite[Theorem 5.1]{Hoy:Steenrod}, the operations $P^i$ also satisfy the Adem relations, and $\beta^2=0$, hence, there is a unique ring homomorphism $\theta:A^{\top}\to A^{**}$ defined by sending $P_{\top}^i$ to $P^i$ and $\beta_{\top}$ to $\beta$. In particular, if we let $A^{*,*}_0$ be the $\mathbb{Z}/\ell$-subalgebra of $A^{*,*}$ generated by $\beta$ and the $P^i$, $\theta$ induces an isomorphism of rings $A^{\top} \to A^{*,*}_0$ by \cite[Lemma 4.1 (1)]{lev:ellcoh}. Moreover, the $H^{*,*}$-linear extension of $\theta$
    $$\id_{H^{*,*}}\otimes_{\mathbb{Z}/\ell}\theta: H^{*,*}\otimes_{\mathbb{Z}/\ell}A^{\top}\to A^{*,*}$$
    is an isomorphism of left $H^{*,*}$-modules, still by \cite[Lemma 4.1]{lev:ellcoh}. We similarly define $M_{B0}$ as the $\mathbb{Z}/\ell$-subalgebra of $M_B$ generated by the $P^i$.
    
    The action of $A^{*,*}$ on $H^{*,*}(\MSp)$ restricts to an action of $A^{*,*}_0$ on $H^{0,0}[b_1,b_2,\ldots]=\mathbb{Z}/\ell[b_1,b_2,\ldots] \subset H^{*,*}[b_1,b_2,\ldots]$, where $\beta$ acts trivially, because $\deg\beta=(1,0)$. Let us denote by $H_0^{*,*}(\MSp)$ the bigraded ring $\mathbb{Z}/\ell[b_1,b_2,\ldots]$. The map $\Phi_{\omega}$ of $A^{*,*}$-modules induces then a map $\Phi_{\omega}^0$ of $A_0^{*,*}$-modules, so, finally, we get a map
    $$\Phi^0 \coloneqq \bigoplus_{\omega \in P} \Phi_{\omega}^0:\bigoplus_{\omega \in P}M_{B0}u_{\omega}\to H^{*,*}_0(\MSp).$$
    
    Let us now note that, as left $H^{*,*}$-modules, $M_B$ is isomorphic to $H^{*,*}\otimes_{\mathbb{Z}/\ell} M_{B0}$, and $H^{*,*}(\MSp)$ is isomorphic to $H^{*,*}\otimes_{\mathbb{Z}/\ell}H^{*,*}_0(\MSp)$. In particular, as homomorphisms of $\mathbb{Z}/\ell$-vector spaces, $\Phi_{\omega}$ and $\Phi$ can be written as $$\Phi_{\omega}=\id_{H^{*,*}}\otimes \Phi_{\omega}^0, \; \; \text{and} \; \; \Phi=\id_{H^{*,*}}\otimes(\bigoplus_{\omega \in P} \Phi_{\omega}^0)=\id_{H^{*,*}}\otimes \Phi^0.$$
    Thus, if $\Phi^0$ is an isomorphism of $A^{*,*}_0$-modules, $\Phi$ will be an isomorphism of $\mathbb{Z}/\ell$-vector spaces, and then also an isomorphism of left $A^{*,*}$-modules. It is then enough to prove that $\Phi^0$ is an isomorphism of $A^{*,*}_0$-modules.

    Let us consider again the cohomology of $\MSp^{\top}$, $H^*(\MSp^{\top})=H^*(\BSp^{\top})=\mathbb{Z}/\ell[b_1^{\top},b_2^{\top},\ldots]$. Since every homogeneous element in $H_0^{*,*}(\MSp)$ has bidegree $(4i,2i)$ for some $i$, it is immediate by the presentations detailed above that the map $\rho:H^*(\MSp^{\top})\to H^{*,*}_0(\MSp)$ defined by taking $b_n^{\top}$ to $b_n$ is an isomorphism of graded rings. We want to show that it is also an isomorphism of left $A^{*,*}_0$-modules, using $A^{*,*}_0\simeq A^{\top}$ to define the module structure on $H^*(\MSp^{\top})$.

    Since the action of $\beta$ is trivial, it is enough to show that $\rho$ preserves the action of the power operations, that is, $\rho(P^i_{\top}(b_n^{\top}))=P^i(b_n)$ for all $i$.

    For each positive integer $r$, the action of $P^i$ on $\mathbb{Z}/\ell[b_1,\ldots,b_r]\subset H^{*,*}_0(\MSp)$ is induced by the action on the cohomology of the quaternionic Grassmannian $H^{*,*}(\HGr(r,n))$ as follows. As in the topological case, the canonical map $\Sigma_{\P^1}^{-2r}\Sigma^\infty_{\P^1}\MSp_{2r}\to \MSp$ induces an isomorphism $H^{a,b}(\MSp)\cong H^{a+4r, b+2r}(\MSp_r)$ for $r\gg0$, namely for $r$ bigger than a certain positive integer $r_0$ depending on $(a,b)$. By Proposition~\ref{prop:CohBSpMSp}(2), the restriction by the zero section $z_{2r}^*:H^{*+4r, *+2r}
    (\MSp_{2r})\to H^{*+4r, *+2r}
    (\BSp_r)$ is injective, with image isomorphic to the ideal generated by the mod $\ell$ Borel class $b_r(E^\Sp_{2r})$. Similarly, we have an isomorphism $\BSp_{2r}\simeq \colim_n\HGr(r, n)$, in $\sH(k)$, giving the approximation of $H^{*+4r, *+2r}
    (\BSp_r)$ by $H^{*+4r, *+2r}(\HGr(r, n))$ for $n\gg0$, and we can identify $H^{a,b}(\MSp)$ with the bidegree $(a+4r, a+2r)$ component of the ideal in $H^{*, *}(\HGr(r, n))$ generated by the mod $\ell$ Borel class $b_r(E^\Sp_{r,n})$, for $r,n\gg0$ (depending on $(a,b)$). As the motivic Steenrod operations are $\Sigma^{a,b}$-stable, it follows that this identification of $H^{a,b}(\MSp)$ with $H^{a+4r, b+2r}(\HGr(r, n))$ for $r,n\gg0$ is compatible with the respective $A^{*,*}$-module structures. 

    Thus, $P^i(b_j)\in H^{4(i+j), 2(i+j)}(\MSp)$ corresponds to $$P^i(b_j(E^\Sp_{r,n})\cdot b_r(E^\Sp_{r,n}))\in H^{4(i+j+r), 2(i+j+r)}(\HGr(r, n)), \; \; r,n\gg0.$$ 
    Let us denote by $\text{HFlag}(1^r;n)$ the complete quaternionic flag variety associated to $\HGr(r,n)$ as in \cite[\S 3, p. 147]{panwal:grass}. By the symplectic splitting principle for quaternionic Grassmannians \cite[Theorem 10.1]{panwal:grass}, the pullback of $E^\Sp_{2r,2n}$ by the map $\text{HFlag}(1^r;n)\to \HGr(r,n)$ splits as the orthogonal direct sum of the $r$ universal rank $2$ symplectic subbundles $(\mathcal{U}_2^{(1)},\phi_2^{(1)}) \perp \ldots \perp (\mathcal{U}_2^{(r)},\phi_2^{(r)})$, and the cohomology pullback maps injectively the Borel classes $b_j(E^\Sp_{2r,2n})$ to the elementary symmetric polynomials $e_j(\xi_1,\ldots,\xi_r)$ in the Borel roots $\xi_m\coloneqq b_1(\mathcal{U}_2^{(m)},\phi_2^{(m)}) \in H^{4,2}(\text{HFlag}(1^r;n))$. Then $P^i(b_j(E^\Sp_{r,n})\cdot b_r(E^\Sp_{r,n}))$ is determined by $P^i((e_j \cdot e_r)(\xi_1,\ldots,\xi_r))$. Moreover, by the Cartan formula \cite[Proposition 9.7]{Voev-power}, $P^i((e_j \cdot e_r)(\xi_1,\ldots,\xi_r))$ is completely determined by the values $P^s(\xi_m)$ for $s=0,1,\ldots,i$, but those values are in turn completely determined by the properties of reduced power operations, as follows.
    \begin{itemize}
    \item Since $P^0=\id$ by \ref{lemma:someproperties} (1), $P^0(\xi_m)=\xi_m$. 
    \item $P^1(\xi_m)=0$ by looking at the degree and at the presentation of $H^{**}(\text{HFlag}(1^r;n))$ given by \cite[Theorem 11.1]{panwal:grass}.
    \item Since $\xi_m$ has bidegree $(2\cdot 2,2)$, $P^2(\xi_m)=\xi_m^{\ell}$ by \ref{lemma:someproperties} (2). 
    \item $P^s(\xi_m)=0$ for all $s \ge 3$ by \ref{lemma:someproperties} (3).
    \end{itemize}

    The analogous result works for the topological case. Namely, the same argument shows that $P^i_{\top}(b_j^{\top})$ is determined by the values $P^s_{\top}(\xi_m^{\top})$ for $s=0,1,\ldots,i$, with $\xi_m^{\top}$ the topological Borel roots of the complex symplectic flag manifold. For those values, the same expressions as before are valid, by the axioms of the power operations $P^i_{\top}$ (see for instance \cite[Chapter VI]{steenrod:cohomology}). 

    Thus, if we express $P^i_{\top}(b_n^{\top})$ as a polynomial $p$ in the variables $e^\top_j=e_j(\xi_1^{\top}, \ldots, \xi_r^{\top})$, the same polynomial $p$ in the variables $e_j=e_j(\xi_1,\ldots, \xi_r)$ will give an expression for $P^i(b_n)$, and thus:
    $$\rho(P^i_{\top}(b_n^{\top}))=\rho(p(b_1^{\top},b_2^{\top},\ldots))=p(b_1,b_2,\ldots)=P^i(b_n).$$
    We have then proved that in the following commutative diagram
    $$
    \begin{tikzcd}
        \bigoplus_{\omega \in P}M_B^{\top}u_{\omega}^{\top} \arrow[r, "\Phi^{\top}"] \arrow[d, swap, "\theta'"] & H^*(\MSp^{\top}) \arrow[d, "\rho"]\\
        \bigoplus_{\omega \in P}M_{B0}u_{\omega} \arrow[r, "\Phi^0"] & H^{**}_0(\MSp),
    \end{tikzcd}
    $$
    where $\theta'$ is the map induced by $\theta$, all maps except $\Phi^0$ are isomorphisms of left $A^{*,*}_0$-modules. Thus, so is $\Phi^0$.
    \end{proof}

Because of the decomposition of $H^{*,*}(\MSp)$ of Lemma \ref{lemma:decompmsp}, the algebra $\Ext_{A^{*,*}}(H^{*,*}(\MSp),H^{*,*})$ is generated by the algebra $\Ext_{A^{*,*}}(M_B,H^{*,*})$ and the dual elements of the cohomology classes $u_{\omega}$.

    The algebra $\Ext_{A^{*,*}}(M_B,H^{*,*})$ is completely studied in \cite[Section 5.1]{lev:ellcoh}. In particular, we have the following computation.
  
  \begin{lemma}\label{lem:ExtMB}
 1. \cite[Lemma 5.1 and Lemma 5.4]{lev:ellcoh} There is an isomorphism of trigraded $\mathbb{Z}/\ell$-algebras
    $$\Ext_{A^{*,*}}(M_B,H^{*,*}) \simeq \Ext_B(\mathbb{Z}/\ell,H^{*,*}),$$
 2. \cite[Lemma 5.2]{lev:ellcoh} We have  
    \begin{itemize}
        \item $\underline{t>2u}$: $\Ext_B^{s,(t-s,u)}(\mathbb{Z}/\ell,H^{*,*})=0$.
        \item $\underline{t=2u}$: $\Ext_B^{s,(2u-s,u)}(\mathbb{Z}/\ell,H^{*,*})$ is a polynomial $\mathbb{Z}/\ell$-algebra in generators $\{h_r'\}_{r\ge0}$, with $\deg(h_r')=(1,(1-2\ell^r,1-\ell^r))$. 
        \item $\underline{t<2u}$: $\Ext_B^{0,(2u-1,u)}(\mathbb{Z}/\ell,H^{*,*})$ is $H^{1,1}$ if $u=1$, and $0$ otherwise, and the product map
        $$H^{1,1} \otimes \bigoplus_{s,u}\Ext_B^{s,(2u-s,u)}(\mathbb{Z}/\ell,H^{*,*}) \to \Ext_B^{s,(2u-s+1,u+1)}(\mathbb{Z}/\ell,H^{*,*})$$
        is surjective.
    \end{itemize}
    \end{lemma}

    Now, let us define, for all even non-$\ell$-adic partitions $\omega$, elements $z_{(\omega)} \in \Ext_{A^{*,*}}^{0,(-2|\omega|,-|\omega|)}$ characterized by the relations $(z_{\omega},u_{\omega'})= \delta_{\omega,\omega'}$, where the pairing $(z_{\omega},u_{\omega'})$ is $z_{\omega}(u_{\omega'})\in H^{*,*}$. Let us note that the diagonal map $\delta^\top$ described above can be read in $\bigoplus_{\omega \in P}M_B^{\top}u_{\omega}^{\top}$ through $\Phi^{\top}$, and it induces, via $\theta'$ and $\rho$, the diagonal map
    $$\delta: \bigoplus_{\omega \in P}M_Bu_{\omega} \to \bigoplus_{\omega \in P}M_Bu_{\omega} \otimes_{H^{*,*}} \bigoplus_{\omega \in P}M_Bu_{\omega}.$$

    The expression \eqref{eq:comultiplication} for $\delta^{\top}(u_{\omega}^{\top})$ yields the analogous expression for $\delta(u_{\omega})$ through $\rho$:

    \begin{equation}
    \label{eq:diagonalexpansion}
        \delta(u_{\omega})= \sum_{{(\omega_1,\omega_2)=\omega} \atop{\omega_1 \neq \omega_2}}(u_{\omega_1} \otimes u_{\omega_2}+u_{\omega_2}\otimes u_{\omega_1}) + \sum_{(\omega_1, \omega_1)=\omega}(u_{\omega_1}\otimes u_{\omega_1}).
    \end{equation}

    From the multiplication map
    \begin{multline*}
        \Ext_{A^{*,*}}(H^{*,*}(\MSp),H^{*,*}) \otimes_{H^{*,*}}\Ext_{A^{*,*}}(H^{*,*}(\MSp),H^{*,*}) \to \\ \Ext_{A^{*,*}}(H^{*,*}(\MSp) \otimes_{H^{*,*}}H^{*,*}(\MSp),H^{*,*}) \to \Ext_{A^{*,*}}(H^{*,*}(\MSp),H^{*,*}),
    \end{multline*}
    induced by $\delta$, we see that the pairing $(z_{\omega_1}\cdot z_{\omega_2},u_{\omega})$ can be computed by taking $(z_{\omega_1}\otimes z_{\omega_2},\delta(u_{\omega}))$ and turning $\otimes$ into products in $H^{*,*}$.
    
    Now we write $(\omega_1,\omega_2)$ for the concatenation: namely $((\omega_1,\ldots, \omega_r),(\omega'_1,\ldots, \omega'_s))$ is the sequence $(\omega_1,\ldots, \omega_r,\omega'_1,\ldots, \omega'_s)$, reordered to be non-increasing. Applying $z_{\omega_1}\otimes z_{\omega_2}$ to the expansion \eqref{eq:diagonalexpansion} of $\delta(u_{\omega})$, we see that $(z_{\omega_1}\otimes z_{\omega_2},\delta(u_{\omega}))$ is $1$ if $\omega=(\omega_1,\omega_2)$, and $0$ otherwise. That means $z_{\omega_1}\cdot z_{\omega_2}=z_{(\omega_1,\omega_2)}$.
    
 Lemma~\ref{lemma:decompmsp} and Lemma~\ref{lem:ExtMB}  implies that the $H^{*,*}$-subalgebra
 $$\oplus_{k\ge0} \Ext_{A^{*,*}}^{0,(-4k,-2k)}(H^{*,*}(\MSp),H^{*,*})$$
 of $ \Ext_{A^{*,*}}^{*,(*,*)}(H^{*,*}(\MSp),H^{*,*})$ is the free $H^{*,*}$-module on the elements $z_\omega$, $\omega\in P$, the only indecomposable elements are the elements $z_{(2k)}\in \Ext_{A^{*,*}}^{0,(-4k,-2k)}(H^{*,*}(\MSp),H^{*,*})$ corresponding to the trivial partitions $\omega=(2k)$, with $2k$ not of the form $\ell^i-1$, and the subalgebra generated by the 
 $z_{(2k)}$ is the polynomial algebra over $H^{*,*}$ on the $z_{(2k)}$.
  
Taking $k=0$, the element $z_{(0)}\in \Ext_{A^{*,*}}^{0,(0,0)}(H^{*,*}(\MSp),H^{*,*})$ dual to $u_{(0)}$ is the identity in the $\Ext$-algebra,  and the summand
$\bigoplus_{s,u}\Ext_{A^{*,*}}^{s,(2u-s,s)}(M_Bu_{(0)},H^{*,*})$
of $\oplus_{s,u}\Ext_{A^{*,*}}^{s,(2u-s,s)}(H^{*,*}(\MSp),H^{*,*})$ is the polynomial algebra on generators $\{h_r'\}_{r\ge0}$ corresponding to the polynomial generators $\{h_r'\}_{r\ge0}$ described in Lemma~\ref{lem:ExtMB}. Moreover, we have $h_{r_1}'z_{\omega_1}\cdot h_{r_2}'z_{\omega_2}=h_{r_1}'h_{r_2}'z_{(\omega_1,\omega_2)}$ and sending $x\in \bigoplus_{s,u}\Ext_{A^{*,*}}^{s,(2u-s,s)}(M_Bu_{(0)},H^{*,*})$ to $x\cdot z_\omega\in \bigoplus_{s,u}\Ext_{A^{*,*}}^{s,(2u-s,s)}(M_Bu_{(\omega)},H^{*,*})$ gives an isomorphism of $$\bigoplus_{s,u}\Ext_{A^{*,*}}^{s,(2u-s,s)}(M_Bu_{(0)},H^{*,*}) \xrightarrow{\sim}\bigoplus_{s,u}\Ext_{A^{*,*}}^{s,(2u-s,s)}(M_Bu_{\omega},H^{*,*})$$
for each partition $\omega\in P$.
  
 Putting all the information from Lemma~\ref{lemma:decompmsp} and Lemma~\ref{lem:ExtMB} together with the product structure on $\{z_\omega\}$ described above, we get the following.

    \begin{prop}
    \label{prop:pres.ExtAlgebra}
        The $\mathbb{Z}/\ell$-algebra $\Ext_{A^{*,*}}(H^{*,*}(\MSp),H^{*,*})$ has the following presentation:
        \begin{itemize}
            \item $\Ext_{A^{*,*}}^{s,(t-s,u)}(H^{*,*}(\MSp),H^{*,*})=0$ if $t>2u$.
            \item $\Ext_{A^{*,*}}^{s,(2u-s,u)}(H^{*,*}(\MSp),H^{*,*})$ is polynomial in the generators:
            \begin{itemize}
                \item $1 \in \Ext_{A^{*,*}}^{0,(0,0)}(H^{*,*}(\MSp),H^{*,*})$;
                \item $z_{(2k)}\in \Ext_{A^{*,*}}^{0,(-4k,-2k)}(H^{*,*}(\MSp),H^{*,*})$, for $k\ge 1$, and $2k\neq \ell^i-1 \; \forall \; i \ge 0 $;
                \item $h_r' \in \Ext_{A^{*,*}}^{1,(1-2 \ell^r,1-\ell^r)}(H^{*,*}(\MSp),H^{*,*})$, for $r \ge 0$.
            \end{itemize}
            \item $\Ext_{A^{*,*}}^{0,(2u-1,u)}(H^{*,*}(\MSp),H^{**})$ is $H^{1,1}$ if $u=1$, and $0$ otherwise, and the product map
        \begin{multline*}
            H^{1,1} \otimes_{\Z/\ell} \bigoplus_{s,u}\Ext_{A^{*,*}}^{s,(2u-s,u)}(H^{*,*}(\MSp),H^{*,*}) = \\ \Ext_{A^{*,*}}^{0,(1,1)}(H^{*,*}(\MSp),H^{*,*}) \otimes_{\Z/\ell} \bigoplus_{s,u}\Ext_{A^{*,*}}^{s,(2u-s,u)}(H^{*,*}(\MSp),H^{*,*}) \to \\ \bigoplus_{s,u} \Ext_{A^{*,*}}^{s,(2u-s+1,u+1)}(H^{*,*}(\MSp),H^{*,*})
        \end{multline*}
        is surjective.
        \end{itemize}
    \end{prop}
  
  \begin{rmk} We can compare the discussion given here with the descriptions of $H^{*,*}(\MGL)$ and $\Ext_{A^{*,*}}(H^{*,*}(\MGL),H^{*,*})$, the last one given by \cite[Lemma 5.9]{lev:ellcoh} and \cite[Proposition 5.7]{lev:ellcoh} (see Lemma~\ref{lemma:decompMGL} and Proposition~\ref{prop:ExtMGL} below for the precise statements). In this way, we see that $H^{*,*}(\MSp)$ is a summand of $H^{*,*}(\MGL)$ as $A^{*,*}$-modules, $\Ext_{A^{*,*}}(H^{*,*}(\MSp),H^{*,*})$ is the corresponding $H^{*,*}$-summand of $\Ext_{A^{*,*}}(H^{*,*}(\MGL),H^{*,*})$,  and $\Ext_{A^{*,*}}(H^{*,*}(\MGL),H^{*,*})$ is the polynomial algebra over $\Ext_{A^{*,*}}(H^{*,*}(\MSp),H^{*,*})$ on generators $z_{(k)}\in \Ext_{A^{*,*}}^{0,(-2k,-k)}(H^{*,*}(\MGL),H^{*,*})$ for $k$ odd. All these comparison maps are induced by the natural map $\Phi: \MSp\to \MGL$ discussed in Remark \ref{rmk:MSp-MGLThomClasses}. 
   \end{rmk}

\subsection{Convergence of the Adams spectral sequence}

Here we want to prove a convergence result for the motivic Adams spectral sequence associated to $\MSp$. The procedure will heavily rely on the results of \cite[Section 6]{lev:ellcoh}. For the proof of such results, we will refer to that paper. We start by discussing a general result already exposed in \cite{DugIsa:Adams}.

\subsubsection{A general result for the motivic Adams spectral sequence}
\label{sub:a.s.s}

The motivic Adams spectral sequence is by definition the spectral sequence associated with the Adams tower.

\begin{constr}[Adams tower]
    Let us fix a motivic spectrum $\E \in \SH(k)$ and a motivic commutative ring spectrum $\sE \in \SH(k)$. By letting $\overline{\sE}$ be the homotopy fiber of the unit map $1_k \to \sE$ of $\sE$, we have by definition the distinguished triangle 
    $$\overline{\sE} \to 1_k \to \sE \to \overline{\sE}[1].$$
    By taking the smash product of this triangle with $\overline{\sE}^{\wedge s} \wedge \E$, we get the distinguished triangle
    \begin{equation}
    \label{eq:Adamstriangle}
        \overline{\sE}^{\wedge s+1} \wedge \E \to \overline{\sE}^s \wedge \E \to \sE \wedge \overline{\sE}^s \wedge \E \to \overline{\sE}^{\wedge s+1} \wedge \E[1].
    \end{equation}
    By using the notations $\E_s \coloneqq \overline{\sE}^{\wedge s} \wedge \E$, $W_s \coloneqq \sE \wedge \overline{\sE}^{\wedge s}$, and $W_s(\E) \coloneqq W_s \wedge \E$, we rewrite \eqref{eq:Adamstriangle} as
    $$\E_{s+1} \to \E_s \to W_s(\E) \to \E_{s+1}[1].$$
    In this way, we get a tower of homotopy cofiber sequences
    $$
    \begin{tikzcd}
        \ldots \arrow[r] & \E_{s+1} \arrow[d] \arrow[r] & \E_s \arrow[d] \arrow[r] & \E_{s-1} \arrow[d] \arrow[r] & \ldots \arrow[r] & \E_0 \arrow[d] \arrow[r, equal] & \E \\
        & W_{s+1}(\E) & W_s(\E) & W_{s-1}(\E) & & W_0(\E) & .
    \end{tikzcd}
    $$
    Let $K_s(\E)$ be the homotopy cofiber of $\E_{s+1}\to \E_0$. The maps $\E_{s+1}\to \E_s$ and $\id_{\E_0}$ induce a map $K_s(\E) \to K_{s-1}(\E)$ on the cofiber, and the fiber of such map is the cofiber of $\E_{s+1} \to \E_s$, that is, $W_s(\E)$. In this way, we got a tower  under $\E$ of homotopy fiber sequences:
    \begin{equation}
    \label{eq:Adamstower}
         \begin{tikzcd}
        & & W_2(\E) \arrow[d] & W_1(\E) \arrow[d] & W_0(\E) \arrow[d, equal] \\
        \E \arrow[r] &\ldots \arrow[r] & K_2(\E) \arrow[r] & K_1(\E) \arrow[r] & K_0(\E) \arrow[r] & K_{-1}(\E)=0,
    \end{tikzcd}
    \end{equation}
    that we call the \emph{Adams tower of $\E$ (relative to $\sE$)}.
\end{constr}

\begin{defn}
    The \emph{$\sE$-nilpotent completion of $\E$}, denoted by $\E^{\wedge}_{\sE}$, is the homotopy limit of the Adams tower \eqref{eq:Adamstower}. 
\end{defn}

We want to consider $\sE=H\Z/\ell$ and define the Adams spectral sequence. The argument used in \cite[Example 3.4]{DunRon:Functors} to show that $H\Z$, denoted by $H\Z$ in that paper, is a commutative ring spectrum, works as well for $H\Z/\ell$, by replacing $\Z_{tr}$ with $\Z_{tr}/\ell$. 

\begin{defn}
    The \emph{(mod-$\ell$) Adams spectral sequence for a motivic spectrum $\E$} is the spectral sequence of the Adams tower of $\E$ relative to $H\Z/\ell$. Explicitly, by using the convention $\F^{a,b}\coloneqq \F^{a,b}(\Spec k)$ for $\F \in \SH(k)$, it is the spectral sequence defined by the tower \eqref{eq:Adamstower}, with $\sE=H\Z/\ell$ and $E_1$-term $W_s(\E)^{t,u}$,
    $$E_1^{s,t,u} \coloneqq W_s(\E)^{t,u} \Rightarrow (\E_{H\Z/\ell}^\wedge)^{t,u},$$
    with filtration degree $s$, cohomological bidegree $(t,u)$, and differentials $d_r^{s,t,u}:E_r^{s,t,u} \to E_r^{s+r,t+1,u}$.
\end{defn}

\begin{defn}
\label{defn:WedgeOfCopiesHZ}
    For $\E \in \SH(k)$, we say that $\E$ is a \emph{motivically finite type wedge of copies of $H\Z/\ell$} if there exist $p,q \in \Z$ and a set of bidegrees $\{p_\alpha,q_\alpha\}{_\alpha\in S}$ such that 
    $$\Sigma^{p,q}\E \simeq \oplus_{\alpha \in S} \Sigma^{p_\alpha,q_\alpha} H\Z/\ell,$$
    with the bidegrees $(p_\alpha,q_\alpha)$ satisfying the conditions:
    \begin{enumerate}
        \item $p_\alpha \ge 2q_\alpha \ge 0$ for all $\alpha \in S$,
        \item For all $m \in \Z$, $q_\alpha \le m$ for almost all $\alpha$.
    \end{enumerate}
\end{defn}

The following is \cite[Proposition 6.6]{lev:ellcoh} or \cite[Remark 6.11 and Proposition 6.14]{DugIsa:Adams}.

\begin{prop}
\label{prop:a.s.s}
    Let us suppose that $\E \in \SH(k)$ is a cellular spectrum and $W_s(\E)$ is a motivically finite type wedge of copies of $H\Z/\ell$ for all $s$. Then the $E_2$-page of the Adams spectral sequence for $\E$ has the form
    $$E_2^{s,t,u}= \Ext_{A^{*,*}}^{s,(t-s,u)}(H^{*,*}(\E),H^{*,*}).$$
    If in addition, fixed $(t,u)$, we have $\lim_r^1 E_r^{s,t,u}(\E)=0$ for all $s$, the spectral sequence converges completely to $(\E_{H\Z/\ell}^\wedge)^{t,u}$. That is, the natural map $(\E_{H\Z/\ell}^\wedge)^{t,u} \to \lim_s K_s(\E)^{t,u}$ is an isomorphism, the associated filtration
    $$F_s(\E_{H\Z/\ell}^\wedge)^{t,u} \coloneqq \ker((\E_{H\Z/\ell}^\wedge)^{t,u} \to K_{s-1}(\E)^{t,u}) \subset (\E_{H\Z/\ell}^\wedge)^{t,u}$$
    is exhaustive, and the natural map
    $$F_s(\E_{H\Z/\ell}^\wedge)^{t,u} / F_{s+1}(\E_{H\Z/\ell}^\wedge)^{t,u} \to E_\infty^{s,t,u}(\E)$$
    is an isomorphism for all $s \ge 0$.
\end{prop}

See \cite[Proposition 6.6]{lev:ellcoh} for a proof.

\begin{prop}
\label{prop:a.s.s.multiplication}
    Let $\sX$ be a cellular ring spectrum with multiplication $\mu_{\sX}$, such that each $W_s(\sX)$ is a motivically finite type wedge of copies of $H\Z/\ell$. Let us also suppose that 
    $H^{*,*}(\sX)$ is flat over $H^{*,*}$ and the canonical map $H^{*,*}(\sX)\otimes_{H^{*,*}}H^{*,*}(\sX) \to H^{*,*}(\sX \wedge \sX)$ is an isomorphism, which implies that the product \eqref{eq:tensor_product_ext} on $\Ext_{A^{*,*}}(H^{*,*,}(\sX),H^{*,*})$ discussed in the last subsection is defined. Then, the Adams spectral sequence for $\sE$ has a multiplicative structure induced by $\mu_\sX$ and compatible with the multiplicative structure on $(\sX_{H\Z/\ell}^\wedge)^{*,*}$. Also, the product on the $E_2$-terms induced by the multiplicative structure on the spectral sequence agrees with the associative $\Z/\ell$-algebra structure on $\Ext_{A^{*,*}}^{s,(t-s,u)}(H^{*,*}(\sX),H^{*,*})$ given by the product \eqref{eq:tensor_product_ext} and the isomorphism 
$$E_2^{s,t,u}= \Ext_{A^{*,*}}^{s,(t-s,u)}(H^{*,*}(\sX),H^{*,*})$$    
given by Proposition \ref{prop:a.s.s}.
\end{prop}

See [Proposition 6.7]\cite{lev:ellcoh} for a proof.

\subsubsection{$(\eta,\ell)$-completions}

Let us recall that we have the unstable Hopf map $\eta_u: \A^2 \setminus \{0\} \to \P^1$ defined by $(x,y) \mapsto [x:y]$. We also have $\A^2 \setminus \{0\}\simeq \P^1 \wedge \G_m$ in $\sH_\bullet(k)$ by \cite[\S 3 Example 2.20]{morvoe:homotopytheory}. Then one takes the \emph{stable Hopf map} as the map $\eta: \Sigma^{1,1}\mathbb{S}_k \to \mathbb{S}_k$ given by $\Sigma^{-2,-1}\Sigma^\infty_{\P^1}\eta_u$.

\begin{constr}
\label{constr:eta-completion}
    Let us denote by $\mathbb{S}_k/\eta^n$ the homotopy cofiber of $\eta^n:\Sigma^{n,n}\mathbb{S}_k \to \mathbb{S}_k$. The  diagram
    $$
    \begin{tikzcd}
       \Sigma^{n+1,n+1}\mathbb{S}_k \arrow[d, swap, "\eta"] \arrow[r, "\eta^{n+1}"] & \mathbb{S}_k \arrow[d, equal] \arrow[r] & \mathbb{S}_k/\eta^{n+1} \\
       \Sigma^{n,n}\mathbb{S}_k \arrow[r, swap, "\eta^{n}"] & \mathbb{S}_k \arrow[r] & \mathbb{S}_k/\eta^n
    \end{tikzcd}
    $$
    induces the map $\mathbb{S}_k/\eta^{n+1} \to \mathbb{S}_k/\eta^n$. Let $\E \in \SH(k)$ be a motivic spectrum, and let $\E/\eta^n \coloneqq \mathbb{S}_k/\eta \wedge \E$. We denote by $\E_\eta^\wedge$ the homotopy limit of the tower
    $$ \ldots \to \E/\eta^3 \to \E/\eta^2 \to \E/\eta,$$
    called the \emph{$\eta$-completion of $\E$}.
\end{constr}

Let us fix some notation for the next definition: for $m \in \Z$, let $S\Z/m$ denote the cofiber of the map $\mathbb{S}_k \xrightarrow{\times m} \mathbb{S}_k$ in $\SH(k)$, and for $\E \in SH(k)$, let $\E/m \coloneqq S\Z/m \wedge \E$.  

\begin{defn}
\label{defn:ellCompl}
    For a motivic spectrum $\E \in \SH(k)$, the \emph{$\ell$-adic completion of $\E$}, denoted by $\E_\ell^\wedge$, is the homotopy limit of the tower
    $$\ldots \to \E/\ell^{n+1} \to \E/\ell^n \to \E^/\ell^{n-1} \to \ldots \; .$$
\end{defn}

As a matter of notation, we will use the shortcut $\E^\wedge_{\eta,\ell} \coloneqq (\E_\eta^\wedge)_\ell^\wedge$ to denote the $(\eta, \ell)$-completion of $\E$.

In \cite{Man:completions}, Mantovani studied $\sE$-completions and $\sE$-localizations of spectra, and in particular proved a comparison between the $H\Z/\ell$-nilpotent completion and the $(\eta,\ell)$-completion of a spectrum $\E$, with the only hypothesis of connectedness of $\E$, generalizing some results showed in \cite{ormsby:adams} (in particular the first claim of \cite[Theorem 1]{ormsby:adams}). In order to define connectedness, let us recall that for any motivic spectrum $\E$, we have a bigraded homotopy sheaf $\pi_{n,m}(\E)$ given by the Nisnevich sheafification of the presheaf $U \mapsto \E^{-n,-m}(U)$ on $\Sm/k$. One usually adopts the notations $\pi_p(\E)_q \coloneqq \pi_{p-q,-q}(\E)$ and $\pi_p(\E) \coloneqq \oplus_{q \in \Z} \pi_p(\E)_q$. 

\begin{defn}
    A motivic spectrum $\E \in \SH(k)$ is called \emph{$r$-connected} if for all $p \le r$, $\pi_p(\E)=0$, and it is called \emph{connected} if it is $r$-connected for some integer $r$.
\end{defn}

Combining some of the main results of \cite{Man:completions}, one gets the following.

\begin{thm}[\cite{Man:completions}, Theorem 4.3.5, Example 5.2, Theorem 7.3.4]
\label{thm:Mantovani}
    If $\E \in \SH(k)$ is a connected motivic spectrum, then $\E_{H\Z/\ell}^\wedge \simeq \E_{\eta, \ell}^\wedge$.
\end{thm}

\begin{prop}
\label{prop:mspconnective}
    $\MSp$ is a $(-1)$-connected spectrum.
\end{prop}

This is a symplectic analogue of \cite[Corollary 2.3]{lev:ellcoh}. Before proving this proposition, we also recall the connectedness of spaces:

\begin{defn}
    A pointed space $X \in \Spc_{\bullet}(k)$ is \emph{$r$-connected} if, for each $x \in X$, the Nisnevich stalk $X_x$ is an $r$-connected space, that is, $\pi_p(X_x,*)=\{0\}$ for $1\le p\le r$ and $\pi_0(X_x,*)=\{*\}$ if $r\ge0$. Here $*\in X_x$ is the base-point.
\end{defn}

Connectedness of spaces is linked to connectedness of spectra by the fundamental stable connectivity theorem of Morel:

\begin{thm}[\cite{Morel:connectivity}, Theorem 3]
\label{thm:MorelConnectivity}
    If $X \in \Spc_{\bullet}(k)$ is $r$-connected, then the infinite suspension spectrum $\Sigma_{\P^1}^\infty X \in \SH(k)$ is $r$-connected.
\end{thm}

\begin{proof}[Proof of Proposition \ref{prop:mspconnective}]
    Let us consider the quaternionic Grassmannian $\HGr(n,np)$, and let us take an open cover $\{U_i\}_i$ of $\HGr(n,np)$ that trivializes, as ordinary vector bundle, the tautological symplectic bundle $E_{n,np}^\Sp$. The symplectic structure on $E_{n,np}^\Sp$ will not play any role here. Since $\HGr(n,np)$ is quasi-compact, we may assume that this open cover is finite. For every $i$, we have $\Th(E_{n,np}^\Sp \mid_{U_i})\simeq \Th(\mathcal{O}_{U_i}^{2n})$. By Remark \ref{rmk:Thomspaces}(2), we have
    $$\Th(\mathcal{O}_{U_i}^{2n}) \simeq S^{4n,2n}\wedge U_{i\; +} \simeq S^{2n} \wedge \G_m^{\wedge 2n} \wedge U_{i \; +},$$
    from which we see that $\Th(\mathcal{O}_{U_i^{2n}}) \in \Spc_{\bullet}(k)$ is $(2n-1)$-connected. By Theorem \ref{thm:MorelConnectivity}, $\Sigma_{\P^1}^\infty \Th(\mathcal{O}_{U_i}^{2n}) \in \SH(k)$ is $(2n-1)$-connected. By a Mayer-Vietoris argument for the sheaves $\pi_p(-)$, for $p\le 2n-1$, we see that $\Sigma_{\P^1}^\infty \Th(E_{n,np}^\Sp \mid_{U_i \cup U_j})$ is $(2n-1)$-connected for all $i,j$. By induction on the cardinality of $\{U_i\}_i$, we conclude that $\Sigma_{\P^1}^\infty \Th(E_{n,np}^\Sp)$ is $(2n-1)$-connected. 

    In general, a filtered colimit of $r$-connected spectra is $r$-connected, and let us note that, if $\E \in \SH(k)$ is $r$-connected, $\Sigma^{2,1}\E=S^1 \wedge \G_m \wedge \E$ is $(r+1)$-connected and $\Sigma^{-2,-1}\E$ is $(r-1)$-connected.

    Going back to our situation, since $\Sigma_{\P^1}^\infty \MSp_{2n}$ is the filtered colimit of $\Sigma_{\P^1}^\infty \Th(E^\Sp_{n,np})$, it is still $(2n-1)$-connected. Thus, $\Sigma^{-4n,-2n}\Sigma_{\P^1}^\infty \MSp_{2n}$ is $(-1)$-connected. This holds for all $n$. Therefore, $\MSp$ is a filtered colimit of $(-1)$-connected spectra, and as such, is $(-1)$-connected itself.
\end{proof}

Theorem \ref{thm:Mantovani} and Proposition \ref{prop:mspconnective} give the following.

\begin{corollary}
\label{cor:EtaEllComplMSp}
    In $\SH(k)$, we have $\MSp_{H\Z/\ell}^\wedge \simeq \MSp_{\eta, \ell}^\wedge$.
\end{corollary}

\subsubsection{The spectral sequence for Msp}

Now, we want to consider again the results of \S \ref{sub:a.s.s} with focus on the case $\sX = \MSp$.

\begin{prop}
\label{prop: a.s.s.MSp}
    The Adams spectral sequence for $\MSp$ is of the form
    $$E_2^{s,t,u}= \Ext_{A^{*,*}}^{s,(t-s,u)}(H^{*,*}(\MSp),H^{*,*}) \Rightarrow (\MSp_{H\Z /\ell}^\wedge)^{t,u},$$
    with differentials $d_r^{s,t,u}:E_r^{s,t,u} \to E_r^{s+r,t+1,u}$.
\end{prop}

\begin{proof}
    The result follows from Proposition \ref{prop:a.s.s}, provided that the two conditions of the propositions are satisfied for $\MSp$.

    The fact that $\MSp$ is cellular is Corollary \ref{cor:cellularityofMSp}. Therefore, we just need to show that $W_s(\MSp)$ is a motivically finite type wedge of copies of $H\Z/\ell$ for all $s$, where we remind the reader that we are considering $\sE=H\Z/\ell$ in the definition of $W_s(\MSp)$.

    $W_s(\MSp)=W_s \wedge \MSp = W_s \wedge_{H\Z/\ell} H\Z/\ell \wedge \MSp$. By Proposition \ref{prop:MotiveOfMSp} and Remark \ref{rmk:HZ/ell}, we can write $$H\Z/\ell \wedge \MSp \simeq \oplus_{\alpha \ge 0} \Sigma^{4n_\alpha,2n_\alpha}H\Z/\ell,$$
    for some non-negative integers $n_\alpha$ such that, for all $n$, there are only finitely many indices $\alpha$ with $n_\alpha=n$. Moreover, by \cite[Lemma 6.8]{lev:ellcoh}, we can write
    $$W_s = \oplus_{(p,q) \in S_s}\Sigma^{p,q}(H\Z/\ell)^{r_{p,q}},$$
    where $S_s = \{(p,q) \mid p+s \ge 2q \ge0 \}$, and for each $q$, $r_{p,q}=0$ for almost all $p$. We note that $(4n_\alpha,2n_\alpha)\in S_s$, and we conclude that we can write
    $$W_s(\MSp) \simeq \oplus_{(p,q) \in S_s}\Sigma^{p,q}(H\Z/\ell)^{n_{p,q}},$$
    where, again, $S_s = \{(p,q) \mid p+s \ge 2q \ge0 \}$, and for each $q$, $n_{p,q}=0$ for almost all $p$. In particular, $W_s(\MSp)$ is a motivically finite type wedge of copies of $H\Z/\ell$ for all $s$. This concludes the proof.
\end{proof}

\begin{prop}
\label{prop:a.s.s.MSp2}
    In the Adams spectral sequence for $\MSp$, if we let $E_r^{t,u}\coloneqq \oplus_sE_r^{s,t,u}$, we have $E_2^{2u,u}\simeq E_\infty^{2u,u}$, and the spectral sequence converges completely to $(\MSp_{\eta,\ell}^\wedge)^{2u,u}$.
\end{prop}

\begin{proof}
    For all $s,t,u$, we have $E_2^{s,t,u}= \Ext_{A^{*,*}}^{s,(t-s,u)}(H^{*,*}(\MSp),H^{*,*})$ by Proposition \ref{prop: a.s.s.MSp}. In particular, from Proposition \ref{prop:pres.ExtAlgebra}(1), we can see that $E_2^{2u+1,u}=0$. Hence, $d_r^{2u,u}$ vanishes for all $r \ge 2$. By Proposition \ref{prop:pres.ExtAlgebra}(2), $E_2^{r,2,1}=0$ for all $r \ge 2$. Therefore, $d_r^{0,1,1}$ will vanish for all $r \ge 2$. Finally, by Proposition \ref{prop:pres.ExtAlgebra}(3), $E_2^{0,1,1}=H^{1,1}$.

    By Proposition \ref{prop:a.s.s.multiplication}, the product on $E_2$ given by the multiplicative structure on the spectral sequence is the product of the $\Z/\ell$-algebra $\Ext_{A^{*,*}}(H^{*,*}(\MSp),H^{*,*})$ of Proposition \ref{prop:pres.ExtAlgebra}. In particular, the product
    $$E_2^{0,1,1}\otimes_{\Z/\ell} \bigoplus_s E_2^{s,2u-2,u-1} \to \bigoplus_s E_2^{s,2u-1,u}$$
    is surjective. Thus, from the vanishing of differentials $d_r^{0,1,1}$ and $d_r^{2u,u}$, we obtain the vanishing of differentials $d_r^{2u-1,u}$.

    Now, since the differentials $d_r^{2u,u}$ and $d_r^{2u-1,u}$ are zero for $r\ge2$, we have that $E_{r+1}^{2u,u}=E_r^{2u,u}$ for $r \ge 2$, from which it follows that $E_2^{2u,u}=E_\infty^{2u,u}$ and $\lim_r^1E_r^{s,2u,u}=0$. Hence, by Proposition \ref{prop:a.s.s}, the spectral sequence converges completely to $(\MSp_{H\Z/\ell}^\wedge)^{2u,u}$. Finally, by Corollary \ref{cor:EtaEllComplMSp}, we get $(\MSp_{H\Z/\ell}^\wedge)^{2u,u} \simeq (\MSp_{\eta,\ell}^\wedge)^{2u,u}$, which concludes the proof.
\end{proof}

\section{Constructing symplectic bordism classes}

\subsection{Symplectic Thom classes for virtual symplectic bundles}

Let us first consider a rank $2r$ symplectic vector bundle $(V,\omega)$ over a smooth variety $X\in \Sm/k$, with $p_X:X \to \Spec k$ the structure morphism of $X$. Since $\MSp$ is $\Sp$-oriented, we may apply the symplectic Thom class construction, detailed in Proposition~\ref{prop:K_0SpExtensionTh(-)}. This gives us the isomorphism
\[
\th^{\Sp,f}_{\MSp}(V,\omega):\Sigma^{V-\sO_X^{2r}}p_X^*\MSp=\Sigma^{-4r,-2r}\Sigma^Vp_X^*\MSp\xrightarrow{\sim} p_X^*\MSp.
\]
More generally, Proposition \ref{prop:K_0SpExtensionTh(-)} says that the symplectic Thom class construction descends to $K^\Sp_0(X)$. Thus, for $(v,\omega)\in K^\Sp_0(X)$ a virtual symplectic vector bundle on $X$ with virtual rank $2r$, we have the isomorphism
\[
\th^{\Sp,f}_{\MSp}(v,\omega): \Sigma^{-4r,-2r}\Sigma^vp_X^*\MSp\xrightarrow{\sim} p_X^*\MSp.
\]

We now want to define a symplectic Thom isomorphism for $\MSp$-cohomology.

\begin{defn}[Symplectic Thom Isomorphism] \label{def:symplThomisos} Let $X\in \Sm/k$ and $(v,\omega)\in K_0^\Sp(X)$ of virtual rank $2r$. We   define a natural isomorphism
\[
\Th_\MSp(v,\omega):\MSp^{a,b}(X)\xrightarrow{\sim} \MSp^{4r+a, 2r+b}(p_{X\#}\Sigma^v1_X)
\]
as the composition
\begin{multline*}
\MSp^{a,b}(X)=[p_\#1_X,\Sigma^{a,b}\MSp]_{\SH(X)}\simeq [1_X, p_X^*\Sigma^{a,b}\MSp]_{\SH(X)}\\ \xrightarrow[\sim]{\Sigma^{4r+a, 2r+b}\Sigma^{-v}\th^{\Sp,f}_{\MSp}(v,\omega)_*}
[1_X, \Sigma^{-v}p_X^*\Sigma^{4r+a,2r+b}\MSp]_{\SH(X)}\\
\xrightarrow[\sim]{(1)} [\Sigma^v1_X, p_X^*\Sigma^{4r+a,2r+b}\MSp]_{\SH(X)}\xrightarrow[\sim]{(2)}
[p_{X\#}\Sigma^v1_X, \Sigma^{4r+a,2r+b}\MSp]_{\SH(k)}\\=\MSp^{4r+a,2r+b}(p_{X\#}\Sigma^v1_X),
\end{multline*}
where the isomorphism $(1)$ is given by applying $\Sigma^v:\SH(X)\to \SH(X)$ and the isomorphism $(2)$ is given by the adjunction $p_{X\#}\dashv p_X^*$.
\end{defn}

\begin{defn}\label{def:generalsymplthomclasses} For $X\in \Sm/k$ and $(v,\omega)\in K_0^\Sp(X)$ of virtual rank $2r$, we define the $\MSp$-cohomology class
\[
[(v,\omega)]_\MSp\in \MSp^{4r, 2r}(p_{X\#}\Sigma^v1_X)
\]
by
\[
[(v,\omega)]_\MSp:=\Th_\MSp(v,\omega)(1_{\MSp^{0,0}(X)}),
\]
where $1_{\MSp^{0,0}(X)}\in \MSp^{0,0}(X)$ is the unit.
\end{defn}

In particular, for two symplectic vector bundles $(V,\omega)$, $(V',\omega')$ on $X$ of respective ranks $2r, 2r'$, we have the Thom isomorphism
\[
\Th_\MSp((V,\omega)-(V',\omega')):\MSp^{a,b}(X)\xrightarrow{\sim}
\MSp^{4(r-r')+a, 2(r-r')+b}(p_{X\#}\Sigma^{V-V'}1_X),
\]
and the $\MSp$ class $[(V,\omega)-(V',\omega')]_\MSp\in \MSp^{4(r-r'), 2(r-r')}(p_{X\#}\Sigma^{V-V'}1_X)$.

\begin{rmk}
\label{rmk:thomclass}
    In Definition \ref{def:generalsymplthomclasses}, if the negative contribution is null, the class $[(V,\omega)]_{\MSp}$ is just the symplectic Thom class $\th_\Sp^{\MSp}(V,\omega)$. The notation $[-]_{\MSp}$ is indeed intended to be a generalization of the symplectic Thom class to virtual symplectic vector bundles.
\end{rmk}

Now, let us consider the graded abelian group $\oplus_{v\in K_0^\Sp(X)}\MSp^{2\rnk(v), \rnk(v)}(p_{X\#}\Sigma^v1_X)$. We make this into a
$K_0^\Sp(X)$-graded ring by giving the product 
\begin{multline*}
\MSp^{2\rnk(v), \rnk(v)}(p_{X\#}\Sigma^v1_X)\times \MSp^{2\rnk(v'), \rnk(v')}(p_{X\#}\Sigma^{v'}1_X)\\\to \MSp^{2\rnk(v+v'), \rnk(v+v')}(p_{X\#}\Sigma^{v+v'}1_X)
\end{multline*}
induced by the product maps
\begin{multline*}
[\Sigma^v1_X,\Sigma^{a,b}p_X^*\MSp]_{\SH(X)}\times
[\Sigma^{v'}1_X,\Sigma^{a',b'}p_X^*\MSp]_{\SH(X)}\\\to
[\Sigma^v1_X\wedge_X\Sigma^{v'}1_X, \Sigma^{a+a',b+b'}p_X^*\MSp\wedge_Xp_X^*\MSp]_{\SH(X)}\\\to 
[\Sigma^{v+v'}1_X, \Sigma^{a+a',b+b'}p_X^*\MSp]_{\SH(X)},
\end{multline*}
where the last map is induced by the isomorphism $\Sigma^v1_X\wedge_X\Sigma^{v'}1_X \simeq  
\Sigma^{v+v'}1_X$ and the multiplication map $\mu_{p_X^*\MSp}:p_X^*\MSp\wedge_Xp_X^*\MSp\to
p_X^*\MSp$. We denote this product by
\begin{multline*}
\smile:\oplus_{v\in K_0^\Sp(X)}\MSp^{2\rnk(v), \rnk(v)}(p_{X\#}\Sigma^v1_X)\times \oplus_{v\in K_0^\Sp(X)}\MSp^{2\rnk(v), \rnk(v)}(p_{X\#}\Sigma^v1_X)\\
\to \oplus_{v\in K_0^\Sp(X)}\MSp^{2\rnk(v), \rnk(v)}(p_{X\#}\Sigma^v1_X).
\end{multline*}
Note that the unit $1_{\MSp^{0,0}(X)}\in \MSp^{0,0}(X)$ acts as (left and right) unit for this product.

\begin{prop}\label{prop:additivity} Let us take $X\in \Sm/k$. Then sending $(v,\omega)\in K_0^\Sp(X)$ to $[(v,\omega)]_\MSp\in \MSp^{2\rnk(v), \rnk(v)}(p_{X\#}\Sigma^v1_X)$  defines a multiplicative map
\[
[-]_\MSp:K_0^\Sp(X)\to \oplus_{v\in K_0^\Sp(X)}\MSp^{2\rnk(v), \rnk(v)}(p_{X\#}\Sigma^v1_X).
\]
That is,
\[
[(v,\omega)+(v',\omega')]_\MSp=[(v',\omega')]_\MSp\smile [(v',\omega')]_\MSp
\]
and $[0]_\MSp$ is the unit $1_{\MSp^{0,0}(X)}\in \MSp^{0,0}(X)$. 
\end{prop}

\begin{proof}  The multiplicativity follows directly from the multiplicativity of the symplectic Thom classes, as expressed in Lemma~\ref{lem:SympThomMult}.

The class $[0]_\MSp$ is the image of $1_{\MSp^{0,0}(X)}\in \MSp^{0,0}(X)$ under the symplectic Thom isomorphism for the 0 symplectic vector bundle, which is by construction the map $\th^\Sp_{\MSp\Mod}(0):p_X^*\MSp\to p_X^*\MSp$, namely the identity map. 
\end{proof}

\subsection{MSp classes coming from varieties}

Now that we have the classes $[(v,\omega)]_\MSp\in \MSp^{2\rnk(v), \rnk(v)}(p_{X\#}\Sigma^v1_X)$ for each $(v,\omega)\in K_0^\Sp(X)$, we want to use these to construct classes in the symplectic bordism ring.

\begin{defn}
    A \textit{symplectic variety} $(X,\omega)$ is a variety $X\in \Sm/k$ with a symplectic form $\omega_X$ on the tangent bundle $T_X$.
\end{defn}

Thus, for $(X,\omega_X)$ a symplectic variety of dimension $2n$, we have the class
$$[-(T_X,\omega_X)]_\MSp \in \MSp^{-4n,-2n}(p_{X\#}\Sigma^{-T_X}1_X).$$

Now, for any proper map of smooth $S$-schemes $f:X \to Y$, and any $\sE \in \SH(S)$, there is a proper pushforward map $f_*:\sE^{a,b}(p_{X!}(1_X)) \to \sE^{a,b}(p_{Y!}(1_Y))$ described in Construction \ref{constr:properpushforward}. As we did for defining $\MGL$ classes of variety, we can apply this construction with $\sE=\MSp$, $X$ a smooth proper variety of dimension $2d$ over $k$, and $f$ the structure map $p_X:X\to \Spec k$. Let us note that $p_{\Spec k!}=\id_{\SH(k)}$ and $p_{X!}=p_{X\#}\Sigma^{-T_{X/k}}$. Thus, for $(a,b)=(-4d,-2d)$, we have the map
\begin{equation}
\label{eq:dualitymap}
    p_{X*}:\MSp^{-4d,-2d}(p_{X\#}\Sigma^{-T_X}1_X) \to \MSp^{-4d,-2d}(1_k)=\MSp^{-4d,-2d}(k)=\MSp_{4d,2d}(k),
\end{equation}
and we can give the following.

\begin{defn}
\label{def:symplclass}
    For $(X,\omega_X)$ a smooth symplectic variety of dimension $2d$, we define the class
    \[
    [X,-(T_X,\omega_X)]_{\MSp} \coloneqq p_{X*}[-(T_{X},\omega_X)]_{\MSp} \in \MSp_{4d,2d}(k).
    \]
\end{defn}

\begin{rmk}
\label{rmk:Msp-MGL}
    Let us remember that there is a canonical map $\Phi:\MSp \to \MGL$ in $\SH(k)$, given by the fact that $\MGL$, as a $\GL$-oriented cohomology theory, is in particular $\Sp$-oriented. This map is induced by the maps $\BSp_{2r}\to \BGL_{2r}$, which come from the closed immersions of group schemes $\Sp_{2r}\to \GL_{2r}$. By construction, for a symplectic vector bundle $(V,\omega)$ over $X$, $\Phi_*[(V,\omega)]_{\MSp}=\th^{\MGL}(V)$ (see also Remark \ref{rmk:MSp-MGLThomClasses}), and for a smooth proper symplectic variety $(X,\omega_X)$ of dimension $2n$ over $k$, we will have $\Phi_*[X,-(T_X,\omega_X)]_{\MSp}=[X]_{\MGL} \in \MGL_{4n,2n}(k)$, where $[X]_{\MGL}$ is the algebraic cobordism class of $X$ defined by the $\GL$-orientation, see Definition~\ref{defn:MGLclasses}.
\end{rmk}

The main problem in studying the coefficient ring $\MSp_{*,*}(k)$ is the lack of examples of symplectic varieties.

Nevertheless, since the proper pushforward $p_{X*}$ does not depend on a symplectic structure on the tangent bundle, we can still get a symplectic class when the tangent bundle is not symplectic but has a symplectic twist, that is, when one gives an isomorphism
$$p_{X\#}\Sigma^{T_X}1_X \simeq p_{X\#}\Sigma^{W}1_X,$$
where $W$ is a vector bundle over $X$ with a symplectic form on it. To be more general, we now give the symplectic versions of Definition~\ref{defn:stwist} and Definition~\ref{defn:twistclass}.

\begin{defn}\label{defn:sstwist} Given $Y\in \Sm/k$, a {\em stable symplectic twist of $-T_Y$} is a tuple  $((v, \omega), \vartheta, m)$, where $(v, \omega)\in K_0^\Sp(Y)$ is a virtual symplectic bundle and  $\vartheta$  is an isomorphism $\Sigma^{-T_Y}1_Y\xrightarrow{\sim} \Sigma^{2m,m}\Sigma^{v}1_Y$ in $\SH(Y)$.  
\end{defn}

We construct symplectic classes by using these symplectic twists. 

\begin{defn}\label{defn:SymplecticTwistedClass} Let  $Y\in \Sm/k$ with structure morphism $p_Y:Y\to \Spec k$ proper of dimension $d_Y$  and let  $((v, \omega), \vartheta, m)$ be a stable symplectic twist of $-T_Y$. Let $2r=\rnk(v)$. We write the element $[v,\omega]_\MSp\in \MSp^{4r, 2r}(p_{X\#}\Sigma^v1_Y)$ as a map
\[
[v,\omega]_\MSp:p_{Y\#}\Sigma^v1_Y\to \Sigma^{4r,2r}\MSp
\]
in $\SH(k)$, and we define the class $[Y, (v, \omega), \vartheta, m]_\MSp\in \MSp^{4r+2m,2r+m}(k)$ as the composition
\[
1_k\xrightarrow{p_Y^\vee}p_{Y\#}\Sigma^{-T_Y}1_Y
\xrightarrow{p_{Y\#}\vartheta}
p_{Y\#}\Sigma^{2m,m}\Sigma^{v}1_Y\xrightarrow{\Sigma^{2m,m}[v,\omega]_\MSp}\Sigma^{2m+4r,m+2r}\MSp. 
\]
\end{defn}

The case of the trivial symplectic twist for a proper symplectic variety $X, (T_X,\omega_X)$ of dimension $2d$ over $k$ gives back the class $[X, -(T_X,\omega_X)]_\MSp=[X, -(T_X,\omega_X),\id,0]_\MSp\in \MSp_{4d,2d}(k)$.

In what follows, we construct stable symplectic twists of $-T_Y$ by using the Ananyevskiy isomorphisms \eqref{eqn:AnanIso}. We start by illustrating the trivial case of odd dimensional projective spaces.

\begin{exmp}[Odd dimensional projective spaces]
\label{exmp:symplprojspaces}
Let $X=\mathbb{P}^{2n+1}$ for a non-negative integer $n$. We then have the Euler short exact sequence
$$0 \to \mathcal{O}_X \to \mathcal{O}_X(1)^{\oplus 2n+2} \to T_X \to 0.$$
Thus, $\Sigma^{T_X \oplus \mathcal{O}_X}1_X \simeq \Sigma^{\mathcal{O}_X(1)^{2n +2}}1_X$. Now, following our discussed in \S \ref{subsec:TwistedDegree}, we apply the Ananyevskiy isomorphism \eqref{eqn:AnanIso} to get 
the isomorphism in $\SH(X)$
\begin{multline*}
 \Sigma^{\sO_X(1)^{n+1}\oplus \sO_X(-1)^{n+1}}1_X
 \xrightarrow{\Sigma^{\sO_X(1)^{n+1}}(\Anan_{\sO_X(-1),\ldots,\sO_X(-1)})}\\
 \Sigma^{\sO_X(1)^{n+1}}(\Sigma^{\sO_X(1)^{n+1}}1_X)\simeq
 \Sigma^{T_X\oplus \sO_X}1_X.
 \end{multline*}
 The bundle $\mathcal{O}_X(1)^{n+1}\oplus \mathcal{O}_X(-1)^{n+1}$ is of the form $V \oplus V^{\vee}$, and thus, it has a canonical symplectic structure given by the standard symplectic form.
Let then $(W,\phi_W)$ be the symplectic vector bundle given by $\mathcal{O}_X(1)^{n+1}\oplus \mathcal{O}_X(-1)^{n+1}$ equipped with the standard symplectic form. We thus have the 
stable symplectic twist $(-(W,\phi_W), \vartheta, 1)$ of $-T_X$ given by the isomorphism
\[
\vartheta:=\Sigma^{-T_X- \sO_X(-1)^{n+1}}(\Anan_{\sO_X(-1),\ldots,\sO_X(-1)}):\Sigma^{-T_X}1_X\xrightarrow{\sim} \Sigma^{2,1}\Sigma^{-W}1_X.
\]
Via Definition~\ref{defn:SymplecticTwistedClass}, we obtain the twisted class
\[
[\P^{2n+1}, -(W,\phi_W), \vartheta, 1]_\MSp\in \MSp^{-4n-2, -2n-1}(k)=\MSp_{4n+2, 2n+1}(k).
\]
Nevertheless, this class is uninteresting for our purposes, because its image through the motivic Hurewicz map $h_{H\Z}$ lies in homological bidegree $(4n+2,2n+1)$, and since the motivic homology of $\MSp$ is generated by elements of even weight, this class has trivial homology, hence all cohomological classes on it vanish.
\end{exmp}

We now refine the procedure used in Example \ref{exmp:symplprojspaces} to build a family of symplectic bordism classes from even dimensional irreducible varieties, and we will later show that they have non-trivial characteristic numbers. This follows a classical construction of classes in topological symplectic cobordism, due to Stong \cite[Section 4]{Stong-cobordism}.

\begin{constr}
\label{constr:Stongvars} We work over an infinite field $k$.
Let us consider a direct product of an even number $2r$ of odd dimensional projective spaces $$X=\mathbb{P}^{2n_1 +1} \times_k \ldots \times_k \mathbb{P}^{2n_{2r} +1}.$$
Let $\xi=\mathcal{O}(1,\ldots,1)$ be the line bundle $p_1^*\mathcal{O}(1)\otimes \ldots \otimes p_{2r}^* \mathcal{O}(1)$ over $X$, where $p_j$ is the projection on the $j$-th component. The rank $2$ vector bundle $\xi \oplus \xi$ over $X$ is globally generated. By Proposition \ref{prop:BertiniThm} below, there exists a global section $s$ of $\xi \oplus \xi$, such that its vanishing locus is a closed subvariety $i:Y\hookrightarrow X$ of codimension $2$ of $X$, smooth and proper over $k$. Let $\xi'$ denote the restriction $i^*\xi$. By construction, the normal bundle $N_i$ over $Y$ is $i^*(\xi \oplus \xi)=\xi'\oplus \xi'$. 

From the closed immersion $i:Y \hookrightarrow X$, we get the short exact sequence
\[
0\to T_Y\to i^*T_X\to N_i\to 0
\]
of vector bundles over $Y$, which gives the canonical isomorphism $\Sigma^{-T_Y}1_Y\cong \Sigma^{N_i-i^*T_X}1_Y$. Following Example \ref{exmp:symplprojspaces} relative to the projective space $\P^{2n+1}$, we have a symplectic bundle $(W_j,\phi_{W_j})$ on $\P^{2n_j+1}$, and isomorphisms
\[
\vartheta_j:\Sigma^{-T_{\P^{2n_j+1}}}1_{\P^{2n_j+1}}\xrightarrow{\sim}\Sigma^{2,1}\Sigma^{-W_j}1_{\P^{2n_j+1}};\quad j=1,\ldots 2r,
\]
inducing the isomorphism
\[
\wedge_{j=1}^{2r}\vartheta_j:\Sigma^{-T_X}1_X\xrightarrow{\sim} \Sigma^{4r,2r}\Sigma^{-\oplus_{j=1}^{2r}p_j^*W_j}1_X.
\]
In addition, we have the Ananyevskiy isomorphism
\[
\vartheta_0:=\Sigma^{\xi}\Anan_\xi:\Sigma^{N_i}1_Y=\Sigma^{\xi'\oplus \xi'}1_Y\xrightarrow{\sim} \Sigma^{\xi'\oplus \xi^{\prime\vee}}1_Y
\]
and the standard symplectic bundle $(\xi'\oplus \xi^{\prime\vee},\phi_2)$. Putting this together gives the stable symplectic twist of $-T_Y$
\[
((\xi'\oplus \xi^{\prime\vee},\phi_2)-\oplus_{i=1}^{2r}i^*p_j^*(W_j, \phi_{W_j}), \vartheta_Y, 2r),
\]
with $\vartheta_Y$ being the isomorphism
\[
\Sigma^{-T_Y}1_Y\simeq \Sigma^{N_i-i^*T_X}1_Y\xrightarrow{\vartheta_0\wedge
(\wedge_{j=1}^{2r}i^*\vartheta_j)}\Sigma^{4r,2r}\Sigma^{(\xi'\oplus \xi^{\prime\vee})-\sum_ji^*p_j^*W_j}1_Y.
\]
Let $2n:=\dim_kX=2r+2\sum_{j=1}^{2r}n_j$. So $\dim_kY=2n-2$, and $\rnk((\xi'\oplus \xi^{\prime\vee}-\sum_ji^*p_j^*W_j)=-2n-2r+2$. Via Definition~\ref{defn:SymplecticTwistedClass}, this gives us the twisted class
\[
[Y, (\xi'\oplus \xi^{\prime\vee},\phi_2)-\sum_ji^*p_j^*(W_j,\phi_{W_j}), \vartheta_Y, 2r]_\MSp\in
\MSp^{-4n+4,-2n+2}(k).
\]
\end{constr}

By using this construction, we can define a symplectic bordism class as follows:

\begin{defn}
\label{def:symplecticclasses}
Let $X$ and $Y$ as in Construction \ref{constr:Stongvars} and let $d_Y:=\dim_kY$. Let us denote by
$(v_Y,\omega_Y)$ the virtual symplectic bundle $(\xi'\oplus \xi^{\prime\vee},\phi_2)-\sum_ji^*p_j^*(W_j,\phi_{W_j})$, and by $v_Y$ the virtual bundle $\xi'\oplus \xi^{\prime\vee}-\sum_ji^*p_j^*W_j$. Let $\vartheta_Y:\Sigma^{-T_Y}1_Y\xrightarrow{\sim} \Sigma^{4r,2r}\Sigma^{v_Y}1_Y$ be the isomorphism as described in Construction~\ref{constr:Stongvars}. This gives the stable symplectic twist $((v_Y,\omega_Y), \vartheta_Y, 2r)$ of $-T_Y$, and the symplectic bordism class
\[
[Y, (v_Y,\omega_Y),\vartheta_Y, 2r]_\MSp\in \MSp^{-2d_Y, -d_Y}(k)=\MSp_{2d_Y, d_Y}(k).
\]
Occasionally, we will use the shorthand $[Y, (v_Y,\omega_Y)]_\MSp:=[Y, (v_Y,\omega_Y),\vartheta_Y, 2r]_\MSp$.
\end{defn}

Note that, by construction, $Y$ has even dimension.

In Construction \ref{constr:Stongvars}, we assumed the existence of a global section $s$ of the vector bundle $\xi \oplus \xi$ over $X$ transverse to the zero section, whose vanishing locus is a proper closed subvariety. Since the field $k$ is infinite, the existence of this section comes from the following variant of Bertini's theorem:

\begin{prop}[Bertini's Theorem]
\label{prop:BertiniThm}
    Let $X$ be a smooth projective variety over $k$, and let $L$ be a very ample line bundle over $X$. Then there exists a Zariski open subset $U$ of the projective space $\mathbb{P}(H^0(X,L))$ on $H^0(X,L)$ such that, for each section $s$ in $U$, its vanishing locus $V(s)$ defines a codimension one closed subscheme of $X \times_k k(s)$ that is smooth over $k(s)$. In particular, if $k$ is infinite, there exists a section $s \neq 0$ in $H^0(X,L)$ such that $V(s)$ is a codimension one closed subscheme of $X$, smooth over $k$.
\end{prop}

\begin{proof}
    Since $X$ is projective, the $k$-vector space $H^0(X,L)$ of global sections is finite-dimensional, and we have $\P(H^0(X,L))\simeq \P^N$, with $N=h^0(X,L)-1$. Since $L$ is very ample, there exists a closed immersion $i:X \hookrightarrow \P(H^0(X,L))\simeq \P^N$ such that $L$ is the pullback $i^*\mathcal{O}(1)$ of the line bundle $\mathcal{O}(1)\to \P^N$, and the hyperplane sections of $X$ are exactly the closed subschemes $Z(s)\hookrightarrow X$ given by zero loci of global sections $s:X \to L$. 

    Let $d\coloneqq \dim_k X$. For each point $x \in X$, we consider the projective tangent space $\overline{T}_{(X,x)}$ of $X$ at the point $x$ as a $d$-dimensional projective subspace $\P^d \subset \P^N$. Let us consider the dual space $(\P^N)^*$ of $\P^N$, defined as the space parametrizing the hyperplane divisors in $\P^N$, which is isomorphic to $\P^N$, and take the subset $W \subseteq X \times (\P^N)^*$ defined by
    $$W \coloneqq \{(x,H) \in X \times (\P^N)^* \mid \overline{T}_{(x,X)}\subseteq H \}.$$

    Now, we note that for any $d$-dimensional linear subspace $L\simeq \P^d \subseteq \P^N$, the set of hyperplanes of $\P^N$ containing $L$ can be parametrized by the set of hyperplanes in a complementary linear subspace $L'$. Indeed, let $L'\simeq \P^{N-d-1}$ be a linear subspace of $\P^N$ of dimension $(N-d-1)$ such that $L \cap L'=\{0\}$. We have a bijection of sets
    $$\{H \; \text{hyperplane in} \; \P^N \mid H \supset L \} \xrightarrow{\varphi} \{h \; \text{hyperplane in} \; L'\}$$
    given as follows. For a hyperplane $H \subset \P^N$ containing $L$ one takes $\varphi(H) \coloneqq H \cap L'$, and for a hyperplane $h \subset L'$ one takes $\varphi^{-1}(h) \coloneqq \langle L,h \rangle$, the hyperplane in $\P^N$ generated by $L$ and $h$. Clearly, the set of hyperplanes of $L'$ is parametrized by $\P^{N-d-1}$.

    Therefore, for each point $x \in X$, the fibre $f^{-1}(x) \subset W$ over $x$ through $f:W \hookrightarrow X \times \P^N \xrightarrow{p_1} X$ is isomorphic to $\P^{N-d-1}$, and $W \subset X \times (\P^N)^*$ is closed. From $\dim_kX=d$, we deduce that $W$ has dimension $N-1<N$. In particular, the closed subset $p_2(W)\subset (\P^N)^*\simeq \P^N$ is a proper closed subset. We can then consider the open complement $U \coloneqq \P^N \setminus p_2(W)$ of $\P^N$. 
    
    Let us now take a hyperplane $H \in U$. For every point $x \in X$, the intersection $H \cap \overline{T}_{X,x}$ will be a codimension $1$ linear space in $\overline{T}_{X,x}$. This is true in particular for all $x \in X \cap H$, which means that $X \cap H$ is smooth as a scheme over the field of definition $k(H)$ of $H$. Since $(\P^N)^*\simeq \P^N$, we can identify $(\P^N)^*$ with $\P(H^0(X,L)$, $U$ with an open subspace of $\P(H^0(X,L)$, and hyperplanes $H\in U$ with sections $s$ in $U$, and we will have that, for each $s$, $Z(s)$ is a codimension $1$ subscheme of $X \times_k k(s)$, smooth over $k(s)$.
\end{proof}

 We have then achieved the following:

 \begin{thm}
 \label{thm:symplecticclasses}
    Let $k$ be an infinite field. For every $X \in \Sm/k$ given by a product of an even number of odd dimensional projective spaces, of total dimension $2d+2$, let $\xi$ be the very ample invertible sheaf $\sO_X(1,\ldots, 1)$. Then there is a codimension $2$ irreducible proper subvariety $Y$ of $X$, smooth over $k$, given as the zero subscheme of a section of $\xi\oplus \xi$, that defines a class $[Y, (v_Y,\omega_Y)]_\MSp \in \MSp_{4d,2d}(k)$.
 \end{thm}

 \subsection{Segre numbers}

We now need to recall some classical definitions of characteristic classes.

First, let us fix $X\in \Sm/k$ and a rank $r$ vector bundle $V\to X$, with associated flag bundle $\pi:\text{Fl}(V)\to X$. The pullback $\pi^*V$ admits a canonical filtration
\[
0=F^{r+1}V\subset F^rV\subset\ldots\subset F^1V\subset F^0V=\pi^*V
\]
with subquotients $L_i:=F^iV/F^{i+1}V$ all line bundles.

\begin{defn}
    Let $\sE\in \SH(k)$ be an oriented ring spectrum. For $X \in \Sm/k$ and $V \to X$ as above, the {\em $\sE$-Chern roots} of $V$ are the element $\xi_i^\sE(V):=c_1^\sE(L_i)\in \sE^{2,1}(\text{Fl}(V))$. For $\sE=H\Z$, we write $\xi_i$ for $\xi_i^{H\Z}$, and we call the classes $\xi_i(V)$ simply the \emph{Chern roots of $V$}.
\end{defn}

The {\em splitting principle} states that $\pi^*:\sE^{2*,*}(X)\to \sE^{2*,*}(\text{Fl}(V))$ is injective, with image the elements of the form $P(\xi^\sE_1,\ldots, \xi^\sE_r)$ for $P\in \sE^{2*,*}[x_1,\ldots, x_r]^{\Sigma_r}$ a symmetric polynomial (see for example \cite[\S 3.5]{Pan:oriented} for a discussion on the splitting principle for oriented cohomology theories).

\begin{defn} 1. Let $m(x,t)$ be the formal product $m(x,t):=\prod_{i,j=1}^\infty 1+x_i^jt_j$. Let $I:=(i_1\ge i_2\ge\ldots \ge i_r\ge 0)$ be a partition. We let $|I|:=\sum_j i_j$, and $t_I:=t_{i_1}\cdots t_{i_r}$. Writing $m(x,t)$ as $\sum_I m_I(x)t_I$ gives the {\em monomial symmetric function} $m_I(x_1, x_2,\ldots)$, which is a homogeneous polynomial in $x_1, x_2,\ldots$ of degree $|I|$. \\[2pt]
2. Let $\sE\in \SH(k)$ be an oriented ring spectrum and let $V\to X$ be a rank $r$ vector bundle on $X\in \Sm/k$. For a partition $I$, the {\em $I$-th Conner-Floyd Chern class} $c_I^\sE(V)\in \sE^{2|I|,|I|}(X)$ is the element corresponding to $m_I(\xi_1^\sE(V),\ldots, \xi_r^\sE(V), 0,0,\ldots)$ via the splitting principle. \\[5pt]
For $\sE=H\Z$, we write simply  $c_I(V)$. 
\end{defn}

\begin{defn}
\label{defn:NewtonClasses}
    Let $V\to X$ a vector bundle of rank $r$ on $X\in \Sm/k$. We call the \emph{$n$-th Newton class of $V$} the class $$c_{(n)}(V) \coloneqq \sum_{i=1}^{r}\xi_i(V)^n \in H\mathbb{Z}^{2n,n}(X).$$
    In other words, it corresponds to the Conner-Floyd Chern class $c_{I}(V)$ relative to the partition $I=(n)$ of $n$.
\end{defn}

As for Chern classes, one can also consider universal Newton classes $c_{(n)} \in H\Z^{2n,n}(\MGL)$, as follows. For each $r\ge 0$, we have the flag bundle $\Fl :=\Fl(E_r)\xrightarrow{\pi_r} \BGL_r$ associated to the tautological bundle $E_r \to \BGL_r$. By construction of the flag bundles, the inclusions $i_r:\BGL_r \hookrightarrow \BGL_{r+1}$ induce maps $\Fl(i_r):\Fl_r \to \Fl_{r+1}$, and we have
$$\Fl(i_r)^*(\pi_{r+1}^*E_{r+1})\simeq \pi_r^*(i_r^*E_{r+1})\simeq \pi_r^*(E_r \oplus \mathcal{O}_{\BGL_r})\simeq L_1^{(r)} \oplus \ldots \oplus L_r^{(r)} \oplus \mathcal{O}_{\Fl_r}.$$
Since $\pi_{r+1}^*E_{r+1}\simeq L_1^{(r+1)}\oplus \ldots \oplus L_{r+1}^{(r+1)}$, we can suppose $\Fl(i_r)^*(L_j^{(r+1)}) = L_j^{(r)}$ for all $j<r$, and then $\Fl(i_r)^*\xi_j(E_{r+1})=\xi_j(E_r)$ for all $j<r$. We define the bundle $\Fl \to \BGL$ through $\Fl:=\colim_r(\Fl_0 \xrightarrow{\Fl(i_0)} \Fl_1 \xrightarrow{\Fl(i_1)}\ldots)$, and the class $\xi_j(E_r)$, for some $r>>0$, gives then a class $\xi_j \in \Fl$. The maps $\pi_r:\Fl_r \to \BGL_r$ induce a map $\Fl\to \BGL$ by passing to the colimit, and $H\Z^{*,*}(\Fl)\simeq \lim_rH\Z^{*,*}(\Fl_r)$, thus, the map $\pi^*:H\Z^{2*,*}(\BGL)\to H\Z^{2*,*}(\Fl)$ has the properties stated in the splitting principle.

\begin{defn}
\label{defn:univNewtonClasses}
    We call the \emph{$n$-th universal Newton class} the class $c_{(n)} \in H\Z^{2n,n}(\MGL)\simeq H\Z^{2n,n}(\BGL)$ corresponding to the symmetric function $\sum_i \xi_i^n \in H\Z^{2n,n}(\Fl)$ analogously as in Definition \ref{defn:NewtonClasses}.
\end{defn}

We also have the following.

\begin{defn}
\label{defn:SegreNumber}
    For a variety $X$, smooth and proper over $k$ of dimension $n$, the \emph{Segre number of $X$} is the number
    $$s_n(X) \coloneqq \text{deg}_k(c_{(n)}(T_X)) \in \mathbb{Z}.$$
\end{defn}

\begin{lemma}
\label{lemma:newtonclasses}
    \begin{enumerate}
        \item[(1)] If $V\to X$ is a vector bundle on $X\in \Sm/k$, and $n>\dim_kX$, then 
        $c_{(n)}(V)=0$.
        \item[(2)] Newton classes are additive: if $V_1, V_2 \to X$ are two vector bundles, we have
        $$c_{(n)}(V_1 \oplus V_2) = c_{(n)}(V_1)+c_{(n)}(V_2).$$
        \item[(3)] Newton classes are natural: given $V\to X$ a vector bundle and $f:Y\to X$ a morphism in $\Sm/k$, we have $c_{(n)}(f^*(V))=f^*(c_{(n)}(V))$ in $H\Z^{2n,n}(Y)$.
        \item[(4)] If $X = X_1 \times X_2$ is the product of two smooth and proper varieties, both of dimension at least $1$, and $n=\dim_kX$, then $c_{(n)}(T_X)=0$. In other words, Segre numbers vanish on decomposable varieties.
    \end{enumerate}
\end{lemma}

\begin{proof}
    (1) follows from the fact that for $X\in \Sm/k$,  $H\mathbb{Z}^{2n,n}(X)=\CH^n(X)=0$ for $n>\dim_kX$. (2) follows immediately from the definition of Newton classes and the fact that if $V_1$ has Chern roots $\xi_1, \ldots, \xi_r$ and $V_2$ has Chern roots $\xi_{r+1}, \ldots, \xi_{r+s}$, then $V_1\oplus V_2$ has Chern roots $\xi_1, \ldots, \xi_{r+s}$. (3) follows from the naturality of the Chern roots with respect to pullback. 
    (4) follows from the other three, since $T_X\simeq p_1^*T_{X_1}\oplus p_2^*T_{X_2}$, with $p_i:X \to X_i$ the projections. 
\end{proof}

\begin{rmk}
\label{rmk:NewtonClasses} For a partition $I=i_1\ge i_2\ge \ldots\ge i_s\ge 0$, let $I'=(i'_1,i'_2, \ldots, i'_{i_1}, 0,0,\ldots)\in \N^\infty$ be the tuple defined by
\[
i'_m=\#\{j\mid i_j=m\}.
\]
This gives $t_I=t^{I'}=\prod_{j=1}^{i_1}t_j^{i'_j}$. The generating function $m(x,t)=\sum_Im_I(x)t_I$ can be rewritten as $m(x,t)=\sum_Im_I(x)t^{I'}$. If we define $\langle I'\rangle:= \sum_ij\cdot i'_j$, we have  $|I|=\langle I'\rangle$. Given a vector bundle $V\to X$ on some $X\in \Sm/k$, we have the \emph{Conner-Floyd Chern polynomial} 
\[
c^\CF(V)=\sum_Ic_I(V)t_I=\sum_Ic_I(V)t^{I'}.
\]
Now, for $0\to V'\to V\to V''\to 0$ an exact sequence of vector bundles on $X$, the Chern roots of $V$ are those of $V'$ together with those of $V''$, which easily implies the relation
\[
c^\CF(V)= c^\CF(V')\cdot c^\CF(V'').
\]
This shows that the assignment $V\mapsto c^\CF(V)$ extends to a group homomorphism
\[
c^\CF(-):K_0(X)\to (1+\oplus_JH\Z^{2\langle J\rangle, \langle J\rangle}(X)t^J)^\times.
\]
In particular, for each partition $I$ (for instance, $I=(n)$), we have a well-defined map (of sets)
\[
c_I:K_0(X)\to H\Z^{2|I|,|I|}(X).
\]
 \end{rmk}

 \begin{lemma}
\label{lemma:segrenumbers}
    Let $Y, X$ be as in Construction \ref{constr:Stongvars}, with $i:Y \hookrightarrow X$ the closed immersion. Let $p_X:X \to \Spec k$ be the structure map of $X$, and let $2d \coloneqq 2n-2=\dim_kY$. Also, let $\alpha \coloneqq c_1(\xi) \in H\mathbb{Z}^{2,1}(X)$. Then
    $$s_{(2d)}(Y)= (-2)\cdot\deg_k(\alpha^{2d+2}).$$
\end{lemma}

\begin{proof}
    By the definitions of the Segre number and the degree map, we have:
    $$s_{(2d)}(Y) =\text{deg}_k(c_{(2d)}(T_Y))=p_{Y*}(c_{(2d)}(T_Y)),$$
    after making the canonical identification $H\Z^{0,0}(\Spec k)=\Z$.
    
    The short exact sequence
    $$0 \to T_Y \to i^*T_X \to N_i \to 0$$
    gives $c_{(2d)}(T_Y) + c_{(2d)}(N_i)=c_{(2d)}(i^*T_X)$, but since $X$ is decomposable, Lemma \ref{lemma:newtonclasses}(3)-(4) gives $c_{(2d)}(i^*T_X)=i^*c_{(2d)}(T_X)=0$. Then $$c_{(2d)}(T_Y)=-c_{(2d)}(N_i)=-c_{(2d)}(i^*(\xi \oplus \xi))=-i^*c_{(2d)}(\xi \oplus \xi).$$
    Moreover, $c_{(2d)}(\xi \oplus \xi)=(c_1(\xi))^{2d}+(c_1(\xi))^{2d}=2\alpha^{2d}$. We have then
    $$\text{deg}_k(c_{(2d)}(T_Y))= \text{deg}_k(-2i^*\alpha^{2d}).$$
    
    We now claim that $i_*i^*x = \alpha^2 x$ for all $x \in H\mathbb{Z}^{4d,2d}(X)$. To see this, let $1^Y\in H\mathbb{Z}^{0,0}(Y)$, $1^X\in H\mathbb{Z}^{0,0}(X)$   be the respective units. By the projection formula (Proposition~\ref{prop:PushPull}(2)) we have 
  \[
  i_*i^*x=i_*(1^Y\cdot i^*x)= i_*(1^Y)\cdot x.
  \]
  Next, we recall that $Y$ is the zero-locus of a global section $s$ of $\xi \oplus \xi$ that is transverse to the zero section $s_0$. This gives us the transverse cartesian diagram
\[
\xymatrix{
Y\ar[r]^i\ar[d]^-i&X\ar[d]^{s_0}\\
X\ar[r]^-s&\xi \oplus \xi.
}
\]
Thus, by the push-pull formula (Proposition~\ref{prop:PushPull}(1)) and Lemma~\ref{lem:FirstChernClassFacts}(1), we have
\[
i_*(1^Y)=i_*i^*(1_X)=s^*s_{0*}(1_X)=c_2(\xi \oplus \xi).
\]
By the Whitney sum formula \eqref{enum:ChernClassAxioms}(2), we have $c_2(\xi \oplus \xi)=c_1(\xi)^2=\alpha^2$, 
so $ i_*i^*x=\alpha^2\cdot x$, as claimed.
  
  Thus
 \begin{align*}
 \text{deg}_k(c_{(2d)}(T_Y))&=  p_{Y*}(c_{(2d)}(T_Y))\\
 &=p_{Y*}(-i^*c_{(2d)}(\xi \oplus \xi))\\
 &=p_{X*}i_*(-i^*c_{(2d)}(\xi \oplus \xi))\\
&=p_{X*}(-\alpha^2\cdot 2\cdot\alpha^{2d})\\
&=(-2)\cdot\deg_k(\alpha^{2d+2})
\end{align*}    
\end{proof}

We now have all the necessary information needed to refine Theorem \ref{thm:symplecticclasses} via the following result, which follows the ideas of \cite[Section 4]{Stong-cobordism}.

\begin{thm}
\label{thm:symplclasses2} Let $k$ be an infinite field.
    Let $\ell$ be an odd prime different from $\text{char}(k)$, and let $\nu_{\ell}$ denote the $\ell$-adic valuation. Then there exists a family $\{(Y_{2d}, (v_{2d}, \omega_{2d}))\}_{d \ge1}$, with $Y_{2d}\in \Sm/k$ proper of dimension $2d$ over $k$, and $(v_{2d}, \omega_{2d})\in K_0^\Sp(Y_{2d})$ a virtual symplectic bundle defining a symplectic twist of $T_{Y_{2d}}$, thereby giving  a class $[Y_d,(v_{2d}, \omega_{2d})]_{\MSp} \in \MSp_{4d,2d}(k)$, such that
    $$\nu_{\ell}(s_{2d}(Y_{2d}))=
    \begin{cases}
        0, \; \; \; \text{if} \;\;  2d \neq \ell^i-1 \; \forall \; i \\
        1, \; \; \; \text{if} \; \; 2d=\ell^r -1, \; r \ge 1.
    \end{cases}$$
\end{thm}

\begin{proof}
    
Let $d$ be any positive integer such that $2d \neq \ell^i-1$ for all $i$. Let
$$2d+2= a_0 +a_1\ell + \ldots +a_r \ell^r$$
be the $\ell$-adic expansion of $2d+2$. Let us define
$$X_{2d+2} \coloneqq (\mathbb{P}^1)^{\times a_0} \times (\mathbb{P}^{\ell})^{\times a_1} \times \ldots \times (\mathbb{P}^{\ell^r})^{\times a_r},$$
where all products are products over $\Spec k$. Then $\dim(X_{2d+2})=2d+2$. Let $Y_{2d} \subset X_{2d+2}$ be the zero locus of a general section of the vector bundle $\xi \oplus \xi \to X_{2d+2}$ as in Construction \ref{constr:Stongvars}. Then $s_{(2d)}(Y_{2d})=\deg_k (-2\alpha^{2d+2})$ by Lemma \ref{lemma:segrenumbers}. 
Let $\alpha_{i,j}\coloneqq c_1(p_{i,j}^*\mathcal{O}(1))$, where $p_{i,j}$ is the projection of $X_{2d+2}$ onto the $j$-th component of $(\mathbb{P}^{\ell^i})^{\times a_i}$, so that 
$$\alpha=(\alpha_{0,1}+\alpha_{0,2}+\ldots +\alpha_{0,a_0}+\alpha_{1,1}+\ldots + \alpha_{1,a_1}+\ldots +\alpha_{r,a_r}).$$
In particular:
$$\alpha^{2d+2}=(\alpha_{0,1}+\alpha_{0,2}+\ldots +\alpha_{0,a_0}+\alpha_{1,1}+\ldots + \alpha_{1,a_1}+\ldots +\alpha_{r,a_r})^{a_0 +a_1\ell +\ldots +a_r \ell^r}.$$

Since $X_{2d+2}$ is a product of projective spaces, we have K\"unneth formula
$$H\Z^{2*, *}(X)\simeq H^{2*,*}(\P^1)^{\otimes a_0}\otimes\ldots\otimes H^{2*,*}(\P^{\ell^r})^{\otimes a_r},$$
with all tensor products over $\Z$, and the isomorphism given by taking the respective pullbacks and cup products. This follows by repeated application of the projective bundle formula (Theorem \ref{thm:PBF}). For all projective spaces $\P^n$, the top degree term in $H^{2*,*}(\P^n)$ is $H^{2n,n}(\P^n)\simeq \Z$, with generator $c_1(\mathcal{O}_{\P^n}(1))^n$, and $p_{\P^n*}(c_1(\mathcal{O}_{\P^n}(1))^n)=1\in H\Z^{0,0}(\Spec k)\simeq \Z$, where $1$ is the unit. Thus, through a series of uses of the push-pull formula and the projection formula (Proposition \ref{prop:PushPull}(1)-(2)), we see that the top degree term in $H\Z^{2*, *}(X_{2d+2})$ is $H\Z^{4d+4, 2d+2}(X_{2d+2})\simeq \Z$, with generator 
\[
\alpha_*^*:=\alpha_{0,1} \cdot\alpha_{0,2}\cdots\alpha_{0,a_0} \cdot \alpha_{1,1}^{\ell} \cdots \alpha_{1,a_1}^{\ell} \cdots \alpha_{r,1}^{\ell^r} \cdots \alpha_{r,a_r}^{\ell^r},
\]
and $\deg_k(\alpha_*^*)=1$. Therefore, $\deg_k(2\alpha^{2d+2})$ is the coefficient of 
$$\alpha_{0,1} \cdot \alpha_{0,2}\cdots\alpha_{0,a_0} \cdot \alpha_{1,1}^{\ell} \cdots \alpha_{1,a_1}^{\ell} \cdots \alpha_{r,1}^{\ell^r} \cdots \alpha_{r,a_r}^{\ell^r}$$
in the expansion of $2\alpha^{2d+2}$.

The multinomial theorem gives the combinatorial formula
$$(x_1 +x_2 + \ldots x_m)^n=\sum_{i_1 + \ldots +i_m=n}\frac{n!}{i_1!i_2!\ldots i_m!} x_1^{i_1}x_2^{i_2}\ldots x_m^{i_m}.$$
Moreover, multinomial coefficients satisfy the identity
$$\frac{n!}{i_1!i_2!\ldots i_m!} =\binom{i_1}{i_1} \binom{i_1+i_2}{i_2}\cdots \binom{i_1+i_2+\ldots +i_m}{i_m}.$$
Thus, $\deg_k(2\alpha^{2d+2})$ is the coefficient
\begin{multline*}
     2 \cdot \frac{(2d+2)!}{1!1!\cdots 1!\ell! \cdots \ell! \ell^{2}! \cdots \ell^r!}= \\ = 2 \binom{1}{1}\binom{2}{1} \cdots \binom{a_0}{1} \binom{a_0+\ell}{\ell}\binom{a_0+2\ell}{\ell} \cdots \binom{a_0+a_1\ell}{\ell} \binom{a_0+a_1+\ell^2}{\ell^2} \cdots\binom{2d+2}{\ell^r}.
\end{multline*}
The product of the first $a_0$ binomial coefficients is $a_0!$. The product of the following $a_1$ binomial coefficients, reduced modulo $\ell$, is $a_1!$. And so on. Iterating this, we get
$$\deg_k(2\alpha^{2d+2}) \equiv\ 2a_0!a_1! \cdots a_r! \; \; \; \text{mod}\;\ell.$$
In particular, $\ell$ does not divide $\deg_k(2\alpha^{2d+2})$, so $\nu_{\ell}(s_{2d}(Y_d))=0$.

Let now $d$ be a positive integer such that $2d=\ell^r-1$ for some $r$. In this case, let us define
$$X_{2d+2} \coloneqq \mathbb{P}^1 \times (\mathbb{P}^{\ell^{r-1}})^{\ell},$$
so that $\dim(X_{2d+2})=\ell^r+1=2d+2$, and take $Y_{2d} \subset X_{2d+2}$ obtained again as in Construction \ref{constr:Stongvars}. Now we let $p_0$ be the projection of $X_{d+2}$ on the first factor $\mathbb{P}^1$, and $p_{1,j}:X_{2d+2} \to \mathbb{P}^{\ell^{r-1}}$ be the projection on the $j$-th component of $(\mathbb{P}^{\ell^{r-1}})^{\ell}$. As before, we write 
$$\alpha=\alpha_0 +\alpha_{1,1} +\ldots + \alpha_{1,\ell},$$
and then
$$\alpha^{2d+2}=(\alpha_0 +\alpha_{1,1} +\ldots + \alpha_{1,\ell})^{\ell^r+1}.$$
So, $\deg_k(2\alpha^{2d+2})$ is now the coefficient of
$$\alpha_0 \cdot \alpha_{1,1}^{\ell^{r-1}} \cdot \alpha_{1,2}^{\ell^{r-1}} \cdots \alpha_{1,\ell}^{\ell^{r-1}}$$
in the expansion of $2\alpha^{\ell^r+1}$. By using the multinomial theorem as before, we get
$$\deg_k(2\alpha^{2d+2})=2\binom{1}{1}\binom{1+\ell^{r-1}}{\ell^{r-1}}\binom{1+2\ell^{r-1}}{\ell^{r-1}} \cdots \binom{1+\ell^r}{\ell^{r-1}}.$$
Let us note that:
$$\binom{1+j\ell^{r-1}}{\ell^{r-1}}=\binom{j\ell^{r-1}}{\ell^{r-1}}\frac{1+j\ell^{r-1}}{1+j\ell^{r-1}-\ell^{r-1}}.$$
In the right hand term, the fraction does not contain the factor $\ell$ in either the nominator or denominator, while the binomial coefficient contains the factor $\ell$ only when $j=\ell$ and never contains the factor $\ell^2$. We conclude that $\ell$ divides $\deg_k(2\alpha^{2d+2})$ and $\ell^2$ does not, that is, $\nu_{\ell}(s_{2d}(Y_{2d}))=1$.
\end{proof}

From now on, we will adopt the following notation.

\begin{defn}
\label{defn:SymplecticY's}
    For $\ell$ a fixed odd prime different from $\text{char}(k)$, we denote by
    $$[Y_{2d}, (v,\omega)]_{\MSp} \in \MSp_{4d,2d}(k)$$
    the symplectic bordism class associated as in Definition \ref{def:symplecticclasses} to the variety $Y_{2d}$ defined in Theorem \ref{thm:symplclasses2}. 
\end{defn}

\begin{rmk}
\label{rmk:ComparingNewtonClasses}
    $s_{2d}(Y_{2d})$ also computes $\text{deg}_k(c_{(2d)}(i^*W-(\xi' \oplus \xi'^{\vee})))$. Indeed, Lemma \ref{lemma:newtonclasses}(2)-(3) gives 
    \begin{multline*}
        c_{(2d)}(\xi' \oplus \xi'^{\vee})=i^*c_{(2d)}(\xi) + i^*c_{(2d)}(\xi^{\vee})= i^*\alpha^{2d}+i^*(-\alpha)^{2d}=\\i^*(2\alpha^{2d})=i^*c_{(2d)}(\xi \oplus \xi)=c_{(2d)}(N_i)=-c_{(2d)}(T_{Y_{2d}}),
    \end{multline*}
    and also, Lemma \ref{lemma:newtonclasses}(1) gives
    $$c_{(2d)}(i^*W)=i^*c_{(2d)}(W)=0$$ because $W$ is the direct sum of pullbacks $p_j^*W_j$ on $Y_{2d}$, where $W_j$ is a vector bundle over a space $\P^{2n_j+1}$ of dimension less than $2d$.
\end{rmk}

\subsection{Detecting generators of the symplectic cobordism ring}

In Section \ref{chapter:A.S.S}, we studied the $\Z/\ell$-algebra $\Ext_{A^{*,*}}(H^{*,*}(\MSp),H^{*,*})$ and the convergence of the motivic mod $\ell$ Adams spectral sequence for $\MSp$ by following an analogous procedure used in \cite[Section 5-6]{lev:ellcoh} for $\MGL$ and $\MSL$. We now briefly recall the original results for $\MGL$ in order to be able to compare the spectral sequences.

\subsubsection{The Adams spectral sequence for $\MGL$}

The $\Z/\ell$-algebra $\Ext_{A^{*,*}}(H^{*,*}(\MGL),H^{*,*})$ and the Adams spectral sequence for $\MGL$ are studied in \cite[Sections 5-6]{lev:ellcoh}. As a first step, they present the structure of the mod $\ell$ motivic cohomology $H^{*,*}(\MGL)$ as an $A^{*,*}$-module, giving the following result.

\begin{lemma}[\cite{lev:ellcoh}, Lemma 5.9]
\label{lemma:decompMGL}
Let $P'$ denote the set of all partitions that are not $\ell$-adic (see Definition \ref{defn:l-adicpartition}). Then
    $$H^{*,*}(\MGL) \simeq \bigoplus_{P'}M_B u_\omega.$$
\end{lemma}

This leads to the following presentation of $E_2$-page of the Adams spectral sequence, namely, the trigraded $\Z/\ell$-algebra $\Ext_{A^{*,*}}(H^{*,*}(\MGL),H^{*,*})$.

\begin{prop}[\cite{lev:ellcoh}, Proposition 5.7]
\label{prop:ExtMGL}
We have:
    \begin{itemize}
            \item $\Ext_{A^{*,*}}^{s,(t-s,u)}(H^{*,*}(\MGL),H^{*,*})=0$ if $t>2u$.
            \item $\Ext_{A^{*,*}}^{s,(2u-s,u)}(H^{*,*}(\MGL),H^{*,*})$ is polynomial in the generators:
            \begin{itemize}
                \item $1 \in \Ext_{A^{*,*}}^{0,(0,0)}(H^{*,*}(\MGL),H^{*,*})$;
                \item $z'_{(k)}\in \Ext_{A^{*,*}}^{0,(-2k,-k)}(H^{*,*}(\MGL),H^{*,*})$, for $k\ge 1$, and $k\neq \ell^i-1 \; \forall \; i \ge 0 $;
                \item $h_r' \in \Ext_{A^{*,*}}^{1,(1-2 \ell^r,1-\ell^r)}(H^{*,*}(\MGL),H^{*,*})$, for $r \ge 0$.
            \end{itemize}
            \item $\Ext_{A^{*,*}}^{0,(2u-1,u)}(H^{*,*}(\MGL),H^{**})$ is $H^{1,1}$ if $u=1$, and $0$ otherwise, and the product map
        $$   H^{1,1} \otimes_{\Z/\ell} \bigoplus_{s,u}\Ext_{A^{*,*}}^{s,(2u-s,u)}(H^{*,*}(\MGL),H^{*,*}) \to \bigoplus_{s,u} \Ext_{A^{*,*}}^{s,(2u-s+1,u+1)}(H^{*,*}(\MGL),H^{*,*})$$
        is surjective.
        \end{itemize}
\end{prop}

Let us recall that Lemma \ref{lemma:decompmsp} provides a decomposition of $H^{*,*}(\MSp)$, analogous to that of $H^{*,*}(\MGL)$ as described above in Lemma~\ref{lemma:decompMGL}, by taking the direct sum over the set $P$ of even non $\ell$-adic partitions, rather than over all non $\ell$-adic partitions. The resulting presentation of the $E_2$-page of the Adams spectral sequence for $\MSp$ is Proposition \ref{prop:pres.ExtAlgebra}.

Going back to $\MGL$, we now recall that $\MGL$ is a cellular spectrum by Theorem \ref{thm:mglcellular}. Moreover, by \cite[Proposition 6.10]{lev:ellcoh}, $W_s(\MGL)$ is a motivically finite type wedge of copies of $H\Z/\ell$. These two conditions, thanks to Proposition \ref{prop:a.s.s}, lead to the following result.

\begin{thm}[\cite{lev:ellcoh}, Theorem 6.11]
\label{thm:a.s.sMGL}
The motivic mod $\ell$ Adams spectral sequence for $\MGL$ is of the form
$$E_2^{s,t,u}=\Ext_{A^{*,*}}^{s,(t-s,u)}(H^{*,*}(\MGL),H^{*,*}) \Rightarrow (\MGL_{H\Z/\ell}^\wedge)^{t,u}.$$
\end{thm}

By \cite[Corollary 2.3]{lev:ellcoh}, $\MGL$ is a $(-1)$-connected spectrum. Thus, by Theorem \ref{thm:Mantovani}, we have $\MGL_{H\Z/\ell}^\wedge \simeq \MGL_{\eta,\ell}^\wedge$. Moreover, the spectrum $\MGL$ is $\eta$-complete by \cite[Lemma 2.1]{Rondigs:Hopf}, namely $\MGL^\wedge_\eta = \MGL$. Lastly, \cite[Lemma 6.3]{lev:ellcoh} shows that $(\MGL^\wedge_\ell)^* \simeq (\MGL^*)^\wedge_\ell$, where the right-hand term is the $\ell$-adic completion of the ring $\MGL^*$. By putting this together, one gets the following.

\begin{prop}
    We have $(\MGL_{H\Z/ \ell}^\wedge)^* \simeq (\MGL^*)^\wedge _\ell$.
\end{prop}

The Adams spectral sequence for $\MGL$ is studied in \cite[Section 6.5]{lev:ellcoh}, where it is shown that the differentials $d_r$ vanish for $r\ge2$, giving the following corollary.

\begin{corollary}[\cite{lev:ellcoh}, Proposition 6.12]
    Let $E(\MGL)$ be the spectral sequence of Theorem \ref{thm:a.s.sMGL}, and $E_r^{t,u}(\MGL)\coloneqq \oplus_sE_r^{s,(t,u)}$. Then $E_2^{2u,u}(\MGL)\simeq E_\infty^{2u,u}(\MGL)$, and the spectral sequence converges completely to $(\MGL_\ell^\wedge)^{2u,u}$.
\end{corollary}

\begin{rmk}
\label{rmk:liftingGenerators}
    As noted in \cite[Section 6.6]{lev:ellcoh}, along the proof of Theorem B, the Adams spectral sequence for $\MGL$ has associated filtration $F^*(\MGL^*)^\wedge_\ell$, graded by $\oplus_u E_\infty^{2u,u}(\MGL)\simeq \oplus_u E_2^{2u,u}(\MGL) \coloneqq \oplus_u \oplus_s E_2^{s,(2u,u)}(\MGL)$, which is a graded $\Z/\ell$-algebra, with $E_2^{0,(0,0)}(\MGL)=\Z/\ell$ and $F^m((\MGL^0)_\ell^\wedge)=(\ell^m)\Z_\ell$. Thus, the polynomial generators $z'_{(k)}$ and $h'_r$ of Proposition \ref{prop:ExtMGL} lift, respectively, to polynomial generators
    $$\tilde{z}'_{(k)} \in F^0((\MGL^{-2k,-k})^\wedge_\ell) \; \; \; \text{and} \; \; \; \tilde{h}'_r \in F^1((\MGL^{-2(\ell^r-1),-(\ell^r-1)})_\ell^\wedge)$$
    for the graded $\Z/\ell$-algebra $(\MGL^*)^\wedge_\ell$, and these generators are unique up to decomposable elements and elements in $\ell \cdot (\MGL^*)^\wedge_\ell$.
\end{rmk}

\subsubsection{A generating criterion}

For this subsection, let us denote by $p$ the exponential characteristic of the base field $k$, that is, if $k$ has characteristic zero, then $p=1$, otherwise $p=\chr k$.

Let $h:= h_{H\Z}:\MGL_{*,*}(\Spec k) \to H\Z_{*,*}(\MGL)$ be the motivic Hurewicz map as defined in Remark \ref{rmk:HurewiczMap}.

For any cohomological class $\alpha \in H\mathbb{Z}^{a,b}(\MGL)$ and any homological class $\gamma \in H\mathbb{Z}_{c,d}(\MGL)$, we have the pairing $\langle \alpha,\gamma \rangle \in H\mathbb{Z}^{a-c,b-d}(\Spec k)$, which is the pairing \eqref{eqn:Pairing} with $\sE=H\Z$.

\begin{defn}
    For $x \in \MGL^{-n}=\MGL^{-2n,-n}(\Spec k)=\MGL_{2n,n}(\Spec k)$, and $c_{(n)} \in H\Z^{2n,n}(\MGL)$ the $n$-th universal Newton class (Definition \ref{defn:univNewtonClasses}), we denote by $c_{(n)}(x)$ the integer
$$c_{(n)}(x) := \langle c_{(n)},h(x) \rangle \in H\Z^{0,0}(\Spec k)=\Z.$$
\end{defn}

Let us review, again from \cite[Section 6.6]{lev:ellcoh}, a generating criterion for $(\MGL^*)^\wedge_\ell$.

Recalling Theorem \ref{thm:comparisonLazard}, and since we only consider the subring generated by homogenous elements, the $\ell$-adic completion $(\MGL^*)^\wedge_\ell$ is the graded ring $\MGL^* \otimes \Z_\ell \simeq \Z_\ell[x_1, x_2, \ldots]$, with $x_i$ in degree $-i$. Thus, the $\Z_\ell$-module of indecomposable elements in $(\MGL^*)^\wedge_\ell$ is the sum $\oplus_{i\ge 1}\Z_\ell \cdot [x_i]$. Since, by Proposition \ref{prop:MGL-classes}, $\MGL_{2d,d}(\Spec k)[1/p]$ is generated by classes $[X]_{\MGL}$ of schemes of dimension $d$, and Newton classes vanish on decomposable elements by Lemma \ref{lemma:newtonclasses} (4), we have that $c_{(n)}(-)$ vanishes on decomposable elements of $(\MGL^*)^\wedge_\ell$. Also, the $\ell$-adic valuation of $c_{(d)}(x_d)$ will be $1$ when $d=\ell^r-1$, for some $r \ge 1$, and 0 otherwise. One then gets the following.

\begin{prop}[\cite{lev:ellcoh}, from proof of Theorem B, Section 6.6]
\label{prop:CriterionMGL}
    A family of element $\{x_d'\}_{d \ge 1}$, with $x_d' \in (\MGL^{-d})^\wedge_\ell$ form a family of polynomial generators of $(\MGL^*)^\wedge_\ell$ if and only if
    \begin{equation*}
        \nu_\ell(c_{(d)}(x_d'))=
        \begin{cases}
            1 \; \; \; \; \text{for} \; \; d= \ell^r-1, \; r\ge 1 \\
            0 \; \; \; \text{for} \; \; d \neq \ell^r-1 \; \forall r \ge 1.
        \end{cases}
    \end{equation*}
\end{prop}

Let us recall that the canonical map $\Phi:\MSp \to \MGL$ induces, by composition, a map $\Phi_*:\MSp^{*,*}\to \MGL^{*,*}$ between the respective coefficient rings. Similarly, by taking $(\eta,\ell)$-completions and then considering the diagonal of the respective coefficient rings, it also induces the map
$$\Phi_*:(\MSp_{\eta,\ell}^\wedge)^* \to (\MGL_\ell^\wedge)^*\simeq (\MGL^*)_\ell^\wedge.$$

\begin{prop}\label{prop:PolynomialMSp}The map $\Phi_*:(\MSp_{\eta, \ell}^\wedge)^* \to (\MGL^*)_\ell^\wedge=\Z_\ell[x_1, x_2,\ldots]$ is injective, with image the polynomial subring with generator $x_2, x_4,\ldots$.
\end{prop}

\begin{proof} The map $\Phi: \MSp \to \MGL$ induces a map between the two respective Adams spectral sequences. By looking at the decompositions of $H^{*,*}(\MSp)$ and $H^{*,*}(\MGL)$ given by Lemmas \ref{lemma:decompmsp} and \ref{lemma:decompMGL}, we see that the induced map between the two respective $E_2$-pages is an isomorphism on the summands of the $\Ext$-algebras corresponding to the duals of the summands $M_B u_\omega$, with $w\in P$, namely with $\omega$ being an even (necessarily non-$\ell$ adic) partition. In particular, the $E_2$ page for $\MSp$ is a split summand of the $E_2$ page for $\MGL$. Since both spectral sequences degenerate at $E_2$, this implies, using the complete convergence, that the map 
$\Phi_*:(\MSp_{\eta, \ell}^\wedge)^* \to (\MGL^*)_\ell^\wedge$ induced by $\Phi$ is injective.

We can consider the filtration $F^*(\MSp_{\eta,\ell}^\wedge)^*$ associated to the spectral sequence $E(\MSp)$ for $\MSp$. Analogously to the case of $F^*(\MGL^*)_\ell^\wedge$ in Remark \ref{rmk:liftingGenerators}, the generators $z_{(2k)}$ and $h_r$ of $E_2(\MSp)$ lift to generators $\tilde{z}_{(2k)}$ and $\tilde{h}_r$ of the graded $\Z_\ell$ algebra $(\MSp_{\eta,\ell}^\wedge)^*$, with $r\ge 0$, $k\ge 1$, $2k \neq \ell^i-i$ for any $i \ge 0$,  uniquely up to sums with decomposable elements and elements in $\ell \cdot (\MSp_{\eta,\ell}^\wedge)^*$. We see that, through the inclusion $\Phi_*:(\MSp_{\eta, \ell}^\wedge)^* \hookrightarrow (\MGL^*)_\ell^\wedge$, $\tilde{z}_{(2k)}$ and $\tilde{h}_r$ map to the polynomial generators $\tilde{z}'_{(2k)}$ and $\tilde{h}'_r$ respectively, up to decomposable elements and elements in $\ell \cdot (\MGL^*)^\wedge_\ell$. Indeed, $z_{(2k)}$ is the dual of $u_{(2k)}$,  hence it belongs to the summand of $E_2(\MSp)$ corresponding to the summand $M_B u_{(2k)}$, while the elements $h_r$ are generators of the $\Z/\ell$ algebra $\Ext_B(\Z/\ell,H^{*,*})\simeq \Ext_{A^{*,*}}(M_B,H^{*,*})$, so they belong to the summand of $E_2(\MSp)$ corresponding to the summand $M_B u_{(0)}$. 

Thus, the $\Z_\ell$-algebra generators $\tilde{z}_{(2k)}$ and $\tilde{h}_r$ of $(\MSp_{\eta, \ell}^\wedge)^*$ map to polynomial generators of the $\Z_\ell$-algebra $(\MGL^*)_\ell^\wedge=\Z_\ell[\{\tilde{z}'_{(k)}, \tilde{h}'_r\}_{k,r}]$, mapping to (possibly new) polynomial generators in all even degrees. Thus $\Phi_*$ identifies $(\MSp_{\eta, \ell}^\wedge)^*$ with this polynomial subalgebra of $(\MGL^*)_\ell^\wedge$.
\end{proof}

From the generating criterion for $(\MGL^*)^\wedge_\ell$ of Proposition \ref{prop:CriterionMGL}, we can obtain the following one.

\begin{prop}
\label{prop:CriterionMSp}
    A family of element $\{y_{2d}'\}_{d \ge 1}$, with $y_{2d}' \in (\MSp^\wedge_{\eta,\ell})^{-2d}$ form a family of polynomial generators of $(\MSp_{\eta,\ell}^\wedge)^*$ if and only if
    \begin{equation*}
        c_{(2d)}(\Phi_* y_{2d}')=
        \begin{cases}
            \lambda \in \Z_\ell^\times & \text{for} \; \; 2d \neq \ell^r-1 \; \forall r \ge 1 \\
            \lambda' \cdot \ell, \; \lambda' \in \Z_\ell^\times & \text{for} \; \; 2d= \ell^r-1, \; r\ge 1.
        \end{cases}
    \end{equation*}
\end{prop}

\begin{proof} We retain the notation used in the proof of Proposition~\ref{prop:PolynomialMSp}. 

From Remark \ref{rmk:liftingGenerators}, we see that the elements $\tilde{h}_r'$, for $r \ge 1$, give generators of $(\MGL^*)^\wedge_\ell$ of degrees $1-\ell^r$ respectively, and the elements $\tilde{z}_{(k)}'$ give generators in the remaining non-zero degrees. Thus, by Proposition \ref{prop:CriterionMGL}, we have
$$\nu_\ell(c_{(k)}(\tilde{z}_k'))=0 \; \; \; \text{and} \; \; \; \nu_\ell(c_{(\ell^r-1)}(\tilde{h}'_r))=1.$$
Therefore, we also have 
\begin{equation}
\label{eq:l-adic}
    \nu_\ell(c_{(2k)}(\Phi_* \tilde{z}_{2k}))=0 \; \; \; \text{and} \; \; \; \nu_\ell(c_{(\ell^r-1)}(\Phi_* \tilde{h}_r))=1.
\end{equation}
If now $y\in (\MGL^n)^\wedge_\ell$ is a polynomial generator, then modulo $\ell\cdot (\MGL^n)^\wedge_\ell$ and decomposable elements we must have that $y$ is a $\Z_\ell$-unit times a generator $\tilde{h}_r'$ (if $n=1-\ell^r$) or $\tilde{z}_k'$ (if $n$ is not of the form $1-\ell^r$). Since Newton classes vanish on decomposable elements, we see that, for arbitrary $y\in (\MGL^n)^\wedge_\ell$, $y$ is a polynomial generator if and only if $\nu_\ell(c_{(n)}(y))=1$ if $n$ is of the form  $1-\ell^r$, and $\nu_\ell(c_{(n)}(y))=0$ if $n$ is not of this form. 

From Proposition~\ref{prop:PolynomialMSp}, we see that the sub-$\Z_\ell$-algebra $(\MSp^\wedge_{\eta,\ell})^*$ is the polynomial sub-$\Z_\ell$-algebra of $(\MGL^*)^\wedge_\ell$
with polynomial generators the $\tilde{h}'_r$, $r\ge1$ and the $\tilde{z}_k'$ for $k$ even (and non-$\ell$-adic). Applying the above criterion, we see that an element $y_{2d}' \in (\MSp^\wedge_{\eta,\ell})^{-2d}$, $d \ge 1$, is a polynomial generator of $(\MSp^\wedge_{\eta,\ell})^*$ if and only if  
 $$
 \nu_\ell(c_{(2d)}(\Phi_* y_{2d}'))=
        \begin{cases}
            1 \; \; \; \; \text{for} \; \; 2d= \ell^r-1, \; r\ge 1 \\
            0 \; \; \; \text{for} \; \; 2d \neq \ell^r-1 \; \forall r \ge 1.
        \end{cases}
$$
This gives the result.
\end{proof}

Let $X=\prod_{i=1}^{2r}\mathbb{P}^{2n_i +1}$ and $Y\subset X$ as in Construction \ref{constr:Stongvars}, with $\dim_kY=2d=\dim_kX-2$. From Definition \ref{def:symplecticclasses}, we have the stable symplectic twist  $((\xi'\oplus \xi^{\prime\vee},\phi_2)-\oplus_{i=1}^{2r}i^*p_j^*(W_j, \phi_{W_j}), \vartheta_Y, 2r)$ of $-T_Y$, giving the twisted class
\[
 [Y,(v_Y,\omega_Y)]_{\MSp} \in \MSp_{4d,2d}(k);\quad (v_Y,\omega_Y)=
 (\xi'\oplus \xi^{\prime\vee},\phi_2)-\oplus_{i=1}^{2r}i^*p_j^*(W_j, \phi_{W_j}).
 \]
  In order to simplify the notation, we introduce the following shorthand.
 \begin{defn}
 \label{defn:convention}
 We let $-T_Y'\in K_0(Y)$ be the virtual vector bundle $\mathcal{O}_Y^{2r}+v_Y$, giving the canonical isomorphism $\Sigma^{4r, 2r}\Sigma^{v_Y}1_Y \simeq \Sigma^{-T_Y'}1_Y$.
 \end{defn} 

 In particular, the isomorphism $\vartheta_Y$ of Construction \ref{constr:Stongvars} can be read as $$\vartheta_Y:\Sigma^{-T_Y}1_Y \xrightarrow{\sim}\Sigma^{-T_Y'}1_Y.$$

 Let us recall that, by Remark \ref{rmk:NewtonClasses}, $c_{(2d)}(-T_Y)$ and $c_{(2d)}(-T_Y')$ are defined.

\begin{prop}
    We have
    $$c_{(2d)}(\Phi_* [Y,(v_Y,\omega_Y)]_\MSp)=(-1)^{n_Y}\deg_k(c_{(2d)}(-T_Y)),$$
    where $n_Y=1 + \Sigma_{i=1}^{2r}(n_i+1)$
\end{prop}

\begin{proof} We first note that, for $x \in \MGL^{-n}=\MGL^{-2n,-n}(\Spec k)=\MGL_{2n,n}(\Spec k)$, and $c_{(n)} \in H\Z^{2n,n}(\MGL)$, we have
\[
c_{(n)}(x) := \langle c_{(n)},h(x) \rangle =c_{(n)}\circ x\in H\Z^{0,0}(\Spec k)=\Z
\]
by Remark \ref{rmk:HurewiczMap}.

Next, expanding the full definition of $[Y,(v_Y,\omega_Y)]_\MSp$ as $[Y,(v_Y,\omega_Y),\vartheta_Y, 2r]_\MSp$, since the map $\Phi:\MSp\to \MGL$ is induced by the maps $\BSp_{2r}\to \BGL_{2r}$, we have
\[
\Phi_* [Y,(v_Y,\omega_Y),\vartheta_Y, 2r]_\MSp =[Y, v_Y+\sO_Y^{2r},\vartheta_Y]_\MGL\in 
\MGL^{-2d,-d}(\Spec k),
\]
where $v_Y\in K_0(Y)$ is the image of $(v_Y,\omega_Y)\in K_0^\Sp(Y)$ under the evident map 
$K_0^\Sp(Y)\to K_0(Y)$ forgetting the symplectic structure. Thus
\[
c_{(2d)}\circ \Phi_* [Y,(v_Y,\omega_Y),\vartheta_Y, 2r]_\MSp=c_{(2d)}\circ [Y, -T_Y',\vartheta_Y]_\MGL.
\]
Furthermore, $-T_Y'$ has virtual rank $-2d$ and $\vartheta_Y$ is a composition of $2r+1$ isomorphisms of the form $\Anan_{e,-L_1,\ldots, -L_s}$, $2r$ for the factors $\P^{2n_1+1}$ and one coming from the rank $2$ bundle $\xi'\oplus \xi'$. In particular, for each factor $\P^{2n_i+1}$ of $X$ we have $s=n_i+1$, and for $\xi'\oplus \xi'$ we have $s=1$. Thus, the sum of all the indices $s$ relative to the isomorphisms $\Anan_{e,-L_1,\ldots, -L_s}$ is $n_Y$, and it follows from Corollary~\ref{cor:TwistClassComp} that 
\[
c_{(2d)}\circ [Y, -T_Y',\vartheta_Y]_\MGL=(-1)^{n_Y}\cdot\deg_k (c_{(2d)}(-T_Y'))
=(-1)^{n_Y}\cdot\deg_k (c_{(2d)}(-T_Y')).
\]

Finally, up to isomorphisms, $-T_Y$ and $-T_Y'$ are both (virtual) sums in $K_0(Y)$ of classes of line bundles, where the only difference is that some of the line bundles $L_i$ appearing in $-T_Y$ get replaced with $L_i^\vee$. But if $-T_Y=\sum_i\epsilon_i[L_i]$, with $\epsilon_i\in\{\pm1\}$, then we have by additivity of the Newton classes that  $c_{(2d)}(-T_Y)=\sum_i\epsilon_i\cdot c_1^{H\Z}(L_i)^{2d}$ and $c_{(2d)}(-T_Y')=\sum_i\epsilon_i\cdot c_1^{H\Z}(L_i^{\otimes \tau_i})^{2d}$ for suitable $\tau_i\in\{\pm1\}$.
Since $H\Z$ has additive formal group law (Example \ref{exmp:orientedHZ}), we have 
\[
c_1(L_i^{\otimes -1})^{2d} =(-c_1(L_i))^{2d}=c_1(L_i)^{2d},
\]
so $c_{(2d)}(-T_Y')=c_{(2d)}(-T_Y)$, completing the proof. 
\end{proof}

Let us now consider the varieties $Y_{2d}$ constructed in Theorem \ref{thm:symplclasses2}, and their associated symplectic classes $[Y_{2d}]_\MSp \coloneqq [Y_{2d},(v_Y,\omega_Y)]_\MSp$ (Definition \ref{defn:SymplecticY's}).

Still by Remark \ref{rmk:NewtonClasses} and additivity of Newton classes we have $c_{(2d)}(-T_{Y_{2d}})=-c_{(2d)}(T_{Y_{2d}})$, hence 
\begin{equation}
\label{eq:generatingconditions}
   \nu_\ell(c_{(2d)}(\Phi_*[Y_{2d}]_\MSp))=\nu_\ell(s_{2d}(Y_{2d})). 
\end{equation}
The identity \eqref{eq:generatingconditions} and Theorem \ref{thm:symplclasses2} tell us that the symplectic classes $[Y_{2d}]_\MSp \in \MSp^{-4d,-2d}$ satisfy the generating criterion for $(\MSp^\wedge_{\eta,\ell})^*$ established in Proposition \ref{prop:CriterionMSp}.

We have then achieved the following.

\begin{thm}
\label{thm:ClassOfGenerators}
    Let $\ell$ be a fixed odd prime different from $\chr(k)$, and let $[Y_{2d}]_\MSp \coloneqq [Y_{2d},(v_Y,\omega_Y)]_\MSp \in \MSp^{-4d,-2d}(\Spec k)$ be the symplectic classes of Definition \ref{defn:SymplecticY's}. Then the images of the classes $[Y_{2d},(v_Y,\omega_Y)]_\MSp$ in $(\MSp_{\eta,\ell}^\wedge)^*$ give a family of polynomial generators for $(\MSp_{\eta,\ell}^\wedge)^*$ over $\Z_\ell$.
\end{thm}

This theorem also allows us to prove the following.

\begin{prop}
\label{prop:eta-ellCompl}
    For each odd prime $\ell$ different from $\chr k$, we have $((\MSp^\wedge_\eta)^*)^\wedge_\ell \simeq (\MSp_{\eta,\ell}^\wedge)^*$.
\end{prop}

\begin{proof}
     Let us use the notation $(-)_{r-\tors}$ for the $r$-torsion elements of a ring. To show that there is an injective map $((\MSp^\wedge_\eta)^*)^\wedge_\ell\hookrightarrow (\MSp_{\eta,\ell}^\wedge)$, we can follow the proof of \cite[Lemma 6.13 (3)]{lev:ellcoh} for $\MSL$. In particular, for all $s,t \in Z$, $n \ge 0$, there is a diagram
    $$
    \begin{tikzcd}
        0 \arrow[r] & \MSp^{s,t}/\ell^n \arrow[r] \arrow[d] & (\MSp/\ell^n)^{s,t} \arrow[r] \arrow[d] & (\MSp^{s+1,t})_{\ell^n-\tors} \arrow[r] \arrow[d] & 0 \\
        0 \arrow[r] & \MSp^{s,t}/\ell^{n-1} \arrow[r] & (\MSp/\ell^{n-1})^{s,t} \arrow[r] & (\MSp^{s+1,t})_{\ell^{n-1}-\tors} \arrow[r] & 0,
    \end{tikzcd}
    $$
    from which, by considering $\eta$-completions and taking the limit over $n$, we get the exact sequence
    \begin{multline*}
        0 \to \lim_n((\MSp_\eta^\wedge)^{s,t}/\ell^n) \to \lim_n((\MSp_\eta^\wedge /\ell^n)^{s,t}) \to \lim_n((\MSp_\eta^\wedge)^{s+1,t}) \\ \to \lim_n^1((\MSp_\eta^\wedge)^{s+1,t}/\ell^n.
    \end{multline*}
    This gives us an injection $((\MSp^\wedge_\eta)^*)^\wedge_\ell \simeq \lim_n((\MSp_\eta^\wedge)^{2s,s}/\ell^n) \hookrightarrow \lim_n((\MSp_\eta^\wedge /\ell^n)^{2s,s}) \simeq (\MSp_{\eta,\ell}^\wedge)^*$. To show the surjectivity, it is sufficient to show that $(\MSp_{\eta,\ell}^\wedge)^*$, seen as a subalgebra of $(\MGL_{\eta,\ell}^\wedge)^*$ through $\Phi_*$, is generated, as $\Z_\ell$-module, by the images $\Phi_*[Y_{2d}, (v_Y,\omega_Y)]_\MSp$. But we already know this, from Theorem \ref{thm:ClassOfGenerators}. This concludes the proof.
\end{proof}

From Proposition \ref{prop:CriterionMSp}, by working on the $\ell$-torsion of $(\MSp_\eta^\wedge)^*$ for each $\ell$ separately, we can also prove the following global criterion.

\begin{thm}
\label{thm:FinalResult}
    Let $({\overline{\MSp}^\wedge_\eta})^*[1/2p]$ denote the quotient of the graded ring $({\MSp^\wedge_\eta})^*[1/2p]$ by its maximal subgroup that is $\ell$-divisible for all odd primes $\ell \neq p$. Then $({\overline{\MSp}^\wedge_\eta})^*[1/2p]$ is a polynomial ring over $\Z[1/2p]$, and a family $\{y_{2d}'\}_{d \ge 1}$, with $y'_{2d} \in ({\overline{\MSp}^\wedge_\eta})^{-2d}[1/2p]$, forms a family of polynomial generators for $({\overline{\MSp}^\wedge_\eta})^*[1/2p]$ if and only if 
    \begin{equation*}
        c_{(2d)}(\Phi_* y_{2d}')=
        \begin{cases}
            \lambda \in \Z[1/2p]^\times & \text{for} \; \; 2d \neq \ell^r-1, \; \; \text{for all prime} \;  \ell \neq 2,p; \; \forall r \ge 1 \\
            \lambda' \cdot \ell, \; \lambda' \in \Z_\ell^\times & \text{for} \; \; 2d= \ell^r-1, \; \; \ell \; \text{prime}, \; \ell \neq 2,p; \; r\ge 1.
        \end{cases}
    \end{equation*}
\end{thm}

\begin{proof} $(\overline{\MSp}_\eta^\wedge)^*[1/2p]$ is a $\Z[1/2p]$-module. Then for all odd prime $\ell$ different from $p$, we have 
    $$(\overline{\MSp}_\eta^\wedge)^*[1/2p] \otimes_{\Z[1/2p]}\Z_\ell \simeq  (\overline{\MSp}_\eta^\wedge)^* \otimes_\Z \Z_\ell \simeq ((\MSp_\eta^\wedge)^*)^\wedge_\ell \simeq (\MSp_{\eta,\ell}^\wedge)^*,$$
    where the last isomorphism is Proposition \ref{prop:eta-ellCompl}.

    By using Proposition \ref{prop:CriterionMSp} as a generating criterion for $(\overline{\MSp}_\eta^\wedge)^*[1/2p] \otimes_{\Z[1/2p]}\Z_\ell$, for each $\ell$, we obtain that an element $y'_{2d} \in {(\overline{\MSp}^\wedge_\eta})^{-2d}[1/2p]$ is a generator if and only if
    \begin{equation*}
    c_{(2d)}(\Phi_* y_{2d}')=
        \begin{cases}
            \lambda \in \Z[1/2p]^\times & \text{if} \; \; 2d \neq \ell^r-1, \; \; \text{for all prime} \;  \ell \neq 2,p; \; \forall r \ge 1 \\
            \lambda' \cdot \ell, \; \lambda' \in \Z_\ell^\times & \text{if} \; \; 2d= \ell^r-1, \; \; \ell \; \text{prime}, \; \ell \neq 2,p; \; r\ge 1.
        \end{cases}
    \end{equation*}

\end{proof}

\begin{rmk}
    Let us note that Theorem \ref{thm:symplclasses2} also provides a family $\{[Y_{2d}, (v_Y,\omega_Y)]_\MSp\}_{d\ge 1}$ of polynomial generators for $({\overline{\MSp}^\wedge_\eta})^*[1/2p]$. If $2d=\ell^r-1$ for some odd prime $\ell \neq p$ and some $r \ge 1$, one takes the variety $Y_{2d} \subset \P^1 \times (\P^{\ell^{r-1}})^{\times \ell}$ constructed in the proof of Theorem \ref{thm:symplclasses2} for that specific $\ell$, and its respective symplectic class $[Y_{2d}, (v_Y,\omega_Y)]_\MSp$. If $2d \neq \ell^r-1$ for all $\ell,r$, one can take the variety $Y_{2d}\subset (\P^1)^{\times (2d+2)}$ constructed in the same proof for any $\ell > 2d+2$.
\end{rmk}

\printbibliography

\end{document}
