\documentclass[preprint,showpacs,preprintnumbers,amsmath,amssymb,nofootinbib]{revtex4}

\usepackage{amsthm}
\newtheorem{thm}{Theorem}[section]
%(or
%\newtheorem{Thm}{Theorem} )
%\newtheorem{corollary}[thm]{Corollary}  % here [thm] is the same name as above
\newtheorem{corollary}%[section]
{Corollary}[section]
\newtheorem{proposition}%[section]
{Proposition}[section]

\newtheorem{lemma}{Lemma}
\newtheorem{definition}{Definition}[section]
\usepackage{mathrsfs}
\usepackage{etex}
\usepackage{amssymb,amsthm,amscd,amsbsy,array}
\usepackage{bm}% bold math
\usepackage{amsmath,dsfont}
%\usepackage{ulem} % underline, strikethrough, etc.
\usepackage{soul} % underline, strikethrough, etc.
\usepackage{graphics,graphicx,xcolor}
%\usepackage[all]{xy}
%% COLORS

\usepackage[colorlinks=true, pdfstartview=FitV, linkcolor=blue, citecolor=blue, urlcolor=blue]{hyperref} % hyperref
\newcommand{\cyan}{\textcolor{cyan}}
\newcommand{\magenta}{\textcolor{magenta}}
\newcommand{\red}{\textcolor{red}}
\newcommand{\yellow}{\textcolor{yellow}}
\newcommand{\blue}{\textcolor{blue}}
\newcommand{\green}{\textcolor{green}}
\newcommand{\orange}{\textcolor{orange}}
\newcommand{\purple}{\textcolor[rgb]{0.5,0.0,0.5}}
\newcommand{\gb}{\quad\colorbox{green}}
\newcommand{\rb}{\colorbox{red}}
\newcommand{\yb}{\colorbox{yellow}}
\newcommand{\blueb}{\colorbox{blue}}
\newcommand{\dgreen}{\textcolor[rgb]{0,0.5,0}}

\newenvironment{redtext}{\color{red}}
{\ignorespacesafterend}
\newenvironment{bluetext}{\color{blue}}{\ignorespacesafterend}
\newenvironment{greentext}{\color{green}}{\ignorespacesafterend}
\newenvironment{magentatext}{\color{magenta}}{\ignorespacesafterend}
\newenvironment{cyantext}{\color{cyan}}{\ignorespacesafterend}
\newenvironment{orangetext}{\color{orange}}
{\ignorespacesafterend}

\newcommand{\bmagenta}{\begin{magentatext}}
\newcommand{\emagenta}{\end{magentatext}}
\newcommand{\bcyan}{\begin{cyantext}}
\newcommand{\ecyan}{\end{cyantext}}
\newcommand{\bblue}{\begin{bluetext}}
\newcommand{\eblue}{\end{bluetext}}
\newcommand{\bred}{\begin{redtext}}
\newcommand{\ered}{\end{redtext}}
\newcommand{\mc}[1]{{\textcolor{red}{#1}}}
\newcommand{\tred}[1]{\mc{\tt #1}}
\newcommand{\bgreen}{\begin{greentext}}
\newcommand{\egreen}{\end{greentext}}
\newcommand{\borange}{\begin{orangetext}}
\newcommand{\eorange}{\end{orangetext}}

%\numberwithin{equation}{section} 
%\renewcommand{\theequation}{\thesection.
%\arabic{equation}}
%\let\ssection=\section
%\renewcommand{\section}{\setcounter{equation}{0}\ssection}
\newcommand{\beq}{\begin{equation}}
\newcommand{\eeq}{\end{equation}}
\newcommand{\bec}{\begin{center}}
\newcommand{\ec}{\end{center}}

\newcommand{\ex}{{\mathrm{exo}}}
\newcommand{\ma}{{\mathrm{mag}}}

%%\setlength{\voffset}{-1.0truecm}
%\hoffset=-10mm
%\textwidth=180mm  %125mm
%\textheight=%245mm
%250mm %220mm %185mm
%\parindent=8mm
%\evensidemargin=1pt
%\oddsidemargin=0pt

%%%\frenchspacing

%%%%%%%%%%%%%%%%%%%%%%%%%%%%%%%%%%%%%%%%%%%%%%%%%%%%%%%%%%%%%%%%%%%%%%%%%%%%%%
%%%%%%%%%%%%%%%%%%%%%%%%%%%%%%% Various macros %%%%%%%%%%%%%%%%%%%%%%%%%%%%%%%
%%%%%%%%%%%%%%%%%%%%%%%%%%%%%%%%%%%%%%%%%%%%%%%%%%%%%%%%%%%%%%%%%%%%%%%%%%%%%%

\def\STOP{\bigskip\rb{\large \bf \textcolor{yellow}{stop here}}\bigskip}

\newcommand{\SLe}{{Sturm{\strut}-Liouville\;}}
\newcommand{\PT}{{P\"oschl{\strut}-Teller\;}}
\newcommand{\dPT}{{derived{\strut} P\"oschl{\strut}-Teller\;}}
\newcommand{\dGauss}{{derived{\strut} Gauss\;}}
\newcommand{\scarf}{{scarf\;}}
\newcommand{\GW}{{gravitational wave\,}}
%\newcommand{\BH}{{Black Hole\,}}
\newcommand{\BH}{{black hole\,}}
\newcommand{\KN}{{Kerr-Newman\,}}
\newcommand{\BHs}{{black holes\,}}

%\def\GW{{gravitational wave\,}}
\newcommand{\ba}{{\mathbf{a}}}
\newcommand{\bA}{{\mathbf{A}}}
\newcommand{\bB}{{\mathbf{B}}}
\newcommand{\cA}{{\mathcal{A}}}
\newcommand{\bbA}{{\mathbb{A}}}
\newcommand{\balpha}{\boldsymbol{\alpha}}
\newcommand{\bbgamma}{\boldsymbol{\gamma}}
\newcommand{\bsigma}{\boldsymbol{\sigma}}
\newcommand{\bTheta}{\boldsymbol{\Theta}}
\newcommand{\ad}{\mathrm{ad}}
\newcommand{\Ad}{\mathrm{Ad}}
\newcommand{\AdS}{\mathrm{AdS}}
\newcommand{\aff}{\mathfrak{aff}}
\newcommand{\rA}{\mathrm{A}}
\newcommand{\bb}{{\mathbf{b}}}
\newcommand{\vB}{{\mathbf{B}}}
\newcommand{\vE}{{\mathbf{E}}}
\newcommand{\vD}{{\mathbf{D}}}
\newcommand{\vH}{{\mathbf{H}}}
\newcommand{\vj}{{\mathbf{j}}}
\newcommand{\bbeta}{\boldsymbol{\beta}}
\newcommand{\bpi}{\boldsymbol{\pi}}
\newcommand{\tbeta}{\widetilde{\beta}}
\newcommand{\bone}{\boldsymbol{1}}
\newcommand{\fb}{\mathfrak{b}}
\newcommand{\tB}{\widetilde{B}}
\newcommand{\tbB}{\widetilde{\vB}}
\newcommand{\NU}{\mathrm{NU}}
\newcommand{\rnu}{\mathrm{nu}}
\newcommand{\calS}{{\mathcal{S}}}
\newcommand{\calA}{{\mathcal{A}}}

\newcommand{\CCarr}{{\mathrm{CCarr}}}
\newcommand{\SCarr}{{\mathrm{SCarr}}}
\newcommand{\ccarr}{{\mathfrak{ccarr}}}
\newcommand{\scarr}{{\mathfrak{schcarr}}}
\newcommand{\be}{{\bm{e}}}
\newcommand{\bx}{{\bm{x}}}
\newcommand{\bX}{{\bm{X}}}
\newcommand{\bxi}{{\bm{\xi}}}
\newcommand{\bzeta}{{\bm{\zeta}}}
\newcommand{\baromega}{\bar{\omega}}
\newcommand{\wpsi}{\widetilde{\psi}}
\newcommand{\mB}{{\mathscr{B}}}
\newcommand{\mT}{{\mathscr{T}}}
\newcommand{\mH}{{\mathscr{H}}}
\newcommand{\mE}{{\mathscr{E}}}
\newcommand{\spin}{{\mathrm{S}}}

%%%
\newcommand{\gE}{\mathfrak{E}}
\newcommand{\gA}{\mathfrak{A}}
\newcommand{\gG}{\mathfrak{G}}
\newcommand{\gB}{\mathfrak{B}}
\newcommand{\gC}{\mathfrak{C}}
\newcommand{\gP}{\mathfrak{P}}
\newcommand{\gD}{\mathfrak{D}}
\newcommand{\bc}{{\mathbf{c}}}
\newcommand{\bC}{{\mathbf{C}}}
\newcommand{\bbC}{\mathbb{C}}
\newcommand{\cC}{{\mathcal{C}}}
\newcommand{\cD}{{\mathcal{D}}}
\newcommand{\BMS}{{\mathrm{BMS}}}
\newcommand{\rC}{{\mathrm{C}}}
\newcommand{\coad}{\mathrm{coad}}
\newcommand{\Coad}{\mathrm{Coad}}
\newcommand{\Cor}{\mathrm{Cor}}
\newcommand{\GammaU}{{}^{U}\!\Gamma}
\newcommand{\gammaU}{{}^{U}\!\gamma}
\newcommand{\cNc}{\mathfrak{cnc}}
\newcommand{\cGal}{\mathrm{CG}}
\newcommand{\cgal}{\mathfrak{cg}}
\newcommand{\cga}{\mathfrak{cga}}
\newcommand{\sla}{\mathfrak{sl}}
\newcommand{\tgamma}{\widetilde{\gamma}}
\newcommand{\Carr}{{\mathrm{Carr}}}
\newcommand{\carr}{{\mathfrak{carr}}}
\newcommand{\ConfCarr}{{\mathcal{CC}}}
\newcommand{\confcarr}{{\mathfrak{ccarr}}}
\newcommand{\hGamma}{{\widehat{\Gamma}}}
\newcommand{\pSch}{{\mathrm{Sch}}} % Projective-Schr\"odinger
\newcommand{\psch}{{\mathfrak{sch}}} % Projective-Schr\"odinger
\def\cSch{{{\rm c-Sch}}}
\def\tSch{{\widetilde{\rm Sch}}}
\def\Barg{{{\rm Barg}}}
\newcommand{\barg}{\mathfrak{barg}}
\newcommand{\Gal}{\mathrm{Gal}}
\newcommand{\gal}{\mathfrak{gal}}
\newcommand{\Eucl}{\mathrm{Eucl}}
\newcommand{\Poinc}{\mathrm{Poinc}}
\newcommand{\poinc}{\mathfrak{poinc}}
\def\csch{{{\rm c-sch}}}
\def\aand{{\quad\text{\small and}\quad}}
\def\where{{\quad\text{\small where}\quad}}
\def\with{{\quad\text{\small with}\quad}}
\def\ie{{\;\text{\small i.e.}\;}}
\def\but{{\quad\text{\small but}\quad}}

\newcommand{\bd}{{\mathbf{D}}}
\newcommand{\bD}{{\mathbf{D}}}
\newcommand{\bbD}{\mathbb{D}}
%\newcommand{\cD}{{\mathcal{D}}}
\renewcommand{\d}{\mathrm{d}}
\newcommand{\rD}{{\mathrm{D}}}
\newcommand{\rB}{{\mathrm{B}}}
\newcommand{\Diff}{{\mathrm{Diff}}}
\newcommand{\diag}{\mathrm{diag}}
\newcommand{\rdiv}{\mathrm{div}}
\newcommand{\Div}{\mathrm{Div}}
\newcommand{\fc}{\mathfrak{c}}

\newcommand{\rE}{{\mathrm{E}}}
\newcommand{\cE}{{\mathcal{E}}}
\newcommand{\cB}{{\mathcal{B}}}
\newcommand{\cG}{{\mathcal{G}}}
\newcommand{\Ein}{{\mathrm{Ein}}}
\newcommand{\veta}{\boldsymbol{\eta}}
\newcommand{\fe}{\mathfrak{e}}
\newcommand{\trE}{\widetilde{\rE}}
\newcommand{\tbE}{\widetilde{\vE}}
\newcommand{\tE}{\widetilde{E}}

\newcommand{\bF}{{\bf F}}
\newcommand{\cF}{{\mathcal{F}}}
\newcommand{\tcF}{\widetilde{\mathcal{F}}}
\newcommand{\hcF}{\widehat{\mathcal{F}}}
\newcommand{\sF}{{\mathcal{F}}}
\newcommand{\hF}{{\widehat{F}}}
\newcommand{\wcF}{{\widetilde{\cal F}}}

\newcommand{\bg}{{\mathbf{g}}}
\newcommand{\bG}{{\bf G}}
\newcommand{\rg}{\mathrm{g}}
\newcommand{\BG}{{G}}
\newcommand{\hrg}{\widehat{\rg}}
\newcommand{\rG}{\mathrm{G}}
\newcommand{\hG}{{\widehat{G}}}
\newcommand{\fg}{\mathfrak{g}}
\newcommand{\fG}{\mathfrak{G}}
\newcommand{\bgamma}{\boldsymbol{\gamma}}
\newcommand{\CGal}{{\mathrm{CGal}}}
\newcommand{\gl}{{\mathrm{gl}}}
\newcommand{\GL}{{\mathrm{GL}}}
\newcommand{\PGL}{\mathrm{PGL}}
%\newcommand{\GW}{\mathrm{GW}}
%\newcommand{\GW}{{gravitational wave}}

\newcommand{\hp}{\hat{\bp}}
\newcommand{\bh}{\mathbf{h}}
\newcommand{\bH}{{\bf H}}
\newcommand{\rh}{h}
\newcommand{\cH}{{\mathcal{H}}}
\newcommand{\Hom}{\mathrm{Hom}}
\newcommand{\fh}{\mathfrak{h}}
\newcommand{\fH}{\mathfrak{H}}
\newcommand{\cI}{{\mathcal{I}}}
\newcommand{\id}{\mathrm{id}}
\newcommand{\Id}{\mathrm{Id}}
\renewcommand{\Im}{\mathrm{Im}}
\newcommand{\bj}{{\mathbf{j}}}
\newcommand{\bJ}{{\mathbf{J}}}
\newcommand{\cJ}{{\mathcal{J}}}
\newcommand{\sJ}{{\mathsf{J}}}
\newcommand{\bk}{\mathbf{k}}
\newcommand{\bK}{{\mathbf{K}}}
\newcommand{\rk}{\mathrm{k}}
\newcommand{\cK}{{\mathcal{K}}}
\newcommand{\bkappa}{\boldsymbol{\kappa}}
\newcommand{\bL}{{\bf L}}
\newcommand{\rl}{\mathrm{\ell}}
%\newcommand{\cL}{{\mathcal{L}}}
\newcommand{\cL}{{\mathscr{L}}}
\newcommand{\cS}{{\mathscr{S}}}
\newcommand{\tLambda}{\widetilde{\Lambda}}
\newcommand{\bM}{{\mathbf{M}}}
\newcommand{\cM}{\mathcal{M}}
\newcommand{\tM}{\widetilde{M}}
\newcommand{\hM}{\widehat{M}}
\newcommand{\bn}{{\mathbf{n}}}
\newcommand{\bN}{{\mathbf{N}}}
\newcommand{\bbN}{\mathbb{N}}
\newcommand{\cN}{\mathcal{N}}
\newcommand{\nablaU}{{}^{U}\!\nabla}
\newcommand{\New}{\mathrm{New}}
\newcommand{\new}{\mathfrak{new}}
\newcommand{\bO}{{\bf O}}
\newcommand{\rO}{{\mathrm{O}}}
\newcommand{\ro}{{\mathrm{o}}}
\newcommand{\cO}{{\mathcal{O}}}
\newcommand{\CO}{\mathrm{CO}}
\newcommand{\bomega}{{\boldsymbol{\omega}}}
\newcommand{\bOmega}{{\boldsymbol{\Omega}}}
\newcommand{\bsell}{{\boldsymbol{\ell}}}
\newcommand{\bp}{{\mathbf{p}}}
\newcommand{\wbp}{\widehat{{\bf p}}}
\newcommand{\bP}{{\bf P}}
\newcommand{\cP}{\mathcal{P}}
\newcommand{\bbP}{\mathbb{P}}
\newcommand{\oP}{{\overline{P}}}
\newcommand{\Proj}{\mathrm{Proj}}
\newcommand{\proj}{\mathfrak{proj}}
\newcommand{\PSL}{\mathrm{PSL}}

\newcommand{\bq}{{\bf q}}
\newcommand{\bQ}{{\bf Q}}
\newcommand{\cQ}{\mathcal{Q}}
\newcommand{\oQ}{{\overline{Q}}}

\newcommand{\bu}{\mathbf{u}}
\newcommand{\hg}{\hat{g}}
\newcommand{\br}{{\bm{r}}}
%\newcommand{{x}}{{\bm{X}}}
\newcommand{\bR}{{\bf R}}
\newcommand{\cR}{\mathcal{R}}
\newcommand{\bbR}{\mathbb{R}}
\newcommand{\bbS}{\mathbb{S}}
\newcommand{\rR}{\mathrm{R}}
\newcommand{\Ric}{\mathrm{Ric}}
\newcommand{\RP}{\bbR\mathrm{P}}
\newcommand{\grad}{{\vnabla}}
\newcommand{\rot}{\vnabla\times}
\renewcommand{\Re}{\mathrm{Re}}
\newcommand{\dive}{{\vnabla\cdot}}
\newcommand{\fs}{\mathfrak{s}}

\newcommand{\bs}{{\bf s}}
\newcommand{\bS}{{\bf S}}
%\newcommand{\cS}{{\mathcal{S}}}
\newcommand{\SA}{\mathrm{SA}}
\newcommand{\Sch}{\mathrm{Sch}}
\newcommand{\sch}{\mathfrak{sch}}
\newcommand{\sign}{\mathrm{sign}}
\newcommand{\SE}{\mathrm{SE}}
\newcommand{\SL}{\mathrm{SL}}
\newcommand{\Sl}{\mathfrak{sl}}
\newcommand{\SO}{\mathrm{SO}}
\newcommand{\ISO}{\mathrm{ISO}}
\newcommand{\SIM}{\mathrm{SIM}}
\newcommand{\ISIM}{\mathrm{ISIM}}
\newcommand{\DISIM}{\mathrm{DISIM}}
\newcommand{\Sp}{\mathrm{Sp}}
%\newcommand{\rso}{\mathrm{so}}
\newcommand{\se}{\mathfrak{se}}
%\newcommand{\soo}{\mathfrak{so}}
\newcommand{\rso}{\mathfrak{so}}
\newcommand{\SU}{\mathrm{SU}}
\newcommand{\Surf}{\mathrm{Surf}}
\newcommand{\sv}{\mathfrak{sv}}

\newcommand{\bbg}{\mathbb{g}}
\newcommand{\hnabla}{\widehat{\nabla}}
\newcommand{\hX}{Y%\widehat{X}
}
\newcommand{\Conf}{{\mathrm{Conf}}}
\newcommand{\conf}{{\mathrm{conf}}}
\newcommand{\cT}{\mathcal{T}}
\newcommand{\bT}{{\bf T}}
\newcommand{\rT}{\mathrm{T}}
\newcommand{\bbT}{\mathbb{T}}
\newcommand{\Tr}{\mathrm{Tr}}
\newcommand{\htheta}{{\widehat{\theta}}}
\newcommand{\dt}{\dot{t}}

\newcommand{\bU}{{\bf U}}
\newcommand{\rU}{\mathrm{U}}
\newcommand{\wU}{{\widetilde{U}}}
\newcommand{\cU}{{\mathcal{U}}}

\newcommand{\bv}{{\bf v}}
\newcommand{\bV}{{\bf V}}
\newcommand{\bW}{{\bf W}}
\newcommand{\bw}{{\bf w}}
\newcommand{\cV}{{\mathcal{V}}}
\newcommand{\wV}{{\widetilde{V}}}
\newcommand{\wQ}{{\widetilde{Q}}}
%\newcommand{\Vect}{\mathrm{Vect}}
%\newcommand{\Ver}{{\mathcal{V}}}
%\newcommand{\Ver}{{\mathrm{Ver}}}
%\newcommand{\ver}{\mathfrak{v}}
%\newcommand{\ver}{\mathfrak{ver}}
\newcommand{\vol}{\mathrm{vol}}
\newcommand{\Vol}{\mathrm{Vol}}


\newcommand{\rW}{\mathrm{W}}
\newcommand{\wW}{{\widetilde{W}}}
%\newcommand{{x}}{{\bm{x}}}
\newcommand{\cX}{{\cal X}}
\newcommand{\hx}{{\widehat{x}}}
\newcommand{\wX}{{\widetilde{X}}}
\newcommand{\oX}{{\overline{X}}}
\newcommand{\oxi}{{\overline{\xi}}}
\newcommand{\dx}{\dot{x}}
\newcommand{\ddx}{\ddot{x}}
\def\vnabla{{\bm{\nabla}}}
\def\bnabla{{\bm{\nabla}}}
%%% \newcommand{\xx}{\xi}
\newcommand{\xx}{x}

\newcommand{\by}{{\bf y}}
\newcommand{\bY}{{\bf Y}}
\newcommand{\dy}{\dot{y}}
\newcommand{\wY}{{\widetilde{Y}}}
\newcommand{\oY}{{\overline{Y}}}

\newcommand{\bz}{{\bf z}}
\newcommand{\bZ}{{\bf Z}}
\newcommand{\cZ}{{\mathcal{Z}}}
\newcommand{\wC}{{\widetilde{\gC}}}
%\newcommand{\wG}{{\widetilde{G}}}
\newcommand{\wG}{{\gB}}
%\newcommand{\wZ}{{\widetilde{Z}}}
\newcommand{\bbZ}{\mathbb{Z}}
\newcommand{\fz}{\mathfrak{z}}
\newcommand{\fZ}{\mathfrak{Z}}
\newcommand{\oz}{{\overline{z}}}
\newcommand{\oZ}{{\overline{Z}}}
\def\smallover\#1/\#2{\hbox{$\textstyle\frac{\#1}{\#2}$}} %

\def\Ort{{\rm O}}
\def\orth{{\rm o}}
\def\valpha{{\bm{\alpha}}}
\def\vbeta{{\bm{\beta}}}
\def\vb{{\bm{b}}}
\def\vgamma{{\bm{\gamma}}}
\def\bGamma{{\bm{\Gamma}}}
\def\vomega{{\bm{\omega}}}
\def\vpi{{\bm{\pi}}}
\def\bu{{\bm{u}}}
\def\bv{{\bm{v}}}
\def\bp{{\bm{p}}}
\def\bP{{\bm{P}}}
%\def\bg{{\bm{g}}}
\def\bJ{{\bm{J}}}
\def\bq{{\bm{q}}}
\def\vC{\mathbf{flatC}}
\def\vX{\mathbf{X}}
\def\parag{\hfil\break} %%%%% paragraph
\def\kikezd{\parag\underbar}

\def\vq{\mathbf{q}}
\def\vp{\mathbf{p}}
%%
\def\bequ{\begin{enumerate}}
\def\eenu{\end{enumerate}}
\def\bitem{\begin{itemize}}
\def\eitem{\end{itemize}}

\def\beq{\begin{equation}}
\def\eeq{\end{equation}}
\def\beqa{\begin{eqnarray}}
\def\eeqa{\end{eqnarray}}
\def\nn{\nonumber}
\def\barray{\left(\begin{array}}
\def\earray{\end{array}\right)}
\def\barraynb{\begin{array}}
\def\earraynb{\end{array}}
\def\ort{{\rm o}}
\def\Ort{{\rm O}}
\def\IR{{\mathbb{R}}} %%%%% Reals
\def\II{{\mathbb{I}}} %%%%%
\def\IF{{\mathbb{F}}} %%%%% emtensor
\def\IZ{{\mathbb{Z}}}
\def\IG{{{G}}} % Gal-space
\def\IC{{\mathbb{C}}} %
\def\IN{{{N}}} % Newton-space
\def\IS{{\mathbb{S}}} % Round sphere
\def\IB{{{B}}} % Bargmann-space
\def\IE{{{E}}}% Euclidean-space
\def\GW{{gravitational wave\;}}
\def\GWs{{gravitational waves\;}}
\def\check{\quad{\gb{\fbox{\texttt{CHECK}}\;}}\quad}
\def\elsewhere{{\quad\gb{\texttt{PUT ELSEWHERE}\quad}}}
\def\complete{\quad{\gb{\fbox{\texttt{COMPLETE}}\;}}\quad}
\def\?{{\,\gb{\fbox{\texttt{??}}\;}}\,}
\def\Comm{{\kikezd{Comment\;}}}
\def\p{{\partial}}
\def\vr{\mathbf{r}}
\def\vv{\mathbf{v}}
\def\vx{\mathbf{x}}
\def\vQ{\mathbf{Q}}
\def\vR{\mathbf{R}}
\def\Rarrow{{\quad\Rightarrow\quad}}
\def\FL{{Friedmann-Lema\^{\i}tre\,}}
\def\RConc{{$\Rarrow$ \purple{\bf Conclusion\,}}}

\def\STOP{\bigskip\rb{\large \bf \textcolor{yellow}{stop here}}\bigskip}
\def\NO{\bigskip\rb{\large \bf \textcolor{yellow}{NO\, !}}}
\def\clarify{\bigskip\gb{\large \bf{CLARIFY}}\bigskip}
\def \p{{\partial}}
\def\bO{\Bbb O}
\def\bK{\Bbb K}
\def\bZ{\Bbb Z}
\newcommand{\bE}{{\mathbf{E}}}
\def\bP{\mathbb P}
\def\bI{\mathbb I}
\def\benu{\begin{enumerate}}
\def\eenu{\end{enumerate}}
\def\bitem{\begin{itemize}}
\def\eitem{\end{itemize}}


%%%%%%%%%%%%%%%%%%%%%%%%%%%%% MACROS % C %%%%%%%%%%%%%%%%%%%%%%%%%%%%%%%%%%%%%

\def\pa{\partial} \def\lrpa{\partial^\leftrightarrow}
\def\pab{\bar\partial} \def\pah{\hat\partial}
\def\dpo{d {\hskip -0.02cm{+}} {\hskip -0.05cm{1}}}
\newcommand{\C}[1]{{\cal \#1}}
\def\CC{{\cal C}}\def\CP{{\cal P}}\def\CD{{\cal D}}\def\CL{{\cal L}}


%%%%%%%%%%%%%%%%%%%%%%%%%%%%%%%%%%%%%%%%%%%%%%%%%%%%%%%%%%%%%%%%%%%%%%%%%%%%%%

\def\bP{{\rm{\bf P}}}
\newcommand{\slPi}{/ {\hskip-0.27cm{\Pi}}}


\newcommand{\labeq}[1] {\label{eq:\#1}}
\newcommand{\eqn}[1] {(\ref{eq:\#1})}
\newcommand{\labfig}[1] {\label{fig:\#1}}
\newcommand{\fig}[1] {\ref{fig:\#1}}
\newcommand{\vM}{{\mathbf{M}}}
\newcommand{\vN}{{\mathbf{N}}}
%%%\input amssym.def
%%%\input amssym.tex

%%%%%%%%%%%%%%%%%%%%%%%%%%%%%%%%%%%%%%%%%%%%%%%%%%%%%%%%%%%%%%%%%%%%%%%%%%%%%%
%%%%%%%%%%%%%%%%%%%%%%%%%%%%%%%%%%%%%%%%%%%%%%%%%%%%%%%%%%%%%%%%%%%%%%%%%%%%%%

\newcommand{\const}{\mathop{\rm const.}\nolimits}
\newcommand{\half }{\smallover{1}/{2}}
\newcommand{\scirc}{{\small\circ}}
\def\implies{\Rightarrow}
\def\isom{\stackrel{\simeq}\longrightarrow}
\def\mod#1{\left|{#1}\right|}
\newcommand{\la}{{\langle}}
\newcommand{\ra}{{\rangle}}
\def\smallover#1/#2{\hbox{$\textstyle\frac{#1}{#2}$}} %
\def\smallcirc{{\raise 0.5pt \hbox{$\scriptstyle\circ$}}}
\def\cabove(#1){\stackrel{\smallcirc}{#1}}
\def\ccabove(#1){\,\stackrel{\smallcirc\smallcirc}{#1}\,}
\def\cccabove(#1){\stackrel{\,\smallcirc\smallcirc\smallcirc}{#1}\,}
%\def\ddd(#1){\stackrel{\cdot \cdot \cdot}{#1}}
\def\2{{\smallover1/2}}
\def\sigmaf{{\sigma_{free}}}
\def\omegaf{{\omega_{free}}}
\def\cA{{\cal A}}
\def\boxit#1{
\vbox{\hrule\hbox{\vrule\kern4pt
\vbox{\kern5pt#1\kern5pt}\kern4pt\vrule}\hrule}
} %%%%% boxit (Knuth)


%%%%%%%%%%%%%%%%%%%%%%%%%%%%%%%%%%%%%%%%%%%%%%%%%%%%%%%%%%%%%%%%%%%%%%%%%%%%%%
%%%%%%%%%%%%%%%%%%%%%%%%%%%%%%%%%%%%%%%%%%%%%%%%%%%%%%%%%%%%%%%%%%%%%%%%%%%%%%

\newcommand{\mybox}[1]{\fbox{$\;\displaystyle{#1}\;$}}
\newcommand{\bigbox}[1]{\fbox{%
\rule[-20pt]{0pt}{45pt}$\;\;\displaystyle{#1}\;\;$}
}
\newcommand{\medbox}[1]{\fbox{%
\rule[-10pt]{0pt}{25pt}$\;\;\displaystyle{#1}\;\;$}%
}
%\renewcommand{\theequation}{\thesection.\arabic{equation}}
%\let\ssection=\section
%\renewcommand{\section}
%{\setcounter{equation}{0}\ssection}

%%%%%
%\begin{subequations}
%\begin{align}
%\label{Seqsc}
%\end{align}
%\label{g0eqns}
%\end{subequations}
\def\besub{\begin{subequations}}
%\def\balign{\begin{align}}
\def\esub{\end{subequations}}

%%%%%%%%%%%%%%%%%%% FIGtemplate
%\begin{figure}[h]
%\includegraphics[scale=.54]{}\vskip-3mm\caption{\textit{\small 
%}
%\label{} }
%\end{figure}
%%%%%%%%%%%%%%
%%%%%%%%%%%%%%%%%%%%%%%%%%%%%%%%%%%%%%%%%%%%%%%%%%%%%%%%%%%%%

%%%%%%%%%%%%%%%%
\begin{document}
%%%%%%%%%%%%%%%%

\preprint{\texttt{arXiv: 2502.0136v2 [gr-qc]}}


\title{Flyby-induced displacement: analytic solution}

\author{
P.-M. Zhang$^{1}$\footnote{corresponding author.  
 zhangpm5@mail.sysu.edu.cn},
Z.~K.~Silagadze$^{2}$\footnote{silagadze@inp.nsk.su},
and P.~A. Horvathy$^{3}$\footnote{horvathy@univ-tours.fr}
}

\affiliation{
${}^1$School of Physics and Astronomy, Sun Yat-sen University, Zhuhai 519082, (China)
\\
${}^2$ Budker Institute of Nuclear Physics and Novosibirsk State University, 630 090, Novosibirsk, (Russia)  
\\
${}^{3}$ Institut Denis Poisson\\ CNRS-UMR 7013
Tours University \\ Parc de Grandmont F-37200 
Tours (France)
\\
%%%%%%%
%\yb{\fbox{Sila-Zh-H-ArXiv-v2}}%\\
}
\date{\today}

\pacs{
04.20.-q  Classical general relativity;\\
}

\begin{abstract}
Approximating the derivative-of-the-Gaussian profile
proposed by Gibbons and Hawking by the hyperbolic Scarf potential, the scattering of particles by a gravitational wave generated by flyby is described analytically by following the Nikiforov-Uvarov method applied to the Scarf potential. Pure displacement arises when the wave zone contains an integer number of half-waves. The results confirm the prediction of Zel'dovich and Polnarev. %
\bigskip

\noindent{Key words: gravitational waves, displacement memory effect, flyby, Nikiforov-Uvarov method
}
\end{abstract}
\maketitle

\tableofcontents
\goodbreak

%%%%%%%%%%%%%%%%%%%%%%%%%%%%%%%%%%%
%\section{Introduction}\label{Intro}
%%%%%%%%%%%%%%%%%%%%%%%%%%%%%%%%%%%

%%%%%%%%%%%%%%%%%%%%%%%%%%%%%%%%%%%
\section{Analytic solution for flyby}\label{AnalFly}
%%%%%%%%%%%%%%%%%%%%%%%%%%%%%%%%%%%


Zel'dovich and Polnarev  suggested that
 gravitational waves generated by flyby could be observed  through the {displacement  memory effect} (DM)
 \cite{ZelPol,DM-1,Jibril}. Since then, and with the discovery that gravitational memory effects are closely related to soft graviton theorems and Bondi-Metzner-Sachs symmetries at future null infinity, the field has grown considerably \cite{BMS1,BMS2,BMS3,BMS4}. Zel'dovich - Polnarev suggestion was confirmed numerically  for the derived-of-the-Gaussian profile 
$\cA^{dG} \propto d\big(\exp[-U^2]\big)/dU$ 
 proposed by Gibbons and Hawking \cite{GibbHaw71} provided the parameters take some ``magic'' values \cite{DM-1}. Their  profile does not support an analytic solution, however it can be approximated by the derived P\"oschl-Teller potential \cite{PTeller,Chakra,DM-1},
\begin{equation}
    \cA^{dPT}(U)= \frac{\; d}{dU}\left(\frac{g}{2\cosh^{2}U}\right)=-g\frac{\sinh{U}}{\cosh^3{U}}\,,
    \label{dPTprof}
\end{equation} 
which can be studied semi-analytically using confluent Heun functions \cite{DM-1,DM-2}. 
 This led us to conjecture a sort of quantisation condition: \emph{DM ariseses when the wave zone contains an integer number $m$ of half-waves}. 
In this note we present yet another approximation of the flyby profile, namely by the hyperbolic Scarf potential \cite{scarfpot},
\begin{equation}
    \cA^{scarf}(U)=-2g\frac{\sinh{U}}{\cosh^2{U}}\,,
    \label{scarfpot}
\end{equation}
for which an analytic solution is found; it \emph{confirms} the above conjecture.
 
For small values of $U$ it behaves as \eqref{dPTprof} but for large $U$ it falls off more slowly.  However dilating $U$ appropriately almost perfect overlapping is obtained, as depicted in FIG.\ref{scarfProfFig}.  
%%%%%%%%%%%%%%%%%%% FIG scarfProfFig
\begin{figure}[h]
\includegraphics[scale=.42]{scarf-dPT2.pdf}
\vskip-4mm
\caption{\textit{\small The  derived \blue{\PT} profile \eqref{dPTprof} and its  \red{Scarf potential}  alter-ego 
\eqref{scarfpot} are both good approximations of the \dgreen{derived-Gaussian} one proposed by Gibbons and Hawking  \cite{GibbHaw71} to describe flyby.} 
\label{scarfProfFig} 
}
\end{figure}
%%%%%%%%%%%%%%

In detail, the geodesics  are, in Brinkmann coordinates, determined by the a Sturm-Liouville-type transverse equations \cite{DM-2},
\beq
\dfrac{d^2\!X^i}{dU^2}-\frac{1}{2}\cA X^i = 0\,,
\label{geoX1X2}
\eeq
$i=1,2$, where $U$ behaves as non-relativistic time and $\cA=\cA(U)$ is the profile of the \GW \cite{DBKP,DGH91}.
These equations are completed with the DM boundary conditions,
\beq
\frac{d\bX}{dU}\Big|(U=-\infty) = 0 = \frac{d\bX}{dU}\Big|(U=+\infty)\,.
\label{DMcond}
\eeq

 Below we show that the hyperbolic Scarf potential \cite{scarfpot,Scrf1}, eqn.  \eqref{scarfpot}, allows us to find an analytic solution which confirms the quantisation conjecture above.
Eqn.\eqref{geoX1X2} can be viewed indeed as a Schr\"odinger equation for the (non-normalizable) zero-energy ground state \cite{DM-1}.
%%%%%%%%%%%%%%%%%%%% 
%%%%%%%%%%%%%%%
 The solution is obtained by switching to $t=\sinh{U}$ which takes \eqref{geoX1X2} into \footnote{For comparison, putting $t=\tanh U$ carries \eqref{geoX1X2}  with \PT profile $\cA^{PT}=(g/2)\cosh^{-2}U$ into a form 
%$$
%(1-t^2)\frac{d^2X}{dt^2}-2t\frac{dX}{dt}-\frac{g}{4}X=0\,,
%$$
which can be solved by Legendre polynomials \cite{DM-1}.}
%%%%%%%%%%%%%%% 
\begin{equation}
    (1+t^2)\frac{d^2X}{dt^2}+t\frac{dX}{dt}-\frac{gt}{1+t^2}X=0\,,
    \label{eq3}
\end{equation}
which is of the generalised hypergeometric type, and can be solved analytically by the Nikiforov-Uvarov method \cite{NikiUvar}. Since a fairly detailed description of this simple, yet elegant and powerful method can be found elsewhere \cite{NikiUvar,NU1,NU2,NU3,NU4}, we only provide a brief summary of the algorithm in the Appendix. 

%%%%%%%%%%%%%%%%%%%%%%%%%%%%%%%%%%%%%%%%%%%%%%%%%%%%
%\section{Transverse geodesics by the Nikiforov-Uvarov method}\label{Transgeo}
%%%%%%%%%%%%%%%%%%%%%%%%%%%%%%%%%%%%%%%%%%%%%%%%%%%%

The Nikoforov-Uvarov method (recalled in the Appendix) applies to second-order differential equations of the generalised hypergeometric type of the form,
\begin{equation}
u^{\prime\prime}+\frac{\pi_1(z)}{\sigma(z)}u^\prime+\frac{\sigma_1(z)}{\sigma^2(z)}u=0\,.
\label{eq4} 
\end{equation}
For Eq. \eqref{eq3} in particular,
\begin{equation}
\sigma=1+t^2,\;\;\;\pi_1=t,\;\;\;\sigma_1=-gt\,.
\label{eq32}
\end{equation}
Then, 
\begin{equation}
\sigma_3=\frac{t^2}{4}+gt+k(1+t^2)=t^2\left(k+\frac{1}{4}\right)+gt+k\,,
\label{eq33}
\end{equation}
Therefore, the condition $\det(\sigma_3)
=g^2-4k\left(k+\frac{1}{4}\right)=0$ gives
\begin{equation}
k=\frac{\pm\sqrt{1+16g^2}-1}{8},
\label{eq35}
\end{equation}
and
\begin{equation}
|g|=2\sqrt{k\left(k+\frac{1}{4}\right)}\,.
\label{eq36}
\end{equation}
Thus we have
\begin{equation}
\sigma_3=\frac{g^2}{4k}\left [t^2+\frac{4k}{g}t+\frac{4k^2}{g^2}\right]=\frac{g^2}{4k}\left(t+\frac{2k}{g}\right)^2\,
\quad
\pi=\frac{t}{2}\pm \sqrt{\sigma_3}= \frac{t}{2}\pm\frac{|g|}{2\sqrt{k}}\left |t+\frac{2k}{g}\right|\,. 
\label{eq37}
\end{equation}
Then
\begin{equation}
\tau=\pi_1+2\pi=2t\pm \frac{|g|}{\sqrt{k}}\left |t+\frac{2k}{g}\right|.
\label{eq38}
\end{equation}
To have real solutions, we assume $k>0$, that is we choose upper sign in Eq.\eqref{eq35}:
$ 
k=\frac{\sqrt{1+16g^2}-1}{8}.   
$ 
On the other side, the quantization condition Eq.\eqref{eq24} requires
\begin{equation}
\lambda= -n(n+1)\mp\epsilon\,\frac{n|g|}{\sqrt{k}},
\label{eq40}
\end{equation}
since
$\tau^\prime=2\pm\epsilon\,\frac{|g|}{\sqrt{k}}$ with  
$\epsilon=\sign\left(t+\frac{2k}{g}\right)\,.$
Therefore we get, after some algebra,
%%%%%%
$$
k=\lambda-\pi^\prime=-n^2-n-\frac{1}{2}\mp\epsilon\frac{|g|}{\sqrt{k}}\left(n+\frac{1}{2}\right ),
$$
and using Eq. \eqref{eq36} for $|g|/\sqrt{k}$, we get 
%%%%%%%%%%%%%
\begin{equation}
\sqrt{k+\frac{1}{4}}=\mp\epsilon\left(n+\frac{1}{2}\right)
\label{eq41}
\end{equation}
and we see that we must choose the lower signs in Eq.\eqref{eq37} and Eq.\eqref{eq38} to get $\mp\epsilon=1$. Then the solution of Eq.\eqref{eq41} is
\begin{equation}
k_n=n(n+1),
\label{eq42}
\end{equation}
and Eq.\eqref{eq36} determines the ``magic" strengths of the  Scarf potential Eq.\eqref{scarfpot}, for which we have polynomial solutions for $y_n(t)$ labeled by the integer ``quantum'' number $n$,
\begin{equation}
\medbox{
|g_n|=(2n+1)\sqrt{n(n+1)}\;.
}
\label{eq43}
\end{equation}

It remains to find the gauge function $\varphi(t)$ and the corresponding hypergeometric type polynomials $y_n$. Because of Eq. \eqref{eq42} and Eq. \eqref{eq43}, we have,
\begin{equation}
\pi(t)=-nt-\sign(g)\sqrt{n(n+1)}, \;\;\;\; \tau(t)=-(2n-1)t-2\sign(g)\sqrt{n(n+1)}\,.
\label{eq44}
\end{equation}
The equation for the gauge function,
\begin{equation}
\frac{d\varphi}{dt}= -\frac{nt+\sign(g)\sqrt{n(n+1)}}{1+t^2}\,,
\label{eq45}
\end{equation}
can be easily integrated and we get (up to irrelevant additive constant which translates to a multiplicative constant in the solution $X(t)$),
\begin{equation}
\varphi(t)=-\frac{n}{2}\ln(1+t^2)+\sign(g)\,\sqrt{n(n+1)}\arctan{t}\,.
\label{eq46}
\end{equation}
Therefore, the corresponding geodesic trajectory is given by
\begin{equation}
\medbox{
X_n(t)=(1+t^2)^{-\frac{n}{2}}\,e^{-\sign(g)\,\sqrt{n(n+1)}\arctan{t}}\, y_n(t)\,.
}
\label{eq47}
\end{equation}
The polynomial $y_n(t)$ is given by the Rodrigues formula Eq. \eqref{eq31} in which we need to find the weight function $\rho(t)$, for which we found,
\begin{equation}
\rho(t)=(1+t^2)^{-\left(n+\frac{1}{2}\right)} e^{-2\sign(g)\,\sqrt{n(n+1)}\arctan{t}}\,.
\label{eq49}
\end{equation}
Therefore
\begin{equation}
\medbox{
y_n(t)=\frac{B_n}{\rho(t)}\left [(1+t^2)^n\rho(t)\right]^{(n)}\,.}
\label{eq50}
\end{equation}
The normalization constants $B_n$ should be determined from the initial conditions. 
The first three polynomials are
$
y_1(t)\sim \tau(t)=-t-2\sqrt{2}\,\sign(g),
\; y_2(t)\sim 2t^2+8\sqrt{6}\,\sign(g) t+23, 
\;y_3(t)\sim -6t^3-72\sqrt{3}\,\sign(g)t^2-423t-172\sqrt{3}\,\sign(g)\,.
$ 
They can be expressed in terms of Routh-Romanovski polynomials \cite{scarfpot,Raposo} as $y_n(t)=R_n^{(\alpha,\beta)}(t)$ with $\alpha=-2\,\sign(g)\sqrt{n(n+1)}$ and $\beta=-n+\frac{1}{2}$.

For (half)wave numbers $n=1,2,3,4$  the analytic solutions \eqref{eq47} reproduce almost identically those half-numerical ones shown in FIG.s \# 8 \& 9 of \cite{DM-2} for the derived \PT case. For $n\geq5$, though,  DM requires to find a high number of decimals and the numerical procedure becomes tedious. Plotting instead the analytic solution \eqref{eq47}-\eqref{eq50} yields FIG.\ref{n=5-6}. 
%%%%%%%%%%%%%%%%%%%%%%%%%%%%%%%%

%%%%%%%%%%%%%%%%%%% FIG flybygeo5-6
\begin{figure}[h]
\includegraphics[scale=.35]{fn5.pdf}\;
\includegraphics[scale=.35]{fn6.pdf}
\vskip-3mm\caption{\textit{\small  DM geodesics for the Scarf profile \eqref{scarfpot} for wave numbers $n=5$ and $n=6$.
}
\label{n=5-6}
}
\end{figure}
%%%%%%%%%%%%%%%%%

%\medskip

%%%%%%%%%%%%%%%%%%%%%%%%%%%%%%%%
%\section{Conclusion}\label{Concl}
%%%%%%%%%%%%%%%%%%%%%%%%%%%%%%%%

In conclusion, our results show that the profile proposed by Hawking and Gibbons for flyby \cite{GibbHaw71} can be approximated by the Scarf potential \eqref{scarfpot} in FIG.\ref{scarfProfFig} which supports  analytic solutions, \eqref{eq47}-\eqref{eq50}, labeled by an integer $n$, \eqref{eq43}, which can be constructed by the Nikiforov-Uvarov method. The  displacement effect (DM) arises when the wave zone contains $n$ half-waves as illustrated in FIG.\ref{n=5-6}, extending an observation made  \cite{DM-2} for the Gaussian/\PT profile. 

Further physical applications of the  Scarf potential which include supersymmetry \cite{scarfpot,Compean}, non-central potentials \cite{Dutt} and even quarks (QCD), are reviewed in ref. \cite{Raposo}. Note that, in the supersymmetry literature, the Scarf hyperbolic potential has recently attracted much attention \cite{Scrf1}. This potential was discovered by Gendenshtein and is therefore referred also  to as the Gendenstein potential \cite{Gendenon,Natanson}. 
See also \cite{Hait,Bhattacharya23,Bhattacharya:2025ljc} for  related work. 

For  isotropic profile we have a ``universal model"~: every plane \GW  can  be mapped locally conformally to free Minkowski space by following  Niederer, or Arnold \cite{Niederer,Arnold,Takagi}, consistently with the vanishing of the Cotton or Weyl tensor \cite{Inomata,Dhasmana,GibbonsDark,ZZH21}.

However  \GWs are isotropic only in 1d, and their anisotropy in  d=2 dim implies that we arrive to unchartered territory \cite{Lukash,Inomata,ZZH21} and it is not clear whether we still have a (conformal) ``universal ancestor".
\goodbreak 

\begin{acknowledgments}\vskip-4mm
The authors are grateful to Gary Gibbons for discussions. PMZ was partially supported by the National Natural Science Foundation of China (Grant No. 11975320).
\end{acknowledgments}
\goodbreak

%%%%%%%%%%%%%%%%%%%%%%%%%%%
\begin{thebibliography}{99}
%%%%%%%%%%%%%%%%%%%%%%%%%%%

\bibitem{ZelPol}
Ya. B. Zel'dovich and A. G. Polnarev,
``Radiation of gravitational waves by a cluster of superdense stars,"
Astron. Zh. {\bf 51}, 30 (1974)
[Sov. Astron. {\bf 18} 17 (1974)].

%\cite{Zhang:2024uyp}
\bibitem{DM-1}
%\cite{Zhang:2024uyp}
%\bibitem{Zhang:2024uyp}
P.~M.~Zhang and P.~A.~Horvathy,
``Displacement within velocity effect in gravitational wave memory,''
Annals of Physics. \textbf{470} (2024) 169784 doi:10.1016/j.aop.2024.169784. [arXiv:2405.12928 [gr-qc]].


%\cite{BenAchour:2024ucn}
\bibitem{Jibril}
J.~Ben Achour and J.~P.~Uzan,
``Displacement versus velocity memory effects from a gravitational plane wave,''
JCAP \textbf{08} (2024), 004
doi:10.1088/1475-7516/2024/08/004
[arXiv:2406.07106 [gr-qc]].
%3 citations counted in INSPIRE as of 17 Sep 2024

\bibitem{BMS1}
L.~Bieri and A.~Polnarev,
``Gravitational wave displacement and velocity memory effects,''
Class. Quant. Grav. \textbf{41}, no.13, 135012 (2024)
doi:10.1088/1361-6382/ad4dfe
[arXiv:2402.02594 [gr-qc]].

\bibitem{BMS2}
A.~I.~Harte, T.~B.~Mieling, M.~A.~Oancea and E.~Steininger,
``Gravitational wave memory and its effects on particles and fields,''
Phys. Rev. D \textbf{111}, no.2, 024034 (2025)
doi:10.1103/PhysRevD.111.024034
[arXiv:2407.00174 [gr-qc]].

\bibitem{BMS3}
K.~Mitman, M.~Boyle, L.~C.~Stein, N.~Deppe, L.~E.~Kidder, J.~Moxon, H.~P.~Pfeiffer, M.~A.~Scheel, S.~A.~Teukolsky and W.~Throwe, \textit{et al.}
``A review of gravitational memory and BMS frame fixing in numerical relativity,''
Class. Quant. Grav. \textbf{41}, no.22, 223001 (2024)
doi:10.1088/1361-6382/ad83c2
[arXiv:2405.08868 [gr-qc]].

\bibitem{BMS4}
S.~Kumar,
``Displacement memory and BMS symmetries,''
{\it in} R.~Ruffini, G.~Vereshchagin (Eds.) {\sl The Sixteenth Marcel Grossmann Meeting.}
World Scientific, Singaspore, 2023.
doi:10.1142/9789811269776\_0094
[arXiv:2109.13082 [gr-qc]].

%\cite{Gibbons:1971wsk}
\bibitem{GibbHaw71}
G.~W.~Gibbons and S.~W.~Hawking,
``Theory of the detection of short bursts of gravitational radiation,'' Phys. Rev. D \textbf{4} (1971), 2191-2197
doi:10.1103/PhysRevD.4.2191


%\cite{Poschl:1933zz}
\bibitem{PTeller}
G.~P\"oschl and E.~Teller,
``Bemerkungen zur Quantenmechanik des anharmonischen Oszillators,''
Z. Phys. \textbf{83} (1933), 143-151
doi:10.1007/BF01331132
%366 citations counted in INSPIRE as of 12 May 2024

%\cite{Chakraborty:2019yxn}
\bibitem{Chakra}
I.~Chakraborty and S.~Kar,
``Geodesic congruences in exact plane wave spacetimes and the memory effect,''
Phys. Rev. D \textbf{101} (2020) no.6, 064022
doi:10.1103/PhysRevD.101.064022
[arXiv:1901.11236 [gr-qc]].
%18 citations counted in INSPIRE as of 22 May 2024

%\cite{Zhang:2024tey}
\bibitem{DM-2}
P.~M.~Zhang, Q.~L.~Zhao, J.~Balog and P.~A.~Horvathy,
``Displacement memory for flyby,''
Annals Phys. \textbf{473} (2025), 169890
doi:10.1016/j.aop.2024.169890
[arXiv:2407.10787 [gr-qc]].
%4 citations counted in INSPIRE as of 23 Jan 2025

\bibitem{scarfpot}
D.~E.~Alvarez-Castillo and M.~Kirchbach,
``The real exact solutions to the hyperbolic scarf potential,''
Rev. Mex. Fis. E \textbf{53}, 143-154 (2007)
[arXiv:quant-ph/0603122 [quant-ph]].

\bibitem{Scrf1}
Y.~C.~Acar, L.~Acevedo \c{S}.~and Kuru,
``Unusual isospectral factorizations of shape invariant Hamiltonians with scarf II potential," Phys. Scr. \textbf{98}, 125229 (2023)
[arXiv:2309.06044 [math-ph]].

%\cite{Duval:1984cj}
\bibitem{DBKP}
C.~Duval, G.~Burdet, H.~P.~Kunzle and M.~Perrin,
``Bargmann Structures and Newton-cartan Theory,''
Phys. Rev. D \textbf{31} (1985), 1841-1853
doi:10.1103/PhysRevD.31.1841
%360 citations counted in INSPIRE as of 09 May 2024

\bibitem{DGH91}
%\cite{Duval:1984cj}
%\bibitem{Duval:1984cj}
C. Duval, G.W. Gibbons, P. Horvathy,
``Celestial mechanics, conformal structures and gravitational waves,''
 Phys. Rev. {\bf D43} (1991) 3907 %-3922
[hep-th/0512188].

\bibitem{NikiUvar}
A.~F.~Nikiforov and V.~B.~Uvarov. {\sl Special Functions of Mathematical Physics: A Unified
Introduction with Applications.} Springer Basel AG, 1988.

\bibitem{NU1}
L.~Ellis, I.~Ellis, C.~Koutschan and S.~K.~Suslov,
``On Potentials Integrated by the Nikiforov-Uvarov Method,''
[arXiv:2303.02560 [quant-ph]]. 

\bibitem{NU2}
C.~Berkdemir, ``Application of the Nikiforov-Uvarov Method in Quantum Mechanics",
{\it in} M.~R.~Pahlavani (Ed.) {\sl Theoretical Concepts of Quantum Mechanics.} InTech, Rijeka, 2012. 
doi: 10.5772/33510.

\bibitem{NU3}
S.~K.~Suslov, J.~M.~Vega-Guzm\'{a}n and K.~Barley, ``An Introduction to Special Functions with Some Applications to Quantum Mechanics", {\it in} M.~Foupouagnigni, W.~Koepf (Eds) {\sl Orthogonal Polynomials.} Birkh\"{a}user, Cham, 2020.
%doi: 10.1007/978-3-030-36744-2_21.

\bibitem{NU4}
C.~Tezcan and R.~Sever, ``A General Approach for the Exact Solution of the Schr\"{o}dinger Equation," Int. J. Theor. Phys. {\textbf 48} (2009), 337–350. 
doi: 10.1007/s10773-008-9806-y

\bibitem{Ismail}
M.E.H. Ismail, {\sl Classical and Quantum Orthogonal Polynomials in One Variable.} Cambridge University Press, 2005.

\bibitem{Compean}
C. B. Compean and M. Kirchbach: ``The trigonometric Rosen-Morse potential in supersymmetric quantum mechanics and its exact solutions'', J. Phys. A:Math.Gen., Vol. 39, (2006), pp. 547-557.

\bibitem{Dutt}
R. Dutt, A. Gangopadhyaya and U. P. Sukhatme,
``Non-Central potentials and spherical harmonics using supersymmetry and shape invariance'', Am. J. Phys., \textbf{65}, (1997) 400-403 [hep-th/9611087].

%\cite{Raposo:2007qma}
\bibitem{Raposo}
A.~P.~Raposo, H.~J.~Weber, D.~E.~Alvarez-Castillo and M.~Kirchbach,
``Romanovski polynomials in selected physics problems,''
Central Eur. J. Phys. \textbf{5} (2007) no.3, 253-284
doi:10.2478/s11534-007-0018-5
[arXiv:0706.3897 [quant-ph]].
%25 citations counted in INSPIRE as of 02 Feb 2025

\bibitem{Gendenon}
G.~Natanson, ``On history of the Gendenshtein (scarf II) potential,
Researchgate preprint, 2018.
doi:10.13140/RG.2.2.20949.73444/1

\bibitem{Natanson}
G.~Natanson, ``On history of the Gendenshtein (scarf II) potential,
Researchgate preprint, 2018.
doi:10.13140/RG.2.2.20949.73444/1


%\cite{Hait:2022ukn}
\bibitem{Hait}
A.~Hait, S.~Mohanty and S.~Prakash,
``Frequency space derivation of linear and nonlinear memory gravitational wave signals from eccentric binary orbits,''
Phys. Rev. D \textbf{109} (2024) no.8, 084037
doi:10.1103/PhysRevD.109.084037
[arXiv:2211.13120 [gr-qc]].
%12 citations counted in INSPIRE as of 07 Feb 2025

%\cite{Bhattacharya:2023wzl}
\bibitem{Bhattacharya23}
S.~Bhattacharya, D.~Bose, I.~Chakraborty, A.~Hait and S.~Mohanty,
``Gravitational memory signal from neutrino self-interactions in supernova,''
Phys. Rev. D \textbf{110} (2024) no.6, L061501
doi:10.1103/PhysRevD.110.L061501
[arXiv:2311.03315 [gr-qc]].
%2 citations counted in INSPIRE as of 07 Feb 2025

%\cite{Bhattacharya:2025ljc}
\bibitem{Bhattacharya:2025ljc}
S.~Bhattacharya and S.~Ghosh,
``Displacement memory and B-memory in generalised Ellis-Bronnikov wormholes,''
[arXiv:2502.03007 [gr-qc]].
%0 citations counted in INSPIRE as of 08 Feb 2025

%\cite{Niederer:1973tz}
\bibitem{Niederer}
U.~Niederer,
``The maximal kinematical invariance group of the harmonic oscillator,''
Helv. Phys. Acta \textbf{46} (1973), 191-200
PRINT-72-4208.

\bibitem{Arnold}
 V. I. Arnold, {\sl Mathematical Methods of Classical Mechanics''} (Springer:
New York, 1989). https://doi.org/10.1007/978-1-4757-2063-1

\bibitem{Takagi}
Shin Takagi, 
``Equivalence of a Harmonic Oscillator to a Free Particle,'' Prog. Theor. Phys. 84 (1990), 1019-1024. 
Progress Letters https://doi.org/10.1143/ptp/84.6.1019.


%\cite{Zhao:2021tsz}
\bibitem{Inomata}
Q.~Zhao, P.~Zhang and P.~A.~Horvathy,
``Time-Dependent Conformal Transformations and the Propagator for Quadratic Systems,''
Symmetry \textbf{13} (2021) no.10, 1866
doi:10.3390/sym13101866
[arXiv:2105.07374 [quant-ph]].
%5 citations counted in INSPIRE as of 06 Feb 2025

%\cite{Dhasmana:2021qvw}
\bibitem{Dhasmana}
S.~Dhasmana, A.~Sen and Z.~K.~Silagadze,
``Equivalence of a harmonic oscillator to a free particle and Eisenhart lift,''
Annals Phys. \textbf{434} (2021), 168623
doi:10.1016/j.aop.2021.168623
[arXiv:2106.09523 [quant-ph]].
%18 citations counted in INSPIRE as of 05 Feb 2025

%\cite{Gibbons:2014zla}
\bibitem{GibbonsDark}
G.~W.~Gibbons,
``Dark Energy and the Schwarzian Derivative,''
[arXiv:1403.5431 [hep-th]].
%22 citations counted in INSPIRE as of 05 Feb 2025

%\cite{Zhang:2021ssp}
\bibitem{ZZH21}
P.~Zhang, Q.~Zhao and P.~A.~Horvathy,
``Gravitational waves and conformal time transformations,''
Annals Phys. \textbf{440} (2022), 168833
doi:10.1016/j.aop.2022.168833
[arXiv:2112.09589 [gr-qc]]. 
%8 citations counted in INSPIRE as of 04 Feb 2025

%\cite{Elbistan:2020dxz}
\bibitem{Lukash}
M.~Elbistan, P.~M.~Zhang, G.~W.~Gibbons and P.~A.~Horvathy,
``Lukash plane waves, revisited,''
JCAP \textbf{01} (2021), 052
doi:10.1088/1475-7516/2021/01/052
[arXiv:2008.07801 [gr-qc]].
%8 citations counted in INSPIRE as of 08 Feb 2025

\end{thebibliography}
%%%%%%%%%%%%%%%%%%%%%

\appendix
\section{\bf Compendium on the Nikoforov-Uvarov method
}\label{Appendix}

As it was already mentioned in the text, the Nikoforov-Uvarov method applies to second-order differential equations of the generalised hypergeometric type of the form
\begin{equation}
u^{\prime\prime}+\frac{\pi_1(z)}{\sigma(z)}u^\prime+\frac{\sigma_1(z)}{\sigma^2(z)}u=0\,,
\label{eq4bis} 
\end{equation}
where the prime denotes $d/dz$. $z$ can be complex. $\pi_1(z)$ is a  polynomial at most of the first degree, and $\sigma(z)$, $\sigma_1(z)$ are polynomials at most of the second degree \cite{NikiUvar}. 

The set of solutions of Eq. \eqref{eq4bis} is invariant under  ``gauge" transformations $u(z) \to y(z)$,
$ 
u(z)=e^{\varphi(z)}y(z),    
$ 
if the gauge function $\varphi(z)$ satisfies the equation
\begin{equation}
\varphi^\prime(z) =\frac{\pi(z)}{\sigma(z)}\,
\label{eq7}
\end{equation}
where $\pi(z)$ is a polynomial at most of the first degree. In this case for $y(z)$ 
we obtain also a generalised hypergeometric-type equation:
\begin{equation}
y^{\prime\prime}+\frac{\pi_2(z)}{\sigma(z)}y^\prime+\frac{\sigma_2(z)}{\sigma^2(z)}y=0\,,
\label{eq8}    
\end{equation}
where
$
\pi_2(z)=\pi_1(z)+2\pi(z)
$ 
is a polynomial  of at most of the first degree, and
\begin{equation}
\sigma_2(z)=\sigma_1(z)+\pi^2(z)+\pi(z)\left [\pi_1(z)-\sigma^\prime(z)\right]+\pi^\prime(z)\sigma(z)
\label{eq10}
\end{equation}
is a polynomial of at most of the second degree. 
Choosing  $\pi(z)$ so that
\begin{equation}
\sigma_2(z)=\lambda\sigma(z)\,   
\label{eq11}
\end{equation}
where $\lambda$ is a constant, Eq. \eqref{eq4} is simplified to an hypergeometric type equation,
\begin{equation}
\sigma(z)y^{\prime\prime}+\pi_2(z)y^\prime+\lambda y=0\,.
\label{eq12}
\end{equation}

In the light of Eq. \eqref{eq10}, Eq. \eqref{eq11} means
\begin{equation}
 \pi^2+\pi[\pi_1-\sigma^\prime]+\sigma_1-k\sigma=0,
 \label{eq13}
\end{equation}
where
$
k=\lambda-\pi^\prime
$ 
is another constant.
Eq. \eqref{eq13} is  solved by,
\begin{equation}
\pi=\frac{\sigma^\prime-\pi_1}{2}\pm\sqrt{\sigma_3(z)}
\where
\sigma_3(z)=\left(\frac{\sigma^\prime-\pi_1}{2}\right)^2-\sigma_1+k\sigma\,.
\label{eq15}
\end{equation}
$\pi$ is a polynomial only when
$\sigma_3(z)$ is the square of a first order polynomial. Hence it has a double root and its discriminant is zero,
$
\Delta(\sigma_3)=0\,,
$
which determines the constant $k$, and hence 
the constant $\lambda$.

We want to find polynomial solutions of Eq. \eqref{eq12}. It can be shown that derivatives of $y(z)$, $v_n(z)=y^{(n)}(z)$, are also generalised hypergeometric-type functions,
\begin{equation}
\sigma v_n^{\prime\prime}+\tau_n(z)v_n^\prime +\mu_nv_n=0\,,
\label{eq20}
\end{equation}
and we get the recurrence relations 
\begin{equation}
\tau_n(z)=\sigma^\prime(z)+\tau_{n-1}(z),\;\;\;\;\mu_n=\mu_{n-1}+\tau_{n-1}^\prime\,,
\label{eq21}
\end{equation}
with initial values
$\tau(z)=\tau_0(z)=\pi_2(z),\;\mu_0=\lambda\,.
$ 
If $y(z)=y_n(z)$ is a polynomial of order $n$, then $v_n=\mathrm{const}$ and  Eq. \eqref{eq20} is satisfied only if $\mu_n=0$. But the repeated application of the recurrence relations Eq. \eqref{eq21}  produces
\begin{equation}
\tau_n(z)=n\sigma^\prime(z)+\tau(z),\;\;\;\mu_n=\lambda+n\tau^\prime+\frac{1}{2}n(n-1)\sigma^{\prime\prime}\,.
\label{eq23}
\end{equation}
Getting a polynomial solution $y(z)=y_n(z)$ of Eq. \eqref{eq12} 
thus requires the ``quantization condition'',
\begin{equation}
\lambda=\lambda_n=- n\tau^\prime-\frac{1}{2}n(n-1)\sigma^{\prime\prime}.
\label{eq24}
\end{equation}
It can be shown \cite{NikiUvar} that, when the quantization condition Eq. \eqref{eq24} is satisfied, the corresponding polynomial solution of Eq. \eqref{eq12} is given by the Rodrigues formula
\begin{equation}
y_n(z)=\frac{B_n}{\rho(z)}\left[\sigma^n(z)\rho(z)\right]^{(n)},
\label{eq31}
\end{equation}
where $B_n$ is some (normalization) constant, and the weight functions $\rho(z)$ satisfies Pearson's equation \cite{Ismail}  
\begin{equation}
(\sigma\rho)^\prime=\rho\tau\,.
\label{eq25}
\end{equation}

%%%%%%%%%%%%%%
%%%%%%%%%%%%%%
\end{document}
%%%%%%%%%%%%%%
%%%%%%%%%%%%%%

