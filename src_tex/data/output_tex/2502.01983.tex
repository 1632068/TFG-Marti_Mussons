\begin{figure}
\begin{center}
\begin{tikzpicture}[scale=0.6,decoration={
    markings,
    mark=at position 0.70 with {\arrow{>}}}]
\begin{scope}[shift={(0,0)}]
%\draw[thin,yellow] (0,0) grid (4,4);

\node at (5.75,2.4) {additive};
\node at (5.75,1.65) {lines};
\node at (5.75,0.5) {$\langle a,b\rangle$};

\node at (2,3.5) {$a+b$};
\draw[thick,dashed] (0,3) -- (4,3);
\draw[thick,dashed] (0,0) -- (4,0);

\node at (0.5,-0.5) {$a$};
\node at (3.5,-0.5) {$b$};

\draw[thick,postaction={decorate}] (0.5,0) .. controls (0.75,0.75) and (1.3,1.25) .. (2,1.25);

\draw[thick,postaction={decorate}] (3.5,0) .. controls (3.25,0.75) and (2.7,1.25) .. (2,1.25);

\draw[thick,<-] (2,2) -- (2,1.25);
\draw[thick] (2,3) -- (2,2);
\end{scope}

\begin{scope}[shift={(9,0)}]
%\draw[thin,yellow] (0,0) grid (4,4);
\draw[thick,dashed] (0,3) -- (4,3);
\draw[thick,dashed] (0,0) -- (4,0);

\node at (5.75,1.5) {$-\langle a,b\rangle$};

\node at (0.5,3.5) {$a$};
\node at (3.5,3.5) {$b$};

\draw[thick,postaction={decorate}] (2,1.75) .. controls (1.6,1.75) and (0.75,2.25) .. (0.5,3);

\draw[thick,postaction={decorate}] (2,1.75) .. controls (2.4,1.75) and (3.25,2.25) .. (3.5,3);

\draw[thick] (2,1) -- (2,1.75);
\draw[thick,->] (2,0) -- (2,1);

\node at (2,-0.5) {$a+b$};
\end{scope}

\begin{scope}[shift={(18,0)}]
%\draw[thin,yellow] (0,0) grid (4,4);

\draw[thick,dashed] (0,3) -- (4,3);
\draw[thick,dashed] (0,0) -- (4,0);

\node at (5.75,1.9) {virtual};
\node at (5.75,1.1) {crossing};

\draw[thick,postaction={decorate}] (0.75,0) -- (3.25,3);

\draw[thick,postaction={decorate}] (3.25,0) -- (0.75,3);

\node at (0.6,3.5) {$b$};
\node at (3.4,3.5) {$a$};

\node at (0.6,-0.5) {$a$};
\node at (3.4,-0.5) {$b$};
\end{scope}

\begin{scope}[shift={(0,-5.5)}]
%\draw[thin,yellow] (0,0) grid (4,4);

\node at (0.85,3.4) {$ca$};
\draw[thick,dashed] (0,3) -- (4,3);
\draw[thick,dashed] (0,0) -- (4,0);
\node at (3.15,-0.4) {$a$};

\draw[thick,->] (3,0) -- (1.67,2);
\draw[thick] (1.67,2) -- (1,3);

\draw[line width=0.50mm,red] (1,0) decorate [decoration={snake,amplitude=0.15mm}] {-- (3,3)};

\draw[line width=0.50mm,red] (0.9,0.75) -- (1.5,0.75);

\draw[line width=0.50mm,red] (2.1,2.5) -- (2.65,2.5);

\node at (0.6,0.40) {$c$};

\node at (3,2.3) {$c$};

\end{scope}


\begin{scope}[shift={(9,-5.5)}]
%\draw[thin,yellow] (0,0) grid (4,4);

\node at (6,2.5) {merging of};
\node at (6,1.75) {multiplicative};
\node at (6,1.00) {lines};

\draw[thick,dashed] (0,3) -- (4,3);
\draw[thick,dashed] (0,0) -- (4,0);

\node at (2.75,2.2) {$c_1c_2$};

\draw[line width=0.50mm,red] (0.75,0) decorate [decoration={snake,amplitude=0.15mm}] {.. controls (0.85,0.75) and (1,0.85) .. (2,1.25)};

\draw[line width=0.50mm,red] (3.25,0) decorate [decoration={snake,amplitude=0.15mm}] {.. controls (3.15,0.75) and (3,0.85) .. (2,1.25)};
 
\draw[line width=0.50mm,red] (2,1.25) decorate [decoration={snake,amplitude=0.15mm}] {-- (2,3)};
 
\draw[line width=0.50mm,red] (0.30,0.45) -- (0.80,0.45);

\draw[line width=0.50mm,red] (2.6,0.45) -- (3.2,0.45);

\draw[line width=0.50mm,red] (1.5,2) -- (2,2);

\node at (0.5,0.85) {$c_1$};

\node at (3.5,0.85) {$c_2$};
\end{scope}


\begin{scope}[shift={(18,-5.5)}]
%\draw[thin,yellow] (0,0) grid (4,4);

\draw[thick,dashed] (0,3) -- (4,3);
\draw[thick,dashed] (0,0) -- (4,0);

\draw[line width=0.50mm,red] (2,0) decorate [decoration={snake,amplitude=0.15mm}] {-- (2,3)};

\draw[thick,fill,red] (2.15,1.5) arc (0:360:1.5mm);

\draw[line width=0.50mm,red] (2,2.45) -- (2.5,2.45);

\draw[line width=0.50mm,red] (1.5,0.55) -- (2,0.55);

\node at (1.1,0.5) {$c$};

\node at (3.15,2.40) {$c^{-1}$};

\node at (6,2.00) {swapping the};
\node at (6,1.25) {co-orientation};
\end{scope}

\end{tikzpicture}
\end{center}
    \caption{Upper left: Whenever two black additive lines merge, we evaluate $\langle a,b\rangle$ at the additive vertex. Upper middle: whenever two additive lines split, we evaluate $-\langle a,b\rangle$. Upper right: we are allowed to have virtual crossing whenever two additive lines cross but the intersection of these two lines is virtual, so there is no corresponding evaluation for these two additive lines. Bottom left: whenever a red line is to the left of a black line, then we rescale the value $a$ of the additive line by $c\in \mathbf{k}^*$ of the multiplicative line.  Bottom middle: multiplicative lines can merge, resulting in multiplication of their weights. Bottom right: we may have a red vertex on the multiplicative network, resulting in the swapping of co-orientations of the multiplicative line.}
    \label{fig_1001}
\end{figure}