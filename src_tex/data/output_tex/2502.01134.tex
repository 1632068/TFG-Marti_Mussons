\section{Introduction}
A homotopy class $[S]$ of surfaces in a $3$-manifold $M$ is  \textbf{filling} if for every $S' \in [S]$, $M-S'$ consists of contractible components. Let $\tm$ be the universal cover. A surface $S$ in a hyperbolic manifold is $(1+\epsilon)$-\qf if its limit set $\partial\ts \subset \partial \tm$ is a $(1+\epsilon)$-quasicircle (but not a $(1+\epsilon')$-quasicircle for any $\epsilon'<\epsilon$). In this paper, we prove
\begin{theorem}\label{almost all quasi-Fuchsian are filling}
	Let $M$ be a closed hyperbolic $3$-manifold. There exists an $\epsilon_0>0$ such that every homotopy class of  $(1+\epsilon)$-\qf surfaces with $0<\epsilon\leq \epsilon_0$ is filling, and the set of incompressible embedded surfaces is decomposed into 
	\begin{enumerate}
		\item at most finitely many totally geodesic surfaces, and
		\item surfaces with quasi-Fuchsian constant $\geq 1+\epsilon_0$. 
	\end{enumerate}
\end{theorem}
This implies that any sequence of surfaces in $M$ is eventually self-intersecting or uniformly away from being geodesic\footnote{Use Huang-Wang [\citenum{hwAlmostFuchsianManifolds}, Theorem 1.2] which bounds supreme principal curvature from below by $\delta_0$.}. There are countably infinitely many $(1+\epsilon)$-\qf surfaces for $\epsilon<\epsilon_0$ in $M$ by \cite{kmImmersingAlmostGeodesic}. We show all but finitely many \cbility classes of Fuchsian surfaces are filling in \Cref{theorem large Fuchsian filling}. We say a surface $S\subset M$ \textit{separates} two distinct points $p, q \in  \partial \widetilde{M}$ if $S$ has a cover $\widetilde{S}$ such that  $p, q$ belong to different components of $\partial \widetilde{M}-\partial\widetilde{S}$. 
\begin{cor}\label{all but finitely many separate any pairs of distinct points}
	With notations as in \Cref{almost all quasi-Fuchsian are filling}. Every homotopy class of $(1+\epsilon)$-\qf surfaces with $0<\epsilon\leq \epsilon_0$ separates any pair of distinct points at $\partial \tm$. 
\end{cor}
Two $\pi_1$-injective surfaces $S_1, S_2 \subset M$ are \textit{commensurable} if there exists a $g\in \pi_1(M)$ such that $g\pi_1(S_1)g^{-1}\cap\pi_1(S_2)$ is of finite index in both $\pi_1(S_1)$ and $\pi_1(S_2)$. This means their preimages in $\tm$ and quasiconformal constants are the same. A commensurability class of surfaces is \textit{filling} if each of its homotopy classes is filling. We always assume surfaces are orientable and $\pi_1$-injective unless stated otherwise. A surface is \textit{Fuchsian} or \textit{totally geodesic} if any geodesic on $S$ with respect to its induced Riemannian metric is also a geodesic on $M$. A sequence of \eqf minimal surfaces is  \textit{asymptotically Fuchsian} if $\epsilon\rightarrow 0$.
The proof of \Cref{almost all quasi-Fuchsian are filling} uses \Cref{limit of almost geodesic surfaces converge to dense subset} and \Cref{dense implies filling}. 
\begin{theorem}\footnote{Independently, Al Assal-Lowe \cite{flInprep} obtain a similar result with different perspectives and applications.}\label{limit of almost geodesic surfaces converge to dense subset}
	Let $M$ be a closed hyperbolic $3$-manifold. If $S_j$ is a sequence of asymptotically Fuchsian surfaces in distinct commensurability classes, then the Grassmann $2$-plane bundle $\gt S_i \rightarrow \gt M$ in the Hausdorff metric. \footnote{On the other hand, Al Assal \cite{afLimitsAsymptoticalFuchsian} shows that the measure limits achieve every convex combination of $\sltr$-invariant ergodic measures.}
\end{theorem}
Lowe in [\citenum{lbDeformationsTotallyGeodesic}, Theorem 1.6] proves similar density results for asymptotically Fuchsian surfaces for a family of negatively curved metrics with some assumption on the limiting circles. \Cref{dense implies filling} then establishes filling by the following argument. Suppose there exist nontrivial $[\gamma_i]\in \pi_1(M-S_i)$. Then a subsequence of $(x_i,v_i) $ in the unit tangent bundle $T^1\gamma_i$ of the geodesic $\gamma_i$ converges to $(x,v)$ tangent to a geodesic $\gamma$. There exist $(p_j, \Pi_j) \in \gt S_j \rightarrow (x, \Pi)\perp (x,v)$. Since $\epsilon_j \rightarrow 0$, large $S_j$ are nearly orthogonal to $\gamma_j$, and then $\partial\tsj$ separates $\partial \tgj$ which means $\pi_1(M-S_j)$ cannot contain $\gamma_j$. We need to use \cite{saMinimalDiscsHyperbolicSpace} which controls the norms of the second fundamental forms of $\tsj$ and \cite{sgArzelaAscoliSubmanifolds} to prove $\tsj$ converge to a toally geodesic plane $\tp$. The proof is heavily inspired by Calegari-Marques-Neves \cite{cmnCountingminimal} and Kahn-Markovic-Smilga \cite{kmsGeometricallyTopologicallyRandomsurface}. A \textit{geodesic plane} in $M$ is the image of a totally geodesic plane $\tp \subset \hy \rightarrow M$. The proof of \Cref{limit of almost geodesic surfaces converge to dense subset} uses crucially \cite{rmRatner'stheorem,snRatnerTheorem} which implies a geodesic plane in $M$ is either dense or a closed Fuchsian surface, and uses \cite{msErgodicInvariantMeasure} which implies that a sequence of Fuchsian surfaces in distinct commensurability classes equidistribute (become uniformly dense in $\gt M$). Since $S_i$ are minimal, their Grassmann bundles are connected closed sets, and their Hausdorff limit is a connected closed subset $\gt S$ of $\gtm$. Any finite union of $\gt \mH_j$ for $\mH_j$ Fuchsian is a disconnected closed set. Thus $\gt S$ which is closed and $\psltr$-invariant must contain a plane which is dense in $\gtm$. 
\Cref{almost all quasi-Fuchsian are filling} is inspired by \Cref{theorem large Fuchsian filling}. Let $\gc(S)$ denote the smallest genus of an orientable surface in $M$ \cble \, to $S$.
\begin{theorem}\label{theorem large Fuchsian filling}
	Let $M$ be a closed hyperbolic $3$-manifold. Then there exists a constant $g_0$ such that every Fuchsian surface $S$ with $\gc(S) \geq g_0$ is filling. Moreover, the number of components of $M-S$ tends to infinity as $\gc(S)\rightarrow \infty$. 
\begin{figure}[h]
\centering
\includegraphics[scale=0.2]{filling-nbhd.png}
\caption{The filling nature of nearly geodesic surfaces.}
\label{filling-nbhd}
\end{figure}
\end{theorem}
This figure is inspired by [\citenum{kmsGeometricallyTopologicallyRandomsurface}, Figure 1]. The $(1+\epsilon)$-\qf minimal surfaces lie below the line because of Seppi's curvature estimates \cite{saMinimalDiscsHyperbolicSpace}.
In a hyperbolic manifold, by the Gauss-Bonnet theorem and the Gauss equation, the area $T$ of a minimal surface $S$ is $\leq 4\pi(g-1)$, with equality if and only if $S$ is geodesic. \Cref{filling fuchsian surface intersects every geodesic} proves a filling Fuchsian surface necessarily has some transverse intersection with all geodesics in $M$. In various proofs, it is helpful for the author to view transverse intersections and filling as an ``open" property of $\gtm$ and the space of $\pi_1$-injective surfaces in $M$. By \cite{raGeodesicSurfacesHyperbolic,bfmsTotallyGeodesicSubmanifolds}, \Cref{theorem large Fuchsian filling} is nontrivial only when $M$ is a closed arithmetic hyperbolic $3$-manifold which contains a Fuchsian surface. 

An outline of the proof of \Cref{theorem large Fuchsian filling} is the following. Let $S_i$ be a sequence of Fuchsian surfaces in distinct \cbility classes. \cite{msErgodicInvariantMeasure} implies that the $2$-plane bundles $\gt S_i$  equidistribute in $\gtm$ as $\gc(S_i)$ grows (thus for every open subset of $\gtm$, $\gt S_i$ must eventually all intersect it, which means $S_i$ approach every point of $M$ in every direction uniformly). Thus $S_i$ have to intersect the thin neighborhoods of all the sides of a Voronoi tessellation constructed by a maximum number of points $\Lambda_n=\{p_j\}$ whose pairwise distance is $\geq 1/n$. This also uses an idea from \cite{dsCountingGeodesics} which detects filling submanifolds by constructing flow boxes using sufficiently fine triangulations. \Cref{transverse intersection is an open property} says the subset of $\gtm$ which corresponds to transversally intersecting a curve $\gamma$ in $M$ contains an open set of $\gtm$ of measure $\geq \min\{2\inj(M), \ell(\gamma)\}.$ Thus eventually $S_i$ have to separate every pair of points in $\Lambda_n$. This shows that the number of components of $M-S_i$ has to tend to infinity. \Cref{filling fuchsian surface intersects every geodesic} proves if a Fuchsian surface $S$ intersects a geodesic $\gamma$ transversally, then $S$ separates the endpoints $\partial \tg \subset \partial \hy$ and $\pi_1(M-S)$ does not contain $[\gamma]$. Combined with the equidistribution of $\gt S_i$, this implies large $S_i$ have to transversely intersect every closed geodesics, and thus proves large Fuchsian surfaces are filling. \cite{lmPolynomialEffectiveDensity} points out the rate of equidistribution and hence $g_0$ depends only on the rate of mixing, $\vol(M)$, and the injectivity radius. 

 One way to visualize \Cref{theorem large Fuchsian filling} and its transition to \Cref{almost all quasi-Fuchsian are filling} is the following. Let $D$ be a convex polyhedron fundamental domain for $M$ in $\hy$. The generators of $\pi_1 \rightarrow \Gamma \leq \psltc$ are given by finitely many side-pairings of $\partial D$, i.e., $\{ g\in \Gamma: g\bd \cap \bd \neq \emptyset\} $. 
  The preimages of large filling Fuchsian surface give rise to some geodesic planes that cut $D$ transversally  and $\partial D$ into components so that any geodesic that passes through $D$ has to intersect one of the planes transversally. 

In addition to being an $SL_2(\R)$-orbit in $\hy$, a geodesic plane can be defined as a submanifold with principal curvature $0$. A slight perturbation to these geodesic planes gives rise to minimal planes with small principal curvatures and limit sets with small quasi-Fuchsian constants. Since the polyhedron fundamental domain is compact, finitely many evenly distributed almost-geodesic planes remain candidates for filling. In other words, although the proof of \Cref{theorem large Fuchsian filling} relies on the rigidity of $\sltr$-orbits, the topological filling nature persists for nearby surfaces which are not orbits of homogeneous subgroups. \Cref{almost all quasi-Fuchsian are filling} is a manifestation of certain stable topological properties on the space of $\pi_1$-injective surfaces of a hyperbolic $3$-manifold. 

Kahn-Markovic \cite{kmImmersingAlmostGeodesic}, in resolving the surface subgroup conjecture, proves the following dense distribution regarding the quasi-Fuchsian surfaces. 

\begin{theorem}[\citenum{bwBoundaryCriterionCubulation}, Corollary 4.2]\label{separation of geodesics}
	Let $M$ be a closed hyperbolic $3$-manifold. Each pair of distinct points $p, q \in \partial \widetilde{M}$ is separated by a quasi-Fuchsian surface $F$ in $ M$. 
\end{theorem}
This theorem along with other important results of Agol, Bergeron-Wise, and Sageev have important consequences in geometric group theory and virtual properties of $3$-manifolds, which we refer the readers to \cite{bwBoundaryCriterionCubulation}, \cite{afw3manifoldGroups}, and the references therein. 
In \Cref{filling surface intersects every geodesic}, we prove that a homotopy class of filling surfaces separates any pair of distinct points at $\partial \hy$. \Cref{almost all quasi-Fuchsian are filling} then gives \Cref{all but finitely many separate any pairs of distinct points}. 

Although the property of being filling sorts out a subclass of submanifolds of a hyperbolic $3$-manifold, it also reflects certain topological complexity of the ambient manifold. Recall from Thurston [\citenum{twThurstonThreemanifold}, Definition 3.8.1] that a geometric $3$-manifold is a Riemannian $3$-manifold supporting one of the eight model geometries. 
\begin{theorem}\label{filling 3-manifold is hyperbolic}
	If $M$ is a geometric $3$-manifold and admits a $\pi_1$-injective filling surface, then $M$ is hyperbolic, Euclidean, or $\hp \times \R$.
\end{theorem}
In the proof we explicitly construct some filling surfaces; see \Cref{fillingflat} for the Euclidean manifolds and \Cref{fillinghyperbolic} for the $\hp \times \R$ ones.

Before moving on to motivations, we talk about an elementary inequality relating the volume of a hyperbolic $3$-manifold and the area of its filling surfaces. In a general Riemannian $3$-manifold $M$, the filling surface cuts $M$ into balls and thus naturally the area of $S$ and the volume of $M$ satisfy the isoperimetric inequality for $M$. Hyperbolic manifolds can be characterized as manifolds satisfying linear isoperimetric inequalities. Denote $\rank(M)$ the smallest number of generators of $\pi_1(M)$. 
 \begin{theorem}\label{isoperimetric inequality filling surface has area at least volume}
	Let $M$ be a closed hyperbolic $3$-manifold and $S$ be a filling $\pi_1$-injective surface. Then 
	\begin{equation}\label{area of S and M}
		\area(S) > \vol(M),
	\end{equation}
	and there is an absolute constant $C$ such that 
	\begin{equation}\label{rank of a filling surface and hyperbolic manifold}
		\rank(S)>C\rank(M).
	\end{equation}
\end{theorem}
Current estimates give $C$ an order of $10^{-9}$. The qualitative linearity between the rank of $S$ and $M$ in \eqref{rank of a filling surface and hyperbolic manifold} contrasts other geometries: we construct a sequence of $3$-manifolds $M_j$ with $\hp\times \R$ geometry such that $\rank(M_j) \rightarrow \infty$ but all admit a filling torus in \Cref{example filling torus in rank tend to infinity}. The proof of \eqref{area of S and M} is a natural consequence of the linear isoperimetric inequality for hyperbolic $3$-space. The proof of \eqref{rank of a filling surface and hyperbolic manifold} relies on \eqref{area of S and M} and \cite{bglsCountingArithmeticLattices} which bound the rank from below by the volume for symmetric space. Cooper \cite{cdPresentationLengthVolume} bounds the volume by the presentation length which inspires \eqref{rank of a filling surface and hyperbolic manifold}. 

\subsection{The motivations from hyperbolic geometry and random geodesics}
The motivations for \Cref{theorem large Fuchsian filling} are twofold: one from the statistics of the topological properties of closed geodesics on a random hyperbolic surface, in particular, the work of Wu and Xue \cite{wxPrimeGeodesicTheorem}, and the other from the rigidity of unipotent flows. We start by describing the motivation from random hyperbolic surfaces, which is also related to the linear isoperimetric inequality of hyperbolic geometry and the Gauss-Bonnet theorem. 

The polar coordinate for the $2$-dimensional hyperbolic metric is 
$$g=\di r^2 +\sinh^2 r \di \theta^2.$$
A ball of radius $r$ has an area of $2\pi (\cosh r-1)$ and a perimeter of $2\pi \sinh r$. Let a contractible region $B$ with area equal to $\area(B_r)$. The linear isoperimetric inequality 
$$ \area(B) \leq  \ell (\partial B_r).$$
 A genus-$g$ hyperbolic surface $S_g$ has an area of $4\pi(g-1)$. Using the linear isoperimetric inequality, its fundamental domain $D\subset \hp$ has a boundary length of at least $4\pi(g-1)$, and thus a filling geodesic must have a length of at least $2\pi(g-1)$. On $S_g$, there are only finitely many closed geodesics of length less than $2\pi(g-1)$. In [\citenum{wxPrimeGeodesicTheorem}, Question on p. 5], Wu and Xue ask the following.  
\begin{q}
	As $g \rightarrow \infty$, on a generic closed hyperbolic surface, are
	most closed geodesics of length significantly greater than $g$ filling?
\end{q}
Dozier and Sapir, in this inspiring paper \cite{dsCountingGeodesics}, prove that almost every closed geodesic of length much greater than $g(\log g)^2$ on a large genus surface is filling. In particular, in [\citenum{dsCountingGeodesics}, Section 5.1], they construct a sufficiently fine triangulation $\mathcal{D}$ and flow boxes centered at the edges of $\mathcal{D}$ so that \textit{most} sufficiently long geodesics pass through all these boxes and hence are filling. Similar ideas apply to surfaces in $3$-manifolds and detect filling, and for \Cref{theorem large Fuchsian filling} the crucial difference is that a large Fuchsian surface \textit{has to} get close to the face of a sufficiently fine triangulation, due to rigidity of $\sltr$-invariant orbit explained below. 

\subsection{The motivations from the rigidity of unipotent flows and the work of Mozes-Shah, Ratner, and Shah}\label{motivationRatner}
There is a huge variety of invariant sets of geodesic flows on the unit tangent bundle $TS$ of a hyperbolic surface, i.e., dense [\citenum{khModernDynamics}, 17.5, Corollary 18.3.5], closed geodesics, geodesic laminations [\citenum{bmIntroductionGeometricTopology}, 8.3], and sets with Hausdorff dimension greater than $1$ but less than $3$ [\citenum{lsVariationsTheorem}, Theorem 1.3]. The flexibility of such invariant sets is also reflected by the existence of arbitrarily long closed geodesics supported on a proper subsurface of $S$, which are never filling. Although Fuchsian surfaces have principal curvature $0$ and geometrically generalize geodesics on hyperbolic surfaces, it is their connection with dynamic system that gives rise to filling topological nature in \Cref{theorem large Fuchsian filling} and \Cref{almost all quasi-Fuchsian are filling}. 

Let $G$ be a connected Lie group and $U$ a unipotent one-parameter subgroup of $G$. In contrast with geodesic flows, there is a strong rigidity for the orbits of horocycle flow ($U$-orbits): Hedlund's theorem \cite{hgHorocyclesFlowMinimal}, which says the horocycle flow on a closed hyperbolic surface is minimal, implies that the projection of a horocycle $H \subset \hp$ to a closed hyperbolic surface is always dense. 
Shah \cite{snRatnerTheorem} classifies the orbit closure of $SO(n-1,1)$ for actions on quotients of $SO(n,1)$ for $n\geq 3$; Etienne Ghys pointed out the implications in geometry: the projection of a codimension-one geodesic hyperplane $P\subset \mathbb{H}^n$ to a closed hyperbolic $n$-manifold $M$ is either dense or a closed totally geodesic hypersurface. For $n \geq 3$, the stark contrast between $SO(n-1, 1)$ and geodesic $SO(1,1)$ relies crucially on the behavior of the orbits of nontrivial unipotent $1$-parameter subgroups contained in $SO(n-1,1)$, as can be seen in [\citenum{snRatnerTheorem}, Section 4-6]. Margulis \cite{mgDiscreteSubgroupsErgodicTheory} shows how to effectively study orbit closures in homogeneous spaces of Lie groups by analyzing minimal closed invariant sets of the action of unipotent subgroups. Some other important works include, for example, Dani-Smillie, Dani, Dani-Margulis, which we refer the readers to the references in e.g. \cite{mgDiscreteSubgroupsErgodicTheory,snRatnerTheorem,snUniformlydistributedOrbits,rmRatner'stheorem}. 

 Ratner \cite{rmRatner'stheorem} classifies finite ergodic $U$-invariant measure and proves every unipotent subgroup of a connected Lie group $G$ is strictly measure rigid, which establishes Raghunathan’s measure conjecture. Shah proves a theorem [\citenum{snUniformlydistributedOrbits}, 1.1] which describes geometrically a subset of points whose $U$-orbits are uniformly distributed with respect to a $G$-invariant measure; in particular, combining with the measure classification \cite{rmRatner'stheorem}, he verifies a stronger form of Raghunathan’s orbit closure conjecture for cocompact quotients of $G$ [\citenum{snUniformlydistributedOrbits}, 1.3]. 
 \cite{msErgodicInvariantMeasure} implies that a sequence of Fuchsian surfaces $S_i$ in distinct \cbility classes becomes equidistributed, i.e., the probability measure induced by the inclusion of $S_i$ on the Grassmann bundle $\gtm$ converges to the uniform Liouville measure on $\gtm$, building on \cite{rmRatner'stheorem}. There is a rapid development of effective Ratner's theorem. As mentioned in \cite{lmPolynomialEffectiveDensity}, the rate of equidistribution, i.e., the rate that a large Fuchsian surface approaches a point from different directions only depends on coarse properties like rate of mixing, volume, and injectivity radius. Thus, it is possible to effectively compute the $g_0$ in \Cref{theorem large Fuchsian filling} which guarantees every Fuchsian surface not commensurable with a surface of genus $<g_0$ is filling (it does not necessarily give the smallest such $g_0$). Doing so requires techniques entirely different from the current paper and we postpone it to a future project. Readers are referred to \cite{lmPolynomialEffectiveDensity,lmwEffectiveEquidistribution,lyEffectiveRatner}, and the references therein. 
 
Calegari-Marques-Neves \cite{cmnCountingminimal} initiated the studies of the distribution of minimal surfaces in negatively curved manifolds using homogeneous dynamics and rigidity results. Lowe \cite{lbDeformationsTotallyGeodesic} studies the minimal foliations of Grassmann bundles of negatively curved manifolds. He also proves various quantitative density and measure convergence results for asymptotically Fuchsian surfaces in non-homogeneous metrics, e.g. [\citenum{lbDeformationsTotallyGeodesic}, Theorem 1.9]. Jiang \cite{jrMinimalSurfaceEntropyCusped} generalizes the minimal surface entropy results of \cite{cmnCountingminimal} to higher dimensional and  noncompact finite-volume hyperbolic manifolds. Thurston in [\citenum{twThurstonThreemanifold}, Remark after Corollary 8.8.6] points out that unlike closed hyperbolic $3$-manifolds, a noncompact finite-volume hyperbolic manifold might have infinitely many homotopy classes of $\pi_1$-injective surfaces of bounded genus. In \cite{jrMinimalSurfaceEntropyCusped}, Jiang proves that the number of homotopy classes of nearly geodesic surfaces of bounded genus remains finite. 

An outline of the paper is the following. In \Cref{prelim}, we recall some backgrounds on \qf and minimal surfaces, and the metric and measure on Grassmann bundles. In \Cref{proofs}, we prove \Cref{theorem large Fuchsian filling}, \Cref{almost all quasi-Fuchsian are filling}, and \Cref{all but finitely many separate any pairs of distinct points}. In \Cref{other}, we classify geometric $3$-manifolds containing a filling $\pi_1$-injective surface, construct explicit filling surfaces in Euclidean and $\hp\times \R $ geometry, and prove \Cref{isoperimetric inequality filling surface has area at least volume}. 