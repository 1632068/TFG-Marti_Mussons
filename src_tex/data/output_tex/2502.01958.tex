\documentclass[12pt,a4paper]{article}
%\usepackage[cp1251]{inputenc}
\usepackage[utf8]{inputenc}
\usepackage[T2A]{fontenc}
\usepackage[english,russian]{babel}
\usepackage{amscd,amssymb}
\usepackage{amsfonts,amsmath,array}
%\usepackage[dvips]{graphicx}
\usepackage{longtable,wrapfig}
\usepackage{amsthm}
\usepackage{tikz}

\textheight= 25cm \textwidth= 16cm \hoffset= -1.cm
\voffset= -2.4cm

\newtheorem{mydef}{Определение}
\newtheorem{theorem}{Теорема}
\newtheorem{notation}{Обозначение}
\newtheorem{corollary}{Следствие}[theorem]
\newtheorem{lemma}{Лемма}
\newtheorem{observation}{Наблюдение}
\renewcommand{\qed}{$\blacksquare \vspace{5mm}$}

\title{\large\bf Хроматическое число плоскости при разбиении дугами ограниченной кривизны}
\author{Георгий Соколов, Всеволод Воронов}

\begin{document}
	\maketitle
	\vspace{20 pt}

\begin{abstract}
\end{abstract}

\section{Введение}

\subsection{История и мотивировка задачи}

В работе рассматривается одна из модификаций задачи Хадвигера-Нельсона о хроматическом числе плоскости. В классической постановке задачи требуется найти наименьшее число цветов, которое потребуется для такой раскраски евклидовой плоскости $\mathbb{R}^2$, чтобы любые две точки, находящиеся на единичном расстоянии, имели разный цвет. Это число называется хроматическим числом плоскости и обозначается $\chi(\mathbb{R}^2)$. Из работ \cite{deGrey} известно, что
\[
    5 \leq \chi(\mathbb{R}^2) \leq 7.
\]

Существует несколько постановок задачи, в которых вводятся более жесткие ограничения на раскраску. В частности, можно задать вопрос: какое число цветов потребуется, если плоскость разбита на некоторые одноцветные области? Кроме того, можно определить вид этих областей. Если требовать, чтобы множества точек, раскрашенных в каждый цвет, были измеримыми, то возникает задача об измеримом хроматическом числе плоскости, $\chi_{mes}(\mathbb{R}^)$, для которого было показано \cite{falconer1981realization}, что 
\[
    5 \leq \chi_{mes}(\mathbb{R}^2) \leq 7.    
\]
Усиливая ограничения, можно потребовать, чтобы границы между областями были непрерывными образами отрезков (жордановыми дугами). Обозначим такое хроматическое число $\chi_{map}(\mathbb{R}^2)$. Из работ \cite{wormald,townsend2005colouring} известно, что
\[
    6 \leq \chi_{map}(\mathbb{R}^2) \leq 7.    
\]

Далее, можно предположить, что границы являются ломаными, одноцветные области многоугольниками, и, кроме того, выполняются некоторые дополнительные условия, например, ограничение на площадь одноцветных областей. В работе \cite{coulson2004chromatic} было независимо дано доказательство для этого частного случая 
\[
    6 \leq \chi_{poly}(\mathbb{R}^2) \leq 7.    
\]

Разумеется,
$\chi_{\{1\}}(\mathbb{R}^2) \leq \chi_{mes}(\mathbb{R}^2) \leq \chi_{map}(\mathbb{R}^2) \leq \chi_{poly}(\mathbb{R}^2) $.

В работе \cite{manta2021triangle} было показано, что для раскраски плоскости, разбитой на треугольники, необходимо и достаточно семи цветов.

В 2003 Экзо предложил постановку задачи о хроматическом числе плоскости с запрещенным интервалом расстояний. В этом случае требуется, чтобы отсутствовала пара одноцветных точек, расстояние между которыми принадлежит интервалу $(1 - \varepsilon, 1+\varepsilon)$. Обозначим соответствующее хроматическое число $\chi_{(1 - \varepsilon, 1+\varepsilon)}(\mathbb{R}^2)$. В 2023 было показано, что при  $0<\varepsilon<1.15$ выполнено  равенство 

\[
    \chi_{(1 - \varepsilon, 1+\varepsilon)}(\mathbb{R}^2)=7.
\]

В настоящей работе это равенство распространено на случаи $\chi_{poly}(\mathbb{R}^2)$ и $\chi_{map}(\mathbb{R}^2)$ при некоторых дополнительных условиях. В первом случае любая ограниченная область должна содержать конечное число многоугольников, во втором мы требуем дополнительно, чтобы любая единичная окружность пересекалась с границами областей по конечному множеству точек.

\subsection{Формулировка основных результатов}

\begin{mydef}
    Правильной $k$-раскраской плоскости или правильной раскраской плоскости в $k$ цветов будем называть такую раскраску $c: \mathbb{R}^2 \to \{ 1,2, \dots , k\}$, что для любых точек $x,y \in \mathbb{R}^2$, находящихся на единичном евклидовом расстоянии 1, выполнено $c(x) \neq c(y)$. 
\end{mydef}

Далее мы всюду рассматриваем правильные раскраски плоскости. Для краткости будем иногда опускать слово ``правильная''.

\begin{mydef}
    Жорданова (правильная) раскраска плоскости --- такая раскраска, что %любые две точки на расстоянии $1$ покрашены в разные цвета и 
    плоскость разбивается на вершины, границы и области, где вершины --- точки, границы --- гладкие кривые с концами в вершинах и области --- связные части, на которые плоскость разбивается вершинами и границами, причём все точки одной области покрашены в один цвет. Кроме того, все границы являются границами между двумя областями, все вершины являются концами границ. Любые две области, имеющие общую границу, покрашены по-разному.
\end{mydef}


\begin{mydef}
    Жорданова раскраска плоскости называется локально конечной, если любое ограниченное подмножество плоскости пересекает лишь конечное число областей и границ.
\end{mydef}

Далее мы везде будем рассматривать локально конечные раскраски. Для краткости будем часто опускать это условие.

%думаю, лучше локальную конечность сделать отдельным определением

% Или можно так определить
%\begin{mydef}
%    Жорданова раскраска плоскости --- такая раскраска, что любые две точки на расстоянии $1$ покрашены в разные цвета и плоскость разбивается на вершины, границы и области, где вершины --- точки, границы --- гладкие кривые с концами в вершинах и области --- связные части, на которые плоскость разбивается вершинами и границами, причём все точки одной области покрашены в один цвет. Кроме того, все границы являются границами между двумя областями, все вершины являются концами границ, граница каждой области является объединением конечного числа границ и существует такая константа $C > 0$, что все области имеют площадь не меньше $C$.
%\end{mydef}

\begin{mydef}
    Многоугольная раскраска плоскости --- такая жорданова раскраска, что все области являются многоугольниками.
\end{mydef}

\begin{mydef}
    Степень вершины --- количество границ, концами которых является вершина.
\end{mydef}

\begin{mydef}
	Будем называть мультицветом точки множество цветов, в которые покрашены области, в замыкании которых находится эта точка; будем называть цветностью точки мощность её мультицвета. Точки цветности $1$, $2$ и $3$ будем называть одноцветными, двухцветными и трёхцветными соответственно.
\end{mydef}

Заметим, что все точки внутри областей одноцветные, точки на границе областей  двухцветные,% или одноцветными (если области, разделённые границей, покрашены в один цвет)
цветность вершины не превосходит её степени. Цветность может быть меньше степени вершины, если цвета некоторых областей, замыкания которых содержат вершину, совпадают.

Основным результатом работы являются следующие теоремы.

\begin{theorem} \label{t1}
	Не существует локально конечной многоугольной раскраски плоскости в $6$ цветов.
\end{theorem}

\begin{theorem} \label{t2}
	Не существует локально конечной жордановой раскраски плоскости в $6$ цветов, при которой любая граница пересекает любую окружность радиуса $1$ не более, чем в конечном числе точек, и нет трёхцветных вершин степени больше $3$.
\end{theorem}

% Так пока не получилось
%\begin{theorem} \label{t2}
%	Не существует жордановой раскраски плоскости в $6$ цветов, при которой ни у какой границы ни в какой точке кривизна не равна $1$ и нет трёхцветных вершин степени больше $3$.
%\end{theorem}

\begin{observation} \label{l0.1}
    Если в жордановой раскраске любая граница пересекает любую окружность радиуса $1$ не более, чем в конечном числе точек,то
    \begin{enumerate}
        \item Границы не содержат дуги окружности радиуса $1$.
        \item У любой точки есть окрестность, содержащаяся в объединении областей, в замыкании которых лежит эта точка, границ, кончающихся в ней и этой точки, если она является вершиной.
        \item Любая окружность радиуса $1$ разбивается на конечное число одноцветных дуг.
    \end{enumerate}
\end{observation}

В частности, эти свойства выполнены для любой многоугольной раскраски.

% Или это всё очевидно и не надо доказывать? -- да, пожалуй
% \textbf{Доказательство леммы \ref{l0.1}.}
% \begin{enumerate}
%     \item Если граница содержит дугу окружности радиуса $1$, она пересекает эту окружность в бесконечном числе точек.
%     \item Зафиксируем точку и рассмотрим произвольную её окрестность. По определению жордановой раскраски она пересекает лишь конечное число областей и границ. Любая вершина является концом границы, поэтому эта окрестность содержит конечное число вершин. Расстояние между точкой и любой не содержащей её в замыкании областью, не кончающейся в ней границей и не совпадающей с ней вершиной больше нуля, поэтому $\varepsilon$-окрестность с $\varepsilon$ меньшим, чем минимальное из этих расстояний, содержится в объединении областей, в замыкании которых лежит эта точка, границ, кончающихся в ней, и этой точки, если она является вершиной.
%     \item Окружность радиуса $1$ пересекает конечное число границ и содержит конечное число вершин, каждая граница пересекает её в конечном числе точек, поэтому вершины и пересечения с границами разбивают окружность на конечное число дуг. Каждая из этих дуг одноцветна, так как содержится в одной области.
% \end{enumerate} \qed


Далее по умолчанию считаем, что плоскость раскрашена в $6$ цветов жордановой раскраской, удовлетворяющей свойствам из Наблюдения \ref{l0.1}. Рассматривая окрестность точки, будем всегда брать достаточно малую окрестность, чтобы она содержалась в объединении областей, в замыкании которых лежит эта точка и границ, кончающихся в ней.

\section{Простейшие свойства раскраски}

Далее будем обозначать $T_1(x) \subset \mathbb{R}^2$ окружность единичного радиуса с центром в точке $x$. Для краткости иногда будем говорить, что точки $u$, $v$ соединены непрерывной кривой $uv$, если из контекста понятно, о какой кривой идет речь, или если рассматривается произвольная непрерывная кривая с концами в точках $u$, $v$.

\begin{lemma} \label{l1.1}
	Если две точки $u$ и $v$ одного цвета соединены непрерывной одноцветной кривой $uv$, то не существует точек этого цвета, лежащих на расстоянии меньше $1$ от одной из них и больше $1$ от другой.
\end{lemma}

\textbf{Доказательство леммы \ref{l1.1}}. Если найдется такая точка $z$, то $T_1(z) \cap uv \neq \emptyset$, и раскраска не является правильной.%Не нарушая общности, считаем, что эта кривая первого цвета. Допустим, существует точка $C$ первого цвета на расстоянии меньше $1$ от одной из точек $A$ и $B$ и больше $1$ от другой. Тогда на окружности радиуса $1$ с центром $C$ нет точек первого цвета. Но одна из точек $A$ и $B$ лежит внутри этой окружности, а другая снаружи, следовательно соединяющая их непрерывная кривая первого цвета  пересекает эту окружность. Получили противоречие. \qed

%такие вещи можно сокращать

\begin{lemma} \label{l1.2}
	Если две точки $u$ и $v$ лежат на границе двух одноцветных областей, то никакая точка $z$, для которой $\|z-u\|<1$, $\|z-v\|>1$, не покрашена ни в один из цветов этих областей.
\end{lemma}

\textbf{Доказательство леммы \ref{l1.2}}. Пусть такая точка $z$ найдется. Предположим для определенности, что $u,v$ лежат на границе областей цветов 1, 2.  Тогда в произвольно малых окрестностях $u$, $v$ найдутся точки $u_1$, $v_1$ цвета 1 и точки $u_2, v_2$ цвета 2. Можно выбрать эти точки так, чтобы $\|z-u_i\|<1$, $\|z-v_i\|>1$, $i=1,2$. Остается применить для пар точек $u_1, v_1$ и $u_2, v_2$ Лемму \ref{l1.1}. %Допустим противное. Не нарушая общности считаем, что одна из областей, на границе которых лежат $A$ и $B$ покрашена в первый цвет и существует точка $C$ первого цвета, лежащая на расстоянии меньше $1$ от $A$ и больше $1$ от $B$. В любой окрестности точек $A$ и $B$ есть точки $A'$ и $B'$, лежащие в одноцветной области цвета $1$ и поэтому соединённые непрерывной кривой цвета $1$. Взяв достаточно маленькую окрестность, получим, что расстояние между $A'$ и $C$ меньше $1$, а между $A'$ и $C$ --- больше $1$, что противоречит предыдущей лемме. \qed

\begin{lemma} \label{l1.3}
	Если точка является одноцветной, то никакая точка на расстоянии $1$ от неё не содержит в своём мультицвете её цвет.
\end{lemma}

\textbf{Доказательство леммы \ref{l1.3}}. Предположим противное. Пусть $A$ --- одноцветная точка (не нарушая общности считаем, что она покрашена в цвет $1$), $U_A$ --- её окрестность цвета $1$, $B$ --- точка на расстоянии $1$ от $A$, в мультицвет которой входит $1$. Пусть $C$ и $D$ --- такие точки в $U_A$, что $C$ лежит на расстоянии меньше $1$ от $B$, а $D$ --- на расстоянии больше $1$. По определению мультицвета в любой окрестности $B$ есть точка цвета $1$. При достаточно маленькой окрестности эта точка тоже будет на расстоянии меньше $1$ от $C$ и больше $1$ от $D$, что противоречит лемме \ref{l1.1}. \qed

\begin{lemma} \label{l1.4}
	Если точки $A$ и $B$ расположены на расстоянии $1$ друг от друга и точка $A$ соединена непрерывными кривыми цвета $1$ с такими точками $C$ и $D$, что расстояние от $B$ до $C$ больше $1$, а до $D$ меньше $1$ (причём сама точка $A$ может быть покрашена в другой цвет), то $B$ не содержит цвет $1$ в своём мультицвете.
\end{lemma}

\textbf{Доказательство леммы \ref{l1.4}}. Допустим, $B$ имеет цвет $1$ в своём мультицвете. Тогда в любой окрестности $B$ есть точка цвета $1$, не лежащая на окружности с центром $A$ и радиусом $1$. Пусть $E$ --- такая точка в окрестности $B$ достаточно малой для того, чтобы расстояние между $E$ и $C$ было больше $1$, а между $E$ и $D$ --- меньше $1$. Тогда окружность с центром $E$ и радиусом $1$ пересекает кривую $CAD$, причём точка пересечения не совпадает с $A$ (так как по выбору $E$ расстояние между $A$ и $E$ не равно $1$) и следовательно покрашена в цвет $1$. Получили, что это точка и точка $E$ лежат на расстоянии $1$ и покрашены в один цвет, что противоречит правильности раскраски. \qed

\begin{lemma} \label{l1.5}
	Не существует двух двухцветных точек с одинаковым мультицветом на расстоянии $1$.
\end{lemma}

\textbf{Доказательство леммы \ref{l1.5}}. Как обычно доказываем от противного, считаем, что эти точки имеют мультицвет $\{1, 2\}$. Пусть $A$ и $B$ --- эти точки, $A$ покрашена в цвет $1$, в $B$ --- в цвет $2$ (они покрашены в разные цвета, так как находятся на расстоянии $1$). Пусть $U_A$ и $U_B$ --- окрестности точек $A$ и $B$ соответственно, в которых не встречаются никакие цвета, кроме $1$ и $2$. Тогда пересечение $U_B$ с окружностью радиуса $1$ с центром $A$ покрашено в цвет $2$. Пусть точки $C$ и $D$ лежат на этом пересечении по разные стороны от $B$. Тогда по лемме \ref{l1.1} симметрическая разность кругов радиуса $1$ с центрами $C$ и $D$ не содержит точек цвета $2$ и следовательно пересечение этой симметрической разности с $U_A$ покрашено в цвет $1$. Но это пересечение содержит в качестве внутренних точек точки на окружности радиуса $1$ с центром $B$, что по лемме \ref{l1.3} противоречит тому, что $1$ содержится в мультицвете $B$. \qed

\begin{lemma} \label{l1.6}
    Пусть $\omega$ --- окружность радиуса $1$ с центром в $O$, точки $A$ и $B$ лежат на $\omega$ на расстоянии $1$ друг от друга, причём и у $A$, и у $B$ одна полуокрестность в $\omega$ покрашена в цвет $1$, а другая --- в цвет $2$. Пусть также границы между одноцветными областями не содержат дуг окружности радиуса $1$. Тогда и внутри $\omega$, и снаружи $\omega$ к точкам $A$ и $B$ примыкают области цвета, отличного от $1$ и $2$.
\end{lemma}

\textbf{Доказательство леммы \ref{l1.6}}.
Пусть полуокрестность $A$ на $\omega$ со стороны точки $B$ покрашена в цвет $1$. Тогда полуокрестность $A$ с противоположной от $B$ стороны покрашена в цвет $2$, полуокрестность $B$ со стороны $A$ --- в цвет $1$ и полуокрестность $B$ с противоположной от $A$ стороны --- в цвет $2$. Тогда никакая точка из пересечения окрестности $B$ в $\mathbb{R}^2$ с кругом радиуса $1$ с центром $A$ не может быть покрашена в цвет $2$, так как иначе получается противоречие с леммой \ref{l1.4} для этой точки, точки в дальней от $A$ полуокрестности $B$ на $\omega$ и точки цвета $1$ в окрестности $A$. Аналогично никакая точка из пересечения окрестности $B$ с внешностью круга радиуса $1$ с центром в $A$ не может быть покрашена в цвет $1$. Поскольку дуга окружности с центром $A$ не является границей одноцветных областей, пересечение любой окрестности $B$ и с внутренностью, и с внешностью круга радиуса $1$ с центром $O$ содержит цвета, отличные от $1$ и $2$. \qed

\section{Точки на границе более, чем трёх одноцветных областей}

В этом разделе мы докажем, что в любой раскраске плоскости в $6$ цветов с границами, являющимися жордановыми кривыми без дуг радиуса $1$, нет четырёхцветных точек. Более того, если границы областей являются ломаными, то можно так перекрасить окрестности точек, лежащих на границе более, чем трёх одноцветных областей, что раскраска останется правильной и любая точка будет лежать в замыкании не более, чем трёх одноцветных областей.

Заметим, что в статье Triangle colorings require at least seven colors (надо сделать нормальную ссылку, пока не понял, как тут ссылки делаются), в которой доказывается похожее утверждение, на раскраску накладывается значительно более сильное условие: разные одноцветные области, покрашенные о один цвет, не могут иметь общую точку границы.

\begin{theorem} \label{t3}
	В жордановой раскраске в $6$ цветов, удовлетворяющей свойствам из леммы \ref{l0.1}, не существует точек цветности больше $3$. 
\end{theorem}

\textbf{Доказательство теоремы \ref{t3}}

Очевидно не существует пятицветных точек (так как иначе окружность радиуса $1$ с центром в пятицветной точке вся, кроме конечного числа точек, должна быть покрашена в один цвет).

Допустим, $O$ --- четырёхцветная точка, $\omega$ --- окружность радиуса $1$ с центром $O$. Не нарушая общности считаем, что $O$ имеет мультицвет $\{3, 4, 5, 6\}$. Тогда $\omega$ разбивается на одноцветные дуги цветов $1$ и $2$. Пусть $A_1$ --- какая-то точка на границе одноцветных дуг, $A_2, A_3, A_4, A_5, A_6$ --- такие точки на $\omega$, что $A_1A_2A_3A_4A_5A_6$ --- правильный шестиугольник. Тогда точки $A_2, A_3, A_4, A_5, A_6$ тоже находятся на границе одноцветных дуг. По лемме \ref{l1.6} и снаружи $\omega$, и внутри $\omega$ к любому $A_i$ примыкает область цвета, отличного от $1$ и $2$, причём из леммы \ref{l1.4} для центра круга и $A_i$ получаем, что цвета областей внутри и снаружи круга различаются. Из отсутствия пятицветных точек следует, что изнутри и снаружи круга радиуса $1$ с центром $O$ к каждой точке $A_i$ примыкает ровно по одной области цветов, отличных от $1, 2$. Не нарушая общности можно считать, что к точке $A_1$ примыкают области цветов $3$ и $4$. По лемме \ref{l1.3} никакая точка на окружности с центром $A_2$ и радиусом $1$ не может быть одноцветной точкой цвета $1$ или $2$. Дуги окружностей не могут быть границами одноцветных областей, поэтому в пересечении окружности с центром $A_2$ и радиусом $1$ с окрестностью $A_1$ есть одноцветные точки цветов $3$ и $4$ и следовательно по лемме \ref{l1.3} цвета $3$ и $4$ не входят в мультицвет $A_2$. Значит к точке $A_2$ примыкают области цветов $5$ и $6$. Аналогично получаем, что к точкам $A_3$ и $A_5$ примыкают области цветов $3$ и $4$, а к $A_4$ и $A_6$ --- цветов $5$ и $6$. Значит для каждого из цветов $3, 4, 5, 6$ найдутся две точки на расстоянии (в смысле длины дуги между ними) $\frac{2\pi}{3}$, к которым области этого цвета примыкают с одной и той же стороны (либо изнутри окружности, либо снаружи). Пусть $\alpha_i$ --- полуплоскость, содержащая $A_i$, границей которой является проходящая через $O$ прямая, перпендикулярная $OA_i$. Тогда если граница какого-то цвета примыкает к $A_i$ изнутри окружности, то покрашенная в этот цвет часть окрестности $O$ лежит в $\alpha_i$, а если примыкает снаружи --- то в $\alpha_{i+3}$. Но из этого замечания и того, что для любого цвета есть две точки на расстоянии $\frac{2\pi}{3}$, к которым области этого цвета примыкают с одной и той же стороны, следует, что каждая пересечение окрестности $O$ с каждым из цветов $3, 4, 5, 6$ лежит в секторе размера $\frac{\pi}{3}$. Получили противоречие. \qed

\begin{figure}
    \centering
    \input{images/4_colors.tikz}
    \caption{4-color point}
    \label{fig:enter-label}
\end{figure}




\begin{lemma} \label{l6.1}
	Есть при правильной раскраске плоскости точки $A, O, B$ таковы, что они не лежат на одной прямой и все внутренние точки отрезков $OA$ и $OB$ покрашены в цвет $1$, то существуют такие точки $C$ и $D$ на отрезках $OA$ и $OB$ соответственно, что раскраска останется правильной, если перекрасить все внутренние точки треугольника $COD$ в цвет $1$.
\end{lemma}

\textbf{Доказательство леммы \ref{l6.1}}. Пусть $I$ --- точка на биссектрисе угла $AOB$ на расстоянии $1$ от $O$. Пусть $E$ --- точка на отрезке $OA$, расстояние от $I$ до которой меньше $1$ (такая точка существует, так как угол $AOI$ меньше $\pi / 2$). Пусть $\varepsilon$ такое, что расстояния от $O$ до $A$ и до $B$ больше $\varepsilon$, а расстояние от $I$ до $E$ меньше $1 - \varepsilon$. Положим $C, D$ --- точки на расстоянии $\varepsilon$ от $O$ на отрезках $OA$ и $OB$. Проверим, что точки $C$ и $D$ подходят под условие. Допустим, это не так. Тогда какая-то точка на расстоянии $1$ от точки внутри треугольника была покрашена в цвет $1$, обозначим её как $F$. Единичная окружность с центром $F$ пересекает внутренность треугольника $COD$, но не пересекает его стороны $OC$ и $OD$, следовательно она дважды пересекает сторону $CD$ в каких-то точках $G$ и $H$, причём вся дуга $GH$ лежит внутри треугольника. Пусть $K$ --- середина дуги $GH$. Тогда вектора $KF$ и $OI$ равны, так как они перпендикулярны отрезку $CD$ и имеют длину $1$. Значит $OK=IF$, поэтому расстояние от $I$ до $F$ меньше $\varepsilon$ и следовательно расстояние от $F$ до $E$ меньше $1$. Но тогда $O$ лежит на расстоянии больше $1$ от $F$, а $E$ --- на расстоянии меньше $1$, причём точка $F$ и отрезок $OE$ покрашены в цвет $1$, что противоречит правильности изначальной раскраски. \qed

\begin{lemma} \label{l6.2}
	Если плоскость покрашена многоугольниками так, что $O$ --- трёхцветная точка, в которой сходится более трёх отрезков границы, то можно перекрасить некоторую окрестность $O$ так, что раскраска останется правильной раскраской многоугольниками, количество отрезков границы, кончающихся в $O$ уменьшится и новых точек, в которых сходится более трёх отрезков границы, не появится.
\end{lemma}

\textbf{Доказательство леммы \ref{l6.2}}. Рассмотрим достаточно маленькую окрестность $O$ чтобы все границы в ней были отрезками с концом $O$. Выберем точки $A$ и $B$ из разных областей одного цвета в этой окрестности так, что они не лежат с $O$ на одной прямой и являются внутренними точками своего цвета, и применим для них лемму \ref{l6.1}. \qed

\begin{theorem} \label{t4}
	Если существует многоугольная раскраска в $6$ цветов, то существует такая многоугольная раскраска в $6$ цветов, что в круге радиуса $100$ нет точек, в которых сходятся более трёх отрезков границы.
\end{theorem}

\textbf{Доказательство теоремы \ref{t4}}. Фиксируем произвольный круг радиуса $100$ и пока в нём есть трёхцветные точки, в которых сходится более трёх отрезков границы, применяем к ним лемму \ref{l6.2}. \qed

\section{Трёхцветная полоска вокруг трёхцветной точки}

Пусть $O$ --- трёхцветная точка, в которой сходятся три границы между одноцветными областями. Не нарушая общности, считаем, что она имеет мультицвет $\{4, 5, 6\}.$ Изучим раскраску окрестности окружности радиуса $1$ с центром в этой точке. Мы докажем, что если границы между одноцветными областями не содержат дуги окружности радиуса $1$, то в любой окрестности этой окружности есть деформированное кольцо (то есть фигура, во внутренности которой содержится замкнутая кривая, внутри которой лежит $O$), раскрашенное в цвета $1, 2, 3$.

Рассмотрим окружность доскаточно маленького радиуса $\delta$ с центром $O$. Зафиксируем точки $A_{45}, A_{46}$ и $A_{56}$ пересечения этой окружности с границами областей цветов $4$ и $5$, цветов $4$ и $6$ и цветов $5$ и $6$ соответственно (если какая-то граница пересекает эту окружность несколько раз, выберем любую точку пересечения).

\begin{lemma} \label{l5}
	Зафиксируем луч из точки $O$. Пусть $\alpha$ --- второй по величине из трёх углов между этим лучом и лучами $OA_{45}, OA_{46}, OA_{56}$. Тогда интервал этого луча с концами на расстоянии $1$ и $\delta \cos \alpha + \sqrt{1 - \delta^2 \sin^2\alpha}$ от $O$ покрашен в цвета $1, 2, 3$.
\end{lemma}

\textbf{Доказательство леммы \ref{l5}}. Точка, расположенная на рассматриваемом луче на расстоянии $\delta \cos \alpha + \sqrt{1 - \delta^2 \sin^2\alpha}$ будет на расстоянии $1$ от второй по удалению от неё из точек $A_{45}, A_{46}, A_{56}$, точки, расположенные ближе к $O$ на этом луче будут на расстоянии, меньшем единицы, хотя бы от двух из этих трёх точек, а точки, расположенные на луче на больше от $O$ расстоянии --- на расстоянии больше единицы хотя бы от двух из этих точек. Поэтому при $\delta \cos \alpha + \sqrt{1 - \delta^2 \sin^2\alpha} > 1$ точки из интервала  $(1,\delta \cos \alpha + \sqrt{1 - \delta^2 \sin^2\alpha})$ будут лежать на расстоянии больше единицы от точки $O$ и меньше единицы от двух из точек $A_{45}, A_{46}, A_{56}$ и следовательно по лемме \ref{l1.2} не могут быть покрашены ни в один из цветов $4, 5, 6$ (так как на границе каждого из этих цветов лежит точка $O$ и хотя бы одна из любых двух из этих трёх точек). Аналогично при  $\delta \cos \alpha + \sqrt{1 - \delta^2 \sin^2\alpha} < 1$ точки из интервала  $(\delta \cos \alpha + \sqrt{1 - \delta^2 \sin^2\alpha}, 1)$ лежат на расстоянии меньше единицы от $O$ и больше единицы от двух из трёх точек и поэтому не могут быть покрашены в цвета $4, 5, 6$. \qed

\begin{figure}
    \centering
    \input{images/3_color_interval.tikz}
    \label{fig:enter-label}
\end{figure}


Применив эту лемму для всех $\alpha$ мы получим, что любой в окрестности окружности радиуса $1$ существует раскрашенная в цвета $1, 2, 3$ отрытая полоска, которая сжимается в точку (возможно выколотую) в тех направлениях, для которых в обозначениях леммы $\delta \cos \alpha + \sqrt{1 - \delta^2 \sin^2\alpha} = 1$.

\begin{mydef}
	Будем называть полученную фигуру, покрашенную в цвета $1, 2, 3$ полоской с дырками, а направления, для которых $\delta \cos \alpha + \sqrt{1 + \delta^2 \sin^2\alpha} = 1$ --- плохими.
\end{mydef}

%Наверное надо дать более строгое определение полоски с дырками (с учётом того, что в разделе 3.3 я хочу наклеивать на дырки <<заплатки>> и продолжать называть то, что получится, полоской с дырками), но пока я не понимаю, как его правильно сформулировать.

%\subsection{Объединение нескольких полосок}

Осталось избавиться от выколотых точек в плохих направлениях. Заметим, что для разных $\delta$ мы получаем разные полоски с дырками и на самом деле объединение таких полосок для разных $\delta$ должно быть покрашено в цвета $1, 2, 3$. Поэтому достаточно проверить, что никакое направление не является плохом для всех маленьких $\delta$, выбрать для каждого плохого направления по такому значению $\delta$, что оно не является плохим, и объединить исходную полоску с полученными таким образом полосками для всех этих $\delta$. Чтобы проверить это, заметим, что если направление является плохом, то точка, расположенная на расстоянии $1$ от $O$ в этом направлении, находится на расстоянии $1$ от второй по удалённости точки пересечения границы с окружностью радиуса $\delta$. Если это выполнено для всех достаточно маленьких $\delta$, то эта граница является дугой окружности радиуса $1$. Получаем следующее утверждение.

\begin{theorem} \label{t5}
	Если точка $O$ является трёхцветной, из неё выходят границы между каждой парой цветов из её мультицвета и никакая из этих границ не является ни в какой окрестности $O$ дугой окружности радиуса $1$, то в любой окрестности окружности радиуса $1$ с центром $O$ есть покрашенная в цвета $1, 2, 3$ область, замыкание которой содержит эту окружность, а внутренность --- замкнутую кривую, внутри которой лежит $O$.  
\end{theorem}

% Не очень понимаю, как правильно сформулировать и строго и коротко доказать теорему и первое следствие, пока так написал. Следствие сформулировано так, что из него, как я понимаю, должна следовать комплиментарность частей этих кривых.

\textbf{Доказательство теоремы \ref{t5}}. Для того, чтобы получить такую область в $\varepsilon$-окрестности окружности, пересечём полученное перед формулировкой теоремы объединение полосок с дырками с внутренностью круга радиуса $1+\varepsilon$ и внешностью круга радиуса $1-\varepsilon$ с центром в $O$. \qed

\begin{corollary} \label{f3.1}
	В условиях теоремы \ref{t3} для любого $\varepsilon$ существует такая кусочно-гладкая замкнутая кривая, что
	\begin{enumerate}
		\item $O$ лежит внутри этой кривой.
		\item Эта кривая лежит во внутренности объединения цветов $1, 2, 3$.
		\item Эта кривая лежит в $\varepsilon$-окрестности окружности с центром $O$ и радиусом $1$.
		\item Для любой точки на этой кривой угол между касательной в этой точке и направлением на точку $O$ лежит в интервале $(1 - \varepsilon, 1 + \varepsilon)$.
	\end{enumerate}
\end{corollary}

\textbf{Доказательство следствия \ref{f3.1}}. Снаружи от окружности радиуса $1$ возьмём дуги окружности радиуса из интервала $(1, 1 + \varepsilon)$, внутри --- дуги окружности радиуса из интервала $(1 - \varepsilon, 1)$, склеим их отрезками, наклонёнными на угол меньше $\varepsilon$ к касательной в их точке пересечения с этими дугами. \qed

\begin{corollary} \label{f3.2}
	При раскраске в $6$ цветов, при которой одноцветные области ограничены жордановыми кривыми, не содержащими дуг окружностей радиуса $1$, не существует двух трёхцветных точек с непересекающимися мультицветами таких, что расстояние между ними меньше $2$ и из каждой из них выходят границы между каждой парой цветов из её мультицвета.
\end{corollary}

\textbf{Доказательство следствия \ref{f3.2}}. При достаточно маленьком $\varepsilon$ кривые из следствия \ref{f3.1} пересекаются и точки пересечения не могут быть покрашены ни в один из шести цветов. \qed

\section{Раскраска круга с центром в трёхцветной точке}

В этом разделе мы докажем следующую теорему.

\begin{theorem} \label{t10}
	Если круг радиуса $3$ раскрашен в $6$ цветов так, что одноцветные области ограничены жордановыми кривыми, не содержащими дуг радиуса $1$, причём нет точек цветности хотя бы $4$ и трёхцветных точек, в которых сходится больше трёх границ областей, то в нём найдётся либо две трёхцветные точки на расстоянии от $1$ до $2$ с одинаковым мультицветом, либо две трёхцветные точки на расстоянии меньше $2$ с непересекающимися мультицветами.
\end{theorem}

% Отсутствие четырёхцветных точек включаем в условие, так как обсуждается раскраска конечного круга, а не всей плоскости, поэтому её нельзя получить из теоремы 3

Зафиксируем раскраску круга радиуса $3$, удовлетворяющую условиям теоремы.

Легко проверить, что на расстоянии не более $1$ от любой точки есть точка цветности хотя бы $3$. Зафиксируем трёхцветную точку $O$ на расстоянии не более $1$ от центра круга. Не нарушая общности, будем считать, что эта точка имеет мультицвет $\{4, 5, 6\}$. Пусть $\omega$ --- окружность радиуса $1$ с центром $O$, а $U$ --- область из теоремы \ref{t3} цветов $1, 2, 3$, в замыкании которой лежит $\omega$. 

\subsection{Раскраска окружности в 3 цвета}

\begin{mydef}
	Будем называть псевдоцветом точки на окружности $\omega$ такой цвет из $\{1, 2, 3\}$, что он встречается в пересечении $U$ с любой окрестностью этой точки.
\end{mydef}

Заметим, что любая точка из $\omega$ имеет хотя бы один псевдоцвет, так как $U$ покрашена в цвета $\{1, 2, 3\}$ и $\omega$ содержится в замыкании $U$; точка имеет два псевдоцвета, если лежит на проходящей через $U$ границе между цветами; никакая точка не может иметь $3$ псевдоцвета, так как иначе она является трёхцветной точкой с мультицветом $\{1, 2, 3\}$ и лежит на расстоянии $1$ от трёхцветной точки $O$ с мультицветом $\{4, 5, 6\}$, что противоречит следствию \ref{f3.2}. Окружность $\omega$ пересекается с границами многоугольника в конечном числе точек, поэтому она разбивается на конечное число дуг с одинаковым псевдоцветом, на границах которых находятся точки с двумя псевдоцветами.

\begin{mydef}
	В этом разделе будем называть раскраской окружности в $3$ цвета разбиение окружности на конечное число дуг и такую раскраску дуг в эти $3$ цвета, что одноцветные дуги не граничат друг с другом. Будем называть внутренние точки одноцветных дуг одноцветными точками и говорить, что они покрашены в цвет дуги, а границы между дугами --- двуцветными точками и будем говорить, что они покрашены в цвета обоих кончающихся в них дуг. Будем называть раскраску правильной, если никакие одноцветные точки на расстоянии $1$ друг от друга не покрашены в один цвет и никакие две двухцветные точки на расстоянии $1$ друг от друга не покрашены в одну пару цветов.
\end{mydef}

Заметим, что в правильной раскраске на расстоянии один от двуцветной точки не может быть одноцветной точки одного из её цветов (так как тогда в окрестростях этих точек есть две одноцветные точки одного цвета на расстоянии $1$ друг от друга), но может быть две двуцветные точки с различными (но пересекающимися) парами цветов на расстоянии $1$ друг от друга.

\begin{lemma} \label{l5.1}
	Раскраска точек $\omega$ в их псевдоцвета является правильной раскраской $\omega$.
\end{lemma}

\textbf{Доказательство леммы \ref{l5.1}}
Очевидно никакие две точки с одним одинаковым псевдоцветом не могут лежать на расстоянии $1$ друг от друга (так как иначе в их окрестностях на $\omega$ есть одноцветные точки на расстоянии $1$). Проверим, что никакие две точки на расстоянии $1$ друг от друга не могут иметь одинаковое множество из двух псевдоцветов. Допустим, $A$ и $B$ --- точки псевдоцвета $\{1, 2\}$ на расстоянии $1$ друг от друга. Тогда по лемме \ref{l1.6} и внутри, и снаружи $\omega$ к $B$ примыкают области цветов, отличных от $1$ и $2$. Пересечение окрестности $B$ с внутренностью или с внешностью $\omega$ лежит в полосе, покрашенной в цвета $1, 2, 3$, поэтому одна из примыкающих к $B$ областей покрашена в цвет $3$. Но это противоречит тому, что на $\omega$ нет точек, в любой окрестности которых есть все три цвета из $1, 2, 3$. \qed

\begin{mydef}
	Правильную раскраску окружности в $3$ цвета будем называть циклической, если никакая одноцветная дуга не граничит с обоих сторон с дугами одного и того же цвета (то есть при движении по окружности цвета дуг меняются как $1, 2, 3, 1, 2, 3, \ldots$ или $1, 3, 2, 1, 3, 2, \ldots$).
\end{mydef}

\begin{lemma} \label{l5.2}
	Если окружность радиуса $1$ правильно раскрашена в три цвета, то можно так перекрасить некоторые одноцветные дуги окружности, что раскраска останется правильной и станет циклической, причём при этом перекрашивании не появится новых двуцветных точек, а старые двуцветные точки либо станут внутренними точками одноцветных дуг, либо не изменят цвет.
\end{lemma}

\textbf{Доказательство леммы \ref{l5.2}}

	Покажем, что если раскраска не является циклической, то можно так перекрасить некоторые одноцветные дуги, что раскраска останется правильной и число одноцветных дуг уменьшится, причём старые двуцветные точки либо станут внутренними точками одноцветных дуг, либо не изменят цвет. Повторяя такое перекрашивание, пока раскраска не станет циклической, получим искомое перекрашивание.
	
	Рассмотрим минимальную по длине одноцветную дугу, граничащую с обоих сторон с дугами одного цвета (такая дуга найдётся, так как раскраска не циклическая; если таких дуг несколько, рассмотрим любую из них). Не нарушая общности, можем считать, что эта дуга покрашена в цвет $1$, а соседние с ней дуги --- в цвет $2$. Пусть $A_1$ --- начало, а $B_1$ --- конец этой дуги при движении по часовой стрелке. Если ни одна точка дуги $A_1B_1$ не находится на расстоянии $1$ от точки цвета $2$, то можно перекрасить дугу $A_1B_1$ в цвет $2$. Иначе пусть $C_1$ --- точка на $A_1B_1$, на расстоянии $1$ от которой есть точка $D_1$ цвета $2$. Не нарушая общности, будем считать, что $D_1$ расположена в направлении движения по часовой стрелке от $C_1$. Пусть $A_2, \ldots A_6$ --- такие точки на $\omega$, что $A_1A_2A_3A_4A_5A_6$ --- правильный шестиугольник, вершины которого перечисленны в направлении обхода по часовой стрелке. Аналогично определим $B_1, \ldots, B_6$. Тогда $D_1$ лежит на дуге $A_2B_2$ и покрашена в цвет $2$. Содержащая $D_1$ дуга цвета $2$ вложена в дугу $A_2B_2$, так как дуга $A_1B_1$ с обоих сторон граничит с дугами цвета $2$, и граничит с обоих сторон с дугами цвета $3$, так как иначе один из её концов имеет мультицвет $\{1, 2\}$ и находится на расстоянии $1$ либо от внутренней точки дуги $A_1B_1$ (что противоречит замечанию после определения правильной раскраски окружности), либо от её конца, который тоже имеет мультицвет $\{1, 2\}$ (что противоречит определению правильной раскраски). Таким образом, содержащая $D_1$ дуга цвета $2$ граничит с обоих сторон с дугами одного цвета, поэтому по выбору дуги $A_1B_1$ она должна быть не короче её и следовательно совпадает с дугой $A_2B_2$. Если дугу $A_2B_2$ нельзя перекрасить в цвет $3$ так, чтобы раскраска осталось правильной, то на ней существует точка $C_2$, находящаяся на расстоянии $1$ от точки $D_2$ цвета $3$, причём точка $D_2$ должна находиться на дуге $A_3B_3$, так как дуга $A_1B_1$ не содержит точек цвета $3$. Повторяя предыдущие рассуждения, получаем, что, если никакую дугу нельзя перекрасить в цвет граничащих с ней дуг, то дуга $A_3B_3$ покрашена в цвет $3$ и граничит с дугами цвета $1$, дуга $A_4B_4$ покрашена в цвет $1$ и граничит с дугами цвета $2$, дуга $A_5B_5$ покрашена в цвет $2$ и граничит с дугами цвета $3$, дуга $A_6B_6$ покрашена в цвет $3$ и граничит с дугами цвета $1$. Но тогда можно сделать циклическую перекраску дуг $A_iB_i$: перекрасить дуги $A_1B_1$ и $A_4B_4$ в цвет $2$, дуги $A_2B_2$ и $A_5B_5$ --- в цвет $3$, а дуги $A_3B_3$ и $A_6B_6$ --- в цвет $1$. \qed

\begin{lemma} \label{l5.3}
	В любой циклической правильной раскраске окружности, кроме её разбиения на $6$ одноцветных дуг длины $\frac{\pi}{3}$, для любой пары цветов найдутся $3$ двуцветные точки с таким мультицветом, попарные расстояния между которыми больше $1$.
\end{lemma}

Доказательство такое же, как для интервального хроматического числа (в этой части доказательства для интервального хроматического числа не использовалось то, что запрещён интервал, а не одно расстояние). 

\begin{lemma} \label{l5.4}
	На $\omega$ не может быть трёх точек $A, B, C$ таких, что пересечение окрестности $B$ с внешностью $\omega$ лежит в области из теоремы \ref{t3}, точка $B$ находится на расстоянии $1$ от $A$ и $C$ и имеет псевдоцвета $\{1, 2\}$, причём полуокрестность $A$ со стороны $B$ покрашена в цвет $1$, а полуокрестность $C$ со стороны $B$ --- в цвет $2$.
\end{lemma}

\textbf{Доказательство леммы \ref{l5.4}}

	Пересечение окрестности $B$ с внешностью круга с центром $A$ и радиусом $1$ не содержит точек цвета $1$, а её пересечение в внешностью круга с центром $B$ и радиусом $1$ --- точек цвета $2$, поэтому пересечение любой окрестности $B$ с внешностью $\omega$ содержит точку цвета, отличного от $1$ и $2$, по условию это пересечение лежит в области из теоремы \ref{t3} и следовательно эта точка покрашена в цвет $3$, что противоречит отсутствию точки с мультицветом $\{1, 2, 3\}$ на $\omega$. \qed

\begin{lemma} \label{l5.5}
	На $\omega$ не существует таких точек с двумя псевдоцветами $A_1, \ldots, A_6$, что эти точки являются вершинами правильного шестиугольника, у $A_1$ и $A_4$ полуокрестности на окружности покрашены в цвета $1$ и $2$, у $A_2$ и $A_5$ --- в цвета $2$ и $3$, у $A_3$ и $A_6$ --- в $3$ и $1$, причём в первый цвет из этих пар у $A_i$ покрашена полуокрестность со стороны $A_{i-1}$, а во второй --- со стороны $A_{i+1}$.
	%Среди вершин любого вписанного в $\omega$ правильного шестиугольника найдётся вершина, пересечение окрестности которой с внешностью $\omega$ лежит в области из теоремы \ref{t3}.
\end{lemma}

\textbf{Доказательство леммы \ref{l5.5}}

Предположим противное. По построению области из теоремы \ref{t3} для каждого из трёх углов, на которые окрестность $O$ разбивается отрезками $OA_{45}, OA_{46}, OA_{56}$, существует дуга с центром в пересечении биссектрисы этого угла с $\omega$, пересечение окрестности которой с внешностью $\omega$ лежит в области из теоремы \ref{t3}, причём длина этой дуги равна $\pi - \alpha - \varepsilon$, где $\alpha$ --- величина угла, а $\varepsilon$ зависит от $\delta$ (радиуса окружности $A_{45}A_{46}A_{56}$) и стремится к $0$ при $\delta \rightarrow 0$. Если не все три угла равны, то напротив меньшей из них такая дуга имеет длину больше $\frac{\pi}{3}$ и следовательно содержит одну из вершин $A_1, \ldots, A_6$, что противоречит лемме \ref{l5.4}. Если все три угла равны, то точки $A_1, \ldots, A_6$ должны быть концами дуг длины $\frac{\pi}{3}$ с центрами в продолжениях выгодящих из $O$ границ областей (иначе получаем аналогичное противоречие с леммой \ref{l5.4}). Рассмотрим вершины, лежащие на дуге с центром в пересечении $\omega$ с продолжением $OA_{56}$. Не нарушая общности, считаем, что это вершины $A_1, A_2$. Пусть $U_1$ и $U_2$ --- пересечения окрестностей $A_1$ и $A_2$ соответственно и внешностей двух кругов радиуса $1$ с центрами в соседних вершинах шестиугольника. Аналогично доказательству леммы \ref{l5.4} получаем, что $U_1$ и $U_2$ покрашены в цвета $4, 5, 6$. Но окружности с центром в точках в $U_1$ и $U_2$ пересекают области цветов $5$ и $6$ в окрестности $O$, поэтому они покрашены в цвет $4$. Аналогично доказательству леммы \ref{l5.1} получаем, что тогда граница между цветами $4$ и $2$ в окрестности $A_1$ должна быть дугой окружности радиуса $1$, что противоречит тому, что границы одноцветных областей не содержат дуг окружности радиуса $1$. \qed

\begin{lemma} \label{l5.6}
	Для любой пары цветов из $\{1, 2\}, \{2, 3\}$ и $\{3, 1\}$ на $\omega$ есть три точки с таким псевдоцветом, находящиеся на расстоянии больше $1$ друг от друга.
\end{lemma}

\textbf{Доказательство леммы \ref{l5.6}}

Рассмотрим построенную по псевдоцветам правильную раскраску $\omega$. Проведя перекраску из леммы \ref{l5.2}, получим циклическую раскраску. По лемме \ref{l5.5} эта раскраска не может быть разбиением на $6$ одноцветных дуг длины $\frac{\pi}{3}$, поэтому по лемме \ref{l5.3} найдутся $3$ двухцветные точки с требуемым мультицветом на расстоянии больше $1$ друг от друга. Поскольку при перекрашивании не появлялись новые двухцветные точки, эти точки --- искомые точки в исходной раскраске. \qed

\subsection{Трёхцветные точки в круге}

\textbf{Доказательство теоремы \ref{t10}}

После применение леммы \ref{l5.6} завершаем доказательство как в случае интервального хроматического числа (при этом используется, что ни в какой точке не сходится более трёх границ одноцветных областей).

\section{Комплиментарные кривые}

\begin{mydef}
	Две кривые $\gamma_1, \gamma_2$ называются комплиментарными кривыми мультицвета $C$, где $C$ --- множество из трёх цветов, если все их точки, кроме быть может конечного числа, являются внутренними точками объединения цветов из $C$, кривые одноцветны в некоторой проколотой окрестности каждой точки, не являющейся внутренней точкой объединения цветов из $C$ и существуют такие их непрерывные параметризации $\gamma_1, \gamma_2: [0, 1] \rightarrow \mathrm{R}^2$, что для любого $t \in [0, 1]$ выполнено $||\gamma_1(t) - \gamma_2(t)|| = 1$. 
\end{mydef}


\begin{mydef}
    Пусть  прямая $l =\{x \mid x \in\mathbb{R}\}$, не принадлежащая рассматриваемой плоскости, раскрашена в 3 цвета, причем цвет точки $x\in l$ определен формулой $c(x) = (\lfloor x \rfloor \operatorname{mod} 3) +1$. Рассмотрим непрерывное отображение $f: \gamma([0,1]) \to l$, сохраняющее цвет, т.е. $c(f(x))=c(x)$ всюду, кроме, быть может, конечного числа точек, и, кроме того, $f(0), f(1) \in \mathbb{Z}$. Назовем индексом кривой $\gamma$ число 
    \[\operatorname{Ind(\gamma)}=f(1)-f(0).\]
    %или \lfloor f(1) \rfloor - \lfloor f(0) \rfloor, если не требовать целые значения
\end{mydef}

%посторонний вопрос: что будет с 4-раскрасками триангуляции сферы

\begin{lemma}
    Если  $\gamma_1 \sim \gamma_2$ то 
    \[
        |\operatorname{Ind} \gamma_1 - \operatorname{Ind} \gamma_2|\leq 1.
    \]
\end{lemma}

Достаточно показать, что непрерывные функции $f_1: \gamma_1([0,1]) \to l$, $f_2: \gamma_2([0,1]) \to l$ можно выбрать таким образом, что  выполнено

\[
    |f_1(\gamma_1(t)) - f_2(\gamma_2(t))| < 2, \quad t \in [0,1].
\]

%конструктивно, что происходит при переходе через двуцветную точку.

%единичное расстояние на раскрашенной плоскости ограничивает расстояние на прямой

\begin{corollary}
    Пусть кривые $\gamma_1$, $\gamma_2$, ... , $\gamma_{s}$ раскрашены в 3 цвета и удовлетворяют условиям

    (a) их концы не являются двуцветными, 
    
    (b) $c(\gamma_i(0)) \neq c(\gamma_i(1))$, $i=1,2, \dots, s$,

    (c) $\gamma_i \sim \gamma_{i+1}$, $i=1,2, \dots, s-1$,

Тогда индексы $\gamma_1$, $\gamma_2$, ... , $\gamma_{s}$ либо все положительны, либо все отрицательны. 
\label{ind_sequence}
\end{corollary}

В самом деле, из условия (b) следует, что индексы отличны от нуля, а из предыдущей леммы --- что знаки $\operatorname{Ind} \gamma_i$, $\operatorname{Ind} \gamma_{i+1}$ совпадают. По индукции получаем требуемое.





%Комплиментарные кривые обладают теми же свойствами, какими и в доказательстве для интервального хроматического числа. В силу леммы \ref{l1.5} одновременная смена цвета в противоположном направлении невозможна, а в остальном доказательство повторяет доказательство для интервального хроматического числа.

%Я не очень понимаю, что надо писать в этом разделе, сослаться на статью про интервальное хроматическое число формально нельзя, но и повторять почти дословно доказательство как-то странно.

% можно здесь несколько обобщить утверждение (в той статье я это делать не буду).
% а именно, написать, что будет, если есть цикл из нескольких дуг, в котором при замыкании меняется направление обхода, т.е. $x_1 y_1 \equiv x_2 y_2$, ... ,  $x_k y_k \equiv y_1 x_1$, и на каждой дуге индекс не ноль.

% и можно писать про ``tripod'' вместо пары пересекающихся окружностей, т.е. рассматривать только их часть, которая нужна для этого рассуждения


\section{Две трёхцветные точки с совпадающими мультицветами}

\begin{theorem} \label{t7}
	При раскраске в $6$ цветов, при которой одноцветные области ограничены жордановыми кривыми, не содержащими дуг окружностей радиуса $1$, не существует двух трёхцветных точек с совпадающими мультицветами таких, что расстояние между ними лежит в интервале $(1, 2)$ и из каждой из них выходят границы между каждой парой цветов из её мультицвета.
\end{theorem}

\textbf{План доказательства теоремы \ref{t7}.} Как я понимаю, из следствия \ref{f3.1} легко получить существование параметризации для трёх пар (деформированных) дуг из доказательства для интервального хроматического числа. Остальные условия из определения комплиментарности включены в эти следствия. 

Пусть нашлись такие трехцветные точки $u_1$, $u_2$, что $1 < \|u_1-u_2\|<2$. Выберем некоторую одноцветную точку $p$ в окрестности пересечения окружностей $T_1(x_1)$, $T_1(x_2)$ и одноцветные точки $q_1$, $q_2$, $q_3$ в окрестности пересечений $T_1(p)$ с окружностями $T_1(x_1)$, $T_1(x_2)$. (Fig. ). Такой выбор возможен благодаря условию, наложенному на границы между областями.

Назовем треножником тройку кривых $p q_1$, $p q_2$, $p q_3$ с общим концом $p$, для которых выполнены условия 
\[
   \forall x \in p q_i: \quad|T_1(x) \cap p q_j| = 1, \quad i,j \in \{1,2,3\},\quad j \neq i.
\]

Нетрудно видеть, что при этих условиях кривые комплементарны:
\[
    q_1 p \sim p q_2, \quad q_2 p \sim p q_3, \quad q_3 p \sim p q_1.
\]

Но тогда они не могут быть правильно раскрашены в 3 цвета, поскольку при такой раскраске получим, что
\[
    \operatorname{Ind} q_1 p = - \operatorname{Ind} p q_1,
\]
а с другой стороны, индексы должны иметь один знак согласно следствию \ref{ind_sequence}.

%Дальнейшее доказательство повторяет доказательство для интервального хроматического числа. Как и в прошлом пункте, я не очень понимаю, что тут правильно писать.

\section{Завершение доказательства теорем \ref{t1} и \ref{t2}}

Осталось собрать вместе полученные утверждения и получить доказательство теорем \ref{t1} и \ref{t2}.

\textbf{Доказательство теоремы \ref{t1}.} Допустим, существует многоугольная раскраска в $6$ цветов. Тогда по теореме \ref{t4} существует такая многоугольная раскраска в $6$ цветов, что в круге радиуса $100$ нет точек, в которых сходятся более трёх отрезков границы. По теореме \ref{t3} в этой раскраске нет четырёхцветных точек, поэтому к любому кругу радиуса $3$ внутри этого круга можно применить теорему \ref{t10}. Значит найдётся две трёхцветные точки на расстоянии от $1$ до $2$ с одинаковыми мультицветами, что противоречит теореме \ref{t7}, или две трёхцветные точки на расстоянии меньше $2$ с непересекающимися мультицветами, что противоречит следствию \ref{f3.2}. \qed

\textbf{Доказательство теоремы \ref{t2}.} Допустим, существует жорданова раскраски плоскости в $6$ цветов, при которой любая граница пересекает любую окружность радиуса $1$ не более, чем в конечном числе точек, и нет трёхцветных вершин степени больше $3$.По теореме \ref{t3} в ней так же нет четырёхцветных точек, поэтому к любому кругу радиуса $3$ внутри этого круга можно применить теорему \ref{t10}. Значит найдётся две трёхцветные точки на расстоянии от $1$ до $2$ с одинаковыми мультицветами, что противоречит теореме \ref{t7}, или две трёхцветные точки на расстоянии меньше $2$ с непересекающимися мультицветами, что противоречит следствию \ref{f3.2}. \qed

\section{Заключение}

Условия Теорем 1,2 позволяют обработать некоторые исключительные случаи и провести остальные рассуждения аналогично тому, как это было сделано при запрещенном интервале расстояний. В указанных формулировках все ``особенности'' образуют локально конечное множество точек плоскости. Если исключить окрестности вершин и рассмотреть ограниченную область, то найдется запрещенный интервал расстояний. Может показаться странным, что существование дуг единичной кривизны (или счетного числа пересечений границы с такой дугой) является принципиальной проблемой. Но это закономерно, поскольку в этом случае ``особенность'' имеет хаусдорфову размерность 1, а не 0. Иными словами, при отсутствии таких дуг площадь областей $S = S(\varepsilon)$, после исключения которых получим раскраску с запрещенным интервалом расстояний $[1-\varepsilon, 1+\varepsilon]$, при $\varepsilon\to 0$ имеет асимптотику
$$S(\varepsilon)=\Omega(\varepsilon^2),$$
а если дуги единичной кривизны присутствуют, то
$$S(\varepsilon)=\Omega(\varepsilon^1).$$

Отметим, что если бы была установлена какая-либо оценка на долю плоскости, которую можно правильно покрасить в 6 цветов с запрещенным интервалом, то из этого следовало бы окончательно, что $\chi_{map}(\mathbb(R)^2)=7.$

%Использованные в доказательстве отображения кривых на прямую можно перенести на окружность и рассмотреть с точки зрения алгебраической топологии. В частности, классы отображений раскрашенной в 3 цвета замкнутой кривой на 3-цветную окружность представляют собой  элемент гомотопической группы $\pi_1(S^1) = \mathbb{Z}$. Аналогично,  4-раскраска триангуляции сферы порождает элемент группы $\pi_2(S^2)=\mathbb{Z}$. Данные конструкции могут иметь самостоятельное значение, но неясно, можно ли их применить в вариантах задачи Хадвигера--Нельсона--Эрдёша.

%Кроме того, можно рассмотреть любую поверхность (или даже CW-комплекс) с такой раскраской ячеек в k цветов, что каждая ячейка граничит ровно с одной ячейкой каждого из остальных цветов. К примеру, 7-раскраска вложения $K_7$ в тор. Тогда раскраска может задавать элементы гомотопических групп в более общем случае.

%При раскраске сферы единичного радиуса $S^2_1\subset\mathbb{R}^3$ используется 7 цветов. Отображение $S^2$ в симплекс $\delta_7$ может быть только стягиваемым. (?) Или нет, если отображается на двумерный "скелет", и стягивать можно только в нем?


\bibliographystyle{abbrv}

\bibliography{main}

%\begin{thebibliography}{9}

%\end{thebibliography}

\end{document} 
