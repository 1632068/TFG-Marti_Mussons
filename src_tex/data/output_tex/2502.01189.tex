%%%%%%%% ICML 2025 EXAMPLE LATEX SUBMISSION FILE %%%%%%%%%%%%%%%%%

\documentclass{article}

\input{preamble}

% The \icmltitle you define below is probably too long as a header.
% Therefore, a short form for the running title is supplied here:
\icmltitlerunning{\dynamictitle}

\begin{document}

\twocolumn[
\icmltitle{
\dynamictitle
}

% It is OKAY to include author information, even for blind
% submissions: the style file will automatically remove it for you
% unless you've provided the [accepted] option to the icml2025
% package.

% List of affiliations: The first argument should be a (short)
% identifier you will use later to specify author affiliations
% Academic affiliations should list Department, University, City, Region, Country
% Industry affiliations should list Company, City, Region, Country

% You can specify symbols, otherwise they are numbered in order.
% Ideally, you should not use this facility. Affiliations will be numbered
% in order of appearance and this is the preferred way.
\icmlsetsymbol{equal}{*}

\begin{icmlauthorlist}
\icmlauthor{Guy Ohayon}{equal,techcs}
\icmlauthor{Hila Manor}{equal,techee}
\icmlauthor{Tomer Michaeli}{techee}
\icmlauthor{Michael Elad}{techcs}
%\icmlauthor{}{sch}
%\icmlauthor{}{sch}
\end{icmlauthorlist}

\icmlaffiliation{techcs}{Faculty of Computer Science, Technion -- Israel Institute of Technology, Haifa, Israel}
\icmlaffiliation{techee}{Faculty of Electrical and Computer Engineering, Technion -- Israel Institute of Technology, Haifa, Israel}

\icmlcorrespondingauthor{Guy Ohayon}{guyoep@gmail.com}
\icmlcorrespondingauthor{Hila Manor}{hila.manor@campus.technion.ac.il}

% You may provide any keywords that you
% find helpful for describing your paper; these are used to populate
% the "keywords" metadata in the PDF but will not be shown in the document
\icmlkeywords{Machine Learning, ICML}

% {\vspace{0.75em}%
% \centering
% \includegraphics[width=1\textwidth]{figures/teaser.pdf}
% \vspace{-1em}
% \captionof{figure}{Our proposed scheme (DDCMs) produces visually appealing image samples with high compression ratios (given at the bottom-right corner of each result). \textbf{Left}: Compression result; \textbf{Middle}: Compressed image synthesis; and \textbf{Right}: Real-world image restoration.  
% \vspace{-0.75em}}
% \label{fig:teaser}
% }
\vskip 0.3in

]



% this must go after the closing bracket ] following \twocolumn[ ...

% This command actually creates the footnote in the first column
% listing the affiliations and the copyright notice.
% The command takes one argument, which is text to display at the start of the footnote.
% The \icmlEqualContribution command is standard text for equal contribution.
% Remove it (just {}) if you do not need this facility.

%\printAffiliationsAndNotice{}  % leave blank if no need to mention equal contribution
\printAffiliationsAndNotice{\icmlEqualContribution} % otherwise use the standard text.

\input{sections/abstract}
\input{sections/intro}
\input{sections/related}
\input{sections/background}
\input{sections/method}
\input{sections/compression}
\input{sections/compr_cond_gen}
\input{sections/discussion}

% \clearpage
\input{sections/ack_and_statement}
\clearpage

\bibliographystyle{icml2025}
\bibliography{citations}


\newpage
\appendix
\onecolumn\input{sections/appendix_random_gen}
\clearpage
\input{sections/appendix_compression}
\clearpage
\section{Compressed Conditional Generation Supplementary}\label{appendix:cond_compression}
\input{sections/score-based}
\clearpage
\input{sections/posterior-sampling}
\clearpage
\input{sections/real-world-face-restoration}
\clearpage
\input{sections/cg_and_cfg}
\clearpage
\input{sections/editing}
\end{document}