\documentclass[11pt,letterpaper]{amsart}

\setlength{\oddsidemargin}{.0in}
\setlength{\evensidemargin}{.0in}
\setlength{\textwidth}{6.5in}
\setlength{\topmargin}{-.3in}
\setlength{\headsep}{.20in}
\setlength{\textheight}{9.in}
\usepackage[leqno]{amsmath}
\usepackage{amsfonts}
\usepackage{amssymb}
\usepackage{amsthm}
\usepackage{amssymb}
\usepackage[all]{xy}
\usepackage{graphicx}
%\usepackage{kpfonts}

\usepackage{xpatch}
\usepackage{amsaddr}
\makeatletter
\xpatchcmd{%
\@maketitle}{%
\ifx\@empty\authors \else \@setauthors \fi
}{%
  \ifx\@empty\authors \else \@setauthors \fi

  \ifx\@empty\addresses \else\@setaddresses\fi
}{\typeout{Patch successful}}{\typeout{Patch failed}}
\makeatother

\DeclareMathOperator*{\mystar}{*}

\makeatletter
%Table of Contents
\setcounter{tocdepth}{3}

% Add bold to \section titles in ToC and remove . after numbers
\renewcommand{\tocsection}[3]{%
  \indentlabel{\@ifnotempty{#2}{\bfseries\ignorespaces#1 #2\quad}}\bfseries#3}
% Remove . after numbers in \subsection
\renewcommand{\tocsubsection}[3]{%
  \indentlabel{\@ifnotempty{#2}{\ignorespaces#1 #2\quad}}#3}
%\let\tocsubsubsection\tocsubsection% Update for \subsubsection
%...

\newcommand\@dotsep{4.5}
\def\@tocline#1#2#3#4#5#6#7{\relax
  \ifnum #1>\c@tocdepth % then omit
  \else
    \par \addpenalty\@secpenalty\addvspace{#2}%
    \begingroup \hyphenpenalty\@M
    \@ifempty{#4}{%
      \@tempdima\csname r@tocindent\number#1\endcsname\relax
    }{%
      \@tempdima#4\relax
    }%
    \parindent\z@ \leftskip#3\relax \advance\leftskip\@tempdima\relax
    \rightskip\@pnumwidth plus1em \parfillskip-\@pnumwidth
    #5\leavevmode\hskip-\@tempdima{#6}\nobreak
    \leaders\hbox{$\m@th\mkern \@dotsep mu\hbox{.}\mkern \@dotsep mu$}\hfill
    \nobreak
    \hbox to\@pnumwidth{\@tocpagenum{\ifnum#1=1\fi#7}}\par% <-- \bfseries for \section page
    \nobreak
    \endgroup
  \fi}
\AtBeginDocument{%
\expandafter\renewcommand\csname r@tocindent0\endcsname{0pt}
}
\def\l@subsection{\@tocline{2}{0pt}{2.5pc}{5pc}{}}
\makeatother

\makeatletter
\renewcommand{\@evenhead}{\hfill\textsc{\small Quasi-isometric rigidity for lamplighters with lamps of polynomial growth} \hfill \small{\thepage}}
\renewcommand{\@oddhead}{\hfill \small \textsc{Vincent Dumoncel} \hfill \small{\thepage}}
\makeatother


\usepackage[backend=biber, style=alphabetic, sorting=nty]{biblatex}

\usepackage[colorlinks=true,
            citecolor=red,
            linkcolor=blue,
            linktocpage=true]{hyperref}

% References here 
\addbibresource{main.bib}

\usepackage{
  amsmath, amsthm, amssymb, mathtools, dsfont, units,          % Math typesetting
  graphicx, wrapfig, subfig, float,                            % Figures and graphics formatting
  listings, color, inconsolata, pythonhighlight,               % Code formatting
  fancyhdr,enumerate, enumitem, tcolorbox, framed, thmtools} 

%\usepackage[T1]{fontenc}
%\usepackage{newpxtext,newpxmath}
%\usepackage[p,osf]{scholax}
%\usepackage[p,osf]{scholax}
% T1 and textcomp are loaded by package. Change that here, if you want
% load sans and typewriter packages here, if needed
%\usepackage{amsmath,amsthm}% must be loaded before newtxmath
% amssymb should not be loaded
%\usepackage[scaled=1.075,ncf,vvarbb]{newtxmath}% need to scale up math package
% vvarbb selects the STIX version of blackboard bold.
\usepackage{setspace}
\usepackage[T1]{fontenc}
\usepackage{heuristica}
\usepackage[heuristica,vvarbb,bigdelims]{newtxmath}
\renewcommand*\oldstylenums[1]{\textosf{#1}}

%Here are some user-defined notations
\newcommand{\RR}{\mathbf R}%bold R
\newcommand{\CC}{\mathbf C}  %bold C
\newcommand{\ZZ}{\mathbf Z}   %bold Z
\newcommand{\QQ}{\mathbf Q}   %bold Q
\newcommand{\rr}{\mathbb R}     %blackboard bold R
\newcommand{\cc}{\mathbb C}    %blackboard bold R
\newcommand{\zz}{\mathbb Z}    %blackboard bold R
\newcommand{\qq}{\mathbb Q}   %blackboard bold Q
\newcommand{\ZZn}[1]{\ZZ/{#1}\ZZ}
\newcommand{\zzn}[1]{\zz/{#1}\zz}
\newcommand{\calM}{\mathcal M}  %calligraphic M
\newcommand{\sm}{\setminus} 
\newcommand{\bfa}{\mathbf a}
\newcommand{\bfb}{\mathbf b}
\newcommand{\bfc}{\mathbf c}
\newcommand{\dist}{\mathrm{d}}

\makeatletter
\newcommand*{\defeq}{\mathrel{\rlap{%
                     \raisebox{0.3ex}{$\m@th\cdot$}}%
                     \raisebox{-0.3ex}{$\m@th\cdot$}}%
                     =}
\makeatother
%improving spacing in tables (space above and below characters in a row)
\newcommand{\tfix}{\rule{0pt}{2.6ex}}
\newcommand{\bfix}{\rule[-1.2ex]{0pt}{0pt}}



%Here are commands with variable inputs 
\newcommand{\intf}[1]{\int_a^b{#1}\,dx}
\newcommand{\intfb}[3]{\int_{#1}^{#2}{#3}\,dx}
\newcommand{\marginalfootnote}[1]{%
        \footnote{#1}
        \marginpar[\hfill{\sf\thefootnote}]{{\sf\thefootnote}}}
\newcommand{\edit}[1]{\marginalfootnote{#1}}


%Here are some user-defined operators
\newcommand{\Tr}{\operatorname {Tr}}
\newcommand{\GL}{\operatorname {GL}}
\newcommand{\SL}{\operatorname {SL}}
\newcommand{\Prob}{\operatorname {Prob}}
\newcommand{\re}{\operatorname {Re}}
\newcommand{\im}{\operatorname {Im}}


%These commands deal with theorem-like environments (i.e., italic)
\theoremstyle{plain}
\newtheorem{theorem}{Theorem}[section]
\newtheorem{corollary}[theorem]{Corollary}
\newtheorem{lemma}[theorem]{Lemma}
\newtheorem{proposition}[theorem]{Proposition}
\newtheorem{conjecture}[theorem]{Conjecture}
\newtheorem{claim}[theorem]{Claim}
%These deal with definition-like environments (i.e., non-italic)
\theoremstyle{definition}
\newtheorem{definition}[theorem]{Definition}
\newtheorem{example}[theorem]{Example}
\newtheorem{question}[theorem]{Question}
\newtheorem{remark}[theorem]{Remark}
\newtheorem{fact}[theorem]{Fact}

%This numbers equations by section
\numberwithin{equation}{section}



%This is for hypertext references
\usepackage{color}
\usepackage{hyperref}

% Dessin 
\usepackage{tikz}
\usepackage{amsmath}
\usetikzlibrary{shapes.geometric, positioning}
\usepackage{graphicx}

\usepackage{xcolor}
\hypersetup{
    colorlinks,
    linkcolor={blue},
    linktoc=page
    %citecolor={blue!50!black},
    %urlcolor={blue!80!black}   
}

\begin{document}
\newcommand{\R}{\mathbb{R}}% Real numbers
\newcommand{\dis}{\displaystyle} 
\newcommand{\Si}{\mathbb{S}}
\newcommand{\intB}{\overset{\circ}{B}}
\newcommand{\Po}{\mathcal{P}}
\newcommand{\B}{\mathcal{B}}
\newcommand{\la}{\langle}
\newcommand{\ra}{\rangle}

\newcommand{\Hi}{\mathcal{H}}
\newcommand{\ind}{\mathbf{1}}

\newcommand{\Q}{\mathbb{Q}}	% Rational numbers
\newcommand{\Z}{\mathbb{Z}}	% Integers
\newcommand{\N}{\mathbb{N}}	% Natural numbers
\newcommand{\M}{\mathcal{M}}	% Moyennes sur X
\newcommand{\F}{\mathcal{F}}
% Calligraphic F for a sigma algebra
\newcommand{\El}{\mathcal{L}}	% Calligraphic L, e.g. for L^p spaces

% Math mode symbols for probability
\newcommand{\pr}{\mathbb{P}}% Probability measure
\newcommand{\E}{\mathbb{E}}	% Expectation, e.g. $\E(X)$
\newcommand{\var}{\text{Var}}	% Variance, e.g. $\var(X)$
\newcommand{\cov}{\text{Cov}}	% Covariance, e.g. $\cov(X,Y)$
\newcommand{\corr}{\text{Corr}}	% Correlation, e.g. $\corr(X,Y)$

\newcommand{\rem}{\backslash\backslash}

% Other miscellaneous symbols
\newcommand{\tth}{\text{th}}	% Non-italicized 'th', e.g. $n^\tth$
\newcommand{\Oh}{\mathcal{O}}	% Big-O notation, e.g. $\O(n)$

\newcommand{\Address}{{
\begin{center}
            \textsc{Vincent Dumoncel - Université Paris Cité}
		\textsc{Institut de Mathématiques de Jussieu-Paris Rive Gauche} 
            \textsc{8, Place Aurélie Nemours} 
            \textsc{75013 Paris, France}
		\textit{E-mail address}: \texttt{vincent.dumoncel@imj-prg.fr}
\medskip
\end{center}
}}



\begin{titlepage}
\setcounter{page}{1}
\title[Quasi-isometric rigidity for lamplighters with lamps of polynomial growth]{Quasi-isometric rigidity for lamplighters with lamps \\ of polynomial growth}
\author{\small{Vincent Dumoncel}}
\address{Institut de Mathématiques de Jussieu-Paris Rive Gauche \\ \texttt{vincent.dumoncel@imj-prg.fr}}


\date{February 2025}
\maketitle

\textsc{Abstract.} A quasi-isometry between two connected graphs is measure-scaling if one can control precisely the sizes of pre-images of finite subsets. Such a notion is motivated by the work of Eskin-Fysher-Whyte on lamplighters over $\Z$ ~\cite{EFW12, EFW13} and the work of Dymarz on biLipschitz equivalences of amenable groups ~\cite{Dym05, Dym10}, and led Genevois and Tessera to introduce the scaling group $\text{Sc}(X)$ of an amenable bounded degree graph $X$ in ~\cite{GT22}. The main result of our article is a rigidity property for quasi-isometries between lamplighters with lamps of polynomial growth. Under assumptions on $G$ and $H$, any such quasi-isometry $N\wr G\longrightarrow M\wr H$ must be measure-scaling for some scaling factor depending on the growth degrees of $N$ and $M$. In particular, the scaling group of such wreath products is reduced to $\lbrace 1\rbrace$. As applications, we obtain additional examples of pairs of quasi-isometric groups that are not biLipschitz equivalent. We also give applications to the quasi-isometric classification of some iterated wreath products, and we exhibit the first example of an amenable finitely generated group $H$ which is \textit{lamplighter-rigid}, in the sense that $\Z/n\Z\wr H$ and $\Z/m\Z\wr H$ are quasi-isometric if and only if $n=m$.

% or \hypersetup{linkcolor=black}, if the colorlinks=true option of hyperref is used

\begin{spacing}{1.3}
\tableofcontents 
\end{spacing}





\input{part1}
\input{part2}
\input{part3}
\input{part4}
\input{part5}
\input{part6}


\begin{spacing}{1.3}
\printbibliography
\end{spacing}




% OLD BIBLOGRAPHY 
\end{titlepage}
\end{document}